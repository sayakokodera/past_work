% "newcommand" collections
%=====================  for general =======================%
\renewcommand\thesubsection{\thesection.\Alph{subsection}}


%=====================  for math =======================%
% symbols
\newcommand{\vech}{\bm{h}}
\newcommand{\vecv}{\bm{v}}
\newcommand{\sumjneqk}{\sum_{j \neq k}}
\newcommand{\Identity}[1]{\bm{I}_{ #1 }}
\newcommand{\compconj}[1]{ #1^{*} }



% some special characters
\newcommand{\ii}{\mathrm{i}}
\newcommand{\dd}{\mathrm{d}}
\newcommand{\ee}{\mathrm{e}}
\newcommand{\dirac}{\delta}

\newcommand{\RR}{{\mathbb{R}}}
\newcommand{\NN}{{\mathbb{N}}}
\newcommand{\CC}{{\mathbb{C}}}
\newcommand{\OO}{{\mathcal{O}}}


\newcommand{\Real}{%  
     \operatorname{Re}
}
\newcommand{\Imag}{%  
     \operatorname{Im}
}


% for two norms (NEW 01.12.18 cf. https://tex.stackexchange.com/questions/107186/how-to-write-norm-which-adjusts-its-size)
\newcommand\norm[1]{\left\lVert#1\right\rVert }

%=====================  for Roman numbers =======================%

\newcommand{\rom}[1]{\uppercase\expandafter{\romannumeral #1\relax}}
% from https://tex.stackexchange.com/questions/23487/how-can-i-get-roman-numerals-in-text

%=====================  for TikZ =======================%
\newcommand{\inputTikZ}[2]{%  
     \scalebox{#1}{\input{#2}}  
}

\newcommand{\Nx}{3}
\newcommand{\Ny}{2}
\newcommand{\ddx}{1.5cm}
\newcommand{\ddy}{\ddx}
\newcommand{\ddz}{1.5cm}
\newcommand{\rCircle}{0.07cm}

%%% draw transducer %%%
\newcommand{\drawTransducer}[4]{ % scale,scaley, x, y in axis
	\draw[draw = none, fill = white] (axis cs: #3 - #1, #4 - #2, 0) -- (axis cs: #3 + #1, #4 - #2, 0) -- (axis cs: #3 + #1, #4 + #2, 0) --  (axis cs: #3 + #1, #4 + #2, -#1) -- (axis cs: #3 - #1, #4 + #2, -#1) -- (axis cs: #3 - #1, #4 - #2, -#1) -- (axis cs: #3 - #1, #4 - #2, 0);
	\draw[] (axis cs: #3 - #1, #4 - #2, 0) -- (axis cs: #3 + #1, #4 - #2, 0) -- (axis cs: #3 + #1, #4 - #2, -#1) -- (axis cs: #3 - #1, #4 - #2, -#1) -- (axis cs: #3 - #1, #4 - #2, 0);
	\draw[] (axis cs: #3 + #1, #4 - #2, 0) -- (axis cs: #3 + #1, #4 + #2, 0) -- (axis cs: #3 + #1, #4 +  #2, -#1) -- (axis cs: #3 - #1, #4 +  #2, -#1) -- (axis cs: #3 - #1, #4 - #2, -#1);
	\draw[] (axis cs: #3 + #1, #4 - #2, -#1) -- (axis cs: #3 + #1, #4 +  #2, -#1);
}

%%% draw video camera %%%
\newcommand{\drawCamera}[6]{ %<rotation origin (x, y, z)>, <start x>, <start y>, <start z> in axis, <camera width [dx]>, <camera height [dz]>
	% base
	\draw[rotate around = {30 : (axis cs: #1)}] (axis cs: #2, #3, #4) rectangle (axis cs: #2 + #5, #3, #4 - #6) ;
	% origin of rotation
	\node[campoint] (rotationorg) at (axis cs: #1) {};
	% "trapezoid" part (tip of the camera)
	\draw[rotate around = {30 : (axis cs: #1)}] (axis cs: #2, #3, #4 - 0.25*#6) -- (axis cs: #2 - 0.5* #6, #3, #4) -- (axis cs: #2 - 0.5* #6, #3, #4-#6) -- (axis cs: #2, #3, #4 - 0.75*#6);	
	% cable
	\draw[rotate around = {30 : (axis cs: #1)}] (axis cs: #2 + #5, #3, #4 - 0.5*  #6) .. controls (axis cs: #2 + 1.5* #5, #3, #4 - 0.5*  #6) and (axis cs: #2 + #5, #3, - 0.5*  #6) .. (axis cs: #2 + 1.5* #5, #3, - #6);	
}

%%%%%%%%%%%%%%%%%%%%%%%%%%%%%%%%%%%%%%%%%%%%%
%=================  for data visualization ===================%
%%%%%%%%%%%%%%%%%%%%%%%%%%%%%%%%%%%%%%%%%%%%%

%%%%%%%%%%%%%%%%%%%%%%%%%%%%%%%%%%%%%%%%%%%%%%
%=================== 1D visualization ======================%
%%%%%%%%%%%%%%%%%%%%%%%%%%%%%%%%%%%%%%%%%%%%%%


% SE offset 
\newcommand{\seoffset}[7]{ % <scale size>, <label font size>, <tick font size>, <fname for xdelta = 0mm>, <fname for xdelta = 2.5mm>, <fname for xdelta = 5mm>, <fname for xdelta = 7.5mm>
\scalebox{#1}{
	\begin{tikzpicture}
            \begin{axis}[
                width = 7cm, 
            	   height = 4cm,
                xlabel = {$\xdelta / \lambda$},
                ylabel = {$\SEdag$},
                label style = {font = #2},
                tick label style = {font = #3},
                legend style ={
                	at={(1.4, 0.8)},
                	nodes={scale=0.95, transform shape},
                	font = #3
                }
                ]
                \input{#7} % 7.5mm
                \input{#6} % 5mm
                \input{#5} % 2.5mm
                \input{#4} % 0mm
                
             % legend
             % To insert the legend title
		   	\addlegendimage{empty legend} 
		   	% Legend entries  
		   	\addlegendentry{\SI{7.5}{\milli\metre}}
             \addlegendentry{\SI{5}{\milli\metre}}
             \addlegendentry{\SI{2.5}{\milli\metre}} %{$1.98 \lambda$}
             \addlegendentry{\SI{0}{\milli\metre}} %{$0.5 \lambda$}
             % Title
             \addlegendentry{$|s_{x} - x|$}        
            \end{axis}
	\end{tikzpicture}
	}
}



% GD PE
\newcommand{\gdpe}[8]{ % <scale size>, <label font size>, <tick font size>, <fname for 0.5 lambda>, <fname for 1mm>, <fname for 2.5mm>, <fname for 5mm>, <fname for 7.5mm>
\scalebox{#1}{
	\begin{tikzpicture}
            \begin{axis}[
                width = 7cm, 
            	   height = 4cm,
            	   xmin = -2.1,
            	   xmax = 2.1,
            	   ymin = -5,
            	   ymax = 8.5,
                xlabel = {$\xdelta / \lambda$},
                ylabel = {$\xdeltaopt / \lambda$}, 
                label style = {font = #2},
                tick label style = {font = #3},
                %y dir = reverse,
                %xtick = {0, 10, ..., 30}, %to customize the axis
                %xticklabel = {0, 5, 10, 15}
                ytick = {-4, -2, ..., 8}, %to customize the axis
                extra x ticks={0.76}, 
                extra x tick style={font = #3, tui_orange},%yshift={-1.2em}
                legend style ={
                	at={(1.4, 0.9)},
                	nodes={scale=0.95, transform shape},
                	font = #3
                }
                ]
              % Reverse the input order, so that 7.5mm away is sent to background
                \input{#8} %7.5mm
                \input{#7} %5mm
                \input{#6} %2.5mm
                \input{#5} %1mm
                \input{#4} %0.5 lambda
                
             % legend
             % To insert the legend title
		   	\addlegendimage{empty legend} 
		   	% Legend entries  
             \addlegendentry{\SI{7.5}{\milli\metre}}
             \addlegendentry{\SI{5}{\milli\metre}}
             \addlegendentry{\SI{2.5}{\milli\metre}} %{$1.98 \lambda$}
             \addlegendentry{\SI{1}{\milli\metre}} %{$0.8 \lambda$}
             \addlegendentry{\SI{0.63}{\milli\metre}} %{$0.5 \lambda$}
             % Title
             \addlegendentry{$|s_{x} - x|$}
             
%             % x = 0.76 line
%	            \addplot[gray, dashed, line width = 1pt, mark = ] coordinates{
%	            			(0.76, 0)
%	            			(0.76, -5)
%            		};
             
            \end{axis}
	\end{tikzpicture}
	}
}


% GD SE
\newcommand{\gdse}[8]{ %  <scale size>, <label font size>, <tick font size>, <fname for 0.5 lambda>, <fname for 1mm>, <fname for 2.5mm>, <fname for 5mm>, <fname for 7.5mm>
\scalebox{#1}{
	\begin{tikzpicture}
            \begin{axis}[
                width = 7cm, 
            	   height = 4cm,
            	   xmin = -2.1,
            	   xmax = 2.1,
            	   ymin = -0.1,
            	   ymax = 1.1,
                xlabel = {$\xdelta / \lambda$},
                ylabel = {$\SEdag$},
                label style = {font = #2},
                tick label style = {font = #3},
                %y dir = reverse,
                %xtick = {0, 10, ..., 30}, %to customize the axis
                extra x ticks={0.8}, 
                extra x tick style={font = #3, tui_orange},%yshift={-1.2em}
                %xticklabel = {0, 5, 10, 15}
                legend style ={
                	at={(1.4, 0.9)},
                	nodes={scale=0.95, transform shape},
                	font = #3
                }
                ]
                % Reverse the input order, so that 7.5mm away is sent to background
                \input{#8} %7.5mm
                \input{#7} %5mm
                \input{#6} %2.5mm
                \input{#5} %1mm
                \input{#4} %0.5 lambda
                
             % legend
             % To insert the legend title
		   	\addlegendimage{empty legend} 
		   	% Legend entries  
             \addlegendentry{\SI{7.5}{\milli\metre}}
             \addlegendentry{\SI{5}{\milli\metre}}
             \addlegendentry{\SI{2.5}{\milli\metre}} %{$1.98 \lambda$}
             \addlegendentry{\SI{1}{\milli\metre}} %{$0.8 \lambda$}
             \addlegendentry{\SI{0.63}{\milli\metre}} %{$0.5 \lambda$}
             % Title
             \addlegendentry{$| s_{x} - x |$}
             
%             % x = 0.8 line
%	            \addplot[gray, dashed, line width = 1pt, mark = ] coordinates{
%	            			(0.8, 0.2)
%	            			(0.8, -0.1)
%            		};
                         
            \end{axis}
	\end{tikzpicture}
	}
}

%%%%%%%%%%%%%%%%%%%%%%%%%%%%%%%%%%%%%%%%%%%%%%
%=================== 2D visualization ======================%
%%%%%%%%%%%%%%%%%%%%%%%%%%%%%%%%%%%%%%%%%%%%%%
% image with both x- & y-labels
\newcommand{\imgbothlabels}[4]{% <scale size>, <label font size>, <tick font size>, <png file name>
\scalebox{#1}{
	\begin{tikzpicture}
            \begin{axis}[
                enlargelimits = false,
                axis on top = true,
                axis equal image,
                unit vector ratio= 0.5 1, % change aspect ratio, one of them should be 1
                point meta min = -1,   
                point meta max = 1,
                xlabel = {$x$ [\SI{}{\milli \metre}]},
                ylabel = {$z$ [\SI{}{\milli \metre}]},
                label style = {font = #2},
                tick label style = {font = #3},
                y dir = reverse,
                xtick = {5, 10, 15},
                ytick = {21, 22.5, 24},
                ]
                \addplot graphics [
                    xmin = 5,
                    xmax = 15,
                    ymin = 20.908125,
                    ymax = 24.4125
                ]{#4};
            \end{axis}
	\end{tikzpicture}
	}
}
            
% img with only x-label
\newcommand{\imgxlabel}[4]{% <scale size>, <label font size>, <tick font size>, <png file name>
\scalebox{#1}{
	\begin{tikzpicture}
            \begin{axis}[
                enlargelimits = false,
                axis on top = true,
                axis equal image,
                unit vector ratio= 0.5 1, % change aspect ratio, one of them should be 1
                point meta min = -1,   
                point meta max = 1,
                xlabel = {$x$ [\SI{}{\milli \metre}]},
                %ylabel = {$y / \dy$},
                label style = {font = #2},
                tick label style = {font = #3},
                xtick = {5, 10, 15},
                yticklabel = \empty,
                y dir = reverse,
                ]
                \addplot graphics [
                    xmin = 5,
                    xmax = 15,
                    ymin = 20.908125,
                    ymax = 24.4125
                ]{#4};
            \end{axis}
	\end{tikzpicture}
	}
}



% img with only x-label and cmap
\newcommand{\imgxlabelwithcmap}[4]{% <scale size>, <label font size>, <tick font size>, <png file name>
\scalebox{#1}{
	\begin{tikzpicture}
            \begin{axis}[
                enlargelimits = false,
                axis on top = true,
                axis equal image,
                unit vector ratio= 0.5 1, % change aspect ratio, one of them should be 1
                point meta min = -1,   
                point meta max = 1,
                colorbar,
                colormap = {mymap}{rgb(0.0pt) = (0, 0.22, 0.39) ; rgb(0.43pt) = (0.91, 0.93, 0.96) ; rgb(0.5pt) = (0.97, 0.97, 0.98) ; rgb(0.57pt) = (0.99, 0.93, 0.87) ; rgb(1.0pt) = (0.94, 0.49, 0) ; } ,
                xlabel = {$x$ [\SI{}{\milli \metre}]},
                %ylabel = {$y / \dy$},
                label style = {font = #2},
                tick label style = {font = #3},
                xtick = {5, 10, 15},
                yticklabel = \empty,
                y dir = reverse,
                ]
                \addplot graphics [
                    xmin = 5,
                    xmax = 15,
                    ymin = 20.908125,
                    ymax = 24.4125
                ]{#4};
            \end{axis}
	\end{tikzpicture}
	}
}

