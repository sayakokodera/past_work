\documentclass[48pt]{tikzposter}

\usepackage{pgfplots}
\pgfplotsset{compat=1.14}
\usepackage{enumitem}
\usepackage{wrapfig}

\usepackage[T1]{fontenc}
\usepackage{fbb}

\usepackage{caption}

%%%% bibliography %%%%%
\usepackage{url}
\usepackage{natbib} % required for the citation example
\usepackage{bibentry} % required for the citation example
%
\nobibliography* % required for the citation example
\bibliographystyle{plain}


\tikzposterlatexaffectionproofoff

% set color

\usepackage{color}
 
\definecolor{fri_gray}{rgb}{0.8, 0.8, 0.8}
\definecolor{fri_green_light}{rgb}{0.76, 0.95, 0.81}
\definecolor{fri_green}{rgb}{0.51, 0.87, 0.78}
\definecolor{tui_orange}{rgb}{0.94, 0.49, 0}
\definecolor{tui_blue}{rgb}{0, 0.22, 0.39}

\definecolor{box_white}{cmyk}{0.0361,0.0251,0.0166,0}
\definecolor{text_black}{cmyk}{0.7979,0.7417,0.6916,0.6554}
\definecolor{tui_orange_dark}{cmyk}{0,0.6,1,0}
\definecolor{tui_orange_light}{cmyk}{0.0000,0.0876, 0.1474, 0.0157}
\definecolor{tui_green_dark}{cmyk}{1,0,0.5,0.2}
\definecolor{tui_green_light}{cmyk}{0.0576,0.0041, 0.0000, 0.0471}
\definecolor{tui_blue_dark}{cmyk}{1.0000,0.5000,0.0000,0.6000}
\definecolor{tui_blue_light}{cmyk}{0.0920,0.0440,0.0000,0.0196}
\definecolor{tui_red_dark}{cmyk}{0.0000,1.0000,1.0000,0.2000}
\definecolor{tui_red_light}{cmyk}{0.0000,0.1107,0.1107,0.0078}



%\pgfplotscreateplotcyclelist{tui_dark}{
%{tui_orange_dark,mark=*},
%{tui_green_dark,mark=square},
%{tui_blue_dark,mark=triangle},
%{tui_red_dark,mark=diamond},
%}
%
%\pgfplotsset{
%    cycle list name=tui_dark,
%}

\tikzstyle{block} = [rectangle, draw, fill=gray!20, 
    text width=5.5em, text centered, rounded corners, minimum height=2em]
\tikzstyle{line} = [draw, line width = 1.5mm, ->]%,> =  triangle 60]
\tikzstyle{line2} = [draw, dashed, line width = 2mm, ->]


\newcommand{\inputTikZ}[2]{%  
     \scalebox{#1}{\input{#2}}  
}



\setlist[itemize]{label=\textcolor{tui_orange_dark}{\textbullet}}


\title{Optimizing Visualization and Documentation Process of Scientific Writing}
\institute{TU Ilmenau, Ultrasonic Imaging Team of EMS Research Group}
\author{Sayako Kodera}
\titlegraphic{
\includegraphics[scale = 2.5]{tu_bw_white.pdf}
}
\usetheme{Basic}

\defineblockstyle{sampleblockstyle}{
    titlewidthscale=0.9,
    bodywidthscale=1,
    titleleft,
    titleoffsetx=0pt,
    titleoffsety=0pt,
    bodyoffsetx=0mm,
    bodyoffsety=15mm,
    bodyverticalshift=10mm,
    roundedcorners=5,
    linewidth=2pt,
    titleinnersep=6mm,
    bodyinnersep=1cm
}{
\draw[
    line width = \blocklinewidth,
    color=framecolor,
    fill=blockbodybgcolor,
    rounded corners=\blockroundedcorners
] (blockbody.south west)
rectangle (blockbody.north east);
\ifBlockHasTitle
\draw[
    line width = \blocklinewidth,
    color=framecolor,
    fill=blocktitlebgcolor,
    rounded corners=\blockroundedcorners
] (blocktitle.south west)
rectangle (blocktitle.north east);
\fi
}

% TODO:
% - adapt colors
% -

% Background Colors
\colorlet{backgroundcolor}{tui_orange_light!70}

\begin{document}

\colorlet{framecolor}{tui_orange_dark}
% Title Colors
\colorlet{titlefgcolor}{tui_blue_dark}
\colorlet{titlebgcolor}{tui_blue_dark}
% Block Colors
\colorlet{blocktitlebgcolor}{tui_blue_dark}
\colorlet{blocktitlefgcolor}{white}
\colorlet{blockbodybgcolor}{white}
\colorlet{blockbodyfgcolor}{black}
% Innerblock Colors
\colorlet{innerblocktitlebgcolor}{white}
\colorlet{innerblocktitlefgcolor}{black}
\colorlet{innerblockbodybgcolor}{white}
\colorlet{innerblockbodyfgcolor}{black}
% Note colors
\colorlet{notefgcolor}{black}
\colorlet{notebgcolor}{white}
\colorlet{noteframecolor}{colorTwo}

\useblockstyle{sampleblockstyle}

\maketitle
\begin{columns}

%%%%%%%%%%%%%%%%%%%%%%%%%%%%%%%%%%%% left side %%%%%

\column{0.5}

\block{Introduction}{
\begin{center}
	\includegraphics[scale = 2]{image/meas_sig_7.pdf}
\end{center}
\begin{itemize}
    \item Ultrasonic testing (UT) is one of the investigation methods in non-destructive testing (NDT)
    \item Discontinuities such as cracks or holes in test objects are localized
    \item Raw measurement data are often (post-)processed to enhance its imaging quality
    \item Data are conventionally visually analyzed
\end{itemize}
}

\block{Motivations and Goals}{
% B-scan
\begin{center}
\begin{minipage}{0.3\colwidth}
\begin{center}
	\inputTikZ{1.8}{civa_bscan_zoomed.tex}	
\end{center}
\end{minipage}
% reco
\begin{minipage}{0.3\colwidth}
\begin{center}
	\inputTikZ{1.8}{civa_reco_zoomed.tex}	
\end{center}
\end{minipage}
\end{center}
%
\begin{itemize}
    \item Documents should have sufficient imaging quality for visual analysis
    \item The imaging quality should remain same even when it's enlarged
    \item Multiple images should be comparable in documents
    \item Changes in the data should be reflected in the document automatically
    \item Changes irrelevant to the data (color map, axis label etc) should be made fast and easily
\end{itemize}
}


\block{Problems}{ 
\hspace{0.3cm}\\
\textbf{\Large Example approach : using plots directly in the document }\\
%
\begin{itemize}
	\item The imaging quality becomes poor, when the image is enlarged 
	\item Keeping the coherency of the figure frames and styles is cumbersome
    \item The entire code should run again for making any changes
\end{itemize}
}


\block{Workflow of the Example Approach}{
\begin{centering}
    \begin{tikzfigure}
        \inputTikZ{0.9}{workflow_serial.tex}
    \end{tikzfigure}
\end{centering}
%
\hspace{0.5cm}
%
\begin{itemize}
	\item The entire process is in a sequential chain 
	\item However, only the processed data is required to be visualized
	\item The data processing part is most time consuming
	\item The documentation process is often not directly linked to the visualization tools
\end{itemize}
}
    

%%%%%%%%%%%%%%%%%%%%%%%%%%%%%%%%%%%% right side %%%%%
\column{0.5}

\block{Solution}{
%\begin{flushleft}
%\textbf{Separate the data processing chain and the visualization chain}
%\end{flushleft}
\section*{Separate the visualization process from data processing}
%
%\hspace{0.2cm}
%
\begin{centering}
    \begin{tikzfigure}
        \inputTikZ{0.9}{workflow_separated.tex}
    \end{tikzfigure}
\end{centering}
%
\hspace{0.5cm}
%
\begin{itemize}
    \item Save the processed data at the end of the data processing chain
    \item Load the saved data in the visualization chain
    \item Convert the data into .tex-files with the bridging program
    \item Both chains can be done solely with Python
    \item Lastly insert the generated .tex file in the main document file
\end{itemize}

\section*{Bridging Program}
\begin{itemize}
	\item Create a .tex script regarding pgfplot \cite{PGFPlotsGallery} \cite{PGFPlotsManual}
	\item Specify axis details in the .tex script
	\item 1D case : use "\textbackslash addplot coordinates" in the script and add x and y values from the loaded data
	\item 2D case : save the loaded data (np.ndarray) as png images with plt.imsave
	\item 2D case : use "\textbackslash addplot graphics" in the script \cite{PGFPlotsManual}
	\item Write a .tex-file with the generated script
\end{itemize}
}

\block{Results}{

\begin{center}
\begin{minipage}{0.35\colwidth}
	\centering
	plt.figsave
	\begin{center}
		\includegraphics[scale=1.1]{image/civa_reco_snippet.png}	
		%\caption{Snippet}
	\end{center}
\end{minipage}
\hspace{0.3cm}
\begin{minipage}{0.35\colwidth}
	\centering
	Our Approach
	\begin{center}
		\inputTikZ{1.8}{civa_reco_zoomed.tex}	
		%\caption{Our Approach}
	\end{center}
\end{minipage}
\end{center}
%
\begin{itemize}
	\item Figure frames are easily adjustable to the environment
	\item Details regarding axis labels and ticks can be modified directly in the document (without running other codes)
\end{itemize}
}


\block{Summary}{
\begin{itemize}
	\item Separate the visualization process from the data processing/generation 
	\item Use visualization bridging programs to convert the data into .tex-files
	\item Changes irrelevant to the data can be made directly in the document
	\item \textbf{less stress with those "last minutes changes"}
\end{itemize}

% reference
\bibliography{EuroScipy}
}
\end{columns}
\end{document}
