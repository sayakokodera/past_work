% mathematical notation
\chapter*{Mathematische Notationen}

\section*{Variablen Typen, Funktionen und Matrizen}
\begin{flushleft}
\normalsize
%\renewcommand{\arraystretch}{1.5}  
\begin{tabular}{l l l}
$\bm{v}$ & Vektor\\
$\bm{v}_i$ & Zugriff auf das $i$-te Skalarelement von $\bm{v}$\\
$\bm{U}$ & 2D Matrix \\
$\bm{U}_{ij} $ & Zugriff auf das Skalarelement in $\bm{U}$, wobei $i$ der Reihe und $j$ der Spalte \\ & entsprechen\\ 
$[\bm{U}]_{i}$ &  Zugriff auf das $i$-te Spalte einer 2D Matrix\\
$\vectorize(\bm{U})$ & Operation, wobei die Matrize $\bm{U}$ vektorisiert wird\\
$\Mdict$  & Tensor\\
$\Mdict(i, j, k)$ & Zugriff auf das Skalarelement in $\Mdict$, wobei $i$ der Reihe, $j$ der Spalte und $k$\\ & der dritten Dimension entsprechen \\
\end{tabular}
\end{flushleft}


\section*{Oft verwendete Bezeichnungen und Symbolen}
\begin{flushleft}
\normalsize
\begin{tabular}{l l l}
$c_0$ & Schallgeschwindigkeit\\
$f_{S}$ & Abtastfrequenz \\
$\lambda$ &  Wellenlänge\\
$\dx$  & Abstand zwischen zwei benachbarten Rasterpunkten auf dem Messraster \\ & entlang der $x$-Achse \\
$\dy$  & Abstand zwischen zwei benachbarten Rasterpunkten auf dem Messraster \\ & entlang der $y$-Achse \\
$\dz$  & Örtliches Abtastintervall der Rekonstruktion entlang der $z$-Achse \\
$\dt$  & Zeitliches Abtastintervall der Messung \\
$\N_x$  & Anzahl der Rasterpunkten auf dem äquidistanten Raster entlang der $x$-\\ & Achse \\
$\N_y$  & Anzahl der Rasterpunkten auf dem äquidistanten Raster entlang der $y$-\\ & Achse \\
$\N_z$  & Anzahl der örtlichen Abtastung der Rekonstruktion entlang der $z$-Achse \\
$\N_t$  & Anzahl der zeitliche Abtastung der Messdaten \\
$\N_{\point}$  & Anzahl der Prüfstellen \\
$\bm{a}$ & einzelner Datenvektor, \gls{ascan} \\
$\bm{C}$ & 2D Referenzdaten \\
$\hat{\bm{C}} $ & 2D fehlerhafte Daten \\
%$\mse$ & \acrfull{mse}\\
%$\rmse$ & \acrfull{rmsenormal}\\
%$\rmse^{\dagger}$ & \acrfull{rmse}\\
\end{tabular}
\end{flushleft}