\chapter{Fazit und Ausblick} \label{chap:conclusion}

\section{Fazit der Arbeit}
% main goal
In dieser Arbeit wurden mögliche Fehlerquellen bei der \acrshort{saft} Rekonstruktion der handaufgenommenen Messdaten festgestellt. Darüber hinaus wurden deren Einflüsse auf die Abbildungsqualität der \acrshort{saft} Rekonstruktion untersucht, so dass Maßnahmen dafür abgeleitet werden können. \par
Da zugrindliegende Annahmen des \acrshort{saft} Algorithmus oft für händische Messungen nicht gelten und deshalb \acrshort{saft} konventionell an maschinell aufgenommenen Messdaten angewendet wird, gibt es kaum Vorarbeiten über dessen Anwendungsmöglichkeit für handaufgenommene Messdaten. Aus diesem Grund haben wir mögliche Fehlerquellen in fünf Faktoren unterteilt: grobe Abtastdichte, ungleichmäßige Verteilung der Prüfstellen, niedrige Abdeckung des Ortsbereiches, Ungenauigkeit bei der Positionsbestimmung und variierender Anpressdruck. Über die erste beide Faktoren existieren bereits Vorarbeiten, wodurch deren Effekte geschätzt werden können. Von daher wurden in dieser Arbeit andere Faktoren, wie Positionsungenauigkeiten in Abhängigkeit von Abdeckungen des Ortsbereiches sowie der variierende Anpressdruck, untersucht. Um deren Einflüsse auf die Abbildungsqualität zu evaluieren, wurden die gewonnene Ergebnisse mit den Referenzdaten verglichen. Als Referenzdaten wurde die Abbildungsqualität der \acrshort{saft}-Rekonstruktion genommen, die man mit maschinell aufgenommenen Daten erreichen würde. \par
% posscan
Um den Effekt der Positionsungenauigkeit zu untersuchen, wurden die Messdaten an den Prüfstellen, die unabhängig vom Messraster auf dem Ortsbereich zufällig verteilt sind, synthetisch generiert. Dann wurden die Messpositionen auf dem äquidistanten Messraster quantisiert, so dass die Messdaten mit dem vorgegebenen \acrshort{saft} Algorithmus verarbeitet werden können. Um die durch Positionsquantisierung hervorgerufene Überbetonung eines bestimmten Bereiches zu vermeiden, wurde die \glsfirst{dr} Methode verwendet. Mit dieser Methode wird nur der erste Beitrag pro Messrasterpunkt aufgenommen und alle anderen werden im System nicht mehr gespeichert. Diese Methode ist für eine Echtzeitrekonstruktion gut geeignet, da die Rechnung der Rekonstruktion damit schneller und weniger rechenaufwendig wird \cite{Krieg18SAFTwithSmartInspect}. Durch die Simulation wurde gezeigt, dass eine Abweichung von bis zu $0.1 \lambda$ vernachlässigbar ist. Außerdem wurde auch dargestellt, dass die Abbildungsqualität der \acrshort{saft} Rekonstruktion der fehlerhaften Messdaten auch durch eine höhere Abdeckung des Ortsbereiches verbessert werden kann. Beispielsweise darf bei einer 10\% Abdeckung eine Abweichung von ca. \SI{0.5}{\milli\metre} nicht überschritten werden, während bei einer 25\% Abdeckung die doppelte Abweichung toleriert werden kann. \par
% zscan
Bei der Untersuchung des nicht konstanten Anpressdrucks wurde die durch den variierenden Abstand zwischen dem Testkörper und dem Prüfkopf verursachte Laufzeitänderung als Fehlerquelle betrachtet. Die Messdaten wurden hier auf dem äquidistanten Messraster generiert und dasselbe Gitter wurde für die \acrshort{saft} Rekonstruktion angenommen. Durch die Simulation wurde gezeigt, dass die vertikale Abstandsabweichung von ca. $0.06 \lambda$ bzw. Laufzeitänderung von \SI{25}{\nano\second} vernachlässigbar ist. Darüber hinaus kann eine Abstandsabweichung von $0.16 \lambda$ ($\hat{=}$ \SI{62.5}{\nano\second} Laufzeitänderung) toleriert werden, wenn der passende Öffnungswinkel, das bedeutet eine angemessene Größe des Prüfkopfes, für das Messszenario ausgewählt ist. Allerdings wurde gezeigt, dass bei einer größeren Laufzeitabweichung der ganze Ortsbereich verrauscht und dadurch die laterale Auflösung der \acrshort{saft} Rekonstruktion erheblich beeinträchtigt wird. Durch den Vergleich der gewonnenen Ergebnisse der beiden Simulationen haben wir festgestellt, dass der gleiche \acrshort{rmse} Wert nicht unbedingt die selbe Abbildungsqualität liefert. Von daher ist \acrshort{rmse} nicht zur Evaluierung der Abbildungsqualität geeignet. \par
% contribution of this thesis
Durch die Simulationen dieser Arbeit wurde gezeigt, dass eine gewisse Ungenauigkeit der händischen Messung bei der \acrshort{saft} Rekonstruktion toleriert werden kann. Mit passenden Maßnahmen können die Einflüsse solcher Faktoren sehr gering gehalten oder sogar vernachlässigt werden. Daraus lässt sich schließen, dass wir von \acrshort{saft} auch bei der Rekonstruktion handaufgenommener Messdaten profitieren können. Die gewonnenen Simulationsergebnisse können als ein Indikator verwendet werden, mit dem die Anpassung eines Messunterstützungssystems, das in dieser Arbeit betrachtet wird, einfacher durchgeführt werden kann.


\section{Ausblick}
Diese Arbeit hat gezeigt, dass wir von \acrshort{saft} auch bei der Rekonstruktion handaufgenommener Messdaten profitieren können, wenn wir die nötigen Maßnahmen für mögliche Fehlerquelle anwenden. Das deutet darauf hin, dass dieser Themenbereich weitere Forschungsmöglichkeiten enthält, um ein Messunterstützungssystem zu entwickeln. Diese Möglichkeiten werden wie folgt aufgelistet. 

\paragraph*{Kombinieren mehrerer Fehlerfaktoren}
In dieser Arbeit werden mögliche Fehlerquellen in fünf Aspekten unterteilt und deren einzelne Effekte wurden unabhängig von den anderen Faktoren durch die Vorarbeiten oder unsere Simulationen evaluiert. Nun können diese Faktoren miteinander kombiniert werden, um eine realitätsnähere Simulation durchzuführen. Dadurch können verlässlichere Toleranzwerte für die in dieser Arbeit betrachteten Systemungenauigkeiten hergeleitet werden.

\paragraph*{Verwendung verschiedener Wichtungsmethoden}
Um das Überbetonungsproblem zu behandeln, wurde die \glsfirst{ta} Methode in dieser Arbeit als Wichtungsmethode berücksichtigt. Allerdings können noch verschiedene Wichtungsmethode für den gleichen Zweck verwendet werden. Es wäre dabei aber sinnvoll, wenn die ausgewählten Wichtungsmethoden die Positionsdichte eines betroffenen Bereiches widerspiegeln würde.


\paragraph*{Andere Fehlermaße}
Für die Evaluierung der Abbildungsqualität der gewonnene Ergebnisse wurde hier \acrfull{rmse} wegen seiner Einfachheit und der Allgemeingültigkeit  verwendet. Dennoch wurde durch die Simulationen gezeigt, dass \acrshort{rmse} für die Evaluierung der Abbildungsqualität nicht unbedingt geeignet ist und deshalb eine anderes Fehlermaß erforderlich ist. Es gibt bereits verschiedene Evaluierungsmethoden für die Abbildungsqualität visualisierter Daten, beispielsweise \acrfull{cnr} oder auch die auf \acrfull{hvs} basierte Methode. Das Problem bei unserem Modell war, dass wir die \acrshort{roi} in der Abbildung schwer definieren konnte, da die Streuer im Testkörper als Punktstreuern betrachtet worden sind. In Zukunft könnte ein realistischeres Messmodell, in dem die Streuer eine begrenzte Größe besitzen, betrachtet werden, wodurch diese Methoden eine verlässlichere Evaluierung ermöglichen könnten.
