
\newglossaryentry{ascan} { name={A-Scan}, description={ein Oszillogramm, das an einer Prüfposition aufgenommen wird und in dem die einzelnen Echos über einer Zeitachse dargestellt sind \cite{WSPraktikumUS1}}, plural={A-Scans}}

\newglossaryentry{bscan} { name={B-Scan}, description={eine Profildarstellung des Testkörpers, wobei die \acrshort{tof} der akustischen Energie entlang der vertikalen Achse dargestellt und die Prüfpositionen entlang der horizontalen Achse präsentiert sind \cite{UTDataPresentation}}, plural={B-Scans}}

\newglossaryentry{cscan} { name={C-Scan}, description={Draufsicht von 3D Daten, wobei die relative Signalamplitude oder die \acrshort{tof} Werte bei jeder Prüfposition durch die Farbenintensität dargestellt sind \cite{UTDataPresentation}. In dieser Arbeit wird die \gls{cscan} Darstellung für die Visualisierung der Daten verwendet, da sich dadurch 3D Daten in eine 2D Darstellung transformieren lassen}, plural={C-Scans}}

\newglossaryentry{normal_transducer} {name={Normal-Prüfkopf}, description={ein Ultraschall Prüfkopf, der im Wesentlichen aus einem piezoelektrischen Schwinger, der infolge der elektrischen Anregung durch Spannungsimpulse mechanische Spannungswellen und damit Longitudinalwellen erzeugt, bestehet \cite{UTNormalPruefkopf}}, long={Normal-Prüfkopf}}

\newglossaryentry{pulse_echo} {name={Puls-Echo-Verfahren}, description={Ein Prüfungsverfahren mit Ultraschall, wobei der in den Testkörper eingefügte Ultraschall von einem Prüfkopf gesendet und die reflektierte Welle vom gleichen Kopf empfangen wird. Die Welle wird oft senkrecht in das Testobjekt eingestrahlt (Siehe Teil \ref{sec:pulse_echo})}, long={Puls-Echo-Verfahren} }

\newglossaryentry{das}{name=DAS, description={Delay and sum, Zeitbereich Rekonstruktions Algorithmus, (siehe Teil \ref{sec:saft_general})},    first={delay and sum (DAS)}, long={delay and sum}, short={DAS}}


\newglossaryentry{tfm}{name=TFM, description={Total Focusing Method, eine Rekonstruktionsmethode, die auf dem gleichen Grundprinzip wie \acrshort{saft} basiert. TFM wird für Messdaten angewendet, die mit mehreren Transducerelementen auf einmal aufgenommen werden},    first={Total Focusing Method (TFM)},  long={Tofal Focusing Method}, short={TFM}}

\newglossaryentry{dr}{name=DR, description={Discard Repetitions, Quantisierungsmethode, die nur einen Beitrag für einen Rasterpunkt nimmt und alle anderen ignoriert (siehe Teil \ref{sec:quantization})},    first={Discard Repetitions (DR)},    long={Discard Repetitions}, short={DR}}

\newglossaryentry{ta}{name=TA, description={Take Average, Quantisierungsmethode, die den Mittelwert aller Beiträge für den selben Rasterpunkt nimmt (siehe Teil \ref{sec:quantization})},    first={Take Average (TA)},    long={Take Average }, short={TA}}



