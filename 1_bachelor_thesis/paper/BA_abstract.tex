
%%% EN %%% 2372 characters (v.190114)
\chapter*{\centering Abstract}
Material inspection is the core tool to ensure the absence of critical flaws in fabrication and enable maintenance of large infrastructure. One of its methods is ultrasonic testing (UT). As the image quality of raw UT data is generally insufficient to visualize the inner structure of objects, its image quality can be improved by post-processing. Techniques used for post-processing are developed to process automatically acquired data, although UT measurements are mostly performed manually. Due to the lack of positional information, post-processing of manual data has been challenging. However, by using the measurement system capable of recording scan positions, post-processing of manual data can be performed, enabling to enhance the visual feedback to the operator in assistance systems. \par 
%
One of the prevalent post-processing methods is the \acrfull{saft}. So far, its application potentials to manually acquired data are little studied. Since manual measurements differ from its automatic counterpart, such as irregular sampling or inaccurate position recognition, its \acrshort{saft} reconstruction is significantly degraded, when such differences are not considered. \par
%
In order to determine the effect of the error sources on the \acrshort{saft} image quality, this thesis compares automatic and manual measurement setups and derives five factors: coarser sampling density, unequal distribution of the scan positions, lower coverage of the inspection area, positional inaccuracy and varying contact pressure. As to the former two factors, there are already relevant studies from which we can assess their impact. Moreover, the former three are related to scan path decisions of the operator, which we cannot influence. Hence, this thesis focuses on the latter three factors by conducting two simulation studies.\par
%
The first investigates the impact of the positional inaccuracy and the spatial coverage, while the second examines the effect of varying contact pressure. For the evaluation, we compare the obtained results with a \acrshort{saft} reconstruction of a simulated automatically acquired data set serving as reference. By showing that the effect of those factors are tolerable or even negligible under suitable conditions, both studies demonstrate the feasibility of applying \acrshort{saft} to manual measurement data.\par
%
In the future, the obtained results can be used as indicators for developing measurement assistance systems.\par


%\chapter*{\centering Abstract original} % original from supervisors
%Material inspection is the core tool to ensure the absence of critical flaws in fabrication and enable maintenance of large infrastructure. One of its methods is ultrasonic testing (UT). As the image quality of raw UT data is generally insufficient to visualize the inner structures of an object, it is often post-processed. Post-processing techniques are developed to process automatically acquired UT measurements. Most of the UT is still performed manually. The post-processing of manually acquired data is challenging due to the lack of positional information. However, enabling the measurement system to record the positional information of the scanning allows to perform post-processing of the data. Such post-processing enhances the visual feedback provided to the engineer as common in assistance systems. \par
%%
%One of the prevalent post-processing methods is the \acrfull{saft} So far, its application potentials to manually acquired data are little studied. Compared to automatic measurements, manually acquired data is sampled in an irregular manner and its sampling positions are not exactly known. \par
%%
%In order to determine the effect of these error sources on the imaging quality, this thesis compares automatic and manual measurement setups and derives five factors: coarser sampling density, unequal distribution of the scan positions, lower coverage of the inspection area, positional inaccuracy and varying contact pressure. As to the former two factors, there are already relevant studies from which we can assess their impact. Moreover, the former three are related to scan path decisions of the operator, which we cannot influence. Hence, this thesis focuses on the latter three factors by conducting two simulation studies.\par
%%
%The first investigates the impact of the positional inaccuracy and the varying spatial coverage, while the second examines the effect of varying contact pressure. For the evaluation, we compare the obtained results with a \acrshort{saft} reconstruction of a simulated automatically acquired data set serving as reference. By showing that the effect of those factors are tolerable or even negligible under suitable conditions, both studies demonstrate the feasibility of applying \acrshort{saft} to manual measurement data.\par
%The obtained results can be used as indicators for developing measurement assistant systems.\par

% EN v190115_01
% motivation : why manual? purpose?
%In the recent years, material inspections are attracting more attentions to prevent accidents and improve public safety. One of its methods is ultrasonic testing (UT). Conventionally UT is performed with automatic measurement systems, nevertheless there are still great needs to take measurements manually. As the image quality of raw UT data is generally insufficient for the better insight of test objects, it is often post-processed. For manual measurements, though, it has long not been an option due to the lack of positional information. However, incorporating measurement assistant systems enables position recording along with measurements and, thus, reconstruction of the obtained data. \par
%% SAFT -> main goals of this thesis
%One of the reconstruction methods is \acrfull{saft}. Since \acrshort{saft} is based on the assumptions which are often not valid for manual measurements, it is mostly applied to the measurement data taken with automated systems. As a result, its application potentials to the manually acquired data was so far little studied. Therefore, it is investigated in this thesis by determining the possible error sources and evaluating their impact on the \acrshort{saft} imaging quality. \par
%% Error Sources
%In order to determine error sources, we compared automatic and manual measurement setups and derived five factors : corse sampling density, unequal distribution of the scan positions, lower coverage of the inspection area, positional inaccuracy and varying contact pressure. As to the former two factors, there are already relevant studies from which we can assess their impact. Moreover, the former three are related to scan path decisions which we cannot influence. Hence, this thesis focuses on the latter three factors by conducting two simulations. \par 
%% simulations
%The first investigates the impact of the positional inaccuracy with the varying spacial coverage, while the second examines the effect of varying contact pressures. For the evaluation, we compared the obtained results with our reference data, which is a \acrshort{saft} reconstruction result of automatically obtained data. By showing that the effect of those factors are tolerated or even negligible under suitable conditions, both simulations demonstrate the great possibilities of \acrshort{saft} application to manual measurement data.\par
%% Ausblick
%The obtained results can be further combined with other factors and used as indicators for developing measurement assistant systems.



%%% EN %%% 2343 characters (v.190113)
%\chapter*{\centering Abstract old}
%This thesis concerns the \acrfull{saft} application to the manual measurement data with its aim to find out the possible error sources and evaluate thier impact on the imaging qaulity of the reconstruction. By comparing automatic and manual measurement setups, we derived 5 factors as error sources : corse sampling density, unequal distribution of the data, lower coverage of the inspection area, positional inaccuracy and varying contact pressure. On the former two factors, there are already relevant studies, thus this thesis focuses on the latter three. \par 
%% posscan
%For investigating the impact of the positional inaccuracy, the measurement was simulated at random scan positions. In addition to that, the areal coverage was also varied, so that its effect can be studied in the same simulation. The positional information was collected during the data generation. The obtained positions are then rounded to the nearest point on the equidistant grid for the reconstruction. With this quantized positional information, the synthetic measurement data were reconstructed. We found out that the positional error up to $\frac{1}{10}$-th wavelength barely alter the measurement data and thus is negligible for the reconstruction. Moreover, the areal coverage can compensate the impact of the positional error and more error can be tolerated when the coverage increases. \par
%% zscan
%The investigation of the effect of varying contact pressures was carried out by simulating the measurement on the equidistant grid. In this thesis the varying pressure was regarded as varying propagation times caused by the change of the distance between the transducer and the surface of the test object. This simulation demonstrated that the deviation up to the double size of the vertical grid spacing $\dz$ is negligible. However, the deviation should be kept under approximately $5 \dz$, in order not to lose the imaging quality of the reconstruction. This value is valid, only when the angle, to be exact the size, of the transducer is properly selected. Furthermore we discovered that it has a strong impact on the lateral resolution, which amplifies the noise over the entire region of the inspection and thus impairs the imaging quality considerably. \par
%% achievement / conclusion
%The obtained results can be combined with other factors and used as indicators for developing a supporting system of manual measurements.


%%% DE %%%
\chapter*{\centering Zusammenfassung} % 2420 Zeichen (v190113)
Die Materialprüfung sichert in der Fertigung den Ausschuss von mit kritischen Fehlern behafteten Bauteilen und ermöglicht die Überwachung der Infrastruktur. Die \acrfull{ut} ist eine ihrer Hauptmethoden. Da die Bildqualität von Ultraschallrohdaten nicht ausreicht um die innere Struktur eines Objektes zu visualisieren, werden die Rohdaten oft nachverarbeitet. Post-processing Techniken existieren zur Verarbeitung von maschinell aufgenommenen \acrshort{ut}-Messdaten. Der Großteil der \acrshort{ut} erfolgt jedoch händisch. Die Nachverarbeitung solcher Messdaten ist aufgrund der fehlenden Positionsinformation schwierig. Wenn jedoch das Messsystem befähigt wird, die Positionsinformation mit zu erfassen, wird eine Nachverarbeitung möglich. Damit kann das visuelle Feedback an den Prüfingenieur, z.B. durch Assistenzsysteme, verbessert werden.\par
Eine der verbreitetsten Nachverarbeitungsmethoden ist die \acrfull{saft}. Zur Anwendung von \acrshort{saft} auf handaufgenommene Prüfdaten existieren kaum Vorarbeiten. Im Vergleich zu maschinell aufgenommenen Daten sind bei handaufgenommenen Daten die Abtastung ungleichmäßig und die Messpositionen nicht exakt bekannt.\par
Um den Einfluss dieser Fehlergrößen zu ermitteln vergleicht die vorliegende Arbeit den maschinellen Messaufbau mit der Handprüfung und leitet fünf Größen her: eine gröbere Abtastung, ungleichmäßige Verteilung der Messpositionen, eine niedrigere Abdeckung der Prüfregion, Positionsungenauigkeiten und variierender Anpressdruck. Zu den ersten zwei Größen existieren bereits Vorarbeiten, aus denen ihr Einfluss ermittelt werden kann. Des Weiteren unterliegen die ersten drei Größen den Messentscheidungen des Prüfingenieurs und können nicht beeinflusst werden. Infolgedessen liegt der Schwerpunkt der Arbeit auf der Untersuchung der letzten drei Größen durch zwei Simulationsstudien.
Die erste untersucht den Einfluss der ungenauen Positionsschätzung sowie variierender Abdeckung, die zweite den Einfluss des Anpressdrucks.
Zur Auswertung werden die Rekonstruktionsergebnisse mit der \acrshort{saft}-Rekonstruktion von simulierten maschinellen Messdaten verglichen. Es wird gezeigt, dass der Einfluss der genannten Faktoren unter geeigneten Bedingungen tolerierbar ist und damit eine \acrshort{saft}-Rekonstruktion handaufgenommener Daten möglich ist.\par
Die Ergebnisse können als Grundlage für die Entwicklung von Assistenzsystemen genutzt werden.



