\chapter{Einleitung} \label{chap:intro}

\section{Motivation} \label{sec:motivation}
% NDT
Um Unfälle mit einer Gefährdung von Menschenleben und sowie schwere Sach- und Umweltschäden zu vermeiden, wird immer mehr Wert auf die alltägliche Sicherheit, beispielsweise in Flugzeugen, Zügen, Brücken, Kraftwerken, Wasserleitungen, Öl-Pipelines oder Gebäuden, gelegt. Dementsprechend werden die Sicherheitsanforderungen ständig erhöht und die Qualitätskontrollen, bzw. \acrfull{shm}, sind nachgefragt. Dafür müssen Bauteile oder Materialien vor und während ihres Betriebs auf verborgene Fehler überprüft und der genereller Zustand kontrolliert werden\cite{DGZfPintro}. \par
Dabei spielen \acrfull{zfp} eine wesentliche Rolle. \acrshort{zfp} ist eine Methode, mit der die Objekte ohne Schaden überprüft werden können. Das heißt, die Funktionalität der geprüften Objekte wird nicht beeinträchtigt. Der Markt der \acrshort{zfp} wächst ständig auf Grund der sich erhöhenden Sicherheitsanforderungen der Regierungen und es wird erwartet, dass der Markt in den USA zwischen 2018 und 2023 mit der kumulierten jährlichen Wachstumsrate von 6.71\% wächst und bis zum Jahr 2023 einen Wert von ca 20.37 Milliarde USD erreicht haben wird \cite{NDTindustrialreport}. \par  
% UT
Eine Prüfmethode der \acrshort{zfp} ist die \acrfull{ut} und diese Methode ist für nahezu alle technischen Werkstoffe einsetzbar \cite{Erhard07ZfPAufgaben}. Die \acrshort{ut} kann nicht nur zur Fehlerprüfung sondern auch zur Wanddickemessung verwendet werden \cite{WSPraktikumUS1}. Beispiele für die \acrshort{ut} Anwendung sind die Schweißnahtprüfung, die Blechprüfung,  die Gussteilprüfung oder die Prüfung von Schmiedestücken oder Maschinenteilen \cite{WSPraktikumUS1}. \par
% automatic 
Derartige Prüfaufgaben werden üblicherweise mit Hilfe von automatisierten Systemen und Prüfrobotern durchgeführt. Mit dem automatisierten Prüfsystem können die Prüfpositionen sehr präzise eingestellt werden und  es lässt sich auch eine vollständige Abdeckung des Ortsbereiches erreichen. Allerdings kann die automatische Prozedur nicht angewendet werden, wenn das Prüfobjekt schlecht zugängig ist oder eine komplexe Geometrie besitzt. Für einen Nischenanwendungsfall wäre es auch unwirtschaftlich, eine passende Maschine zu entwickeln. \cite{Krieg18SAFTwithSmartInspect}. \par
% manual
Bei solchen Fällen wird die \acrshort{ut} Messung mit einem Handprüfgerät durchgeführt. Der Nachteile dabei ist, dass die Qualität des Messergebnisses stark von der Expertise des Prüfers und des von ihm durchgeführten Prüfpfades abhängt. Außerdem sind die Ortsinformationen der Prüfstellen meist nicht verfügbar, was zu mehrmaligen Messaufnahmen an einer gleichen Stelle führen kann. Das ist auch der Grund warum die händisch aufgenommenen Messungen nicht reproduzierbar sind und deren Messdaten sich nicht für die Nachverarbeitung eignen, obwohl die Abbildungsqualität der Rohdaten generell durch die passenden Verarbeitungen wesentlich verbessert werden kann. \par
% SmartInspect : what it should do
Um den Prüfer bei der Entscheidung über den Prüfpfad zu unterstützen, muss das auf den bisherigen Daten basierende Messergebnis dem Prüfer simultan zur Prüfung angezeigt werden \cite{Krieg18SAFTwithSmartInspect}. Darüber hinaus müssen die Ortsinformationen auch bei der Prüfung für den Prüfer verfügbar gemacht werden, um die Abdeckung des Ortsbereiches anzuzeigen und dadurch mehrmalige Aufnahmen von einer gleichen Stelle zu vermeiden \cite{Krieg18SAFTwithSmartInspect}. \par
% SmartInspect : how it works
Solch ein Unterstützungssystem kann realisiert werden, wenn das Messsystem während der Prüfung nicht nur die Messdaten sondern auch die Prüfpositionen aufnimmt, speichert, aktualisiert und simultan zur Prüfung darstellt. Das bedeutet, die Prüfpositionen müssen beispielsweise über eine Kamera ermittelt werden und deren Ortsinformationen müssen im System gespeichert werden. Andererseits muss das Messergebnis bei jeder Aufnahme mit den neu aufgenommenen und bisherigen Messdaten aktualisiert und dem Prüfer angezeigt werden. Bei der Datendarstellung ist es auch erforderlich, die bisherigen Prüfstellen anzuzeigen, so dass die Abdeckung des Ortsbereiches verbessert werden kann \cite{Krieg18SAFTwithSmartInspect}. \par
% SAFT with SmartInspect
Dies ermöglicht nicht nur das Reproduzieren der Messung sondern auch die Rekonstruktion der Messdaten. \acrfull{saft} ist ein etablierter Rekonstruktionsalgorithmus und wird seit Jahren zur Verbesserung der Abbildungsqualität der rohen \acrshort{ut} Messdaten verwendet. Mit den vorhandenen Ortsinformationen der Prüfpositionen kann \acrshort{saft} theoretisch auch für handaufgenommene Messdaten angewendet werden. Wenn eine Echtzeitrekonstruktion während der händischen Messung durchgeführt werden würde und das durch Rekonstruktion verbesserte Bild als Messergebnis dem Prüfer simultan zur Prüfung angezeigt werden könnte, würde das die händische Messung signifikant erleichtern.  \par
% bridge to the thesis
Nun ist die Frage, ob durch die Anwendung von \acrshort{saft} auf die handaufgenommenen Messdaten die Abbildungsqualität ihrer Rekonstruktion, wie bei den gewöhnlichen maschinellen Daten, verbessert werden kann oder nicht. Das bedeutet, dass die Herausforderungen bei der Rekonstruktion handaufgenommener Messdaten festgestellt und deren Einflüsse untersucht werden müssen. Wenn die negativen Auswirkungen solcher Fehlern unter bestimmten Bedingungen unterdrückt werden können, können daraus resultierend Maßnahmen abgeleitet werden, was die Realisierung eines Messunterstützungssystems mit einer Echtzeitrekonstruktion vereinfachen kann. \par


\section{Aufgabenstellung und Ziel der Arbeit} \label{sec:thesis_scope}
% Messaufbau
\begin{figure}[h!]
\begin{center}
\inputTikZ{0.6}{figures/BA_fig_SmartInspect.tex}
\caption{Messaufbau}
\label{fig:smart_inspect}
\end{center}
\end{figure}

% Messaufbau
In dieser Arbeit wurde ein handaugenommenes Messsystem wie in Abbildung \ref{fig:smart_inspect} betrachtet. Ein \gls{normal_transducer} wird direkt auf ein Metallobjekt gelegt und händisch im Ortsbereich bewegt, wobei die Prüfung mit dem \gls{pulse_echo} durchgeführt wird. Bei der Messung sucht sich der Prüfer den Prüfpfad selbst aus und die Ortsinformationen werden durch eine Kamera, die während der Messung Prüfpositionen aufnimmt, ermittelt. Mit den Ortsinformationen werden die Messdaten progressiv mit Hilfe des \acrshort{saft} Algorithmus rekonstruiert. \par
% Problem 
Allerdings ist eine solche Art der Messaufnahme nicht ganz genau und das kann zur Verschlechterung der Abbildungsqualität der Rekonstruktion führen. Solche Fehler können einerseits von den Charakteristiken der händischen Messung hervorgerufen werden und anderseits durch die Ungenauigkeiten des Messsystems verursacht werden. Da wir keinen Einfluss darauf haben, wie die Prüfung vom Prüfer durchgeführt wird, sollte die Systemungenauigkeit möglichst verringert werden, um die Abbildungsqualität der \acrshort{saft} Rekonstruktion zu verbessern. \par
% Ziel
Im Rahmen dieser Arbeit sollen die möglichen Fehlerquellen bei der \acrshort{saft} Rekonstruktion der handaufgenommenen Messdaten festgestellt und deren Einflüsse untersucht werden. Dafür werden Simulationen durchgeführt, um den Effekt der mit der Systemungenauigkeit verbundenen zwei Faktoren zu evaluieren: $($a$)$ die durch die Positionsbestimmung oder der Quantisierung verursachten Positionsungenauigkeit des Prüfkopfes und $($b$)$ die durch beispielsweise Veränderung des Anpressdrucks in der händischen Ankopplung hervorgerufenen Laufzeitänderungen. Die Evaluierung erfolgt durch den Vergleich mit den Referenzdaten. Als Referenzdaten wurde die Rekonstruktionsqualität von \acrshort{saft} genommen, die man mit maschinell aufgenommenen Daten erreichen würde.   \par
% Inhalt
Diese Arbeit ist wie folgt aufgebaut. In Kapitel \ref{chap:basic_ut_saft} werden zunächst die Grundlagen der \acrshort{ut} und des in dieser Arbeit verwendeten \acrshort{saft} Algorithmus erläutert. In Kapitel \ref{chap:soa} wird der Stand der Technik bezüglich der \acrshort{saft} Anwendung an handaufgenommene Messdaten vorgestellt. Im folgenden Kapitel \ref{chap:errorsource} werden die möglichen Fehlerquellen in Bezug auf die Merkmale der händischen Messung und die Ungenauigkeit des Messszenarios diskutiert. Dann werden die Simulationsszenarien, sowie die Referenzdaten dieser Arbeit in Kapitel \ref{chap:scenario} dargestellt. Die Kapitel \ref{chap:posscan} und \ref{chap:zscan} widmen sich jeweils der Simulation der einzelnen Fehlerquellen: die Positionsungenauigkeit und die Laufzeitänderung. In beiden Kapiteln werden das Simulationsprinzip, die exemplarischen Ergebnisse und die Auswertungsergebnisse präsentiert. Schließlich wird in Kapitel \ref{chap:conclusion} das Fazit dieser Arbeit gegeben.

