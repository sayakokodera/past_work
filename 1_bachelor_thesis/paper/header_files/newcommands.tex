% "newcommand" collections

\newcommand{\conv}{\ast}

% some special characters
\newcommand{\ii}{\mathrm{i}}
\newcommand{\dd}{\mathrm{d}}
\newcommand{\ee}{\mathrm{e}}
\newcommand{\dirac}{\delta}

\newcommand{\RR}{{\mathbb{R}}}
\newcommand{\NN}{{\mathbb{N}}}
\newcommand{\CC}{{\mathbb{C}}}
\newcommand{\OO}{{\mathcal{O}}}

% some math convenience definitions
%\newcommand{\vf}[1]{\bm{#1}}
%\newcommand{\tf}[1]{\bm{\mathcal{{#1}}}}
%\newcommand{\ind}[1]{\mathsf{#1} }
%
%\newcommand{\cint}[2]{\int\! #1 \,\dd#2 }
%\newcommand{\cintl}[4]{\int_{#3}^{#4}\! #1 \,\dd#2 }
%
%\newcommand{\mat}[1]{\begin{bmatrix}#1\end{bmatrix}}
%
%\newcommand{\F}[2]{\mathcal{F}_{#1}\left\{#2\right\}}
%\newcommand{\iF}[2]{\mathcal{F}_{#1}^{-1}\left\{#2\right\}}
%
%\newcommand{\vel}[2]{\vf{#1}[#2]}
% for two norms (NEW 01.12.18 cf. https://tex.stackexchange.com/questions/107186/how-to-write-norm-which-adjusts-its-size)
\newcommand\norm[1]{\left\lVert#1\right\rVert}

%=====================  for TikZ =======================%
\newcommand{\inputTikZ}[2]{%  
     \scalebox{#1}{\input{#2}}  
}

\newcommand{\Nx}{3}
\newcommand{\Ny}{2}
\newcommand{\ddx}{1.5cm}
\newcommand{\ddy}{\ddx}
\newcommand{\ddz}{1.5cm}
\newcommand{\rCircle}{0.07cm}

%%% draw transducer %%%
\newcommand{\drawTransducer}[4]{ % scale,scaley, x, y in axis
	\draw[draw = none, fill = white] (axis cs: #3 - #1, #4 - #2, 0) -- (axis cs: #3 + #1, #4 - #2, 0) -- (axis cs: #3 + #1, #4 + #2, 0) --  (axis cs: #3 + #1, #4 + #2, -#1) -- (axis cs: #3 - #1, #4 + #2, -#1) -- (axis cs: #3 - #1, #4 - #2, -#1) -- (axis cs: #3 - #1, #4 - #2, 0);
	\draw[] (axis cs: #3 - #1, #4 - #2, 0) -- (axis cs: #3 + #1, #4 - #2, 0) -- (axis cs: #3 + #1, #4 - #2, -#1) -- (axis cs: #3 - #1, #4 - #2, -#1) -- (axis cs: #3 - #1, #4 - #2, 0);
	\draw[] (axis cs: #3 + #1, #4 - #2, 0) -- (axis cs: #3 + #1, #4 + #2, 0) -- (axis cs: #3 + #1, #4 +  #2, -#1) -- (axis cs: #3 - #1, #4 +  #2, -#1) -- (axis cs: #3 - #1, #4 - #2, -#1);
	\draw[] (axis cs: #3 + #1, #4 - #2, -#1) -- (axis cs: #3 + #1, #4 +  #2, -#1);
}

%%% draw video camera %%%
\newcommand{\drawCamera}[6]{ %<rotation origin (x, y, z)>, <start x>, <start y>, <start z> in axis, <camera width [dx]>, <camera height [dz]>
	% base
	\draw[rotate around = {30 : (axis cs: #1)}] (axis cs: #2, #3, #4) rectangle (axis cs: #2 + #5, #3, #4 - #6) ;
	% origin of rotation
	\node[campoint] (rotationorg) at (axis cs: #1) {};
	% "trapezoid" part (tip of the camera)
	\draw[rotate around = {30 : (axis cs: #1)}] (axis cs: #2, #3, #4 - 0.25*#6) -- (axis cs: #2 - 0.5* #6, #3, #4) -- (axis cs: #2 - 0.5* #6, #3, #4-#6) -- (axis cs: #2, #3, #4 - 0.75*#6);	
	% cable
	\draw[rotate around = {30 : (axis cs: #1)}] (axis cs: #2 + #5, #3, #4 - 0.5*  #6) .. controls (axis cs: #2 + 1.5* #5, #3, #4 - 0.5*  #6) and (axis cs: #2 + #5, #3, - 0.5*  #6) .. (axis cs: #2 + 1.5* #5, #3, - #6);	
}

%%%	 SAFT dictionary %%%
% matrix separations
\newcommand{\linewidththick}{0.1}
\newcommand{\linewidththin}{0.1}
\newcommand{\matrixspacesmall}{0.2}
\newcommand{\matrixspacebig}{0.35}
% single element
\newcommand{\matrixwidth}{2}
\newcommand{\matrixheight}{1.5}
\newcommand{\vectorwidth}{0.5}

% result vector
\newcommand{\resultcolorone}{orange}
\newcommand{\resultcolortwo}{green}

% matrix -1
\newcommand{\matrixonecolorone}{cyan}
\newcommand{\matrixonecolortwo}{yellow}

% matrix 0
\newcommand{\matrixtwocolorone}{violet}
\newcommand{\matrixtwocolortwo}{orange}

% matrix 1
\newcommand{\matrixthreecolorone}{blue}
\newcommand{\matrixthreecolortwo}{green}

% data vector
\newcommand{\datacolorone}{black}
\newcommand{\datacolortwo}{white}


%%% image macro %%%
% cimg with both x- & y-labels
\newcommand{\cimgbothlabels}[4]{% <scale size>, <label font size>, <tick font size>, <png file name>
\scalebox{#1}{
	\begin{tikzpicture}
            \begin{axis}[
                enlargelimits = false,
                axis on top = true,
                axis equal image,
                point meta min = -1,   
                point meta max = 1,
                xlabel = {$x / \dx$},
                ylabel = {$y / \dy$},
                label style = {font = #2},
                tick label style = {font = #3},
                %y dir = reverse,
                %xtick = {0, 20, ..., 100} to customize the axis
                %xticklabel = {0, 20, 40, 60, 80, 100}
                ]
                \addplot graphics [
                    xmin = 0,
                    xmax = 39,
                    ymin = 0,
                    ymax = 39
                ]{#4};
            \end{axis}
	\end{tikzpicture}
	}
}
            
% cimg with only x-label
\newcommand{\cimgxlabel}[4]{% <scale size>, <label font size>, <tick font size>, <png file name>
\scalebox{#1}{
	\begin{tikzpicture}
            \begin{axis}[
                enlargelimits = false,
                axis on top = true,
                axis equal image,
                point meta min = -1,   
                point meta max = 1,
                xlabel = {$x / \dx$},
                %ylabel = {$y / \dy$},
                label style = {font = #2},
                tick label style = {font = #3},
                ytick = \empty,
                %y dir = reverse,
                ]
                \addplot graphics [
                    xmin = 0,
                    xmax = 39,
                    ymin = 0,
                    ymax = 39
                ]{#4};
            \end{axis}
	\end{tikzpicture}
	}
}

% cimg with only y-label
\newcommand{\cimgylabel}[4]{% <scale size>, <label font size>, <tick font size>, <png file name>
\scalebox{#1}{
	\begin{tikzpicture}
            \begin{axis}[
                enlargelimits = false,
                axis on top = true,
                axis equal image,
                point meta min = -1,   
                point meta max = 1,
                %xlabel = {$x / \dx$},
                ylabel = {$y / \dy$},
                label style = {font = #2},
                tick label style = {font = #3},
                xtick = \empty,
                %y dir = reverse,
                ]
                \addplot graphics [
                    xmin = 0,
                    xmax = 39,
                    ymin = 0,
                    ymax = 39
                ]{#4};
            \end{axis}
	\end{tikzpicture}
	}
}

% cimg without labels
\newcommand{\cimgnolabel}[2]{% <scale size>, <label font size>, <tick font size>, <png file name>
\scalebox{#1}{
	\begin{tikzpicture}
            \begin{axis}[
                enlargelimits = false,
                axis on top = true,
                axis equal image,
                point meta min = -1,   
                point meta max = 1,
                %xlabel = {$x / \dx$},
                %ylabel = {$y / \dy$},
                %label style = {font = #2},
                %tick label style = {font = #3},
                xtick = \empty,
                ytick = \empty,
                %y dir = reverse,
                ]
                \addplot graphics [
                    xmin = 0,
                    xmax = 39,
                    ymin = 0,
                    ymax = 39
                ]{#2};
            \end{axis}
	\end{tikzpicture}
	}
}


% simgs with both labels and cmaps
\newcommand{\cimgbothlabelswithcmap}[4]{% <scale size>, <aspect ratio x>, <aspect ratio y>, <label font size>, <tick font size>, <png file name>
\scalebox{#1}{
\begin{tikzpicture}
            \begin{axis}[
            	  enlargelimits = false,
                axis on top = true,
                axis equal image,
                point meta min = 0,   
                point meta max = 1,
                colorbar,
                colormap = {mymap}{rgb(0.0pt) = (0.97, 0.97, 0.98) ; rgb(0.45pt) = (0.51, 0.87, 0.78) ; rgb(0.75pt) = (0.94, 0.49, 0) ; rgb(1.0pt) = (0.75, 0.12, 0.06) ; },
                xlabel = {$x / \dx$},
                ylabel = {$y / \dy$},
                label style = {font = #2},
                tick label style = {font = #3},
                ]
                %\addplot[tui_red, thick, mark = x] coordinates{
                %};
                \addplot graphics [
                    xmin = 0,
                    xmax = 39,
                    ymin = 0,
                    ymax = 39
                ]{#4};
            \end{axis}
            \end{tikzpicture}
            
	}
}

% bscan presentation reflecting the aspect ratio of axis
\newcommand{\bscanwithaspectratio}[6]{% <scale size>, <aspect ratio x>, <aspect ratio y>, <label font size>, <tick font size>, <png file name>
\scalebox{#1}{
\begin{tikzpicture}
            \begin{axis}[
                enlargelimits = false,
                axis on top = true,
                axis equal image,
                unit vector ratio= #2 #3, % change aspect ratio, one of them should be 1
                point meta min = -1,   
                point meta max = 1,
                colorbar,
                colormap = {mymap}{rgb(0.0pt) = (0, 0.22, 0.39) ; rgb(0.47pt) = (0.91, 0.93, 0.96) ; rgb(0.5pt) = (0.97, 0.97, 0.98) ; rgb(0.53pt) = (0.99, 0.93, 0.87) ; rgb(0.6pt) = (0.99, 0.93, 0.87) ; rgb(1.0pt) = (0.75, 0.12, 0.06) ; },
                xlabel = {$x / \dx$},
                ylabel = {$t / \dt$},
                label style = {font = #4},
                tick label style = {font = #5},
                y dir = reverse,
                ytick = {450, 550, 650}, %to customize the axis
                %yticklabel = {450, 550, 650}
                ]
                %\addplot[tui_red, thick, mark = x] coordinates{
                %};
                \addplot graphics [
                    xmin = 0,
                    xmax = 39,
                    ymin = 450,
                    ymax = 700
                ]{#6};
            \end{axis}
            \end{tikzpicture}
            
	}
}

