% Conclusion
% Intro: goal what has been done
In an attempt to ultimately improve their reconstruction results, in this thesis a data-driven hybrid preprocessing scheme is presented for irregularly subsampled \acrshort{ut} measurements. As a part of preprocessing, we aimed to interpolate the missing measurements, which is in essence nonlinear spatio-temporal interpolation. The main goal of this thesis is to establish an interpretable \acrshort{DNN} scheme to benefit from both theory-based approaches and \acrshort*{DNN}, which makes use of its nonlinear mapping capability and fast on-site computing. The primary work of this thesis is formulation of our research problem such that we understand the role \acrshort*{DNN} plays in our preprocessing scheme. \par

%==================================%
% First part = SoA
%==================================%
In the first part of this thesis, measurement systems and data modeling for manual \acrshort{ut} measurements are first presented to describe the considered scenario for this thesis. This follows an overview of conventional \acrshort*{DNN} and its state of the art approaches for spatio-temporal interpolation. Here, the main concern of traditional \acrshort*{DNN} is highlighted, which is its black box nature. This prevents us to understand what the network does and learns, leaving it ungeneralizable and requiring a huge amount of training data to compensate. In the last chapter of the first part, fundamental idea of spatio-temporal interpolation is extensively studied. As a well-established scientific field in spatial data analysis, geostatistics offer profound insight with respect to handling of spatio-temporal data. The emphasis here is placed on the fact that spatio-temporal data cannot be treated as just an extension of spatial data and requires careful examination of their statistical aspects. Through literature survey it becomes more prominent that there are growing interests in spatio-temporal data analysis, yet it still remains a very challenging task. \par

%==================================%
% Second part = detail of the method
%==================================%
In the second part of this thesis, the proposed preprocessing scheme is presented. Based on the examination of our needs and the nature of \acrshort*{ut} data,  the preprocessing of \acrshort*{ut} data is divided into two phases: (1) estimation of field statistics and (2) prediction of measurements. In this way, we can combine geostatistical interpolation technique with a help of \acrshort*{DNN} to obtain better estimates of field statistics. For this purpose, \acrshort*{ut} data are formulated as a geostatistics based model, demonstrating that our problem can be indeed considered as a geostatistical problem, which is one of the contribution of this work. Considering the properties of our measurement data, the estimation process via \acrshort*{DNN} is formulated as a vector-valued nonlinear regression problem, which is well known to be a suitable task for \acrshort*{DNN}. This follows the description of network architecture, training scheme and dataset. \par


%==================================%
% Third part = results
%==================================%
The last part of this thesis is dedicated to examine the validity of the proposed preprocessing scheme. For this purpose, two sets of simulations are conducted. %
% Simulation 1
The first set of simulations, Simulation \rom{1}, focuses on the prediction accuracy of \acrshort{SFKrig} and the estimation performance of \gls{FVnet} compared to the one we could obtain with fully sampled measurements. The obtained results show that \acrshort*{SFKrig} overall outperforms \acrshort{idw}, demonstrating its general effectiveness for \acrshort*{ut} data. In view of \gls*{ascan} prediction, the performance of \gls*{FVnet} is comparable to that of ground truth based prediction. Yet, \gls*{FVnet} estimates only the structrue of \glspl*{FV}, resulting in possible fluctuation of \gls{FV} values. As a result, there may be non-negligible deviation in amplitude between the ground truth based prediction and that of \gls*{FVnet}. This shows as the price we paid by prioritizing the flexibility over including more domain specific knowledge, such as spectral shape. \par

% Simulartion 2
With the second sef ot simulations, Simulation \rom{2}, the effectiveness of the entire preprocessing scheme was investigated. The main purpose here is to evaluate the proposed preprocessing scheme in terms of the resolution of resulting \acrshort{SAFT} reconstructions. Overall, performing interpolation appears to contribute to enhanced resolution, unless the subsampling is extremely low. For a very low coverage, in our case less than 0.61 samples per squared wavelength, the results show that keeping the raw data unprocessed yields better resolution in reconstructions. With increasing number of samples, however, there is observable difference in reconstruction resolution between processed and unprocessed data. Here as well \gls*{FVnet} supported \acrshort*{SFKrig} outperforms \acrshort*{idw} for the subsampling with the same or even higher coverage. This also confirms its applicability to \acrshort*{ut} data. Moreover, the obtained results indicates that the concern regarding deviation in amplitude does not have much effect on the reconstruction results. As a part of Simulation \rom{2}, the effect of strategic sampling was also investigated. Compared to random sampling of the same coverage, the results demonstrate that strategic sampling leads to enhanced resolution in reconstructions. This implies that providing the feedback has indeed positive impact on the measurement quality. \par

%==================================%
% Summary + future work
%==================================%
Overall, the proposed preprocessing scheme exhibits its effectiveness for our \acrshort*{ut} scenario. Yet, there are still rooms for improvement, which can be further investigated. First, amplitude of \glspl*{FV} can be estimated as well by modeling them parametrically. In this way, the trained network includes the \acrshort*{spde} based domain knowledge, leading to more physically aware outputs. Second, the reconstruction results could be incorporated into the loss function during the training phase, which makes the network also aware of our ultimate goal. Another possibility would be tuning the weight regularization parameter of \acrshort*{SFKrig}. Currently a fixed value is used for all frequency components, yet this does not have to be, and adjusting this regularization parameter individually may yield better prediction results. Toward actual online applications, the processing time can be accelerate by efficient parallel computing or progressive approach. \par

