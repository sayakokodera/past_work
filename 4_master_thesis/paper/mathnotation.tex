\section*{Variable Types, array access and functions}
\begin{flushleft}
\normalsize
%\renewcommand{\arraystretch}{1.5}  
\begin{tabular}{l l l}
$\bm{v}$ & Vector\\
$\bm{v}_i$ & Access to the $i$-th scalar element of $\bm{v}$\\
$\bm{U}$ & Matrix \\
$\bm{U}_{ij} $ & Access to the scalar element in $\bm{U}$, where $i$ and $j$ denote its column and row, \\ & respectively \\ 
%$[\bm{U}]_{i}$ &  Access to the $i$-th column of $\bm{U}$\\
$\vectorize{ \bm{U} }$ & Vectorization of a matrix $\bm{U}$\\
%$\Mdict$  & Tensor\\
%$\Mdict(i, j, k)$ & Zugriff auf das Skalarelement in $\Mdict$, wobei $i$ der Reihe, $j$ der Spalte und $k$\\ & der dritten Dimension entsprechen \\
\end{tabular}
\end{flushleft}


\section*{Frequently used terms and symbols}
\begin{flushleft}
\normalsize
\begin{tabular}{l l l}
$c_0$ & Speed of sound in a medium\\
$f_{S}$ & Sampling frequency \\
$\lambda$ &  Wave length\\
$\dx$  & Spatial discretization step along the $x$-axis\\
$\dy$  & Spatial discretization step along the $y$-axis\\
$\dz$  & Spatial discretization step along the $z$-axis \\
$\dt$  & Temporal resolution of the measurement\\
$\N{x}$  & Number of samples along the $x$-axis in reconstruction and ground truth\\
$\N{y}$  & Number of samples along the $y$-axis in reconstruction and ground truth\\
$\N{z}$  & Number of samples along the $z$-axis in reconstruction and ground truth\\
$\N{t}$  & Number of temporal samples of the measurements\\
%$\bm{a}$ & einzelner Datenvektor, \gls{ascan} \\
%$\bm{C}$ & 2D Referenzdaten \\
%$\hat{\bm{C}} $ & 2D fehlerhafte Daten \\
%$\mse$ & \acrfull{mse}\\
%$\rmse$ & \acrfull{rmsenormal}\\
%$\rmse^{\dagger}$ & \acrfull{rmse}\\
\end{tabular}
\end{flushleft}


\section*{Notations for \acrshort*{srf} and \acrshort*{strf}}
%%%%%%%%%%%%%%%%%%%%%%%%%%%%%%%%
%===================== SRF ======================%
%%%%%%%%%%%%%%%%%%%%%%%%%%%%%%%%
\subsection*{ \acrfull{srf} }
\acrshort*{srf} depends on the position $\pos{i}$:
\begin{equation*}
	\SRF{ \pos{i} } .
\end{equation*}

% SoS
For \glssymbol{sos} \acrshort*{srf}: its spatial correlation function depends solely on the lag vector $\lagvec{}$:
\begin{equation*}
	\acf{ Y }{ \lagvec{} } = \Corr{ \SRF{\pos{} }, \SRF{ \pos{} + \lagvec{} } } .
\end{equation*}

% IS
For \glssymbol{is} \acrshort*{srf}: \\
Spatial incremental process for a fixed $\lagvec{ij}$, which depends on the position $\pos{i}$:
\begin{equation*}
	\begin{matrix}
		\SRFinc{ \lagvec{ij} } ( \pos{i} )
				& = & \SRF{ \pos{i} } - \SRF{ \pos{j} } \\
				& = & \SRF{ \pos{i} } - \SRF{ \pos{i} + \lagvec{ij} } .\\
	\end{matrix}
\end{equation*}

% ACF of the incremental process
Since $\SRF{\pos{i}}$ is  \glssymbol*{is}, $ \SRFinc{ \lagvec{ij} } ( \pos{i} ) $ is zero-mean spatial \glssymbol*{sos}. This leads its
spatial correlation function with a fixed $\lagvec{ij}$ to be expressed as a function of the lag vector $ \lagvecinc{}$ independent of $\pos{i}$: 
\begin{equation*}
	\acfunivariate{ X }{ \lagvec{ij} }{ \lagvecinc{ } } = \Corr{ \SRFinc{ \lagvec{ij} } ( \pos{i} ) ,  \SRFinc{ \lagvec{ij} } ( \pos{i} + \lagvecinc{} ) } .
\end{equation*}


%%%%%%%%%%%%%%%%%%%%%%%%%%%%%%%%
%==================== STRF ======================%
%%%%%%%%%%%%%%%%%%%%%%%%%%%%%%%%
\subsection*{ \acrfull{strf} }
\acrshort*{strf} depends on the position $\pos{i}$ and time:
\begin{equation*}
	\SRF{ \pos{i}, t } .
\end{equation*}

% Frequency response
This spatio-temporal process can also be described in terms of the (temporal) frequency as 
\begin{equation*}
	\SRFfreq{ \pos{i}, \omega } = \ft{t}{ \SRF{ \pos{i}, t } } 
\end{equation*}

% ST-SoS
For spatio-temporal \glssymbol*{sos} \acrshort*{strf}: its spatial correlation depends on the spatial lagvector $\lagvec{}$ 
\begin{equation*}
	\acfunivariate{ Y }{ t }{ \lagvec{}} = \Corr{ \SRF{\pos{}, t }, \SRF{ \pos{} + \lagvec{}, t } }, 
\end{equation*}
and its spatio-temporal correlation depends also on the temporal lag $\tau$ as 
\begin{equation*}
	\acf{ Y }{ \lagvec{}, \tau } = \Corr{ \SRF{\pos{}, t }, \SRF{ \pos{} + \lagvec{}, t + \tau } } .
\end{equation*}


% ST-IS
For a spatio-temporal \glssymbol*{is} \acrshort*{strf}: \\
For a fixed $\lagvec{ij}$, its spatial incremental process varies with the position $\pos{i}$
\begin{equation*}
	\SRFinc{ \lagvec{ij} } ( \pos{i}, t ) = \SRF{ \pos{i}, t } - \SRF{ \pos{j}, t} .
\end{equation*}

% ACF of the incremental procss
Analog to the purely spatial case, the incremental process $\SRFinc{ \lagvec{ij} } ( \pos{i}, t ) $ is zero-mean spatio-temporal \glssymbol*{sos}. This means that for a fixed lag vector $\lagvec{ij}$ its spatial correlation depends only on the lag vector $\lagvecinc{}$ and independent of the position $\pos{i}$
\begin{equation*}
	\acfunivariate{ X }{ \lagvec{ ij }, t }{ \lagvecinc{} } = \Corr{ \SRFinc{ \lagvec{ij} } ( \pos{i}, t ) ,  \SRFinc{ \lagvec{ij} } ( \pos{i} + \lagvecinc{}, t ) } .
\end{equation*}

% Spectral representaion
This incremental process can also be expressed in the frequency domain as 
\begin{equation*}
	\SRFincfreq{ \lagvec{ ij } } ( \pos{i}, \omega ) = \ft{t}{ \SRFinc{ \lagvec{ij} } ( \pos{i}, t ) },
\end{equation*}
which depends on the position $\pos{i}$.
%% Output of a narrow band-pass -> normally distributed
%For a single frequency component $\omega_{m}$, the frequency response of the incremental process is asymptotically normally distributed over the space with mean zero as
%\begin{equation*}
%	\SRFincfreq{ \lagvec{ ij } } ( \pos{i}, \omega_{m} ) \sim \CC \NormalDist \left(  0, \sigma^{2} ( \lagvec{ ij }, \omega_{m} )  \right) 
%\end{equation*}

% ACF of the freq. response
For a fixed lag vector $\lagvec{ ij }$ of the increment and a single frequency component $\omega_{m}$, the spatial correlation of the incremental frequency response is defined as 
\begin{equation*}
	\begin{matrix}
		\acfunivariate{ \SFresponse{ X } }{ \lagvec{ ij }, \omega_{m} }{ \lagvecinc{} }  
		& = 
		& \Corr{ \SRFincfreq{ \lagvec{ ij } } ( \pos{i}, \omega_{m} ), \SRFincfreq{ \lagvec{ ij } } ( \pos{i} + \lagvecinc{}, \omega_{m} )  } 
		&  \text{ }
		& \forall 
		& \omega_{m}.\\
	\end{matrix}
\end{equation*}

%\input{tables/table_math_letters.tex}

