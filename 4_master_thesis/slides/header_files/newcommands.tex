% "newcommand" collections
% important!!!! no numbers in the name of commands!!!!!!!!

%=====================  for math =======================%
% symbols

% For font size of subscript
\newcommand{\subscript}[1]{%
	\scaleto{ #1 }{4pt} 
}

% Math operands
% l2 norms
\newcommand{\norm}[1]{%
	\left\lVert#1\right\rVert
}
% Transpose
\newcommand{\TP}[1]{%
	#1^{\T}
}
% Vectorization
\newcommand{\vectorize}[1]{%
	\operatorname{vec} \{ #1 \}
}
% Diagonal matrix
\newcommand{\diag}[1]{%
	\operatorname{diag} \{ #1 \}
}
% Real part
\newcommand{\Real}{%  
	\operatorname{Re}
}
% Imaginary part
\newcommand{\Imag}{%  
	\operatorname{Im}
}
% Column space
\newcommand{\Col}[1]{
	\operatorname{col} \{ #1 \}
}
% Nullspace 
\newcommand{\Null}[1]{
	\operatorname{null} \{ #1 \}
} 
% Expected value
\newcommand{\Expect}[1]{
	\operatorname{E}  \left\lbrace  #1  \right\rbrace 
} 
% Unfolding 
\newcommand{\unfold}[2]{ % tensor, mode
	[ #1 ]_{ ( #2 ) }
} 
% Correlation
\newcommand{\Corr}[1]{
	\operatorname{Corr}  \left\lbrace  #1  \right\rbrace 
} 
%Covariance
\newcommand{\Cov}[1]{
	\operatorname{Cov}  \left\lbrace  #1  \right\rbrace 
} 
% Variance
\newcommand{\Var}[1]{
	\operatorname{Var}  \left\lbrace  #1  \right\rbrace 
}
%% MSE
\newcommand{\MSE}[1]{
	\operatorname{MSE}  \left\lbrace  #1  \right\rbrace 
}
% Minimization
\newcommand{\minimize}[1]{ % variable
	\underset{#1}{ \operatorname{min} }
} 
% Fourier transform (FT)
\newcommand{\ft}[2]{ % domain, argument
	\mathscr{F}_{ #1 } \left\lbrace  #2  \right\rbrace 
}
% Inverse Fourier transform (IFT)
\newcommand{\ift}[2]{ % domain, argument
	\mathscr{F}_{ #1 }^{-1} \left\lbrace  #2  \right\rbrace 
}
% DFT matrix
\newcommand{\DFTmat}[1]{ % size
	\bm{F}_{ #1 } 
}


% symbols
% General
\newcommand{\Identity}[1]{\bm{I}_{#1}} % Identity matrix
\newcommand{\RR}{{\mathbb{R}}}
\newcommand{\CC}{{\mathbb{C}}}
\newcommand{\NN}{{\mathbb{N}}}
\newcommand{\NormalDist}{ \mathscr{N} } % Normal distribution

% Dimensions
\newcommand{\N}[1]{N_{#1}}

% Scan positions & lags
\newcommand{\pos}[1]{
	\bm{s}_{ #1 } 
}
% Prediction position
\newcommand{\pospred}[1]{
	\bm{s}_{ \widetilde{ #1 } } 
}

\newcommand{\lagvec}[1]{
	\bm{h}_{#1} 
}
\newcommand{\lag}[1]{
	h_{#1} 
}
\newcommand{\lagvecinc}[1]{
	\bm{ \vartheta }_{#1} 
}
% Wavenumber 
\newcommand{\wavenum}{
	\bm{k}
}


% Data in space-time domain: ascans, data in 3D, unfolding 
\newcommand{\ascansig}{a}
\newcommand{\ascan}[1]{ % i-th A-Scan (vector), param = i 
	\bm{a}_{#1}
} 
\newcommand{\ascanhat}[1]{ % Estimation of the i-th A-Scan (vector), param = i
	\hat{\ascan{ }}_{#1}
} 
\newcommand{\Amat}[1]{ % 2D representation of the measurement data, param = measurement method
	\bm{A}_{#1}
} 
\newcommand{\Atensor}[1]{ % 3D representation of the measurement data, param = measurement method
	\bm{\mathcal{A}}_{#1}
} 
\newcommand{\Atensorhat}[1]{ % 3D representation of the interpolated/predicted measurement data, param = prediction method
	\hat{\Atensor{ } }_{#1}
} 
\newcommand{\aall}{ % = vec{\Atensor}
	\bm{\alpha}
} 
\newcommand{\aallhat}[1]{ % = vec{\Atensorhat}, param = prediction method
	\hat{\aall}_{#1}
} 

%%%%% Data in space-freq. domain %%%%
% Complex SRF for UT data
\newcommand{\CSRF}{ P }
% Incremental Complex SRF for UT data
\newcommand{\CSRFinc}{ Q }
% Freq. response vector
\newcommand{\frsp}[1]{ % freq. spectrum of the i-th A-Scan (vector), param = i 
	\bm{p}_{#1}
} 
\newcommand{\frsphat}[1]{ % estimate freq. spectrum of the i-th A-Scan (vector), param = i 
	\hat{\fresp{ }}_{#1}
} 
\newcommand{\ftcoeffut}[1]{ % freq. spectrum of the i-th A-Scan (vector), param = i 
	\bm{\pi}_{#1}
} 
\newcommand{\ftcoeffuthat}[1]{ % freq. spectrum of the i-th A-Scan (vector), param = i 
	\hat{\ftcoeffut{ }}_{#1}
} 

\newcommand{\frspinc}[1]{ % = p_{i} - p_{j} = difference b/w two freqeunce responce, param = ij
	\bm{q}_{#1}
} 

\newcommand{\Frsp}{ % freq. spectrum of multiple A-Scans, matrix form
	\bm{P}
} 
\newcommand{\Frspinc}{ % freq. spectrum of multiple incremental process, matrix form
	\bm{Q}
} 
\newcommand{\Ftensor}{ % freq. spectrum of multiple A-Scans in 3D form
	\bm{\mathcal{P}}
} 
\newcommand{\Ftensorhat}[1]{ % estimated freq. spectrum of multiple A-Scans in 3D form, param = prediction method
	\hat{\Ftensor}_{#1}
} 
\newcommand{\frspall}{ % = vec{\Ftensor}
	\bm{\rho}
} 
\newcommand{\frspallhat}[1]{ % = vec{\Ftensorhat},  param = prediction method
	\hat{\frspall}_{#1}
} 


% Pulse, SAFT matrix, spatial approximation
\newcommand{\pulsesig}{h}
\newcommand{\pulse}{\bm{h}}
\newcommand{\pulsehat}{\hat{\pulse}}
\newcommand{\pulsederiv}{\bm{j}}
\newcommand{\pulsederivhat}{\hat{\pulsederiv}}
\newcommand{\SAFTcol}[2]{\pulse_{#1}^{(#2)}}
\newcommand{\SAFThatcol}[2]{\pulsehat_{#1}^{(#2)}}
\newcommand{\SAFT}{\bm{H}}
\newcommand{\SAFThat}{\hat{\SAFT}}
\newcommand{\SAFTdot}{\bm{J}}
\newcommand{\SAFTdotcol}[2]{\pulsederiv_{#1}^{(#2)}}
\newcommand{\SAFTdothat}{\hat{\SAFTdot}}
\newcommand{\SAFTdothatcol}[2]{\pulsederivhat_{#1}^{(#2)}}
\newcommand{\Error}{\bm{E}}
\newcommand{\ErrorDef}{\diag{\xxdelta} \otimes \Identity{\M}}
% Defect map & recos
\newcommand{\defmap}{\bm{b}} 
\newcommand{\reflector}{\bm{r}} 
\newcommand{\reco}{\hat{\defmap}}
\newcommand{\recohat}[1]{\hat{\defmap}_{#1}}
\newcommand{\Reco}{\bm{B}}  
\newcommand{\Recohat}{\hat{\Reco}}  


% Metrics
\newcommand{\SEdag}{\SE^{\dagger}} 
\newcommand{\MSEdag}{\operatorname{MSE}^{\dagger}} 

% else
\newcommand{\refcoeff}{ b }
\newcommand{\noisevec}{\bm{n}} 
\newcommand{\vsound}{c_{0}} 


%%%%%%%%%%%  ST-RF modeling relevant %%%%%%%%%%%%
% Mean
\newcommand{\srfmean}{\mu_{ \subscript{Y} }}
% An SRF
\newcommand{\SRF}[1]{ %position
	Y ( #1 )
}
\newcommand{\SRFhat}[1]{ %position
	\hat{Y} ( #1 )
}
% Increments of an SRF
\newcommand{\SRFinc}[1]{ %lag vector, posision
	X^{ \left[  #1 \right] } 
}
% Spectral representation
\newcommand{\SFresponse }[1]{ % SRF
	Z_{ \subscript{ #1 } } 
}
\newcommand{\SRFfreq}[1]{ % posision
	\SFresponse{ Y } ( #1 )
}
\newcommand{\SRFincfreq}[ 1 ]{ % lag vector, position
	\SFresponse{ X }^{ \left[  #1 \right] } 
}

% Semivariogram
\newcommand{\semivario}[1]{ % independent variable (e.g. lag vector and/or time lag)
	\scaleto{\gamma}{7pt}_{ \subscript{Y} } ( #1 )
}
\newcommand{\semivariohat}[1]{ % independent variable (e.g. lag vector and/or time lag)
	\hat{ \scaleto{ \gamma }{7pt} }_{ \subscript{Y} } ( #1 )
}
\newcommand{\vario}[1]{ % SRF, independent variable (e.g. lag vector and/or time lag)
	2 \scaleto{\gamma}{7pt}_{ \subscript{Y} } ( #1 )
}
% Auto covariance function
\newcommand{\acvfbase}[2]{ % SRF, independent variable (e.g. lag vector and/or time lag)
	\scaleto{c}{6pt}_{ \subscript{#1} } ( #2 )
}
\newcommand{\acvf}[1]{ % independent variable (e.g. lag vector and/or time lag)
	\acvfbase{Y}{ #1 }
}
\newcommand{\acvfhat}[1]{ % independent variable (e.g. lag vector and/or time lag)
	\hat{ \scaleto{ c }{6pt} }_{ \subscript{Y} } ( #1 )
}
\newcommand{\acvfunivariate}[3]{ % SRF,  fixed parameter, variable (e.g. lag vector and/or time lag)
	\scaleto{c}{6pt}_{ \subscript{#1} }^{ \left[ #2 \right] } ( #3 )
}


%%% Stat. moments
% Correlation
\newcommand{\acf}[2]{ %functions, arguments
	\scaleto{\varphi}{7pt}_{ \subscript{#1} } (#2)
} 
% Spatial correlation of the incremental process
\newcommand{\acfunivariate}[3]{ % SRF, fixed parameter, variable 
	\scaleto{\varphi}{7pt}_{ \subscript{ #1 } }^{ \left[ #2 \right]  } (#3)
}
% Base of spectral density function 
\newcommand{\psdbase}[1]{ % noise process
	\Phi_{ \subscript{#1} } 
}
% Spectral density function 
\newcommand{\psd }[2]{ %functions, arguments
	\psdbase{#1} (#2)
} 
% Spectral density function with some fixed parameter values
\newcommand{\psdunivariate }[3]{ %functions, fixed parameter, variable 
	\psdbase{ #1 }^{ \left[ #2 \right] } (#3)
} 

%%%%%% Variogram relevant %%%%%%%
\newcommand{\paramtoest}{\bm{\theta}} 

%%%%%% Kriging relevant %%%%%%%%
% Weights
\newcommand{\weights}{\bm{w}} 
% Vectors of the observed dataset
\newcommand{\srfvec}{
	\bm{y}_{ \scaleto{ N }{3pt} } 
} 
% Vectors of the observed dataset n space-time domain
\newcommand{\strfvec}{% 
	\bm{y} 
} 
% Matrix of the observed dataset in space-time domain
\newcommand{\strfmat}{
	\bm{Y}_{N}
} 
% Vectors of the observed dataset n space-time domain
\newcommand{\freqresvec}[1]{% SRF
	\bm{z}_{ \scaleto{ #1 }{4pt} }  
} 
% Matrix of the observed dataset in space-time domain
\newcommand{\freqresmat}[1]{
	\bm{Z}_{ #1 }
} 
% FT coefficients vector
\newcommand{\ftcoeffvec}[1]{% SRF
	\bm{\zeta}_{ #1 } 
} 
% Max lag function: for smpaling variogram matrix
\newcommand{\maxlagfctsmp}{
	g_{ \scaleto{ \bm{\Gamma} }{4pt}}
} 
% Max lag function: for smpaling variogram matrix
\newcommand{\maxlagfctpred}{
	g_{ \bm{\upsilon} }
} 

%%% LInear  MMSE estimator
% CCF vector for linear MMSE estimate
\newcommand{\ccfvec}[1]{
	\bm{u}_{ \subscript{#1} }
}
% Covariance matrix for linear MMSE estimates
\newcommand{\ccfmat}[1]{% SRF (= indicates which covariance we are talking about)
	\bm{R}_{ \subscript{#1} }
} 

% Variogram vectors of the sampled positions 
\newcommand{\varismpvec}{\bm{\gamma}} 
% Variogram matrices of the sampled positions
\newcommand{\varismpMat}[1]{% subscript
	\bm{\Gamma}_{ \subscript{#1} }
} 
% Variogram vectors of the prediction positions (to the sampled positions) 
\newcommand{\varipredvec}[1]{ %subscript 
	\bm{\upsilon}_{ \subscript{#1} }
} 
% Variogram matrices of the prediction positions (to the sampled positions) 
\newcommand{\varipredMat}[1]{ %subscript
	\bm{\Upsilon}_{ \subscript{#1} }
}  




% FK relevant
% Modified Bessel function
\newcommand{\mBess}[2]{ %order, arguments
	K_{#1} \left(  #2 \right) 
} 
%======= Frequency variogram =======%
% Frequency Variogram
\newcommand{\fv}[3]{ % SRF, freq, lagvec
	\scaleto{\gamma}{7pt}_{ \subscript{ #1 } }^{ \left[  #2 \right] } ( #3 )
}
% Permutation matrix
\newcommand{\Permat}[2]{% subscript, upperscript
	\bm{ J }_{#1}^{#2}
} 
% ML relevant
\newcommand{\FeatMap}[1]{
	\bm{\Phi} ( #1 )
} 
\newcommand{\kernel}[2]{
	k ( #1, #2 )
} 

%%%%%%%%%%%%%%%%%%%%%%%%%%%%%%%%%%%%%%%%%%%%%%%%%%%%%%%%%%%%%%
%=====================  for roman numbers =======================%
%%%%%%%%%%%%%%%%%%%%%%%%%%%%%%%%%%%%%%%%%%%%%%%%%%%%%%%%%%%%%%

\newcommand{\rom}[1]{\uppercase\expandafter{\romannumeral #1\relax}}
% from https://tex.stackexchange.com/questions/23487/how-can-i-get-roman-numerals-in-text

%%%%%%%%%%%%%%%%%%%%%%%%%%%%%%%%%%%%%%%%%%%%%%
%======================================== table macro ====%
%%%%%%%%%%%%%%%%%%%%%%%%%%%%%%%%%%%%%%%%%%%%%%
% table scaling
\newcommand{\inputTable}[2]{%  
	\resizebox{#1}{!}{% Use \textwidth!!! (e.g. 0.9 \textwidth)
	 \input{#2}
	}
}


%%%%%%%%%%%%%%%%%%%%%%%%%%%%%%%%%%%%%%%%%%%%%%
%========================================== for TikZ  ====%
%%%%%%%%%%%%%%%%%%%%%%%%%%%%%%%%%%%%%%%%%%%%%%
\newcommand{\Nx}{3}
\newcommand{\Ny}{2}
\newcommand{\ddx}{1.5cm}
\newcommand{\ddy}{\ddx}
\newcommand{\ddz}{1.5cm}
\newcommand{\rCircle}{0.11cm}
\newcommand{\rCircleCamera}{0.07cm}

%%% draw transducer %%%
%5 3D
\newcommand{\drawTransducer}[4]{ % scale,scaley, x, y in axis
	\draw[draw = none, fill = white] (axis cs: #3 - #1, #4 - #2, 0) -- (axis cs: #3 + #1, #4 - #2, 0) -- (axis cs: #3 + #1, #4 + #2, 0) --  (axis cs: #3 + #1, #4 + #2, -#1) -- (axis cs: #3 - #1, #4 + #2, -#1) -- (axis cs: #3 - #1, #4 - #2, -#1) -- (axis cs: #3 - #1, #4 - #2, 0);
	\draw[] (axis cs: #3 - #1, #4 - #2, 0) -- (axis cs: #3 + #1, #4 - #2, 0) -- (axis cs: #3 + #1, #4 - #2, -#1) -- (axis cs: #3 - #1, #4 - #2, -#1) -- (axis cs: #3 - #1, #4 - #2, 0);
	\draw[] (axis cs: #3 + #1, #4 - #2, 0) -- (axis cs: #3 + #1, #4 + #2, 0) -- (axis cs: #3 + #1, #4 +  #2, -#1) -- (axis cs: #3 - #1, #4 +  #2, -#1) -- (axis cs: #3 - #1, #4 - #2, -#1);
	\draw[] (axis cs: #3 + #1, #4 - #2, -#1) -- (axis cs: #3 + #1, #4 +  #2, -#1);
}

%% 2D
\newcommand{\drawTransducerTwoD}[2]{ % scale, x  1,3
	\draw[draw = none, fill = white] (axis cs: #2 - #1, 0) -- (axis cs: #2 + #1, 0) --  (axis cs: #2 + #1, -#1) -- (axis cs: #2 - #1, -#1) -- (axis cs: #2 - #1, 0);
	\draw[] (axis cs: #2 - #1, 0) -- (axis cs: #2 + #1, 0) -- (axis cs: #2 + #1, -#1) -- (axis cs: #2 - #1, -#1) -- (axis cs: #2 - #1, 0);
	\draw[] (axis cs: #2 + #1, 0) -- (axis cs: #2 + #1, -#1) -- (axis cs: #2 - #1, -#1);
}

%%% draw video camera %%%
%% 3D
\newcommand{\drawCamera}[6]{ %<rotation origin (x, y, z)>, <start x>, <start y>, <start z> in axis, <camera width [dx]>, <camera height [dz]>
	% base
	\draw[rotate around = {30 : (axis cs: #1)}] (axis cs: #2, #3, #4) rectangle (axis cs: #2 + #5, #3, #4 - #6) ;
	% origin of rotation
	\node[campoint] (rotationorg) at (axis cs: #1) {};
	% "trapezoid" part (tip of the camera)
	\draw[rotate around = {30 : (axis cs: #1)}] (axis cs: #2, #3, #4 - 0.25*#6) -- (axis cs: #2 - 0.5* #6, #3, #4) -- (axis cs: #2 - 0.5* #6, #3, #4-#6) -- (axis cs: #2, #3, #4 - 0.75*#6);	
	% cable
	\draw[rotate around = {30 : (axis cs: #1)}] (axis cs: #2 + #5, #3, #4 - 0.5*  #6) .. controls (axis cs: #2 + 1.5* #5, #3, #4 - 0.5*  #6) and (axis cs: #2 + #5, #3, - 0.5*  #6) .. (axis cs: #2 + 1.5* #5, #3, - #6);	
}

%% 2D
\newcommand{\drawCameraTwoD}[5]{ %<rotation origin (x, z)>, <start x>, <start y>, <start z> in axis, <camera width [dx]>, <camera height [dz]> 1, 2, 4, 5, 6 -> 4-> 3, 5->4, 6-> 5
	% base
	\draw[rotate around = {30 : (axis cs: #1)}] (axis cs: #2, #3) rectangle (axis cs: #2 + #4, #3 - #5) ;
	% origin of rotation
	\node[campoint] (rotationorg) at (axis cs: #1) {};
	% "trapezoid" part (tip of the camera)
	\draw[rotate around = {30 : (axis cs: #1)}] (axis cs: #2, #3 - 0.25*#5) -- (axis cs: #2 - 0.5* #5, #3) -- (axis cs: #2 - 0.5* #5, #3-#5) -- (axis cs: #2, #3 - 0.75*#5);	
	% cable
	\draw[rotate around = {30 : (axis cs: #1)}] (axis cs: #2 + #4, #3 - 0.5*  #5) .. controls (axis cs: #2 + 1.5* #4, #3 - 0.5*  #5) and (axis cs: #2 + #4, - 0.5*  #5) .. (axis cs: #2 + 1.5* #4, - #5);	
}


%%%%%%%%%%%%%%%%%%%%%%%%%%%%%%%%%%%%%%%%%%%%%%
%======================================= image macro ====%
%%%%%%%%%%%%%%%%%%%%%%%%%%%%%%%%%%%%%%%%%%%%%%
% TikZ scaling
\newcommand{\inputTikZ}[2]{%<scaling factor>, <name of the tex file>
     \scalebox{#1}{\input{#2}}  
}


%=====================================================%
%====================== Animation =======================%
%=====================================================%

% GCNR  results with animation
\newcommand{\resultsanimate}[4]{ %<scale size>, <font size>, <font size>,<slide page for the mark>, <mark coordinates>
	\scalebox{#1}{
		\begin{tikzpicture}
			\pgfplotsset{ % Fix the x and y range
				xmin = 4.5, xmax = 17.5, 
				ymin = 0.84, ymax = 0.99, 
				scaled ticks=false % to avoid formtting with 10^-2 in y tick
			}
			%
			\begin{axis}[
				width = 10cm, height = 6cm, 
				grid=both,
				grid style={line width=.1pt, draw=gray!10},
				xlabel = {Coverage [$\%$] or \textcolor{tui_red}{samples / $\lambda^{2}$ }},
				ylabel = {$\MGCNR$},
				ylabel style = {font = #2},
				xlabel style = {
					font = #2, 
					yshift={-2em}
				},
				tick label style = {font = #2},
				ytick={0.86, 0.9, 0.94, 0.98},
				% Extra ticks for samples per squared wavelength
				extra x ticks = {5, 7, 10, 12, 15},
				extra x tick labels = {0.44, 0.61, 0.87, 1.04, 1.31},
				extra x tick style = {
					tui_red, 
					font = #2, 
					yshift={-1.5em}
				},
				legend style ={
					at={(1.5, 0.7)},
					nodes={scale=1, transform shape}
				}
				]
				% Plots
				% Only for intial samples                
				\input{figures/coords_1D/simulations/gcnr_smp.tex}
				
				% Marker 1
				\only<#3>{
					\addplot[tui_red, mark = star, mark size = 2pt] coordinates{
						#4
					};
				}   	
				% legend
				% To insert the legend title
				%\addlegendimage{empty legend} 
				% Legend entries  
				\addlegendentry{No preprocessing}
				\addlegendentry{IDW}
				\addlegendentry{SF-Krig.}
				% Title
				%\addlegendentry{$|s_{x} - x|$}     
			\end{axis}
		\end{tikzpicture}
	}
}


% GCNR  results with animation incl. reampling
\newcommand{\resultsanimateResmp}[4]{ %<scale size>, <font size>, <font size>,<slide page for the mark>, <mark coordinates>
	\scalebox{#1}{
		\begin{tikzpicture}
			\pgfplotsset{ % Fix the x and y range
				xmin = 4.5, xmax = 17.5, 
				ymin = 0.84, ymax = 0.99, 
				scaled ticks=false % to avoid formtting with 10^-2 in y tick
			}
			\begin{axis}[
				width = 10cm, height = 6cm, 
				grid=both,
				grid style={line width=.1pt, draw=gray!10},
				xlabel = {Coverage [$\%$] or \textcolor{tui_red}{samples / $\lambda^{2}$ }},
				ylabel = {$\MGCNR$},
				ylabel style = {font = #2},
				xlabel style = {
					font = #2, 
					yshift={-2em}
				},
				tick label style = {font = #2},
				ytick={0.86, 0.9, 0.94, 0.98},
				% Extra ticks for samples per squared wavelength
				extra x ticks = {5, 7, 10, 12, 15},
				extra x tick labels = {0.44, 0.61, 0.87, 1.04, 1.31},
				extra x tick style = {
					tui_red, 
					font = #2, 
					yshift={-1.5em}
				},
				legend style ={
					at={(1.5, 0.7)},
					nodes={scale=1, transform shape}
				}
				]
				
				% Plots
				% Only for intial samples                
				\input{figures/coords_1D/simulations/gcnr_smp.tex}
				% With resampled ones
				\input{figures/coords_1D/simulations/gcnr_resmp.tex} 
				
				% Marker
				\only<#3>{
					\addplot[tui_red, mark = star, mark size = 2pt] coordinates{
						#4
					};
				}   
				
				% legend
				% To insert the legend title
				%\addlegendimage{empty legend} 
				% Legend entries  
				\addlegendentry{No preprocessing}
				\addlegendentry{IDW}
				\addlegendentry{SF-Krig.}
				\addlegendentry{Resmp. IDW}
				\addlegendentry{Resmp. SF-Krig.}
				% Title
				%\addlegendentry{$|s_{x} - x|$}     
				
			\end{axis}
		\end{tikzpicture}
	}
}


%=====================================================%
%======================== 1D =========================%
%=====================================================%


% Variogram models
\newcommand{\plotvariomodels}[3]{ % <scale size>, <label font size>,  <tick font size>, 
	\scalebox{#1}{
		\begin{tikzpicture}
			\begin{axis}[
				height = 8.5cm,
				width = 12cm,
				grid=both,
				grid style={line width=.1pt, draw=gray!10},
				xlabel = {Normalized lag},
				ylabel = {Amplitude},
				label style = {font = #2},
				tick label style = {font = #3},
				legend style ={
					at={(0.95, 0.35)},
					nodes={scale=1, transform shape}
				}
				]
				\input{figures/coords_1D/variograms.tex} % addplot coordinates
				% legend
				\addlegendentry{Spherical}
				\addlegendentry{Exponential}
				\addlegendentry{Mat{\'e}rn}
				\addlegendentry{Tent}
			\end{axis}
		\end{tikzpicture}
	}
}

% Plot: covariance vs variogram
\newcommand{\plotcovvsvari}[3]{ % <scale size>, <label font size>,  <tick font size>, 
	\scalebox{#1}{
		\begin{tikzpicture}
			\begin{axis}[
				height = 8.5cm,
				width = 12cm,
				grid=both,
				grid style={line width=.1pt, draw=gray!10},
				xlabel = {Normalized lag},
				ylabel = {Amplitude},
				label style = {font = #2},
				tick label style = {font = #3},
				legend style ={
					at={(0.95, 0.65)},
					nodes={scale=1, transform shape}
				}
				]
				\input{figures/coords_1D/cov_vs_vari.tex} % addplot coordinates
				% legend
				\addlegendentry{$\acvf{ \lag{} }$}
				\addlegendentry{$\semivario{ \lag{} }$}
			\end{axis}
		\end{tikzpicture}
	}
}


% A-Scan without legend
\newcommand{\plotascan}[4]{ % <scale size>, <label font size>,  <tick font size>, <fname>
	\scalebox{#1}{
		\begin{tikzpicture}
			\begin{axis}[
				height = 3.5cm,
				width = 5.5cm,
				grid=both,
				grid style={line width=.1pt, draw=gray!10},
				xlabel = {$t$ [\SI{}{\micro \second}]},
				ylabel = {Amplitude},
				label style = {font = #2},
				tick label style = {font = #3},
				]
				\input{#4} % addplot coordinates
			\end{axis}
		\end{tikzpicture}
	}
}

% Plot predicted A-Scan vs ground truth
\newcommand{\plotapred}[4]{ % <scale size>, <label font size>,  <tick font size>, <fname>
	\scalebox{#1}{
		\begin{tikzpicture}
			\begin{axis}[
				grid=both,
				grid style={line width=.1pt, draw=gray!10},
				xlabel = {$t$ [\SI{}{\micro \second}]},
				ylabel = {Amplitude},
				label style = {font = #2},
				tick label style = {font = #3},
				legend style ={
					at={(0.95, 0.95)},
					nodes={scale=0.75, transform shape}
				}
				]
				\input{#4} % addplot coordinates
				% legend
				\addlegendentry{Ground truth}
				\addlegendentry{Prediction}
			\end{axis}
		\end{tikzpicture}
	}
}

% FV plots without legend
\newcommand{\plotFV}[4]{ % <scale size>, <label font size>,  <tick font size>, <fname>
	\scalebox{#1}{
		\begin{tikzpicture}
			\begin{axis}[
				grid=both,
				grid style={line width=.1pt, draw=gray!10},
				xlabel = {lag $/ \dx$ },
				ylabel = {Amplitude},
				label style = {font = #2},
				tick label style = {font = #3},
				]
				\input{#4} % addplot coordinates
			\end{axis}
		\end{tikzpicture}
	}
}

% FV plots compariosn
\newcommand{\plotFVcomp}[5]{ % <scale size>, <label font size>,  <tick font size>, <fname>, <legend to compare>
	\scalebox{#1}{
		\begin{tikzpicture}
			\begin{axis}[
				grid=both,
				grid style={line width=.1pt, draw=gray!10},
				xlabel = {lag $/ \dx$ },
				ylabel = {Amplitude},
				label style = {font = #2},
				tick label style = {font = #3},
				legend style ={
					at={(0.45, 0.95)},
					nodes={scale=0.75, transform shape}
				}
				]
				\input{#4} % addplot coordinates
				% legend
				\addlegendentry{Ground truth}
				\addlegendentry{#5}
			\end{axis}
		\end{tikzpicture}
	}
}

% DNN training results
\newcommand{\plotFVtrainingresults}[4]{ % <scale size>, <label font size>,  <tick font size>, <fname>
	\scalebox{#1}{
		\begin{tikzpicture}
			\begin{axis}[
				grid=both,
				grid style={line width=.1pt, draw=gray!10},
				xlabel = {lag $/ \dx$ },
				ylabel = {Amplitude},
				label style = {font = #2},
				tick label style = {font = #3},
				legend style ={
					at={(0.4, 0.95)},
					nodes={scale=0.75, transform shape}
				}
				]
				\input{#4} % addplot coordinates
				% legend
				\addlegendentry{Input}
				\addlegendentry{True output}
				\addlegendentry{Prediction}
			\end{axis}
		\end{tikzpicture}
	}
}

%% For defense
% GCNR  results with animation
\newcommand{\resultsSmpStatic}[2]{ %<scale size>, <font size>,
	\scalebox{#1}{
		\begin{tikzpicture}
			\pgfplotsset{ % Fix the x and y range
				xmin = 4.5, xmax = 17.5, 
				ymin = 0.84, ymax = 0.99, 
				scaled ticks=false % to avoid formtting with 10^-2 in y tick
			}
			%
			\begin{axis}[
				width = 10cm, height = 6cm, 
				grid=both,
				grid style={line width=.1pt, draw=gray!10},
				xlabel = {Coverage [$\%$] or \textcolor{tui_red}{samples / $\lambda^{2}$ }},
				ylabel = {$\MGCNR$},
				ylabel style = {font = #2},
				xlabel style = {
					font = #2, 
					yshift={-2em}
				},
				tick label style = {font = #2},
				ytick={0.86, 0.9, 0.94, 0.98},
				% Extra ticks for samples per squared wavelength
				extra x ticks = {5, 7, 10, 12, 15},
				extra x tick labels = {0.44, 0.61, 0.87, 1.04, 1.31},
				extra x tick style = {
					tui_red, 
					font = #2, 
					yshift={-1.5em}
				},
				legend style ={
					at={(1.5, 0.7)},
					nodes={scale=1, transform shape}
				}
				]
				% Plots
				% Only for intial samples                
				\input{figures/coords_1D/simulations/gcnr_smp.tex}
				
				% legend
				% To insert the legend title
				%\addlegendimage{empty legend} 
				% Legend entries  
				\addlegendentry{No preprocessing}
				\addlegendentry{IDW}
				\addlegendentry{SF-Krig.}
				% Title
				%\addlegendentry{$|s_{x} - x|$}     
			\end{axis}
		\end{tikzpicture}
	}
}



% Simulations: evaluation results
\newcommand{\plotrestulsGCNR}[3]{ % <scale size>, <label font size>,  <tick font size>
	\scalebox{#1}{
		\begin{tikzpicture}
			\begin{axis}[
				height = 8.5cm,
				width = 12cm,
				grid=both,
				grid style={line width=.1pt, draw=gray!10},
				xlabel = {Coverage [$\%$] or \textcolor{tui_red}{samples / $\lambda^{2}$ }},
				ylabel = {Relative $\MGCNR$},
				ylabel style = {font = #2},
				xlabel style = {
					font = #2, 
					yshift={-2.5em}
				},
				tick label style = {font = #3},
				% Extra ticks for samples per squared wavelength
				extra x ticks = {5, 7, 10, 12, 15},
				extra x tick labels = {0.44, 0.61, 0.87, 1.04, 1.31},
				extra x tick style = {
					tui_red, 
					font = #3, 
					yshift={-1.5em}
				},
				% Legend
				legend style ={
					at={(0.95, 0.4)},
					nodes={scale=0.75, transform shape}
				}
				]
				\input{figures/coords_1D/simulations/gcnr.tex} % addplot coordinates
				% legend
				\addlegendentry{#3 No preprocessing}
				\addlegendentry{#3 IDW}
				\addlegendentry{#3 SF-Kriging}
				\addlegendentry{#3 Resmp. IDW}
				\addlegendentry{#3 Resmp. SF-Kriging}
			\end{axis}
		\end{tikzpicture}
	}
}


% Simulations: evaluation results
\newcommand{\plotrestulsMSE}[3]{ % <scale size>, <label font size>,  <tick font size>
	\scalebox{#1}{
		\begin{tikzpicture}
			\begin{axis}[
				height = 8.5cm,
				width = 12cm,
				grid=both,
				grid style={line width=.1pt, draw=gray!10},
				xlabel = {Coverage [$\%$] or \textcolor{tui_red}{samples / $\lambda^{2}$ }},
				ylabel = {$\MSEdag$},
				ylabel style = {font = #2},
				xlabel style = {
					font = #2, 
					yshift={-2.5em}
				},
				tick label style = {font = #3},
				% Extra ticks for samples per squared wavelength
				extra x ticks = {5, 7, 10, 12, 15},
				extra x tick labels = {0.44, 0.61, 0.87, 1.04, 1.31},
				extra x tick style = {
					tui_red, 
					font = #3, 
					yshift={-1.5em}
				},
				% Legend
				legend style ={
					at={(0.95, 0.95)},
					nodes={scale=0.75, transform shape}
				}
				]
				\input{figures/coords_1D/simulations/mse.tex} % addplot coordinates
				% legend
				\addlegendentry{#3 No preprocessing}
				\addlegendentry{#3 IDW}
				\addlegendentry{#3 SF-Kriging}
				\addlegendentry{#3 Resmp. IDW}
				\addlegendentry{#3 Resmp. SF-Kriging}
			\end{axis}
		\end{tikzpicture}
	}
}


% Simulations: re/sampling positions
\newcommand{\plotscanpositions}[5]{ % <scale size>, <label font size>,  <tick font size>, <fname for p_smp>, <fname for p_resmp>
	\scalebox{#1}{
		\begin{tikzpicture}
			\begin{axis}[
				%height = 8.5cm,
				%width = 12cm,
				grid=both,
				grid style={line width=.1pt, draw=gray!10},
				xlabel = {$x$ [\SI{}{\milli \metre}]},
				ylabel = {$y$ [\SI{}{\milli \metre}]},
				label style = {font = #2},
				tick label style = {font = #3},
				xtick={0, 10, 20, 30, 40, 50},
				%xticklabels={0, 5, 10, 15, 20, 25},
				ytick={0, 10, 20, 30, 40, 50},
				y%ticklabels={0, 5, 10, 15, 20, 25},
				]
				\input{#4} % addplot coordinates: p_smp
				\input{#5} % addplot coordinates: p_resmp
			\end{axis}
		\end{tikzpicture}
	}
}


% Scan positions for batch itp
% Simulations: re/sampling positions
\newcommand{\plotbatchitppsmp}[3]{ % <scale size>, <label font size>,  <tick font size>
	\scalebox{#1}{
		\begin{tikzpicture}
			\begin{axis}[
				%height = 8.5cm,
				%width = 12cm,
				grid=both,
				grid style={line width=.1pt, draw=gray!10},
				minor x tick num=1,
				minor y tick num=1,
				xlabel = {$x / \dx$},
				ylabel =  {$y / \dy$},
				label style = {font = #2},
				tick label style = {font = #3},
				xtick={0, 2, ..., 10},
				ytick={0, 2, ..., 10},
				]
				\input{figures/coords_1D/batch_itp/p_smp.tex} % addplot coordinates: p_smp
				\input{figures/coords_1D/batch_itp/p_pred_point.tex} % addplot coordinates: p_pred
			\end{axis}
		\end{tikzpicture}
	}
}


%%%%%%%%%%%%%%%%%%%%%%%%%%%%%%%%%%%%%%%%%%%%%%
%=================== 2D visualization ======================%
%%%%%%%%%%%%%%%%%%%%%%%%%%%%%%%%%%%%%%%%%%%%%%

%%%%%%%%%%%%% Top-view %%%%%%%%%%%%%%%%

% top-view image (i.e. C-Scan) WITHOUT labels
\newcommand{\topviewnolabels}[4]{% <scale size>, <label font size>, <tick font size>, <png file name>
	% !!!!! The same ticks are to be used for both x- and y-ticks!!
	\scalebox{#1}{
		\begin{tikzpicture}
			\begin{axis}[
				enlargelimits = false,
				axis on top = true,
				axis equal image,
				unit vector ratio= 1 1, % change aspect ratio, one of them should be 1
				%xlabel = {$x$ [\SI{}{\milli \metre}]},
				%ylabel = {$y$ [\SI{}{\milli \metre}]},
				label style = {font = #2},
				tick label style = {font = #3},
				%y dir = reverse,
				xtick = {0, 5, 10, ..., 25},
				ytick = {0, 5, 10, ..., 25},
				]
				\addplot graphics [
				xmin = 0,
				xmax = 25,
				ymin = 0,
				ymax = 25
				]{#4};
			\end{axis}
		\end{tikzpicture}
	}
}


% top-view image (i.e. C-Scan) with both x- & y-labels
\newcommand{\topviewbothlabels}[4]{% <scale size>, <label font size>, <tick font size>, <png file name>
	% !!!!! The same ticks are to be used for both x- and y-ticks!!
	\scalebox{#1}{
		\begin{tikzpicture}
			\begin{axis}[
				enlargelimits = false,
				axis on top = true,
				axis equal image,
				unit vector ratio= 1 1, % change aspect ratio, one of them should be 1
				xlabel = {$x$ [\SI{}{\milli \metre}]},
				ylabel = {$y$ [\SI{}{\milli \metre}]},
				label style = {font = #2},
				tick label style = {font = #3},
				%y dir = reverse,
				xtick = {0, 5, 10, ..., 25},
				ytick = {0, 5, 10, ..., 25},
				]
				\addplot graphics [
				xmin = 0,
				xmax = 25,
				ymin = 0,
				ymax = 25
				]{#4};
			\end{axis}
		\end{tikzpicture}
	}
}

% top-view image (i.e. C-Scan) with only x-label
\newcommand{\topviewxlabel}[4]{% <scale size>, <label font size>, <tick font size>, <png file name>
	% !!!!! The same ticks are to be used for both x- and y-ticks!!
	\scalebox{#1}{
		\begin{tikzpicture}
			\begin{axis}[
				enlargelimits = false,
				axis on top = true,
				axis equal image,
				unit vector ratio= 1 1, % change aspect ratio, one of them should be 1
				xlabel = {$x$ [\SI{}{\milli \metre}]},
				%ylabel = {$y$ [\SI{}{\milli \metre}]},
				label style = {font = #2},
				tick label style = {font = #3},
				%y dir = reverse,
				xtick = {0, 5, 10, ..., 25},
				ytick = {0, 5, 10, ..., 25},
				]
				\addplot graphics [
				xmin = 0,
				xmax = 25,
				ymin = 0,
				ymax = 25
				]{#4};
			\end{axis}
		\end{tikzpicture}
	}
}


% top-view image (i.e. C-Scan) with only y-label
\newcommand{\topviewylabel}[4]{% <scale size>, <label font size>, <tick font size>, <png file name>
	% !!!!! The same ticks are to be used for both x- and y-ticks!!
	\scalebox{#1}{
		\begin{tikzpicture}
			\begin{axis}[
				enlargelimits = false,
				axis on top = true,
				axis equal image,
				unit vector ratio= 1 1, % change aspect ratio, one of them should be 1
				%xlabel = {$x$ [\SI{}{\milli \metre}]},
				ylabel = {$y$ [\SI{}{\milli \metre}]},
				label style = {font = #2},
				tick label style = {font = #3},
				%y dir = reverse,
				xtick = {0, 5, 10, ..., 25},
				ytick = {0, 5, 10, ..., 25},
				]
				\addplot graphics [
				xmin = 0,
				xmax = 25,
				ymin = 0,
				ymax = 25
				]{#4};
			\end{axis}
		\end{tikzpicture}
	}
}



% top-view image (i.e. C-Scan) without labels but with cmap
\newcommand{\topviewnolabelwithcmap}[4]{% <scale size>, <label font size>, <tick font size>, <png file name>
	% !!!!! The same ticks are to be used for both x- and y-ticks!!
	\scalebox{#1}{
		\begin{tikzpicture}
			\begin{axis}[
				enlargelimits = false,
				axis on top = true,
				axis equal image,
				unit vector ratio= 1 1, % change aspect ratio, one of them should be 1
				%xlabel = {$x$ [\SI{}{\milli \metre}]},
				%ylabel = {$y$ [\SI{}{\milli \metre}]},
				label style = {font = #2},
				tick label style = {font = #3},
				%y dir = reverse,
				xtick = {0, 5, 10, ..., 25},
				ytick = {0, 5, 10, ..., 25},
				%%%%% For color map %%%%%
				point meta min = -27.5,   
				point meta max = 0,
				colorbar,
				colorbar style={
					title= #2 [\SI{}{\decibel}]
				},
				colormap = {mymap}{rgb(0.0pt) = (0.97, 0.97, 0.98) ; rgb(0.06pt) = (0.99, 0.95, 0.92) ; rgb(0.12pt) = (0.99, 0.93, 0.87) ; rgb(0.75pt) = (0.94, 0.49, 0) ; rgb(0.9pt) = (0.78, 0.29, 0.0) ; rgb(0.97pt) = (0.75, 0.12, 0.06) ; rgb(1.0pt) = (0.61, 0.16, 0.15) ; } ,
				]
				\addplot graphics [
				xmin = 0,
				xmax = 25,
				ymin = 0,
				ymax = 25
				]{#4};
			\end{axis}
		\end{tikzpicture}
	}
}


% top-view image (i.e. C-Scan) with x label and cmap
\newcommand{\topviewxlabelwithcmap}[4]{% <scale size>, <label font size>, <tick font size>, <png file name>
	% !!!!! The same ticks are to be used for both x- and y-ticks!!
	\scalebox{#1}{
		\begin{tikzpicture}
			\begin{axis}[
				enlargelimits = false,
				axis on top = true,
				axis equal image,
				unit vector ratio= 1 1, % change aspect ratio, one of them should be 1
				xlabel = {$x$ [\SI{}{\milli \metre}]},
				%ylabel = {$y$ [\SI{}{\milli \metre}]},
				label style = {font = #2},
				tick label style = {font = #3},
				%y dir = reverse,
				xtick = {0, 5, 10, ..., 25},
				ytick = {0, 5, 10, ..., 25},
				%%%%% For color map %%%%%
				point meta min = -27.5,   
				point meta max = 0,
				colorbar,
				colorbar style={
					title= #2 [\SI{}{\decibel}]
				},
				colormap = {mymap}{rgb(0.0pt) = (0.97, 0.97, 0.98) ; rgb(0.06pt) = (0.99, 0.95, 0.92) ; rgb(0.12pt) = (0.99, 0.93, 0.87) ; rgb(0.75pt) = (0.94, 0.49, 0) ; rgb(0.9pt) = (0.78, 0.29, 0.0) ; rgb(0.97pt) = (0.75, 0.12, 0.06) ; rgb(1.0pt) = (0.61, 0.16, 0.15) ; } ,
				]
				\addplot graphics [
				xmin = 0,
				xmax = 25,
				ymin = 0,
				ymax = 25
				]{#4};
			\end{axis}
		\end{tikzpicture}
	}
}

% top-view image (i.e. C-Scan) with x label and cmap
\newcommand{\topviewbothlabelscmap}[4]{% <scale size>, <label font size>, <tick font size>, <png file name>
	% !!!!! The same ticks are to be used for both x- and y-ticks!!
	\scalebox{#1}{
		\begin{tikzpicture}
			\begin{axis}[
				enlargelimits = false,
				axis on top = true,
				axis equal image,
				unit vector ratio= 1 1, % change aspect ratio, one of them should be 1
				xlabel = {$x$ [\SI{}{\milli \metre}]},
				ylabel = {$y$ [\SI{}{\milli \metre}]},
				label style = {font = #2},
				tick label style = {font = #3},
				%y dir = reverse,
				xtick = {0, 5, 10, ..., 25},
				ytick = {0, 5, 10, ..., 25},
				%%%%% For color map %%%%%
				point meta min = -27.5,   
				point meta max = 0,
				colorbar,
				colorbar style={
					title= #2 [\SI{}{\decibel}]
				},
				colormap = {mymap}{rgb(0.0pt) = (0.97, 0.97, 0.98) ; rgb(0.06pt) = (0.99, 0.95, 0.92) ; rgb(0.12pt) = (0.99, 0.93, 0.87) ; rgb(0.75pt) = (0.94, 0.49, 0) ; rgb(0.9pt) = (0.78, 0.29, 0.0) ; rgb(0.97pt) = (0.75, 0.12, 0.06) ; rgb(1.0pt) = (0.61, 0.16, 0.15) ; } ,
				]
				\addplot graphics [
				xmin = 0,
				xmax = 25,
				ymin = 0,
				ymax = 25
				]{#4};
			\end{axis}
		\end{tikzpicture}
	}
}

%================ For batch interpolation ====================%
% With color map for batch interpolation
% top-view image (i.e. C-Scan) with both labels and cmap
\newcommand{\batchitptopview}[5]{% <scale size>, <label font size>, <tick font size>, <vmax>,  <png file name>
	\scalebox{#1}{
		\begin{tikzpicture}
			\begin{axis}[
				enlargelimits = false,
				axis on top = true,
				axis equal image,
				unit vector ratio= 1 1, % change aspect ratio, one of them should be 1
				xlabel = {$x / \dx$},
				ylabel = {$y / \dy$},
				label style = {font = #2},
				tick label style = {font = #3},
				xtick = {0, 5, 10},
				ytick = {0, 5, 10},
				%yticklabel = \empty,
				%y dir = reverse,
				%%%%% For color map %%%%%
				point meta min = 0,   
				point meta max = #4,
				colorbar,
				colorbar style={
					title= #2 [\SI{}{\decibel}]
				},
				colormap = {mymap}{rgb(0.0pt) = (0.9688, 0.9688, 0.9766) ; rgb(0.35pt) = (0.9883, 0.9297, 0.8672) ; rgb(0.75pt) = (0.9414, 0.4922, 0) ; rgb(1.0pt) = (0.7490, 0.1216, 0.0627); } ,
				]
				\addplot graphics [
				xmin = 0,
				xmax = 10,
				ymin = 0,
				ymax = 10
				]{#5};
			\end{axis}
		\end{tikzpicture}
	}
}


%================= GCNR: target area illustration =============%
% top-view image (i.e. C-Scan) with both x- & y-labels
\newcommand{\topviewMUSEroi}[3]{% <scale size>, <label font size>, <tick font size>
	% !!!!! The same ticks are to be used for both x- and y-ticks!!
	\scalebox{#1}{
		\begin{tikzpicture}
			\begin{axis}[
				enlargelimits = false,
				axis on top = true,
				axis equal image,
				unit vector ratio= 1 1, % change aspect ratio, one of them should be 1
				xlabel = {$x$ [\SI{}{\milli \metre}]},
				ylabel = {$y$ [\SI{}{\milli \metre}]},
				label style = {font = #2},
				tick label style = {font = #3},
				%y dir = reverse,
				xtick = {0, 10, 20, ..., 50},
				xticklabels = {0, 5, 10, ..., 25},
				ytick = {0, 10, 20, ..., 50},
				yticklabels = {0, 5, 10, ..., 25},
				]
				\addplot graphics [
				xmin = 0,
				xmax = 50,
				ymin = 0,
				ymax = 50
				]{images/muse_roi.png};
				% Target area
				\only<7->{
				\input{figures/coords_1D/simulations/target_area.tex}
				}
			\end{axis}
		\end{tikzpicture}
	}
}


%%%%%%%%%%%%% Side-view %%%%%%%%%%%%%%%%

% Side-view image (i.e. B-Scan) WITHOUT labels
\newcommand{\sideviewnolabels}[4]{% <scale size>, <label font size>, <tick font size>, <png file name>
	\scalebox{#1}{
		\begin{tikzpicture}
			\begin{axis}[
				enlargelimits = false,
				axis on top = true,
				axis equal image,
				unit vector ratio= 1 1, % change aspect ratio, one of them should be 1
				point meta min = -1,   
				point meta max = 1,
				%xlabel = {$x$ [\SI{}{\milli \metre}]},
				%ylabel = {$z$ [\SI{}{\milli \metre}]},
				label style = {font = #2},
				tick label style = {font = #3},
				y dir = reverse,
				xtick = {0, 5, 10, ..., 25},
				ytick = {77.3, 83, 88.5},
				]
				\addplot graphics [
				xmin = 0,
				xmax = 25,
				ymin = 77.3,
				ymax = 88.5
				]{#4};
			\end{axis}
		\end{tikzpicture}
	}
}

% Side-view image (i.e. B-Scan) with both x- & y-labels
\newcommand{\sideviewbothlabels}[4]{% <scale size>, <label font size>, <tick font size>, <png file name>
	\scalebox{#1}{
		\begin{tikzpicture}
			\begin{axis}[
				enlargelimits = false,
				axis on top = true,
				axis equal image,
				unit vector ratio= 1 1, % change aspect ratio, one of them should be 1
				point meta min = -1,   
				point meta max = 1,
				xlabel = {$x$ [\SI{}{\milli \metre}]},
				ylabel = {$z$ [\SI{}{\milli \metre}]},
				label style = {font = #2},
				tick label style = {font = #3},
				y dir = reverse,
				xtick = {0, 5, 10, ..., 25},
				ytick = {77.3, 83, 88.5},
				]
				\addplot graphics [
				xmin = 0,
				xmax = 25,
				ymin = 77.3,
				ymax = 88.5
				]{#4};
			\end{axis}
		\end{tikzpicture}
	}
}

% Side-view image (i.e. B-Scan) with only x-label
\newcommand{\sideviewxlabel}[4]{% <scale size>, <label font size>, <tick font size>, <png file name>
	\scalebox{#1}{
		\begin{tikzpicture}
			\begin{axis}[
				enlargelimits = false,
				axis on top = true,
				axis equal image,
				unit vector ratio= 1 1, % change aspect ratio, one of them should be 1
				point meta min = -1,   
				point meta max = 1,
				xlabel = {$x$ [\SI{}{\milli \metre}]},
				%ylabel = {$z$ [\SI{}{\milli \metre}]},
				label style = {font = #2},
				tick label style = {font = #3},
				y dir = reverse,
				xtick = {0, 5, 10, ..., 25},
				ytick = {77.3, 83, 88.5},
				]
				\addplot graphics [
				xmin = 0,
				xmax = 25,
				ymin = 77.3,
				ymax = 88.5
				]{#4};
			\end{axis}
		\end{tikzpicture}
	}
}


% Side-view image (i.e. B-Scan) with only y-label
\newcommand{\sideviewylabel}[4]{% <scale size>, <label font size>, <tick font size>, <png file name>
	\scalebox{#1}{
		\begin{tikzpicture}
			\begin{axis}[
				enlargelimits = false,
				axis on top = true,
				axis equal image,
				unit vector ratio= 1 1, % change aspect ratio, one of them should be 1
				point meta min = -1,   
				point meta max = 1,
				%xlabel = {$x$ [\SI{}{\milli \metre}]},
				ylabel = {$z$ [\SI{}{\milli \metre}]},
				label style = {font = #2},
				tick label style = {font = #3},
				y dir = reverse,
				xtick = {0, 5, 10, ..., 25},
				ytick = {77.3, 83, 88.5},
				]
				\addplot graphics [
				xmin = 0,
				xmax = 25,
				ymin = 77.3,
				ymax = 88.5
				]{#4};
			\end{axis}
		\end{tikzpicture}
	}
}

% Side-view image (i.e. B-Scan) with only x-label with color map
\newcommand{\sideviewxlabelwithcmap}[4]{% <scale size>, <label font size>, <tick font size>, <png file name>
	\scalebox{#1}{
		\begin{tikzpicture}
			\begin{axis}[
				enlargelimits = false,
				axis on top = true,
				axis equal image,
				unit vector ratio= 1 1, % change aspect ratio, one of them should be 1
				point meta min = -1,   
				point meta max = 1,
				xlabel = {$x$ [\SI{}{\milli \metre}]},
				%ylabel = {$z$ [\SI{}{\milli \metre}]},
				label style = {font = #2},
				tick label style = {font = #3},
				y dir = reverse,
				xtick = {0, 5, 10, ..., 25},
				ytick = {77.3, 83, 88.5},
				%%%%% For color map %%%%%
				point meta min = -1,   
				point meta max = 1,
				colorbar,
				colormap = {mymap}{rgb(0.0pt) = (0.04, 0.15, 0.18) ; rgb(0.1pt) = (0, 0.22, 0.39) ; rgb(0.47pt) = (0.91, 0.93, 0.96) ; rgb(0.49pt) = (0.97, 0.98, 0.99) ; rgb(0.5pt) = (0.97, 0.97, 0.98) ; rgb(0.51pt) = (0.99, 0.95, 0.92) ; rgb(0.53pt) = (0.99, 0.93, 0.87) ; rgb(0.8pt) = (0.94, 0.49, 0) ; rgb(0.95pt) = (0.75, 0.12, 0.06) ; rgb(1.0pt) = (0.61, 0.16, 0.15) ; } ,
				]
				\addplot graphics [
				xmin = 0,
				xmax = 25,
				ymin = 77.3,
				ymax = 88.5
				]{#4};
			\end{axis}
		\end{tikzpicture}
	}
}


