\documentclass[11pt,compress,aspectratio=169]{beamer} % selection of font size = 9, 10, 11, 12, 14, ....pt


% specify the language of the document here
%
\newcommand\setslidelanguage{english}
%\newcommand\setslidelanguage{german}

%
% this flag turns on a basic Fraunhofer CI overwrite
% this just changes colors atm, types and more convenient
% theme handling have to be added
%
%\newcommand\setslidetheme{tui} % - alpha
%\newcommand\setslidetheme{tuiiis} % - alpha
%\newcommand\setslidetheme{tuiizfp} % - alpha
%\newcommand\setslidetheme{iis} % - alpha
%\newcommand\setslidetheme{iistui} % - alpha
%\newcommand\setslidetheme{izfp} % - alpha
\newcommand\setslidetheme{izfptui} % - alpha


%
% load header with all includes
%
%%% input encoding, font encoding
\usepackage[T1]{fontenc}
\usepackage[utf8]{inputenc}
\usepackage{lmodern}


%%% beamerstuff - requirements
\usepackage{ifthen}
\usepackage[absolute,overlay]{textpos}
\setlength{\TPHorizModule}{1cm}
\setlength{\TPVertModule}{1cm}


%%% presentation style stuff
% load a lot of custom macros and sets the
% beamer theme
% when new logos and stuff are required
% look there
%%% presentation style stuff


%%%%%%%%%%%%%%%%%%%%%%%%%%%%%%%%%%%%%%%%%%%%%%%%%%%%%%%%%%%%%%%%%%%%%%%%%%%%%%%%%%%%%%%%%%%%%%%%%%%%%%%%%%%%%%%%%%%%%%%%%%%%%%%%%%%%%
% general stuff

% thanks to https://tex.stackexchange.com/questions/2541/beamer-frame-numbering-in-appendix
% this helps us to organize framenumbering s.t. they do not increase for the appendix
\newcommand{\backupbegin}{
   \newcounter{framenumberappendix}
   \setcounter{framenumberappendix}{\value{framenumber}}
}
\newcommand{\backupend}{
   \addtocounter{framenumberappendix}{-\value{framenumber}}
   \addtocounter{framenumber}{\value{framenumberappendix}} 
}


% this is required to omit problems with tikz-externalize
% as I did all that footer/header stuff using tikz
% so we have to prevent tikz externalize from
% externalizing these
% otherwise framecounters and sections will screw up
\newcommand{\donotexternalize}[1]{
\tikzset{external/export=false}
#1
\tikzset{external/export=true}
}


%%%%%%%%%%%%%%%%%%%%%%%%%%%%%%%%%%%%%%%%%%%%%%%%%%%%%%%%%%%%%%%%%%%%%%%%%%%%%%%%%%%%%%%%%%%%%%%%%%%%%%%%%%%%%%%%%%%%%%%%%%%%%%%%%%%%%
% TU Ilmenau stuff

%% Logos
\newcommand\logotuiwhite{\includegraphics[width=1.86cm]{logos/tu_bw_white.pdf}}
\newcommand\logotuicolor{\includegraphics[width=1.86cm]{logos/tu_col_green.pdf}}

\newcommand\settuitextmacros{
	% Sets macros for text formatting
	\newcommand<>{\cihighlight}[1]{\textcolor##2{tui_orange_dark}{##1}}
	\newcommand<>{\cigray}[1]{\textcolor##2{tui_gray_dark}{##1}}
	\newcommand<>{\cihide}[1]{\textcolor##2{white}{##1}}
}


%% Style
\newcommand{\setTUIStyleTwoLogos}{
	\mode<presentation>
		{
			\usepackage{theme/beamerthemetui}
			\setbeamercovered{transparent}
		}
	}
	
	
	
% for enabling 16:9
\newlength{\ruleswidth}
\setlength{\ruleswidth}{\dimexpr(\paperwidth-20mm)\relax}
	
	
%%%%%%%%%%%%%%%%%%%%%%%%%%%%%%%%%%%%%%%%%%%%%%%%%%%%%%%%%%%%%%%%%%%%%%%%%%%%%%%%%%%%%%%%%%%%%%%%%%%%%%%%%%%%%%%%%%%%%%%%%%%%%%%%%%%%%
% Fraunhofer stuff


%% Logos
\newcommand\logoizfpwhite{\raisebox{-6.68mm}{\includegraphics[width=1.86cm]{logos/izfp_85mm_p334_w.pdf}}}
\newcommand\logoizfpcolor{\raisebox{-6.68mm}{\includegraphics[width=1.86cm]{logos/izfp_85mm_p334-eps-converted-to.pdf}}}
\newcommand\logoiiswhite{\raisebox{-6.72mm}{\includegraphics[width=1.86cm]{logos/iis_85mm_p334.pdf}}}
\newcommand\logoiiscolor{\raisebox{-6.72mm}{\includegraphics[width=1.86cm]{logos/iis_85mm_p334.pdf}}}


\newcommand\setfhitextmacros{
	% Sets macros for text formatting
	\newcommand<>{\cihighlight}[1]{\textcolor##2{fraunhofer_orange}{##1}}
	\newcommand<>{\cigray}[1]{\textcolor##2{fraunhofer_silver!70!black}{##1}}
	\newcommand<>{\cihide}[1]{\textcolor##2{white}{##1}}
	% taken from https://tex.stackexchange.com/questions/23034/how-to-get-larger-item-symbols-for-some-lists-in-a-beamer-	
	% presentation and
	% https://tex.stackexchange.com/questions/87133/changing-the-color-of-itemize-item-in-beamer
	\setbeamertemplate{itemize subsubitem}{\tiny\raise0.5pt\hbox{\color{fraunhofer_silver}$ \blacksquare$}}
}

\newcommand{\setFHIStyleTwoLogos}{
	\mode<presentation>
		{
			\usepackage{theme/beamerthemefhi}
			\setbeamercovered{transparent}
		}
	% Big TODO: Frutiger font
}




%%%%%%%%%%%%%%%%%%%%%%%%%%%%%%%%%%%%%%%%%%%%%%%%%%%%%%%%%%%%%%%%%%%%%%%%%%%%%%%%%%%%%%%%%%%%%%%%%%%%%%%%%%%%%%%%%%%%%%%%%%%%%%%%%%%%%
% set everything up using the macros
%% TUI
\ifthenelse{\equal{\setslidetheme}{tui}}
{
		\newcommand\setfirstlogo{\logotuiwhite}
		\newcommand\setsecondlogo{}
		
		\setTUIStyleTwoLogos
		
		\settuitextmacros
}{}
\ifthenelse{\equal{\setslidetheme}{tuiiis}}
{
		\newcommand\setfirstlogo{\logotuiwhite}
		\newcommand\setsecondlogo{\logoiiswhite}
		
		\setTUIStyleTwoLogos
		
		\settuitextmacros
}{}
\ifthenelse{\equal{\setslidetheme}{tuiizfp}}
{
		\newcommand\setfirstlogo{\logotuiwhite}
		\newcommand\setsecondlogo{\logoizfpwhite}
		
		\setTUIStyleTwoLogos
		
		\settuitextmacros
}{}


%% Fraunhofer 
\ifthenelse{ \equal{\setslidetheme}{iis}}
{
		\newcommand\setfirstlogo{\logoiiscolor}
		\newcommand\setsecondlogo{}

		\setFHIStyleTwoLogos
		
		\setfhitextmacros
}{}
\ifthenelse{ \equal{\setslidetheme}{izfp}}
{
		\newcommand\setfirstlogo{\logoizfpcolor}
		\newcommand\setsecondlogo{}
		
		\setFHIStyleTwoLogos
		
		\setfhitextmacros
}{}
\ifthenelse{ \equal{\setslidetheme}{iistui}}
{
		\newcommand\setfirstlogo{\logoiiscolor}
		\newcommand\setsecondlogo{\logotuicolor}

		\setFHIStyleTwoLogos
		
		\setfhitextmacros
}{}
\ifthenelse{ \equal{\setslidetheme}{izfptui}}
{
		\newcommand\setfirstlogo{\logoizfpcolor}
		\newcommand\setsecondlogo{\logotuicolor}

		\setFHIStyleTwoLogos
		
		\setfhitextmacros
}{}




%%% language settings
\ifthenelse{ \equal{\setslidelanguage}{german}}
	{	\usepackage[ngerman]{babel} }
	{	\usepackage[english]{babel} }


%%% general (useful) packages
% math
\usepackage{amsmath}
\usepackage{bm}
\usepackage{mathtools} % For \coloneqq

% graphics
\usepackage{pgfplots}
\pgfplotsset{compat=newest}
\usepackage{animate}
\usepackage{tikz}
\usepackage{multido}
\usetikzlibrary{patterns}
\usetikzlibrary{matrix,arrows,shapes,trees,fit, decorations, spy}
\usetikzlibrary{calc,positioning,scopes,backgrounds}
\usetikzlibrary{tikzmark,decorations.pathreplacing,calligraphy}
\usepgfplotslibrary{colormaps}
\usetikzlibrary{external} % most useful, but requires --shell-escape
\usetikzlibrary{petri} %for tokens in the block diagram for iterative GD
% this can be used if you want to use tikzexternalize
% it precompiles your plots to a subfolder called
% figures/tikz-ext/ and names them according to
% the section 
% see https://tex.stackexchange.com/questions/177292/chapter-and-figure-numbers-in-tikz-externalize-prefix
%
%
%\tikzexternalize[
%    prefix=figures/tikz-ext/,
%    figure name=plot_sec\thesubsection_no,
%]

%% for putting figures at the specific locations
\usepackage[absolute,overlay]{textpos}
\setlength{\TPHorizModule}{1mm}
\setlength{\TPVertModule}{1mm}

% for checkmark in conclusion
\usepackage{bbding}


% settings and default macros
\usepackage[load-configurations=abbreviations,load-configurations=binary,binary-units=true]{siunitx}	%korrektes Setzen von SI-Einheiten
\DeclareSIUnit{\mms}{\milli\squaremetre}
\DeclareSIUnit{\inch}{in}
\DeclareSIUnit{\inchs}{in\squared}
\DeclareSIUnit{\mil}{mil}
\DeclareSIUnit{\Msps}{Msps}
\DeclareSIUnit{\Mbps}{Mbps}
\DeclareSIUnit{\LSB}{LSB}
\DeclareSIUnit{\pFS}{\percent FS}
\DeclareSIUnit{\dBc}{\deci\bel c}
\DeclareSIUnit{\dBm}{\deci\bel m}
\DeclareSIUnit{\dBFS}{\deci\bel FS}
\DeclareSIUnit{\dB}{\deci\bel}
\DeclareSIUnit{\dBi}{\deci\bel i}
\DeclareSIUnit{\hex}{0x}
\DeclareSIUnit{\vp}{\volt_{\text{p}}}
\DeclareSIUnit{\vpp}{\volt_{\text{pp}}}
\DeclareSIUnit{\kb}{\kilo\bit}
\DeclareSIUnit{\kB}{\kilo\byte}
\DeclareSIUnit{\MB}{\mega\byte}




%
% your settings and includes go here
%
%%%% bibliography %%%%%
\usepackage{url}
\usepackage{natbib} % required for the citation example
\usepackage{bibentry} % required for the citation example
%
\nobibliography* % required for the citation example
\bibliographystyle{plain}

% "newcommand" collections
% !!! In newcommands: numbers cannot be used for the command names!!!
%=====================  for general =======================%
\renewcommand\thesubsection{\thesection.\Alph{subsection}}

\newcommand{\ToDo}[1]{%
	\textcolor{red}{#1}
}

%=====================  for math =======================%
% Math operands
% l2 norms
\newcommand{\norm}[1]{%
	\left\lVert#1\right\rVert
}
% Transpose
\newcommand{\TP}[1]{%
	#1^{\T}
}
% Vectorization
\newcommand{\vectorize}[1]{%
	\operatorname{vec} \{ #1 \}
}
% Diagonal matrix
\newcommand{\diag}[1]{%
	\operatorname{diag} \{ #1 \}
}
% Real part
\newcommand{\Real}{%  
     \operatorname{Re}
}
 % Imaginary part
\newcommand{\Imag}{%  
     \operatorname{Im}
}
% Column space
\newcommand{\Col}[1]{
	\operatorname{col} \{ #1 \}
}
% Nullspace 
\newcommand{\Null}[1]{
	\operatorname{null} \{ #1 \}
} 


% symbols
% General
\newcommand{\Identity}[1]{\bm{I}_{#1}} % Identity matrix
\newcommand{\RR}{{\mathbb{R}}}
\newcommand{\CC}{{\mathbb{C}}}

% TLS relevant
\newcommand{\Mone}{\bm{\Omega}}
\newcommand{\MoneDelta}{\Delta \Mone}
\newcommand{\MonePerturb}{\Mone + \MoneDelta}
\newcommand{\Mtwo}{\bm{\Psi}}
\newcommand{\MtwoDelta}{\Delta \Mtwo}
\newcommand{\MtwoPerturb}{\Mtwo + \MtwoDelta}
\newcommand{\Mthree}{\bm{\Upsilon}}
\newcommand{\Mfour}{\bm{T}}
\newcommand{\UU}{\bm{U}} % SVD: U
\newcommand{\VV}{\bm{V}} % SVD: V
\newcommand{\SSigma}{\bm{\Sigma}} % SVD: Sigma
% Newton relevant
\newcommand{\gradf}{g}
\newcommand{\Hess}{\bm{H_{f}}}
\newcommand{\dd}[1]{\bm{d}_{#1}}
\newcommand{\iter}[1]{\xx_{#1}}

% scan positions
% p = scalar, \pp = vector
\newcommand{\pp}{\bm{p}}
\newcommand{\pphat}{\bm{\hat{p}}}
\newcommand{\ppdelta}{\Delta \pp}
% x = scalar, \xx = vector
\newcommand{\xdelta}[1]{\Delta x_{#1}}
\newcommand{\xhat}[1]{\hat{x}_{#1}}
\newcommand{\xopt}{\xhat_{\optimized}}
\newcommand{\xx}{\bm{x}}
\newcommand{\xxdelta}{\Delta \xx}
\newcommand{\xxhat}{\hat{\xx}}
\newcommand{\xxopt}{\xxhat_{\optimized}}
\newcommand{\xxdeltahat}{\Delta \xxhat}
\newcommand{\xxdeltaopt}{\xxdeltahat_{\optimized}}
% scatter position
\newcommand{\scatterer}{\bm{s}}
% A-Scans
\newcommand{\ascansig}{a}
\newcommand{\ascan}[1]{\bm{a}_{#1}} % with \xx
\newcommand{\ascanhat}[1]{\hat{\ascan}_{#1}} % with \xxhat
\newcommand{\Ascan}{\bm{A}} % with \xx
\newcommand{\Ascanhat}{\hat{\Ascan}} % with \xxhat
% Pulse, SAFT matrix, spatial approximation
\newcommand{\pulsesig}{h}
\newcommand{\pulse}{\bm{h}}
\newcommand{\pulsehat}{\hat{\pulse}}
\newcommand{\pulsederiv}{\bm{j}}
\newcommand{\pulsederivhat}{\hat{\pulsederiv}}
\newcommand{\SAFTcol}[2]{\pulse_{#1}^{(#2)}}
\newcommand{\SAFThatcol}[2]{\pulsehat_{#1}^{(#2)}}
\newcommand{\SAFT}{\bm{H}}
\newcommand{\SAFThat}{\hat{\SAFT}}
\newcommand{\SAFTdot}{\bm{J}}
\newcommand{\SAFTdotcol}[2]{\pulsederiv_{#1}^{(#2)}}
\newcommand{\SAFTdothat}{\hat{\SAFTdot}}
\newcommand{\SAFTdothatcol}[2]{\pulsederivhat_{#1}^{(#2)}}
\newcommand{\Error}{\bm{E}}
\newcommand{\ErrorDef}{\diag{\xxdelta} \otimes \Identity{\M}}
% Defect map
\newcommand{\defect}{\bm{b}} 
\newcommand{\defecthat}{\hat{\defect}}
% Reco
\newcommand{\Reco}{\bm{B}}  
\newcommand{\Recohat}{\hat{\Reco}}  

% TLS curve fit
\newcommand{\zz}{\bm{z}}
\newcommand{\zzperturb}{\zz + \Delta \zz}
\newcommand{\XX}{\bm{X}}
\newcommand{\XXperturb}{\XX + \Delta \XX}
\newcommand{\ww}{\bm{w}}
% Iterative EC via Newton
\newcommand{\vonebase}{\alpha}
\newcommand{\voneall}{\bm{\tilde{\vonebase}}}
\newcommand{\vonepart}[1]{\bm{\vonebase}_{#1}}
\newcommand{\vtwobase}{j}
\newcommand{\vtwoall}{\bm{\tilde{\vtwobase}}}
\newcommand{\vtwopart}[1]{\bm{\vtwobase}_{#1}}
% Metrics
\newcommand{\SEdag}{\SE^{\dagger}} 
\newcommand{\MSEdag}{\MSE^{\dagger}} 

% else
\newcommand{\refcoeff}{\beta}
\newcommand{\noisevec}{\bm{n}} 


% Dimensions
\newcommand{\N}{N}
\newcommand{\M}{M}
\newcommand{\K}{K}
\newcommand{\LL}{L}
\newcommand{\I}{I}

%=====================  for roman numbers =======================%

\newcommand{\rom}[1]{\uppercase\expandafter{\romannumeral #1\relax}}
% from https://tex.stackexchange.com/questions/23487/how-can-i-get-roman-numerals-in-text


%=====================  for tables =======================%
\newcommand{\inputTable}[4]{ % <scaling factor>, <file name for the table>, <caption>, <label>
	%\resizebox{#1}{!}{ -> only works in the beamer setting?
		\begin{table}
		\begin{center}
			\input{#2}
			\caption{#3}
			\label{#4}
		\end{center}
		\end{table}
	%}
}

%=====================  for TikZ =======================%
\newcommand{\inputTikZ}[2]{%  
     \scalebox{#1}{\input{#2}}  
}
%%%%%%%%%%%%%%%%%%%%%%%%%%%%%%%%%%%%%%%%%%%%%
%=================  for data visualization ===================%
%%%%%%%%%%%%%%%%%%%%%%%%%%%%%%%%%%%%%%%%%%%%%

%%%%%%%%%%%%%%%%%%%%%%%%%%%%%%%%%%%%%%%%%%%%%%
%=================== 1D visualization ======================%
%%%%%%%%%%%%%%%%%%%%%%%%%%%%%%%%%%%%%%%%%%%%%%

% SE tolerance
\newcommand{\resultSEtolerance}[6]{ % <scale size>, <font size>,  <fname for track 50mm>, <fname for opt 50mm>,  <fname for track 30mm>, <fname for opt 30mm>
\scalebox{#1}{
	\begin{tikzpicture}
            \begin{axis}[
                width = 7cm, height = 4cm,
                xlabel = {Tracking error / $\lambda$}, ylabel = {$\MSEdag$},
                ymin= -0.04, ymax= 0.65,
                label style = {font = #2},
                tick label style = {font = #2},
                xtick = {0, 0.2, ..., 1.01},
                ytick = {0, 0.2, ..., 0.6}, 
                grid=both, grid style={line width=.1pt, draw=gray!20},
                legend style ={
                	at={(1.45, 0.7)},
                	nodes={scale=0.95, transform shape},
                	font = #2
                }
                ]
                \input{#3} % track 50mm
                \input{#4} % opt 50mm
                \input{#5} % track 30mm
                \input{#6} % opt 30mm
                
             % legend
             % To insert the legend title
		   	%\addlegendimage{empty legend} 
		   	% Legend entries  
             \addlegendentry{No correction}
             \addlegendentry{BEC} 
             % Title
             %\addlegendentry{$|s_{x} - x|$}        
            \end{axis}
	\end{tikzpicture}
	}
}

% SE depth 
\newcommand{\resultSE}[7]{ % <scale size>, <font size>, <xlabel>, <ymax>, <xtick>, <fname for track>, <fname for opt>
\scalebox{#1}{
	\begin{tikzpicture}
            \begin{axis}[
                width = 7cm, height = 4cm,
                xlabel = {#3}, ylabel = {$\MSEdag$},
                ymin= -0.04, ymax= #4,
                label style = {font = #2},
                tick label style = {font = #2},
                xtick = {#5},
                ytick = {0, 0.2, ..., #4}, 
                grid=both, grid style={line width=.1pt, draw=gray!20},
                legend style ={
                	at={(1.45, 0.7)},
                	nodes={scale=0.95, transform shape},
                	font = #2
                }
                ]
                \input{#6} % track
                \input{#7} % opt
                
             % legend
             % To insert the legend title
		   	%\addlegendimage{empty legend} 
		   	% Legend entries  
             \addlegendentry{No correction}
             \addlegendentry{BEC} 
             % Title
             %\addlegendentry{$|s_{x} - x|$}        
            \end{axis}
	\end{tikzpicture}
	}
}

% API
\newcommand{\resultAPI}[9]{ % <scale size>, <font size>, <xlabel>, <ymax>, <xtick>, <ytick>, <fname for true>, <fname for track>, <fname for opt>
\scalebox{#1}{
	\begin{tikzpicture}
            \begin{axis}[
                width = 7cm, height = 4cm,
                xlabel = {#3}, ylabel = {$\MAPI$},
                ymin= 16.5, ymax= #4,
                label style = {font = #2},
                tick label style = {font = #2},
                xtick = {#5},
                ytick = {#6},
                grid=both, grid style={line width=.1pt, draw=gray!20},
                legend style ={
                	at={(1.45, 0.8)},
                	nodes={scale=0.95, transform shape},
                	font = #2
                }
                ]
                \input{#7} % true
                \input{#8} % track
                \input{#9} % opt
                
             % legend
             % To insert the legend title
		   	%\addlegendimage{empty legend} 
		   	% Legend entries  
		   	\addlegendentry{Reference}
             \addlegendentry{No correction}
             \addlegendentry{BEC}
             % Title
             %\addlegendentry{$|s_{x} - x|$}        
            \end{axis}
	\end{tikzpicture}
	}
}


% GCNR
\newcommand{\resultGCNR}[9]{ % <scale size>, <font size>, <xlabel>, <ymin>, <xtick>, <ytick>, <fname for true>, <fname for track>, <fname for opt>
\scalebox{#1}{
	\begin{tikzpicture}
            \begin{axis}[
                width = 7cm, height = 4cm,
                xlabel = {#3}, ylabel = {$\MGCNR$},
                ymin= #4, ymax= 0.97,
                label style = {font = #2},
                tick label style = {font = #2},
                xtick = {#5},
                ytick = {#6},  % 1.01 = otherwise the tick does not show up 
                grid=both, grid style={line width=.1pt, draw=gray!20},
                legend style ={
                	at={(1.45, 0.8)},
                	nodes={scale=0.95, transform shape},
                	font = #2
                }
                ]
                \input{#7} % true
                \input{#8} % track
                \input{#9} % opt
                
             % legend
             % To insert the legend title
		   	%\addlegendimage{empty legend} 
		   	% Legend entries  
		   	\addlegendentry{Reference}
             \addlegendentry{No correction}
             \addlegendentry{BEC}
             % Title
             %\addlegendentry{$|s_{x} - x|$}        
            \end{axis}
	\end{tikzpicture}
	}
}

% GD SE
\newcommand{\gdse}[8]{ %  <scale size>, <label font size>, <tick font size>, <fname for 0.5 lambda>, <fname for 1mm>, <fname for 2.5mm>, <fname for 5mm>, <fname for 7.5mm>
\scalebox{#1}{
	\begin{tikzpicture}
            \begin{axis}[
                width = 7cm, 
            	   height = 4cm,
            	   xmin = -2.1,
            	   xmax = 2.1,
            	   ymin = -0.1,
            	   ymax = 1.1,
                xlabel = {$\xdelta / \lambda$},
                ylabel = {$\SEdag$},
                label style = {font = #2},
                tick label style = {font = #3},
                %y dir = reverse,
                %xtick = {0, 10, ..., 30}, %to customize the axis
                extra x ticks={0.8}, 
                extra x tick style={font = #3, tui_orange},%yshift={-1.2em}
                %xticklabel = {0, 5, 10, 15}
                legend style ={
                	at={(1.4, 0.9)},
                	nodes={scale=0.95, transform shape},
                	font = #3
                }
                ]
                % Reverse the input order, so that 7.5mm away is sent to background
                \input{#8} %7.5mm
                \input{#7} %5mm
                \input{#6} %2.5mm
                \input{#5} %1mm
                \input{#4} %0.5 lambda
                
             % legend
             % To insert the legend title
		   	\addlegendimage{empty legend} 
		   	% Legend entries  
             \addlegendentry{\SI{7.5}{\milli\metre}}
             \addlegendentry{\SI{5}{\milli\metre}}
             \addlegendentry{\SI{2.5}{\milli\metre}} %{$1.98 \lambda$}
             \addlegendentry{\SI{1}{\milli\metre}} %{$0.8 \lambda$}
             \addlegendentry{\SI{0.63}{\milli\metre}} %{$0.5 \lambda$}
             % Title
             \addlegendentry{$| s_{x} - x |$}
             
%             % x = 0.8 line
%	            \addplot[gray, dashed, line width = 1pt, mark = ] coordinates{
%	            			(0.8, 0.2)
%	            			(0.8, -0.1)
%            		};
                         
            \end{axis}
	\end{tikzpicture}
	}
}

%%%%%%%%%%%%%%%%%%%%%%%%%%%%%%%%%%%%%%%%%%%%%%
%=================== 2D visualization ======================%
%%%%%%%%%%%%%%%%%%%%%%%%%%%%%%%%%%%%%%%%%%%%%%
% image with both x- & y-labels
\newcommand{\imgbothlabels}[7]{% <scale size>, <label font size>, <tick font size>,<ytick>, <ymin>, <ymax>, <png file name>
\scalebox{#1}{
	\begin{tikzpicture}
            \begin{axis}[
                enlargelimits = false,
                axis on top = true,
                axis equal image,
                unit vector ratio= 0.5 1, % change aspect ratio, one of them should be 1
                point meta min = -1,   
                point meta max = 1,
                xlabel = {$x$ [\SI{}{\milli \metre}]},
                ylabel = {$z$ [\SI{}{\milli \metre}]},
                label style = {font = #2},
                tick label style = {font = #3},
                y dir = reverse,
                xtick = {15, 20, 25},
                ytick = {#4},
                ]
                \addplot graphics [
                    xmin = 15,
                    xmax = 25,
                    ymin = #5,
                    ymax = #6
                ]{#7};
            \end{axis}
	\end{tikzpicture}
	}
}
            
% img with only x-label
\newcommand{\imgxlabel}[7]{% <scale size>, <label font size>, <tick font size>, <ytick>, <ymin>, <ymax>, <png file name>
\scalebox{#1}{
	\begin{tikzpicture}
            \begin{axis}[
                enlargelimits = false,
                axis on top = true,
                axis equal image,
                unit vector ratio= 0.5 1, % change aspect ratio, one of them should be 1
                point meta min = -1,   
                point meta max = 1,
                xlabel = {$x$ [\SI{}{\milli \metre}]},
                %ylabel = {$y / \dy$},
                label style = {font = #2},
                tick label style = {font = #3},
                xtick = {15, 20, 25},
                ytick = {#4},
                %yticklabel = \empty,
                y dir = reverse,
                ]
                \addplot graphics [
                    xmin = 15,
                    xmax = 25,
                    ymin = #5,
                    ymax = #6
                ]{#7};
            \end{axis}
	\end{tikzpicture}
	}
}



% img with only x-label and cmap
\newcommand{\imgxlabelwithcmap}[7]{% <scale size>, <label font size>, <tick font size>, <ytick>, <ymin>, <ymax>, <png file name>
\scalebox{#1}{
	\begin{tikzpicture}
            \begin{axis}[
                enlargelimits = false,
                axis on top = true,
                axis equal image,
                unit vector ratio= 0.5 1, % change aspect ratio, one of them should be 1
                point meta min = -1,   
                point meta max = 1,
                colorbar,
                colormap = {mymap}{rgb(0.0pt) = (0, 0.22, 0.39) ; rgb(0.43pt) = (0.91, 0.93, 0.96) ; rgb(0.5pt) = (0.97, 0.97, 0.98) ; rgb(0.57pt) = (0.99, 0.93, 0.87) ; rgb(1.0pt) = (0.94, 0.49, 0) ; } ,
                xlabel = {$x$ [\SI{}{\milli \metre}]},
                %ylabel = {$y / \dy$},
                label style = {font = #2},
                tick label style = {font = #3},
                xtick = {15, 20, 25},
                ytick = {#4},
                %yticklabel = \empty,
                y dir = reverse,
                ]
                \addplot graphics [
                    xmin = 15,
                    xmax = 25,
                    ymin = #5,
                    ymax = #6
                ]{#7};
            \end{axis}
	\end{tikzpicture}
	}
}


% set color

\usepackage{color}
 
\definecolor{fri_gray}{rgb}{0.8, 0.8, 0.8}
\definecolor{fri_green_light}{rgb}{0.76, 0.95, 0.81}
\definecolor{fri_green}{rgb}{0.51, 0.87, 0.78}
\definecolor{tui_orange}{rgb}{0.94, 0.49, 0}
\definecolor{tui_blue}{rgb}{0, 0.22, 0.39}

\definecolor{box_white}{cmyk}{0.0361,0.0251,0.0166,0}
\definecolor{text_black}{cmyk}{0.7979,0.7417,0.6916,0.6554}
\definecolor{tui_orange_dark}{cmyk}{0,0.6,1,0}
\definecolor{tui_orange_light}{cmyk}{0.0000,0.0876, 0.1474, 0.0157}
\definecolor{tui_green_dark}{cmyk}{1,0,0.5,0.2}
\definecolor{tui_green_light}{cmyk}{0.0576,0.0041, 0.0000, 0.0471}
\definecolor{tui_blue_dark}{cmyk}{1.0000,0.5000,0.0000,0.6000}
\definecolor{tui_blue_light}{cmyk}{0.0920,0.0440,0.0000,0.0196}
\definecolor{tui_red_dark}{cmyk}{0.0000,1.0000,1.0000,0.2000}
\definecolor{tui_red_light}{cmyk}{0.0000,0.1107,0.1107,0.0078}

%%%% TikZ setup %%%%

% for SmartInspect
\tikzstyle{scanpath} = [tui_blue, dotted, ->, shorten >=1mm, shorten <=1mm,]
\tikzstyle{scanpoint} = [circle, draw, black, inner sep = \rCircle, fill = black]
\tikzstyle{campoint} = [circle, draw, black, inner sep = \rCircleCamera, fill = gray]

% for synthetic aperture
\tikzstyle{griddot} = [circle, draw, black, inner sep = 0.01cm, fill = black]

% for drawing block diagram
\tikzstyle{line} = [draw, thick]
\tikzstyle{line1} = [draw, thick, ->]
\tikzstyle{line3} = [draw, gray, thick, dashed, ->]
\tikzstyle{line4} = [draw, tui_blue, very thick, dashed, -]
% For entire system 
\tikzstyle{largebox} = [rectangle, draw, dashed, very thick, 
text width = 9.6cm,  minimum height = 3.3cm
]
\tikzstyle{block1} = [rectangle, draw,  
	text width=3cm, text centered, minimum height = 2cm
	]
\tikzstyle{block2} = [rectangle, draw,  
	text width=4.5cm, text centered, minimum height = 2cm
	]
\tikzstyle{junction} = [circle, draw, fill=black, minimum size = 1pt, scale = 0.5]
% For FVnet
\tikzstyle{block3} = [rectangle, draw,  
text width=2cm, text centered, minimum height = 1cm
]
\tikzstyle{block4} = [rectangle, draw,  
text width=1.3cm, text centered, minimum height = 3cm
]
\tikzstyle{largecircle} = [circle, draw,  
text width=1.5cm, text centered, minimum size = 1pt
]
\tikzstyle{diamblock} = [diamond, draw,  
text width=1.2cm, text centered
]
\tikzstyle{largebox2} = [rectangle, draw, fri_gray_dark!70, very thick, 
text width = 5.3cm,  minimum height = 4.3cm
]
\tikzstyle{largebox3} = [rectangle, draw, fri_gray_dark!70, very thick, 
text width = 10.8cm,  minimum height = 4.3cm
]
\tikzstyle{largebox4} = [rectangle, draw, dashed, fri_gray_dark!70, very thick, 
text width = 7.5cm,  minimum height = 3.2cm
]


% for markers in RMSE
\pgfplotsset{
  every axis plot post/.append style={
    every mark/.append style={line width=3pt}
  }
}
% for posscan simulation flow
\tikzstyle{graydashed} = [draw, gray, thick, dashed]

% math operators

\DeclareMathOperator{\sinc}{sinc}
\DeclareMathOperator{\round}{round}
\DeclareMathOperator{\svd}{svd}
\DeclareMathOperator{\argmin}{argmin}
\DeclareMathOperator{\sgn}{sgn}
\DeclareMathOperator{\zeros}{zeros}

\DeclareMathOperator{\dx}{dx}
\DeclareMathOperator{\dy}{dy}
\DeclareMathOperator{\dz}{dz}
\DeclareMathOperator{\dt}{dt}
\DeclareMathOperator{\dist}{d}
\DeclareMathOperator{\pos}{pos}

\DeclareMathOperator{\Expect}{{{\mathbb E}}}

\DeclareMathOperator{\T}{T}

% Metrics
\DeclareMathOperator{\SE}{SE}
\DeclareMathOperator{\MSE}{MSE}
\DeclareMathOperator{\API}{API}
\DeclareMathOperator{\MAPI}{MAPI}
\DeclareMathOperator{\GCNR}{gCNR}
\DeclareMathOperator{\MGCNR}{MgCNR}
\DeclareMathOperator{\OVL}{OVL}

% else
\DeclareMathOperator{\optimized}{opt}
\DeclareMathOperator{\estimated}{est}
\DeclareMathOperator{\thres}{th}
\DeclareMathOperator{\Frob}{F}
\DeclareMathOperator{\Hermit}{H}
\DeclareMathOperator{\TLS}{TLS}
\DeclareMathOperator{\ROI}{ROI}
\DeclareMathOperator{\CF}{CF}





% PDF-options
\title{Data Driven Hybrid Algorithms for Preprocessing of Manually Acquired Ultrasound NDT Data} 
%\subtitle{CSP Advanced Research Project WS19/20}
\institute{\foreignlanguage{german}{Technische Universität Ilmenau}}
\author{Sayako Kodera}
\date{July 28, 2021}

% beamer goto-button setup
\setbeamercolor{button}{fg=white,bg=blue}
\renewcommand{\beamergotobutton}[1]{%
    \begingroup% keep color changes local
    \setbeamercolor{button}{fg=white, bg=tui_orange}%
    \beamerbutton{\insertgotosymbol#1}% original definition
    \endgroup
    }

%%%%%%%%%%%%%%%%%%%%%%%%%%%%%%%%%%%%%%%%%%%%%%
%================================= tips to use beamer ======%

% \uncover<page-> = the text will be covered (so displayed in light colors) for the particular pages of the slide
% \ony<page-> = the text/images will be hidden for the particular pages of the slide
% \footnotetext[number] = text for footnote wirh the givien "number"
% \footnotemark = displays the mark (i.e. nuber) of the footnote

%%%%%%%%%%%%%%%%%%%%%%%%%%%%%%%%%%%%%%%%%%%%%%
%%%%%%%%%%%%%%%%%%%%%%%%%%%%%%%%%%%%%%%%%%%%%%

\begin{document}
% display only logos but no contact info in footline
\setbeamertemplate{footline}[light]
% print title slide
\begin{frame}[noframenumbering] % noframenumbering prevents the framecounter from increasing for this single slide
 	\titlepage
\end{frame}

% small footline for more content
\setbeamertemplate{footline}[shrunkplain]
%\begin{frame}[noframenumbering] % noframenumbering prevents the %framecounter from increasing for this single slide
%	\frametitle{Table of contents}
%	\vfill
% 	\tableofcontents[part=1]
%\end{frame}

%%%%%%%%%%%%%%%%%%%%%%%%%%%%%%%%%%%%%%%%%%%%%%
\part{content}

%%%%%%%%%%%%%%%%%%%%%%%%%%%%%%%%%%%%%%%%%%%%%%
%%%%%%%%%%%%%%%%%%%%%%%%%%%%%%%%%%%% Sec. 1.1 %%%%%
\section{Background}

\subsection{Measurement Assistance System} 
% small footline with pagenumbers
\setbeamertemplate{footline}[shrunklight]
% full headline for sections in headline
\setbeamertemplate{frametitle}[full]
\begin{frame}[t]
	\frametitle{Assisted Manual Ultrasonic Testing}
	%\fontsize{10pt}
	%======================================== content =====%
	\begin{columns}[t]
		%========================%
		% Image part
		\begin{column}{0.6\textwidth}
			\begin{overprint}
				\centering
				%
				\only<1>{
				\textbf{Conventional manual UT} \\ 
				\vspace*{0.5cm}
				\includegraphics[width= 0.85\textwidth]{images/pic_ut_ascan.jpg}\\
				\vspace*{0.3cm}	
				\hspace*{1.2cm} \cigray{Source: Quality Magazine}
				}
				\only<2>{
					\textbf{3D SmartInspect} \footnotemark \\  \vspace*{0.1cm} 
					\includegraphics[width= 0.7\textwidth]{images/si_labphoto.jpg}\\
					\vspace*{-0.3cm}
				}
				\only<3->{
					\textbf{3D SmartInspect} \footnotemark \\  \vspace*{0.1cm} 
					\includegraphics[width= 0.7\textwidth]{images/si_cscan050219_4.PNG}\\	
				}
			\end{overprint}
		\end{column}
		%=======================%
		% Text part
		\begin{column}{0.4\textwidth}	
		\begin{overprint}
			Problem: reliability\\
			\only<2->{
				\vspace{\topsep}
				\hspace*{-0.3cm} $\rightarrow$ Assistance system
				%
				\begin{itemize}
					\item Position recognition
					\item Data recording
					\item Data visualization
					\item Visual feedback
					\item Post-processing
				\end{itemize}
				%
				\hspace*{0.2cm}
				\only<4->{
					$\rightarrow$ Degraded resolution
				}
			}
		\end{overprint}
		\end{column}	
	\end{columns}	

	\vspace*{0.3cm}
	% footnote
	\only<2->{
		\begin{overprint}
			\footnotetext[1]{%
				A. Omira, %
				\textit{Real-time Reconstruction of Manually Measured Compressed Ultrasound Synthetic Aperture Measurement Data}, %
				2021}
		\end{overprint}
	}
	
	%===================================================%
\end{frame}

\subsection{Motivation}
% small footline with pagenumbers
\setbeamertemplate{footline}[shrunklight]
% full headline for sections in headline
\setbeamertemplate{frametitle}[full]
\begin{frame}[t]
	\frametitle{Motivation: Image Quality Improvement }
	%======================================== content =====%
	%%% autom = Muse %%%
	\only<1>{
		\textbf{Automatic measurement}\\
		%
		\vspace*{0.5cm}
		\centering
		\begin{columns}[c]	
			% Data
			\begin{column}{0.5\textwidth}
				\centering
				Measurement data\footnotemark \addtocounter{footnote}{-1} \\  \vspace*{0.1cm} 
				\inputTikZ{0.8}{figures/Krieg_18SHMNDT/reference_reco_data.tex}
			\end{column}	
			% Reco
			\begin{column}{0.5\textwidth}
				\centering
				SAFT Reconstruction\footnotemark \addtocounter{footnote}{-1} \\ \vspace*{0.1cm}
				\inputTikZ{0.8}{figures/Krieg_18SHMNDT/reference_reco_reco.tex}
			\end{column}	
		\end{columns}	
	}
	
	%%% manual = aciation SCAN 5-3 %%%
	\only<2-3>{
		\textbf{Manual measurement}%
		\hspace*{0.2cm} \cihide<2>{ \cihighlight<3>{$\bm{\because}$ \textbf{accumulation of random factors}} }\\
		%
		\centering
		\begin{columns}[t]	
			% Data
			\begin{column}{0.5\textwidth}
				\centering
				Measurement data\footnotemark \addtocounter{footnote}{-1} \\ \vspace*{0.3cm} %\hspace*{0.3cm}
				\inputTikZ{0.75}{figures/Krieg_18SHMNDT/aviation_SCAN5_3_data_gauss_2.tex}
			\end{column}
			% Reco
			\begin{column}{0.5\textwidth}
				\centering
				SAFT Reconstruction\footnotemark \\ \vspace*{0.3cm}
				\inputTikZ{0.75}{figures/Krieg_18SHMNDT/aviation_SCAN5_3.tex}
			\end{column}	
		\end{columns}	
	}
	\vspace*{0.3cm}
	% footnote
	\only<1>{
		\footnotetext[2]{F. Krieg et al., SAFT processing for manually acquired ultrasonic measurement data with 3D SmartInspect, \textsl{SHM-NDT}, 2018}
	}
	%===================================================%	
\end{frame}


\subsection{Motivation: Undersampling} 
% small footline with pagenumbers
\setbeamertemplate{footline}[shrunklight]
% full headline for sections in headline
\setbeamertemplate{frametitle}[full]
\begin{frame}[t]
\frametitle{Impact of Missing Data} 
%\fontsize{10pt}
%======================================== content =====%
	%%=======  Left: Raw measurement =======%%%
	\begin{textblock}{50}(5, 17)
		\centering
		%
		\begin{overprint}
			\only<1>{
				\centering
				\textbf{ROI}\\
				\vspace*{0.3cm}
				% <scale size>, <label font size>, <tick font size>, <png file name>
				\topviewMUSEroi{0.55}{\Large}{\Large}\\
			}	
			%% Automatic
			\only<2>{
				\textbf{Full sampling}\\
				\vspace*{0.3cm}
				%% C-Scan
				% <scale size>, <label font size>, <tick font size>, <png file name>
				\topviewylabel{0.55}{\Large}{\Large}{figures/tex_png/simulations_logscale/A_true.png}\\
				%% B-Scan
				% <scale size>, <label font size>, <tick font size>, <png file name>
				\sideviewbothlabels{0.45}{\LARGE}{\LARGE}{figures/tex_png/simulations/A_true_sideview.png}\\
			}
			%% Manual
			\only<3>{
				\textbf{Sparse sampling}\\
				\vspace*{0.3cm}
				%% C-Scan
				% <scale size>, <label font size>, <tick font size>, <png file name>
				\topviewylabel{0.55}{\Large}{\Large}{figures/tex_png/simulations_logscale/015/A_smp.png}\\
				%% B-Scan
				% <scale size>, <label font size>, <tick font size>, <png file name>
				\sideviewbothlabels{0.45}{\LARGE}{\LARGE}{figures/tex_png/simulations/015/A_smp_sideview.png}\\
			}
		\end{overprint}
	\end{textblock}
	%
	%
	%%======= Middle: Preprocessing =======%%%
	\begin{textblock}{50}(55, 17)
		\centering
		%
		\only<2->{$\bm{\rightarrow}$\\}
		%
		\begin{overprint}
			%% Automatic
			\only<2>{
				%\imgzdefshallow{1.2}{\scriptsize}{\scriptsize}{figures/pytikz/2D/texpngs/ARP/20mm_track_1lambda.png}\\
				% <scale size>, <label font size>, <tick font size>, <png file name>
			}
			%% Manual
			\only<3>{
				%\imgzdefmiddle{1.2}{\scriptsize}{\scriptsize}{figures/pytikz/2D/texpngs/ARP/30mm_track_1lambda.png}\\
				% <scale size>, <label font size>, <tick font size>, <png file name>
			}
		\end{overprint}
	\end{textblock}
	%
	%
	%%======= Right: Reco =======%%%
	\begin{textblock}{50}(105, 17)
		\centering
		%
		\only<2->{\textbf{Reconstruction}\\}
		%
		\begin{overprint}
			%% Automatic
			\only<2>{
				%% C-Scan
				% <scale size>, <label font size>, <tick font size>, <png file name>
				\topviewnolabelwithcmap{0.55}{\Large}{\Large}{figures/tex_png/simulations_logscale/R_true.png}\\
				%% B-Scan
				% <scale size>, <label font size>, <tick font size>, <png file name>
				\sideviewxlabelwithcmap{0.45}{\LARGE}{\LARGE}{figures/tex_png/simulations/R_true_sideview.png}\\
			}
			%% Manual
			\only<3>{
				%% C-Scan
				% <scale size>, <label font size>, <tick font size>, <png file name>
				\topviewnolabelwithcmap{0.55}{\Large}{\Large}{figures/tex_png/simulations_logscale/015/R_smp.png}\\
				%% B-Scan
				% <scale size>, <label font size>, <tick font size>, <png file name>
				\sideviewxlabelwithcmap{0.45}{\LARGE}{\LARGE}{figures/tex_png/simulations/015/R_smp_sideview.png}\\
			}
		\end{overprint}
	\end{textblock}
	
%===================================================%
\end{frame}


\subsection{Solution: Interpolation} 
% small footline with pagenumbers
\setbeamertemplate{footline}[shrunklight]
% full headline for sections in headline
\setbeamertemplate{frametitle}[full]
\begin{frame}[t]
	\frametitle{Solution: Interpolation as Preprocessing} 
	%\fontsize{10pt}
	%======================================== content =====%
	%========= Text: i.e. arrows =========%
	% A_smp -> A_fk
	\begin{textblock}{50}(32, 18)
		\centering
		$\bm{\rightarrow}$
	\end{textblock}
	% A_fk -> R_fk
	\begin{textblock}{50}(78, 18)
		\centering
		\only<2>{$\bm{\rightarrow}$}
	\end{textblock}
	
	%%=======  Left: Raw measurement =======%%%
	\begin{textblock}{50}(5, 17)
		\centering
		%
		\begin{overprint}
			%% Initial scan positions
			\only<1->{
				\textbf{Sparse sampling}\\
				\vspace*{0.3cm}
				%% C-Scan
				% <scale size>, <label font size>, <tick font size>, <png file name>
				\topviewylabel{0.55}{\Large}{\Large}{figures/tex_png/simulations_logscale/015/A_smp.png}\\
				%% B-Scan
				% <scale size>, <label font size>, <tick font size>, <png file name>
				\sideviewbothlabels{0.45}{\LARGE}{\LARGE}{figures/tex_png/simulations/015/A_smp_sideview.png}\\
			}
%			%% Strategic resampling positions
%			\only<2>{
%				\textbf{Strategic resampling}\\
%				\vspace*{0.3cm}
%				%% C-Scan
%				% <scale size>, <label font size>, <tick font size>, <png file name>
%				\topviewylabel{0.55}{\Large}{\Large}{figures/tex_png/simulations_logscale/015/A_resmp.png}\\
%				%% B-Scan
%				% <scale size>, <label font size>, <tick font size>, <png file name>
%				\sideviewbothlabels{0.45}{\LARGE}{\LARGE}{figures/tex_png/simulations/015/A_resmp_sideview.png}\\
%			}
		\end{overprint}
	\end{textblock}
	%
	%
	%%======= Middle: Preprocessing =======%%%
	\begin{textblock}{50}(55, 17)
		\centering
		%
		\textbf{Preprocessing}\\
		\vspace*{0.3cm}
		%
		\begin{overprint}
			%% A_fk
			\only<1->{
				%% C-Scan
				% <scale size>, <label font size>, <tick font size>, <png file name>
				\topviewnolabels{0.55}{\Large}{\Large}{figures/tex_png/simulations_logscale/015/A_fk_smt.png}\\
				%% B-Scan
				% <scale size>, <label font size>, <tick font size>, <png file name>
				\sideviewxlabel{0.45}{\LARGE}{\LARGE}{figures/tex_png/simulations/015/A_fk_smt_sideview.png}\\
			}
%			%% A_re_fk
%			\only<2>{
%				%% C-Scan
%				% <scale size>, <label font size>, <tick font size>, <png file name>
%				\topviewnolabels{0.55}{\Large}{\Large}{figures/tex_png/simulations_logscale/A_true.png}\\
%				%% B-Scan
%				% <scale size>, <label font size>, <tick font size>, <png file name>
%				\sideviewxlabel{0.45}{\LARGE}{\LARGE}{figures/tex_png/simulations/A_true_sideview.png}\\
%			}
		\end{overprint}
	\end{textblock}
	%
	%
	%%======= Right: Reco =======%%%
	\begin{textblock}{50}(105, 17)
		\centering
		%
		\begin{overprint}
			%% R_smp
			\only<1>{
				\textbf{Reconstruction}\\
				%
				%% C-Scan
				% <scale size>, <label font size>, <tick font size>, <png file name>
				\topviewnolabelwithcmap{0.55}{\Large}{\Large}{figures/tex_png/simulations_logscale/015/R_smp.png}\\
				%% B-Scan
				% <scale size>, <label font size>, <tick font size>, <png file name>
				\sideviewxlabelwithcmap{0.45}{\LARGE}{\LARGE}{figures/tex_png/simulations/015/R_smp_sideview.png}\\
			}
			%% R_fk
			\only<2>{
				\textbf{Reconstruction}\\
				%
				%% C-Scan
				% <scale size>, <label font size>, <tick font size>, <png file name>
				\topviewnolabelwithcmap{0.55}{\Large}{\Large}{figures/tex_png/simulations_logscale/015/R_fk.png}\\
				%% B-Scan
				% <scale size>, <label font size>, <tick font size>, <png file name>
				\sideviewxlabelwithcmap{0.45}{\LARGE}{\LARGE}{figures/tex_png/simulations/015/R_fk_sideview.png}\\
			}
%			%% R_re_fk
%			\only<2>{
%				%% C-Scan
%				% <scale size>, <label font size>, <tick font size>, <png file name>
%				\topviewnolabelwithcmap{0.55}{\Large}{\Large}{figures/tex_png/simulations_logscale/015/R_re_fk.png}\\
%				%% B-Scan
%				% <scale size>, <label font size>, <tick font size>, <png file name>
%				\sideviewxlabelwithcmap{0.45}{\LARGE}{\LARGE}{figures/tex_png/simulations/015/R_re_fk_sideview.png}\\
%			}
		\end{overprint}
	\end{textblock}
	
	%===================================================%
\end{frame}


\subsection{Objectives and Contributions}
% small footline with pagenumbers
\setbeamertemplate{footline}[shrunklight]
% full headline for sections in headline
\setbeamertemplate{frametitle}[full]
\begin{frame}[t]
\frametitle{Objectives and Contributions}
%======================================== content =====%
	%%========= Left: Text =========%%
	\begin{textblock}{90}(10, 20) %-> using textblock avoid shifting of figures b/w slides
	\only<1-3>{
	\textbf{Objectives} 	
	\begin{itemize}
		\item Artefacts reduction in reconstructions\\
					$\rightarrow$ Interpolate missing data as preprocessing\\
					$\hat{=}$ Interpolation of nonlinear spatio-temporal data \\
		\item Fast yet interpretable method
	\end{itemize}
	%\vspace*{0.2cm}
	
	\only<3>{
	\textbf{Solutions}
	\begin{itemize}
		\item Spatial statistical interpolation 
		\item Batch-wise interpolation
		\item Perform in space-frequency domain
		\item Incorporate DNN for fast on-site execution
	\end{itemize}
	}
	}
	\only<4->{
	\textbf{Contributions}
	\begin{itemize}
		\item Spatial statistical modeling of UT data in space-frequency (SF) domain 
		\item Develop a hybrid interpolation scheme in SF-domain\\
					\hspace*{0.2cm} (i) Interpolation via a MMSE estimator \\
					\hspace*{0.5cm} $\rightarrow$ \textbf{SF-Kriging} \\
					\hspace*{0.2cm} (ii) Estimation of spatial statistics via DNN\\
					\hspace*{0.5cm} $\hat{=}$ Vector-valued regression\\
					\hspace*{0.5cm} $\rightarrow$ \textbf{FVnet}
		\item Feedback feature for experimental design
	\end{itemize}
	}
	\end{textblock}
	
	%%========= Right: Image =========%%
	\begin{textblock}{50}(100, 13)
		\centering
		%% R_smp
		\only<1>{
			\textbf{Sparse sampling}
			\vspace*{0.2cm}\\
			% <scale size>, <label font size>, <tick font size>, <png file name>
			\topviewbothlabels{0.55}{\Large}{\Large}{figures/tex_png/simulations_logscale/015/R_smp.png}\\
		}
		%% R_fk
		\only<2-4>{
			\textbf{Preprocessed}
			\vspace*{0.2cm}\\
			% <scale size>, <label font size>, <tick font size>, <png file name>
			\topviewbothlabels{0.55}{\Large}{\Large}{figures/tex_png/simulations_logscale/015/R_fk.png}\\
		}
		%% Variance map 
		\only<5>{
			\textbf{Feedback}
			\vspace*{0.2cm}\\
			% <scale size>, <label font size>, <tick font size>, <png file name>
			\topviewbothlabels{0.55}{\Large}{\Large}{figures/tex_png/simulations/015/rmse_fk.png}\\
		}
		%% R_re_fk
		\only<6->{
			\textbf{Resmpl. + preproc.}
			\vspace*{0.2cm}\\
			% <scale size>, <label font size>, <tick font size>, <png file name>
			\topviewbothlabels{0.55}{\Large}{\Large}{figures/tex_png/simulations_logscale/015/R_re_fk.png}\\
		}
	\end{textblock}
	%
	% A-Scan
	\begin{textblock}{50}(100, 60)
		\centering
		\only<2->{
			% <scale size>, <label font size>,  <tick font size>, <fname>
			\plotascan{0.8}{\large}{\large}{figures/coords_1D/ascan_example.tex}
		}
	\end{textblock}
	
%===================================================%	
\end{frame}

%%%%%%%%%%%%%%%%%%%%%%%%%%%%%%%%%%%%%%%%%%%%%%%%%%%%%%%%%%%%%%%%%%%%%%
%%%%%%%%%%%%%%%%%%%%%%%%%%%%%%%%%%%%%%%%%%%%%%%%%%%%%%%%%%%%%%%%%%%%%%
\section{Method}

\subsection{Method: Overview}
% small footline with pagenumbers
\setbeamertemplate{footline}[shrunklight]
% full headline for sections in headline
\setbeamertemplate{frametitle}[full]
\begin{frame}[t]
	\frametitle{Preprocessing Scheme}
	%======================================== content =====%
	\begin{textblock}{140}(10, 25)
		\centering
		\inputTikZ{0.7}{figures/fig_diag_wholesystem.tex}
	\end{textblock}
	%===================================================%	
\end{frame}


%\subsection{Method: Lists}
%% small footline with pagenumbers
%\setbeamertemplate{footline}[shrunklight]
%% full headline for sections in headline
%\setbeamertemplate{frametitle}[full]
%\begin{frame}[t]
%	\frametitle{Method Lists}
%	%======================================== content =====%
%	\only<1>{
%		\begin{itemize}
%			\item Interpolation of nonlinear ST-data = difficult\\
%						$\rightarrow$ Temporal correlation\\
%						$\Rightarrow$ Vector-valued prediction
%			\item SF-domain: orthogonal Fourier bases\\
%						$\rightarrow$ prediction for single frequency possible\\
%						$\Rightarrow$ scalar-valued prediction
%			\item Linear MMSE predictor in SF-domain\\
%						$\hat{=}$ Weights are determined based on the second-order spatial statistics
%			\item Second-order spatial statistics = FV\\
%						FV is defined based on intrinsic stationarity assuption\\
%						 $\hat{=}$ A weaker assumption of spatial WSS\\
%			\item Intrinsic stationarity $\rightarrow$ stationarity of the increments\\
%						(1) Increments are zero mean\\
%						(2) Variance of the increments is shift invariance\\
%		\end{itemize}
%	}
%	\only<2>{
%		\begin{itemize}
%			\item Nonparametric FV estimation = method-of-moments estimate\\
%						$(-)$ requires adequate samples $\because$ some lags may be missing\\
%			\item Parametric FV estimation\\
%						$\rightarrow$ for each lag and frequency\\
%						$\Rightarrow$ not suitable for online application\\
%			\item Data driven nonparametric FV estimation via DNN\\
%						Property: lattice data within a small batch\\
%						$\rightarrow$ Vector-valued lags \& FVs\\
%						$\Rightarrow$ Vector-valued regression problem
%		\end{itemize}
%	}
%	%===================================================%	
%\end{frame}

\subsection{ST-Interpolation}
% small footline with pagenumbers
\setbeamertemplate{footline}[shrunklight]
% full headline for sections in headline
\setbeamertemplate{frametitle}[full]
\begin{frame}[t]
	\frametitle{ST-Interpolation in SF-domain}
	%======================================== content =====%
	% Overview
	\only<1>{
		\begin{textblock}{140}(10, 25)
			\centering
			\inputTikZ{0.7}{figures/fig_diag_wholesystem_SFKrig.tex}
		\end{textblock}
	}
	%
	%====== Text ======%
	%% ST-domain
	\only<2-3>{
		\begin{textblock}{90}(10, 20)
			\textbf{ST-domain}: temp. correlation $\rightarrow$ vector-valued pred.\\
			\vspace*{0.2cm}
			\hspace*{0.3cm} Sought: %
			\begin{equation*}
				\bm{a}_{\pos{0}} \in \RR^{M}
			\end{equation*}\\
			
			\hspace*{0.3cm} Given: %
			\begin{equation*}
				\bm{A}_{S} = 
				\begin{bmatrix}
					\bm{a}_{\pos{1}} & \bm{a}_{\pos{2}} & \cdots & \bm{a}_{\pos{N}} \\
				\end{bmatrix}
				\in \RR^{M \times N}
			\end{equation*}\\
		\end{textblock}
	}
	
	% SF-Domain
	\only<3>{
		\begin{textblock}{90}(10, 60)
			\textbf{SF-domain}: orthogonal Fourier bases\\
			\vspace*{0.2cm}
			\hspace*{0.3cm} $\Rightarrow$ individual prediction for a single frequency\\
			\begin{equation*}
				\bm{p}_{\pos{0}} = \bm{F}_{M}  \bm{a}_{\pos{0}} \text{ } \in \CC^{M}
			\end{equation*}
		\end{textblock}
	}
	
	% Problem formulation in SF-domain
	\only<4->{
		\begin{textblock}{90}(10, 20)
			\textbf{SF-domain}: $\rightarrow$ set of scalar-valued pred. $ \forall \omega_{m}$\\
			\vspace*{0.2cm}
			\hspace*{0.3cm} Sought: %
			\begin{equation*}
				p_{\pos{0}m} \in \CC
			\end{equation*}\\
			
			\hspace*{0.3cm} Given: %
			\begin{equation*}
				\bm{\pi}_{m}^{S} = 
				\begin{bmatrix}
					p_{\pos{1}m} & p_{\pos{2}m} & \cdots & p_{\pos{N}m} \\ 
				\end{bmatrix}^{\T}
				\in \CC^{M}
			\end{equation*}
		\end{textblock}
	}
	
	%====== Image ======%
	\only<2->{
		\begin{textblock}{50}(100, 20)
			\centering
			Samp. and pred. positions\\
			\vspace*{0.1cm}
			\plotbatchitppsmp{0.6}{\Large}{\Large}
		\end{textblock}
	}
	
	%===================================================%	
\end{frame}

\subsection{SF-Kriging}
% small footline with pagenumbers
\setbeamertemplate{footline}[shrunklight]
% full headline for sections in headline
\setbeamertemplate{frametitle}[full]
\begin{frame}[t]
	\frametitle{Spatial Statistical Approach: SF-Kriging}
	%======================================== content =====%
	%======== Text =======%
	%%% IS
	\only<1-2>{
		\begin{textblock}{90}(10, 18)
			\textbf{Assumptions}: \textbf{intrinsic stationarity}\\
			\vspace*{0.2cm} 
			\hspace*{0.3cm} $\hat{=}$ Stationary assumptions for the increments \\
		\end{textblock}
		
		% 1st assumption
		\begin{textblock}{90}(10, 35)
			(1) Mean of the increments is 0\\
			\begin{equation*}
				\Expect{ p (\bm{s} ) - p ( \bm{s} + \bm{h} ) } = 0 
			\end{equation*}
		\end{textblock}
		
		% 2nd assumption
		\begin{textblock}{90}(10, 55)
			(2) Variance of the increments is shift invariant\\
			\hspace*{0.5cm} $\rightarrow$ function of the spatial lag\\
			\begin{equation*}
				\Var{ p (\bm{s} ) - p ( \bm{s} + \bm{h} ) }  \coloneqq  2 \gamma ( \bm{h} ) 
			\end{equation*}
			
			\only<2>{
				\hspace*{0.3cm} $2 \gamma ( \bm{h} ) = $ \cihighlight{ \textbf{Frequency variogram (FV)} }\\
				\hspace*{1.2cm} (nonnegative, real-valued)
			}
		\end{textblock}
	}
	
	% Kriging
	\only<3->{
		% Linear predictor
		\begin{textblock}{90}(10, 18)
			\textbf{Linear unbiased MMSE predictor (Kriging)}: \\
			\vspace*{0.2cm}
			\hspace*{2cm} $\hat{p}_{\pos{0}m} = \bm{w}_{m}^{\Hermit} \bm{\pi}_{m}^{S} $
		\end{textblock}
		%
		% Weights
		\begin{textblock}{90}(10, 32)
			\only<4->{
				\textbf{Weights}: 2nd order statistics \only<4-8>{= inc. variance}\only<9->{= FVs}\\
				\vspace*{0.2cm}
				\only<5->{\hspace*{2cm} $\bm{w}_{m} = \bm{\Gamma}_{m}^{-1} \bm{\upsilon}_{m} \in \RR^{N}$\\}
			}
		\end{textblock}
		%
		% FV matrix of samples
		\begin{textblock}{80}(8, 48)
			\begin{overprint}
			\only<6-8>{%
				Between the samples:
				%
				\vspace*{-0.2cm}
				\begin{equation*}
					\bm{\Gamma}_{m} = 
					\begin{bmatrix}
						\Var{p_{\pos{1}m} - p_{\pos{1}m}} & \cdots & \Var{p_{\pos{N}m} - p_{\pos{1}m}} \\
						\vdots & \ddots & \vdots \\
						\Var{p_{\pos{1}m} - p_{\pos{N}m}} & \cdots & \Var{p_{\pos{N}m} - p_{\pos{N}m}} \\
					\end{bmatrix}
					\in \RR^{N \times N}
				\end{equation*}
			}
			%
			\only<9->{%
				Between the samples:
				%
				\vspace*{-0.2cm}
				\begin{equation*}
					\bm{\Gamma}_{m} = 
					\begin{bmatrix}
						\gamma (\pos{1} - \pos{1}) & \cdots & \gamma (\pos{N} - \pos{1}) \\
						\vdots & \ddots & \vdots \\
						\gamma (\pos{1} - \pos{N}) & \cdots & \gamma (\pos{N} - \pos{N}) \\
					\end{bmatrix}
					\in \RR^{N \times N}
				\end{equation*}
			}
		\end{overprint}
		\end{textblock}
		%
		% FV vector b/w pred. and samples
		\begin{textblock}{80}(8, 74)
			\begin{overprint}
			\only<7-8>{%
				Between $p_{\pos{0}m}$ and the samples: \only<8>{\cihighlight{$\rightarrow$ unknown}}
				%
				\vspace*{-0.3cm}
				\begin{equation*}
					\bm{\upsilon}_{m} = 
					\begin{bmatrix}
						\Var{p_{\pos{0}m} - p_{\pos{1}m}} & \cdots & \Var{p_{\pos{0}m} - p_{\pos{N}m}} \\
					\end{bmatrix}
					^{\T} \in \RR^{N}
				\end{equation*}
			}
			\only<9->{%
				Between $p_{\pos{0}m}$ and the samples:
				%
				\vspace*{-0.3cm}
				\begin{equation*}
					\bm{\upsilon}_{m} = 
					\begin{bmatrix}
						\gamma (\pos{0} - \pos{1}) & \cdots & \gamma (\pos{0} - \pos{N})\\ 
					\end{bmatrix}
					^{\T} \in \RR^{N}
				\end{equation*}
			}
			%
%			\only<7->{
%				\hspace*{1.5cm} $\rightarrow$ unknown
%			}
			\end{overprint}
		\end{textblock}
	}
	
	%======== Image ========%
	%====== Image ======%
	\begin{textblock}{50}(100, 20)
		\centering
		$\left\lbrace p ( \pos{} ) : \pos{} \subset D \right\rbrace $\\
		\vspace*{0.1cm}
		\plotbatchitppsmp{0.6}{\Large}{\Large}
	\end{textblock}
	%
	% For highlighting the FV estimation
	\begin{textblock}{50}(105, 73)
		\only<10->{
			$\Rightarrow$ \cihighlight{ \textbf{Estimate of $\gamma_{m} ( \lagvec{} )$ } }
		}
	\end{textblock}
	
	%===================================================%	
\end{frame}


\subsection{Frequency Variogram Estimation}
% small footline with pagenumbers
\setbeamertemplate{footline}[shrunklight]
% full headline for sections in headline
\setbeamertemplate{frametitle}[full]
\begin{frame}[t]
	\frametitle{Frequency Variogram Estimation}
	%======================================== content =====%
	% Overview
	\only<1>{
		\begin{textblock}{140}(10, 25)
			\centering
			\inputTikZ{0.7}{figures/fig_diag_wholesystem_FVnet.tex}
		\end{textblock}
	}
	%
	% FV estimation problems & solutions
	\only<2->{
		% Method of moments
		\begin{textblock}{140}(10, 17)
			(\rom{1}) \textbf{Nonparametric estimation}
			%
			\only<3->{
				\begin{itemize}
					\uncover<3-4>{\item Method-of-moments estimate}
					\uncover<3-4>{\item $(-)$ requires adequate samples\\
															$\because$ some lags may be missing}
				\end{itemize}
			}
		\end{textblock}
		%
		% Parametric 
		\begin{textblock}{140}(10, 39)
			(\rom{2}) \textbf{Parametric estimation}
			\only<4->{
				\begin{itemize}
					\uncover<4>{\item Modeling based on SPDEs}
					\uncover<4>{\item Parameter estimation for each lag and frequency\\
															$\rightarrow$ not suitable for online applications}
				\end{itemize}%
			}
		\end{textblock}
		%
		% DNN
		\begin{textblock}{140}(10, 62)
			\only<5->{
				(\rom{3}) \textbf{Data driven approach via DNN (FVnet)}
				\begin{itemize}
					\item Property: lattice data within a small batch\\
								$\rightarrow$ Vector-valued lags and frequency variograms\\
					\item Method-of-moments estimates are available\\
								$\Rightarrow$ Vector-valued regression problem
				\end{itemize}
			}
		\end{textblock}
	}
	%===================================================%	
\end{frame}


%%%%%%%%%%%%%%%%%%%%%%%%%%%%%%%%%%%%%%%%%%%%%%%%%%%%%%%%%%%%%%%%%%%%%%
%%%%%%%%%%%%%%%%%%%%%%%%%%%%%%%%%%%%%%%%%%%%%%%%%%%%%%%%%%%%%%%%%%%%%%
\section{Simulations}
\subsection{Setup}
% small footline with pagenumbers
\setbeamertemplate{footline}[shrunklight]
% full headline for sections in headline
\setbeamertemplate{frametitle}[full]
\begin{frame}[t]
\frametitle{Hybrid Preprocessing: Performance} 
%======================================== content =====%
	%%%% Text %%%%
	%
	\only<1-3>{
		\begin{textblock}{130}(10, 18) 
		\begin{overprint}
			\textbf{Simulation studies}\\
			\begin{itemize}
				\item \cigray<2->{Batch interpolation}
				\item\cihighlight<2->{Reconstruction of subsampled data}
			\end{itemize}
			%
			\only<3>{
				\textbf{Purpose}
				\begin{itemize}
					\item Resolution of reconstructed images
					\item Effectiveness of experimental design
				\end{itemize}
			}
		\end{overprint}
		\end{textblock}
	}
	%
	\only<4>{
		\begin{textblock}{130}(10, 18)
		\begin{overprint}
			\textbf{Scenario and Assumptions}
			\begin{itemize}
				\item Base data = densely scanned measurements \\
							(\SI{0.5}{\milli \metre} grids $\hat{=}$ 8.7 samples $/ \lambda^2$)
				\item ROI = \SI{25}{\milli \metre} $\times$ \SI{25}{\milli \metre}
				\item Subsampling in ROI: $5 \ldots 17\%$ \\
							($\hat{=}$ 0.44 $\ldots$ 1.48 samples $/ \lambda^{2}$)
				\item Moving window interpolation (50\% overlap)
				\item Single batch = \SI{5}{\milli \metre} $\times$ \SI{5}{\milli \metre}
			\end{itemize}
			\vspace*{0.2cm}
		\end{overprint}
		\end{textblock}
	}
	%
	\only<5->{
		% Reference & IDW
		\begin{textblock}{130}(10, 18)
			\only<5->{
				\textbf{Comparison}
				\begin{itemize}
					\item Reference \\
								= Reconstruction of fully sampled data
					\item Inverse Distance Weighting (IDW)
				\end{itemize}
			}
		\end{textblock}
		%
		\begin{textblock}{90}(10, 43) 
			\only<6->{
				\textbf{Evaluation metrics}
				\begin{itemize}
					\item \cigray<7->{Normalized squared error}
					\item \cihighlight<7->{\textit{Generalized Contrast-to-Noise Ratio} ($\GCNR$)}
				\end{itemize}
			}
			\only<7->{
				\hspace*{0.8cm} $\GCNR$ = 1 - $P$(false detection)\\
				\hspace*{1.4cm} $\Rightarrow$ higher $\GCNR$ $\hat{=}$ better resolution
			}
		\end{textblock}
		%
	}
	%
	%%%%% Image %%%%%%
	\begin{textblock}{90}(80, 22) 
	\only<4->{
		\centering
		\textbf{ROI}\\
		% <scale size>, <label font size>, <tick font size>, <png file name>
		\topviewMUSEroi{0.55}{\Large}{\Large}\\
	}	
	\end{textblock}
%===================================================%	
\end{frame}

\subsection{Results 1} 
% small footline with pagenumbers
\setbeamertemplate{footline}[shrunklight]
% full headline for sections in headline
\setbeamertemplate{frametitle}[full]
\begin{frame}[t]
\frametitle{Performance Evaluation for Varying Coverage} 
%\fontsize{10pt}
%======================================== content =====%
	%%%%%%%%%%%%%%%%%%%%%%%%%%%%%%
	%=========== Results: GCNR ==============%
	%%%%%%%%%%%%%%%%%%%%%%%%%%%%%%
	\centering
	%<scale size>, <font size>, <slide page for the mark>, <mark coordinates>
	\resultsSmpStatic{1}{\large}%
%===================================================%
\end{frame}


\subsection{Strategic Resampling}
% small footline with pagenumbers
\setbeamertemplate{footline}[shrunklight]
% full headline for sections in headline
\setbeamertemplate{frametitle}[full]
\begin{frame}[t]
	\frametitle{Experimental Design: Strategic Resampling}
	%======================================== content =====%
	%%========= Left: Text =========%%
	\begin{textblock}{70}(10, 22) %-> using textblock avoid shifting of figures b/w slides
	\begin{overprint}
		\textbf{Feedback for experimental design}
		\begin{itemize}
			\uncover<2->{\item (1) Initial sampling ($= N$)}
			\uncover<3->{\item (2) Interpolation via SF-Kriging}
			\uncover<4->{\item (3) Variance map as feedback}
			\uncover<5->{\item (4) Prioritized resampling}\\
			\uncover<6->{\hspace*{0.4cm} + random resampling\\
										\hspace*{0.4cm} ($= 2N$)}
		\end{itemize}
	
		\vspace*{0.3cm}
		\only<7->{
			$\rightarrow$ Effect on reconstruction resolution
		}
	\end{overprint}
	\end{textblock}

	%%========= Right: Image =========%%
	\begin{textblock}{60}(85, 25) %-> using textblock avoid shifting of figures b/w slides
		\centering
		\only<2->{
			\inputTikZ{0.8}{figures/strategic_resampling_animate.tex}
		}
	\end{textblock}
	
	
	%===================================================%	
\end{frame}


\subsection{Results 2} 
% small footline with pagenumbers
\setbeamertemplate{footline}[shrunklight]
% full headline for sections in headline
\setbeamertemplate{frametitle}[full]
\begin{frame}[t]
	\frametitle{Effectiveness of Strategic Resampling} 
	%\fontsize{10pt}
	%======================================== content =====%
	%%%%%%%%%%%%%%%%%%%%%%%%%%%%%%
	%=========== Animation: GCNR ==============%
	%%%%%%%%%%%%%%%%%%%%%%%%%%%%%%
	% 5%
	\only<1-2>{%
		\centering
		%<scale size>, <font size>, <slide page for the mark>, <mark coordinates>
		\resultsanimateResmp{1}{\large}{2}{(5, 0.8557544375216948) (5, 0.9476274261491666) (5, 0.9452070720887681) }
	}
	% 10%
	\only<4>{%
		\centering
		%<scale size>, <font size>, <slide page for the mark>, <mark coordinates>
		\resultsanimateResmp{1}{\large}{4}{(10, 0.9508049534920663) (10, 0.9594028540162732) (10, 0.9678061257508634) }
	}
	% 15%
	\only<6>{%
		\centering
		%<scale size>, <font size>, <slide page for the mark>, <mark coordinates>
		\resultsanimateResmp{1}{\large}{6}{(15, 0.9661895999905868) (15, 0.9631637529343241) (15, 0.9725018532573202) }
		%{ (12, 0.960993309152738) (12, 0.960977755028799) (12, 0.97133687511399) }
	}
	%%======= Coverage =========%%
	\begin{textblock}{50}(10, 18)
		\only<3>{$N$ = 0.44 samples $/ \lambda^2$}%
		\only<5>{$N$ = 0.87 samples $/ \lambda^2$}%
		\only<7>{$N$ = 1.31 samples $/ \lambda^2$}% 
	\end{textblock}
	
	%%=======  Left: Reco true =======%%%
	\begin{textblock}{35}(5, 25)
		\centering
		\begin{overprint}
			%% R_true
			\only<3, 5, 7>{
				\textbf{Reference}\\
				\vspace*{0.3cm}
				% <scale size>, <label font size>, <tick font size>, <png file name>
				\topviewbothlabels{0.5}{\Large}{\Large}{figures/tex_png/simulations_logscale/R_true.png}\\
				%
				\vspace*{0.3cm}
				$\bm{1}$ % GCNR value
			}
		\end{overprint}
	\end{textblock}
	%
	%
	%%======= Center left: Reco smp or FK (for resampling) =======%%%
	\begin{textblock}{35}(45, 25)
		\centering
		\begin{overprint}
			%%%% Initial sampling positions %%%%
			%% 5% 
			\only<3>{
				\textbf{SF-Krig.}\\
				\vspace*{0.3cm}
				% <scale size>, <label font size>, <tick font size>, <png file name>
				\topviewxlabel{0.5}{\Large}{\Large}{figures/tex_png/simulations_logscale/005/R_fk.png}\\
				%
				\vspace*{0.3cm}
				$\bm{0.893}$ % GCNR value
			}
			%% 10%
			\only<5>{
				\textbf{SF-Krig.}\\
				\vspace*{0.3cm}
				% <scale size>, <label font size>, <tick font size>, <png file name>
				\topviewxlabel{0.5}{\Large}{\Large}{figures/tex_png/simulations_logscale/010/R_fk.png}\\
				%
				\vspace*{0.3cm}
				$\bm{0.959}$ % GCNR value
			}
			%% 15%
			\only<7>{
				\textbf{SF-Krig.}\\
				\vspace*{0.3cm}
				% <scale size>, <label font size>, <tick font size>, <png file name>
				\topviewxlabel{0.5}{\Large}{\Large}{figures/tex_png/simulations_logscale/015/R_fk.png}\\
				%
				\vspace*{0.3cm}
				$\bm{0.978}$ % GCNR value
			}
		\end{overprint}
	\end{textblock}
	%
	%
	%%======= Center right: IDWs (both sampling and resampling) =======%%%
	\begin{textblock}{35}(80, 25)
		\centering
		\begin{overprint}
			%% 5%
			\only<3>{
				\textbf{Resmp. IDW}\\
				\vspace*{0.3cm}
				% <scale size>, <label font size>, <tick font size>, <png file name>
				\topviewxlabel{0.5}{\Large}{\Large}{figures/tex_png/simulations_logscale/005/R_re_idw.png}\\
				%
				\vspace*{0.3cm}
				$\bm{0.953}$ % GCNR value
			}
			%% 10%
			\only<5>{
				\textbf{Resmp. IDW}\\
				\vspace*{0.3cm}
				% <scale size>, <label font size>, <tick font size>, <png file name>
				\topviewxlabel{0.5}{\Large}{\Large}{figures/tex_png/simulations_logscale/010/R_re_idw.png}\\
				%
				\vspace*{0.3cm}
				$\bm{0.964}$ % GCNR value
			}
			%% 15%
			\only<7>{
				\textbf{Resmp. IDW}\\
				\vspace*{0.3cm}
				% <scale size>, <label font size>, <tick font size>, <png file name>
				\topviewxlabel{0.5}{\Large}{\Large}{figures/tex_png/simulations_logscale/015/R_re_idw.png}\\
				%
				\vspace*{0.3cm}
				$\bm{0.969}$ % GCNR value
			}
		\end{overprint}
	\end{textblock}
	
	%%======= Right: FKs (both sampling and resampling) =======%%%
	\begin{textblock}{35}(115, 25)
		\centering
		\begin{overprint}
			%% 5%
			\only<3>{
				\textbf{Resmp. SF-Krig.}\\
				% <scale size>, <label font size>, <tick font size>, <png file name>
				\topviewxlabelwithcmap{0.5}{\Large}{\Large}{figures/tex_png/simulations_logscale/005/R_re_fk.png}\\
				%
				\vspace*{0.3cm}
				$\bm{0.966}$ % GCNR value
			}
			%% 10%
			\only<5>{
				\textbf{Resmp. SF-Krig.}\\
				% <scale size>, <label font size>, <tick font size>, <png file name>
				\topviewxlabelwithcmap{0.5}{\Large}{\Large}{figures/tex_png/simulations_logscale/010/R_re_fk.png}\\
				%
				\vspace*{0.3cm}
				$\bm{0.973}$ % GCNR value
			}
			%% 15%
			\only<7>{
				\textbf{Resmp. SF-Krig.}\\
				% <scale size>, <label font size>, <tick font size>, <png file name>
				\topviewxlabelwithcmap{0.5}{\Large}{\Large}{figures/tex_png/simulations_logscale/015/R_re_fk.png}\\
				%
				\vspace*{0.3cm}
				$\bm{0.983}$ % GCNR value
			}
		\end{overprint}
	\end{textblock}
	
	%===================================================%
\end{frame}


%%%%%%%%%%%%%%%%%%%%%%%%%%%%%%%%%%%%%%%%%%%%%%%%%%%%%%%%%%%%%%%%%%%%%%
%%%%%%%%%%%%%%%%%%%%%%%%%%%%%%%%%%%%%%%%%%%%%%%%%%%%%%%%%%%%%%%%%%%%%%
\section{Summary} 

\subsection{Conclusion (1)}
% small footline with pagenumbers
\setbeamertemplate{footline}[shrunklight]
% full headline for sections in headline
\setbeamertemplate{frametitle}[full]
\begin{frame}
	\frametitle{Conclusion (1)}
	%\vspace*{1cm}
	%\footnotesize
	%======================================== content =====%
	%%%% Motivation + objectives %%%%
	\textbf{Research problem and objectives}
	\begin{itemize}
		\item Direct SAFT reconstruction of manaual UT data\\
					$\rightarrow$ Degraded resolution \\
					$\because$ Accumulation of random factors \\
		\item Sparse and/or irregular spatial subsampling of UT data\\
					$\rightarrow$ Spatial aliasing in SAFT reconstructions\\
					$\Rightarrow$ Appearance of artefacts\\
		\item Goal = artefacts reduction by preventing spatial aliasing\\
					$\rightarrow$ Interpolate missing UT data\\
					$\hat{=}$ Interpolation of nonlinear spatio-temporal data\\
		\item Fast yet interpretable method
	\end{itemize}
	%===================================================%
\end{frame}


\subsection{Conclusion (2)}
% small footline with pagenumbers
\setbeamertemplate{footline}[shrunklight]
% full headline for sections in headline
\setbeamertemplate{frametitle}[full]
\begin{frame}[t]
	\frametitle{Conclusion (2)}
	%======================================== content =====%
	%%%% Achievements + contribution %%%%
	\textbf{Achievements and contributions}
	\begin{itemize}
		\item Extensive study on spatial statistics
		\item Spatial statistical modeling of UT data in space-frequency domain
		\item Problem formulation for a hybrid approach\\
		(i) Space-frequency domain Interpolation via SF-Kriging\\
		(ii) Estimation of second order spatial statistics via FVnet\\
		\hspace*{0.5cm} $\hat{=}$ Vector-valued regression\\ 
		\hspace*{0.5cm} $\rightarrow$ Applicable to other types of spatio-temporal lattice data\\
		\item Extension of point SF-Kriging to multi-point ones
		\item Establishing a preprocessing scheme with a feedback feature
	\end{itemize}
	%===================================================%	
\end{frame}


\subsection{Conclusion (3)}
% small footline with pagenumbers
\setbeamertemplate{footline}[shrunklight]
% full headline for sections in headline
\setbeamertemplate{frametitle}[full]
\begin{frame}[t]
	\frametitle{Conclusion (3)}
	%======================================== content =====%
		\begin{overprint}
		% Findings
		\begin{textblock}{140}(10, 20)
			\textbf{Findings}
			\begin{table}
				\centering
				\inputTable{0.9 \textwidth}{tables/table_conclusion.tex}
			\end{table}
		\end{textblock}
		
		% Future work
		\begin{textblock}{140}(10, 50)
			\only<2>{
				\textbf{Future work}
				\begin{itemize}
					\item Incorporate the neighboring batch information in FVnet
					\item Parameteric estimation of frequency variogram via DNN
					\item Extension to a progressive approach
					%\item Optimal value of the hyper parameter for SF-Kriging
				\end{itemize}
			}
		\end{textblock}
	\end{overprint}
	%===================================================%	
\end{frame}



%%%%%%%%%%%%%%%%%%%%%%%%%%%%%%%%%%%%%%%%%%%%%%%%%%%%%%%%%%%%%%%%%%%%%%
%%%%%%%%%%%%%%%%%%%%%%%%%%%%%%%%%%%%%%%%%%%%%%%%%%%%%%%%%%%%%%%%%%%%%%
%%%%%%%%%%%%%%%%%%%%%%%%%%%%%%%%%%%% Backup %%%%%
%%%%%%%%%%%%%%%%%%%%%%%%%%%%%%%%%%%%%%%%%%%%%%%%%%%%%%%%%%%%%%%%%%%%%%
%%%%%%%%%%%%%%%%%%%%%%%%%%%%%%%%%%%%%%%%%%%%%%%%%%%%%%%%%%%%%%%%%%%%%%
\appendix
\backupbegin
\section{Backup}
\subsection{Backup Start}
% small footline with pagenumbers
\setbeamertemplate{footline}[shrunklight]
% full headline for sections in headline
\setbeamertemplate{frametitle}[full]
\begin{frame}
	%\frametitle{ } 
	%\vspace*{1cm}
	%\footnotesize
	%======================================== content =====%
	\centering
	\huge Backup
	%===================================================%
\end{frame}

% Backup

\subsection{Error Sources}
% small footline with pagenumbers
\setbeamertemplate{footline}[shrunklight]
% full headline for sections in headline
\setbeamertemplate{frametitle}[full]
\begin{frame}[t]
	\frametitle{Accumulation of randomness}
	%======================================== content =====%
	\centering	
	%
	\begin{columns}[t]
		%%%% Text %%%%
		\begin{column}{0.4\textwidth}
			\begin{overprint}
				\vspace*{0.3cm}\\
				System inaccuracy
				\begin{itemize}
					\uncover<2->{\item \cigray<5>{Positional inaccuracy}}
					\uncover<3->{\item \cigray<5>{Inconsistent coupling}}
				\end{itemize}
				
				Path selection
				\begin{itemize}
					\uncover<4->{\item \cihighlight<5>{Spatial undersampling}}
				\end{itemize}
			\end{overprint}
		\end{column}	
		
		%%% image %%%
		% grided	
		\begin{column}{0.3\textwidth}
			%\vspace*{0.5cm}
			%\hspace*{2cm}
			\begin{figure}
				\begin{center}
					\inputTikZ{0.4}{figures/autoUT_illustration.tex}
				\end{center}			
			\end{figure}
			
			% caption
			\begin{center}
				\only<1>{\large Automatic}
				\only<2>{\large Accurate}
				\only<3>{\large Constant}
				\only<4->{\large Full}
			\end{center}	
		\end{column}
		
		% manual
		\begin{column}{0.3\textwidth}
			\begin{figure}
				\begin{center}
					\inputTikZ{0.4}{figures/manualUT_illustration.tex}
				\end{center}			
			\end{figure}	
			
			% caption
			\begin{center}
				\only<1>{\large Manual}
				\only<2>{\large Inaccurate}
				\only<3>{\large Inconsistent}
				\only<4->{\large Incomplete}
			\end{center}
		\end{column}
	\end{columns}
	%===================================================%
\end{frame}


\subsection{ST-Interpolation}
% small footline with pagenumbers
\setbeamertemplate{footline}[shrunklight]
% full headline for sections in headline
\setbeamertemplate{frametitle}[full]
\begin{frame}[t]
	\frametitle{ST-Interpolation in SF-domain}
	%======================================== content =====%
	% Overview
	\only<1>{
		\begin{textblock}{140}(10, 25)
			\centering
			\inputTikZ{0.7}{figures/fig_diag_wholesystem_SFKrig.tex}
		\end{textblock}
	}
	%
	%====== Text ======%
	%% ST-domain
	\only<2-3>{
		\begin{textblock}{90}(10, 20)
			\textbf{ST-domain}: temp. correlation $\rightarrow$ vector-valued pred.\\
			\vspace*{0.2cm}
			\hspace*{0.3cm} Sought: %
			\begin{equation*}
				\bm{a}_{0} \in \RR^{M}
			\end{equation*}\\
			
			\hspace*{0.3cm} Given: %
			\begin{equation*}
				\bm{A}_{S} = 
				\begin{bmatrix}
					\bm{a}_{1} & \bm{a}_{2} & \cdots & \bm{a}_{N} \\
				\end{bmatrix}
				\in \RR^{M \times N}
			\end{equation*}\\
		\end{textblock}
	}
	
	% SF-Domain
	\only<3>{
		\begin{textblock}{90}(10, 60)
			\textbf{SF-domain}: orthogonal Fourier bases\\
			\vspace*{0.2cm}
			\hspace*{0.3cm} $\Rightarrow$ individual prediction for a single frequency\\
			\begin{equation*}
				\bm{p}_{0} = \bm{F}_{M}  \bm{a}_{0} \text{ } \in \CC^{M}
			\end{equation*}
		\end{textblock}
	}
	
	% Problem formulation in SF-domain
	\only<4->{
		\begin{textblock}{90}(10, 20)
			\textbf{SF-domain}: $\rightarrow$ set of scalar-valued pred. $ \forall \omega_{m}$\\
			\vspace*{0.2cm}
			\hspace*{0.3cm} Sought: %
			\begin{equation*}
				p_{0m} \in \CC
			\end{equation*}\\
			
			\hspace*{0.3cm} Given: %
			\begin{equation*}
				\bm{\pi}_{m}^{S} = 
				\begin{bmatrix}
					p_{1m} & p_{2m} & \cdots & p_{Nm} \\ 
				\end{bmatrix}^{\T}
				\in \CC^{M}
			\end{equation*}\\
			
			\vspace*{0.5cm}
			\only<5>{
				$\Rightarrow$ \textbf{Optimal prediction}:
				\begin{equation*}
					\hat{p}_{0m} = \Expect{p_{0m} \text{ } | \text{ } \bm{\pi}_{m}^{S} }
				\end{equation*}
			}
		\end{textblock}
	}
	
	%====== Image ======%
	\only<2->{
		\begin{textblock}{50}(100, 20)
			\centering
			Samp. and pred. positions\\
			\vspace*{0.1cm}
			\plotbatchitppsmp{0.6}{\Large}{\Large}
		\end{textblock}
	}
	
	%===================================================%	
\end{frame}

\subsection{SF-Kriging}
% small footline with pagenumbers
\setbeamertemplate{footline}[shrunklight]
% full headline for sections in headline
\setbeamertemplate{frametitle}[full]
\begin{frame}[t]
	\frametitle{Spatial Statistical Approach: SF-Kriging}
	%======================================== content =====%
	%======== Text =======%
	%%% IS
	\only<1-2>{
		\begin{textblock}{90}(10, 18)
			\textbf{Assumpations}: \textbf{intrinsic stationarity}\\
			\vspace*{0.2cm} 
			\hspace*{0.3cm} $\rightarrow$ closer points $\hat{=}$ similar values \\
		\end{textblock}
		
		% 1st assumption
		\begin{textblock}{90}(10, 35)
			(1) Mean of the increments is 0\\
			\begin{equation*}
				\Expect{ p (\bm{s} ) - p ( \bm{s} + \bm{h} ) } = 0 
			\end{equation*}
		\end{textblock}
		
		% 2nd assumption
		\begin{textblock}{90}(10, 55)
			(2) Variance of the increments is shift invariant\\
			\hspace*{0.5cm} $\rightarrow$ function of the spatial lag\\
			\begin{equation*}
				\Var{ p (\bm{s} ) - p ( \bm{s} + \bm{h} ) }  \coloneqq  2 \gamma ( \bm{h} ) 
			\end{equation*}
			
			\only<2>{
				\hspace*{0.3cm} $2 \gamma ( \bm{h} ) = $ \cihighlight{ \textbf{Frequency variogram (FV)} }
			}
		\end{textblock}
	}
	
	% Kriging
	\only<3->{
		\begin{textblock}{90}(10, 18)
			\textbf{Linear predictor}: \\
			\vspace*{0.2cm}
			\hspace*{2cm} $\hat{p}_{0m} = \bm{w}_{m}^{\Hermit} \bm{\pi}_{m}^{S} $\\
			%			\begin{equation*}
			%				\hat{p}_{0m} = \bm{w}_{m}^{\Hermit} \bm{\pi}_{m} 
			%			\end{equation*}
			%
			\vspace*{0.3cm}
			\textbf{Linear unbiased MMSE predictor (Kriging)}: \\
			\vspace*{0.2cm}
			\hspace*{0.8cm} $\minimize{ \bm{w}_{m} }$    $f (\weights_{m} )$     $\text{s.t.   }  \displaystyle \sum_{i = 1}^{N} w_i = 1 $
			%			\begin{equation*}
			%				\begin{matrix}
			%					\minimize{ \bm{w}_{m} } & & f (\weights_{m}  ) \\
			%				\end{matrix}
			%			\end{equation*}
			\begin{equation*}
				\begin{matrix}
					f( \weights_{m} ) 
					& =& \Expect{ \left| p_{0m} - \hat{p}_{0m} \right|^{2} } \\
					& = & \Var{ p_{0m} - \hat{p}_{0m} }  \\
					& = & \displaystyle - \sum_{i = 1}^{N} \sum_{j = 1}^{N} w_i w_j \gamma_{m} ( \lagvec{ij} ) + 2 \sum_{i = 1}^{N} w_i \gamma_{m} ( \lagvec{0i} )
				\end{matrix}
			\end{equation*}
			%
			\hspace*{0.5cm} with $\lagvec{ij} = \pos{i} - \pos{j}$ and $\lagvec{0i} = \pos{0} - \pos{i}$	
		\end{textblock}
	}
	
	%======== Image ========%
	%====== Image ======%
	\begin{textblock}{50}(100, 20)
		\centering
		$\left\lbrace p ( \pos{} ) : \pos{} \subset D \right\rbrace $\\
		\vspace*{0.1cm}
		\plotbatchitppsmp{0.6}{\Large}{\Large}
	\end{textblock}
	%
	% For highlighting the FV estimation
	\begin{textblock}{50}(105, 70)
		\only<4>{
			$\rightarrow$ \cihighlight{ \textbf{Estimate of $\gamma_{m} ( \lagvec{} )$ } }
		}
	\end{textblock}
	
	%===================================================%	
\end{frame}


\subsection{FVnet}
% small footline with pagenumbers
\setbeamertemplate{footline}[shrunklight]
% full headline for sections in headline
\setbeamertemplate{frametitle}[full]
\begin{frame}[t]
	\frametitle{FVnet: Estimation of Spatial Statistics}
	%======================================== content =====%
	% Overview
	\only<1>{
		\begin{textblock}{140}(10, 25)
			\centering
			\inputTikZ{0.7}{figures/fig_diag_wholesystem_FVnet.tex}
		\end{textblock}
	}
	%
	% Properties
	\only<2-4>{	
		%========= Text =========%
		\begin{textblock}{90}(10, 20)
			\textbf{Properties and assumptions}: \\
			\vspace*{0.2cm}
			\hspace*{0.2cm} $\rightarrow$ within a small batch\\
			%
			\begin{itemize}
				\item Intrinsic stationary
				\item Variability in $x$ and $y$ is negligible\\
				$\rightarrow$ $\gamma_{m} ( \lagvec{} ) = \gamma_{m} ( \lag{} ) $\\
				\item Lattice data\\
				$\rightarrow$ known vector-valued lags $\in \RR^{N_{h}}$ \\
				\item Vector-valued FV $\bm{ \gamma_{m} } \in \RR^{N_{h}}$
				\item There are certain structures in FVs \\
				$\rightarrow$ similar within the neighboring bins
			\end{itemize}
		\end{textblock}
	}
	%
	% Process
	\only<5>{
		\begin{textblock}{140}(10, 20)
			\textbf{Problem formulation for DNN}\\
			\hspace*{0.3cm} $\Rightarrow$ Estimate the structure of $\bm{ \gamma_{m} } \in \RR^{N_{h}}$ \\
			\hspace*{0.3cm} $\hat{=}$ Vector-valued regression problem
		\end{textblock}
		
		\begin{textblock}{140}(10, 40)
			\textbf{Network inputs}
			\begin{itemize}
				\item Input 1 = Fourier coefficients of 3 bins \\
				$\bm{\Pi}_{m} = \left[ \bm{\pi}_{m-1} \text{  } \bm{\pi}_{m-1} \text{  } \bm{\pi}_{m-1} \right] \in \CC^{N \times 3}$\\
				\item Input 2 = Scan positions
			\end{itemize}
		\end{textblock}
	}
	%
	%========= Image =========%
	\only<3-5>{
		\begin{textblock}{50}(90, 25)
			\centering
			\begin{overprint}
				\only<3>{
					\plotFV{0.6}{\Large}{\Large}{figures/coords_1D/batch_itp/fv_signal.tex}
				}
				\only<4->{
					\plotFV{0.6}{\Large}{\Large}{figures/coords_1D/batch_itp/fvnorm_signal.tex}
				}
			\end{overprint}
		\end{textblock}
	}
	
	% Block diag
	\only<6->{
		\textbf{FVnet}
		\begin{textblock}{140}(10, 25)
			\centering
			\inputTikZ{0.7}{ figures/fig_diag_FVnet.tex}
		\end{textblock}
		%
		\begin{textblock}{140}(10, 65)
			(i) $\Phi_{\EFV}: \CC^{M \times 3} \mapsto \RR^{N_{h} \times 3}$\\
			\hspace*{0.3cm} Fourier coeffs. $\mapsto$ smoothed \& normalized method-of-moments estimate of the FVs\\
			%
			\vspace*{0.2cm}
			(ii) $\Phi_{\hist}: \RR^{N \times 2} \mapsto \RR^{N_{h}}$\\
			\hspace*{0.3cm} Sampling positions $\mapsto$ distribution of the available lags 
		\end{textblock}
	}
	%===================================================%	
\end{frame}


\subsection{Parameters: MUSE}
% small footline with pagenumbers
\setbeamertemplate{footline}[shrunklight]
% full headline for sections in headline
\setbeamertemplate{frametitle}[full]
\begin{frame}[t]
	\frametitle{Parameters: MUSE}
	%======================================== content =====%
	\begin{textblock}{140}(10, 25)
		\begin{table}
			\centering
			\inputTable{0.7 \textwidth}{tables/table_params_MUSE.tex}
		\end{table}
	\end{textblock}
	%===================================================%	
\end{frame}

\subsection{Parameters: SF-Kriging}
% small footline with pagenumbers
\setbeamertemplate{footline}[shrunklight]
% full headline for sections in headline
\setbeamertemplate{frametitle}[full]
\begin{frame}[t]
	\frametitle{Parameters: SF-Kriging}
	%======================================== content =====%
	\begin{textblock}{140}(10, 25)
		\begin{table}
			\centering
			\inputTable{0.7 \textwidth}{tables/table_params_itp.tex}
		\end{table}
	\end{textblock}
	%===================================================%	
\end{frame}


\subsection{Parameters: FWM constant}
% small footline with pagenumbers
\setbeamertemplate{footline}[shrunklight]
% full headline for sections in headline
\setbeamertemplate{frametitle}[full]
\begin{frame}[t]
	\frametitle{Parameters: FWM Constant}
	%======================================== content =====%
	\begin{textblock}{140}(10, 25)
		\begin{table}
			\centering
			\inputTable{0.7 \textwidth}{tables/table_params_FWM_const.tex}
		\end{table}
	\end{textblock}
	%===================================================%	
\end{frame}


\subsection{Parameters: FWM variable}
% small footline with pagenumbers
\setbeamertemplate{footline}[shrunklight]
% full headline for sections in headline
\setbeamertemplate{frametitle}[full]
\begin{frame}[t]
	\frametitle{Parameters: FWM Variables}
	%======================================== content =====%
	\begin{textblock}{140}(10, 25)
		\begin{table}
			\centering
			\inputTable{0.7 \textwidth}{tables/table_params_FWM_variables.tex}
		\end{table}
	\end{textblock}
	%===================================================%	
\end{frame}

\subsection{Results: Batch Wise Interpolation}
% small footline with pagenumbers
\setbeamertemplate{footline}[shrunklight]
% full headline for sections in headline
\setbeamertemplate{frametitle}[full]
\begin{frame}[t]
	\frametitle{Results: Batch Wise Interpolation}
	%======================================== content =====%
	\begin{overprint}
		\centering
		\only<1>{
			\textbf{FVnet + SF-Kriging}\\
			\vspace*{0.2cm}
			%
			\centering
			\plotapred{0.8}{\large}{\large}{figures/coords_1D/batch_itp/a_fk_x0y2.tex}
		}
		\only<2>{
			\textbf{IDW}\\
			\vspace*{0.2cm}
			%
			\centering
			\plotapred{0.8}{\large}{\large}{figures/coords_1D/batch_itp/a_idw_x0y2.tex}
		}
	\end{overprint}
	%===================================================%	
\end{frame}


\subsection{Results 1} 
% small footline with pagenumbers
\setbeamertemplate{footline}[shrunklight]
% full headline for sections in headline
\setbeamertemplate{frametitle}[full]
\begin{frame}[t]
	\frametitle{Performance Evaluation for Varying Coverage} 
	%\fontsize{10pt}
	%======================================== content =====%
	%%%%%%%%%%%%%%%%%%%%%%%%%%%%%%
	%=========== Animation: GCNR ==============%
	%%%%%%%%%%%%%%%%%%%%%%%%%%%%%%
	% 7%
	\only<1-2>{%
		\centering
		%<scale size>, <font size>, <slide page for the mark>, <mark coordinates>
		\resultsanimate{1}{\large}{2}{(7, 0.928949239870331) (7, 0.9297401836195587) (7, 0.9200115019620995)}%
	}
	% 12%
	\only<4>{%
		\centering
		%<scale size>, <font size>, <slide page for the mark>, <mark coordinates>
		\resultsanimate{1}{\large}{4}{ (12, 0.9396880203093468) (12, 0.9531296147578115) (12, 0.960993309152738) }
	}
	
	%%=======  Left: Reco true =======%%%
	\begin{textblock}{35}(5, 25)
		\centering
		\begin{overprint}
			%% R_true
			\only<3, 5>{
				\textbf{Reference}\\
				\vspace*{0.3cm}
				% <scale size>, <label font size>, <tick font size>, <png file name>
				\topviewbothlabels{0.5}{\Large}{\Large}{figures/tex_png/simulations_logscale/R_true.png}\\
			}
		\end{overprint}
	\end{textblock}
	%
	%
	%%======= Center left: Reco smp or FK (for resampling) =======%%%
	\begin{textblock}{35}(45, 25)
		\centering
		\begin{overprint}
			%%%% Initial sampling positions %%%%
			%% 7% 
			\only<3>{
				\textbf{No preproc.}\\
				\vspace*{0.3cm}
				% <scale size>, <label font size>, <tick font size>, <png file name>
				\topviewxlabel{0.5}{\Large}{\Large}{figures/tex_png/simulations_logscale/007/R_smp.png}\\
			}
			%% 12%
			\only<5>{
				\textbf{No preproc.}\\
				\vspace*{0.3cm}
				% <scale size>, <label font size>, <tick font size>, <png file name>
				\topviewxlabel{0.5}{\Large}{\Large}{figures/tex_png/simulations_logscale/012/R_smp.png}\\
			}
		\end{overprint}
	\end{textblock}
	%
	%
	%%======= Center right: IDWs (both sampling and resampling) =======%%%
	\begin{textblock}{35}(80, 25)
		\centering
		\begin{overprint}
			%% 7%
			\only<3>{
				\textbf{IDW}\\
				\vspace*{0.4cm}
				% <scale size>, <label font size>, <tick font size>, <png file name>
				\topviewxlabel{0.5}{\Large}{\Large}{figures/tex_png/simulations_logscale/007/R_idw.png}\\
			}
			%% 12%
			\only<5>{
				\textbf{IDW}\\
				\vspace*{0.4cm}
				% <scale size>, <label font size>, <tick font size>, <png file name>
				\topviewxlabel{0.5}{\Large}{\Large}{figures/tex_png/simulations_logscale/012/R_idw.png}\\
			}
		\end{overprint}
	\end{textblock}
	
	%%======= Right: FKs (both sampling and resampling) =======%%%
	\begin{textblock}{35}(115, 25)
		\centering
		\begin{overprint}
			%% 7%
			\only<3>{
				\textbf{SF-Krig.}\\
				% <scale size>, <label font size>, <tick font size>, <png file name>
				\topviewxlabelwithcmap{0.5}{\Large}{\Large}{figures/tex_png/simulations_logscale/007/R_fk.png}\\
			}
			%% 12%
			\only<5>{
				\textbf{SF-Krig.}\\
				% <scale size>, <label font size>, <tick font size>, <png file name>
				\topviewxlabelwithcmap{0.5}{\Large}{\Large}{figures/tex_png/simulations_logscale/012/R_fk.png}\\
			}
		\end{overprint}
	\end{textblock}
	
	%===================================================%
\end{frame}
 

\backupend
%%%%%%%%%%%%%%%%%%%%%%%%%%%%%%%%%%%%%%%%%%%%%%
\end{document}
