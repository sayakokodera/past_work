%%%%%%%%%%%%%%%%%%%%%%%%%%%%%%%%%%%%%%%%%%%%%%%%%%%
%%% Header: declarations, packages, definitions
%%%%%%%%%%%%%%%%%%%%%%%%%%%%%%%%%%%%%%%%%%%%%%%%%%%

\documentclass[10pt,twocolumn,a4paper,conference]{IEEEtran}

% package collection

%%% Language %%%
\usepackage[english]{babel}

%%% pgfplots and TikZ %%%
\usepackage{pgfplots}
\pgfplotsset{compat = 1.14}
\usepgfplotslibrary{colormaps}

\usepackage{tikz}
\usetikzlibrary{patterns}
\usetikzlibrary{positioning}
\usetikzlibrary{arrows}
\usetikzlibrary{snakes}
\usetikzlibrary{petri} %for tokens in the block diagram for iterative GD

%\usepackage{fullpage}

%%% Math %%%
\usepackage{amsmath,amssymb,amsfonts}   % ams packages, useful for typing math
\usepackage{bm}     % bm provides \boldmath for bold fonts in equations (matrices, vectors)

%%% EPS graphics -> unneccesary? %%%
%\usepackage{graphicx}                   % needed to insert EPS graphics via \includegraphics
%\usepackage{psfrag}                     % needed only if text in EPS graphics shall be replaced (e.g., formulas in graphics)

%%% else %%%
\usepackage{siunitx}
\usepackage[font = {small}]{caption}
\usepackage{subcaption}
\usepackage{multicol}
\usepackage{gensymb}

%%% Bibliography %%%
\usepackage{url}
\usepackage[sort]{natbib}
\bibliographystyle{unsrt} %plain -> order by alphabet, unsrt -> order by appearance 
%\bibliographystyle{IEEEsort}

\usepackage{hyperref}
%\usepackage[breaklinks,hidelinks]{hyperref}







% "newcommand" collections
% important!!!! no numbers in the name of commands!!!!!!!!

%=====================  for math =======================%
% symbols
% scan positions: p
\newcommand{\pp}{\bm{p}}
\newcommand{\pphat}{\bm{\hat{p}}}
\newcommand{\ppdelta}{\Delta \pp}
\newcommand{\ppdeltahat}{\Delta \hat{\pp}}
% scan positions: x
\newcommand{\xdelta}{\Delta x}
\newcommand{\xhat}{\hat{x}}
\newcommand{\xopt}{\hat{x}_{\optimized}}
\newcommand{\xdeltaest}{\xdelta_{\estimated}}
\newcommand{\xdeltaopt}{\xdelta_{\optimized}}
\newcommand{\xdeltahat}{\hat{\xdelta}}
\newcommand{\xvec}{\bm{x}}
\newcommand{\xvechat}{\bm{\hat{x}}}
\newcommand{\xdeltavec}{\Delta \xvec}
\newcommand{\xtrackvec}{\xvechat_{\track}}
\newcommand{\xoptvec}{\xvechat_{\optimized}}
\newcommand{\xdeltaoptvec}{\Delta \xvechat_{\optimized}}
% scatter position
\newcommand{\scatterer}{\bm{s}}
% A-Scans
\newcommand{\ascan}{a}
\newcommand{\ascanvec}{\bm{a}}
\newcommand{\ascank}{\bm{a}_{k}}
\newcommand{\ascanvechat}{\bm{\hat{a}}}
\newcommand{\adothat}{\dot{\bm{\hat{a}}}}
\newcommand{\aopt}{\bm{\tilde{a}}}
\newcommand{\Ascan}{\bm{A}}
% Pulse
\newcommand{\pulse}{h}
\newcommand{\pulsevec}{\bm{h}}
\newcommand{\pulsevecdot}{\bm{h}'}
\newcommand{\SAFTcol}[2]{\pulsevec_{#1}^{( #2)}}
% SAFT matrix
\newcommand{\SAFT}{\bm{H}}
\newcommand{\SAFTp}{\SAFT (\pp)}
\newcommand{\SAFTx}{\SAFT (x)}
\newcommand{\SAFTk}{\SAFT_{k}}
\newcommand{\SAFThat}{\hat{\bm{H}}}
\newcommand{\SAFThatk}{\hat{\bm{H}}_{k}}
\newcommand{\SAFTapproxxvec}{\left( \SAFThat + \bm{E} \Jacobianhat \right)}

% Jacobian relevent
\newcommand{\Jacobian}{\bm{J}} 
\newcommand{\Jacobianhat}{\hat{\Jacobian}} 
\newcommand{\Jacobianpartial}{\bm{J}_{\SAFTcol}}
\newcommand{\Deriv}{\bm{D}} 
\newcommand{\Derivcol}{\bm{d}_{l}} 
% else
\newcommand{\reflectivity}{r}
\newcommand{\refcoeff}{\beta}
\newcommand{\defect}{\bm{b}} % should be modified? 
\newcommand{\defectsingle}{b} % should be modified? 
\newcommand{\defecthat}{\hat{\bm{b}}} % should be modified? 
\newcommand{\zvec}{\bm{z}} 
\newcommand{\noisevec}{\bm{n}} 
\newcommand{\fhatpartial}{\hat{f}_{i}}
\newcommand{\taux}{\tau_{l} (x)}
\newcommand{\Identity}{\bm{I}} 
\newcommand{\SEdag}{\SE^{\dagger}} 
\newcommand{\MSEdag}{\MSE^{\dagger}} 

% Math functions/operators
\newcommand{\norm}[1]{%
	\left\lVert#1\right\rVert
}
\newcommand{\vecoperator}[1]{%
	\vectorize \{ #1 \}
}
\newcommand{\diagoperator}[1]{%  
     \operatorname{diag} \{ #1 \}
}
\newcommand{\Real}{%  
     \operatorname{Re}
}
\newcommand{\Imag}{%  
     \operatorname{Im}
}



% some special characters
\newcommand{\ii}{\mathrm{i}}
\newcommand{\dd}{\mathrm{d}}
\newcommand{\ee}{\mathrm{e}}
\newcommand{\dirac}{\delta}

\newcommand{\RR}{{\mathbb{R}}}
\newcommand{\NN}{{\mathbb{N}}}
\newcommand{\CC}{{\mathbb{C}}}
\newcommand{\OO}{{\mathcal{O}}}


%%%%%%%%%%%%%%%%%%%%%%%%%%%%%%%%%%%%%%%%%%%%%%
%======================================== table macro ====%
%%%%%%%%%%%%%%%%%%%%%%%%%%%%%%%%%%%%%%%%%%%%%%
% table scaling
\newcommand{\inputTable}[2]{%  
	\resizebox{#1}{!}{%
	 \input{#2}
	}
}


%%%%%%%%%%%%%%%%%%%%%%%%%%%%%%%%%%%%%%%%%%%%%%
%========================================== for TikZ  ====%
%%%%%%%%%%%%%%%%%%%%%%%%%%%%%%%%%%%%%%%%%%%%%%
\newcommand{\Nx}{3}
\newcommand{\Ny}{2}
\newcommand{\ddx}{1.5cm}
\newcommand{\ddy}{\ddx}
\newcommand{\ddz}{1.5cm}
\newcommand{\rCircle}{0.11cm}
\newcommand{\rCircleCamera}{0.07cm}

%%% draw transducer %%%
%5 3D
\newcommand{\drawTransducer}[4]{ % scale,scaley, x, y in axis
	\draw[draw = none, fill = white] (axis cs: #3 - #1, #4 - #2, 0) -- (axis cs: #3 + #1, #4 - #2, 0) -- (axis cs: #3 + #1, #4 + #2, 0) --  (axis cs: #3 + #1, #4 + #2, -#1) -- (axis cs: #3 - #1, #4 + #2, -#1) -- (axis cs: #3 - #1, #4 - #2, -#1) -- (axis cs: #3 - #1, #4 - #2, 0);
	\draw[] (axis cs: #3 - #1, #4 - #2, 0) -- (axis cs: #3 + #1, #4 - #2, 0) -- (axis cs: #3 + #1, #4 - #2, -#1) -- (axis cs: #3 - #1, #4 - #2, -#1) -- (axis cs: #3 - #1, #4 - #2, 0);
	\draw[] (axis cs: #3 + #1, #4 - #2, 0) -- (axis cs: #3 + #1, #4 + #2, 0) -- (axis cs: #3 + #1, #4 +  #2, -#1) -- (axis cs: #3 - #1, #4 +  #2, -#1) -- (axis cs: #3 - #1, #4 - #2, -#1);
	\draw[] (axis cs: #3 + #1, #4 - #2, -#1) -- (axis cs: #3 + #1, #4 +  #2, -#1);
}

%% 2D
\newcommand{\drawTransducerTwoD}[2]{ % scale, x  1,3
	\draw[draw = none, fill = white] (axis cs: #2 - #1, 0) -- (axis cs: #2 + #1, 0) --  (axis cs: #2 + #1, -#1) -- (axis cs: #2 - #1, -#1) -- (axis cs: #2 - #1, 0);
	\draw[] (axis cs: #2 - #1, 0) -- (axis cs: #2 + #1, 0) -- (axis cs: #2 + #1, -#1) -- (axis cs: #2 - #1, -#1) -- (axis cs: #2 - #1, 0);
	\draw[] (axis cs: #2 + #1, 0) -- (axis cs: #2 + #1, -#1) -- (axis cs: #2 - #1, -#1);
}

%%% draw video camera %%%
%% 3D
\newcommand{\drawCamera}[6]{ %<rotation origin (x, y, z)>, <start x>, <start y>, <start z> in axis, <camera width [dx]>, <camera height [dz]>
	% base
	\draw[rotate around = {30 : (axis cs: #1)}] (axis cs: #2, #3, #4) rectangle (axis cs: #2 + #5, #3, #4 - #6) ;
	% origin of rotation
	\node[campoint] (rotationorg) at (axis cs: #1) {};
	% "trapezoid" part (tip of the camera)
	\draw[rotate around = {30 : (axis cs: #1)}] (axis cs: #2, #3, #4 - 0.25*#6) -- (axis cs: #2 - 0.5* #6, #3, #4) -- (axis cs: #2 - 0.5* #6, #3, #4-#6) -- (axis cs: #2, #3, #4 - 0.75*#6);	
	% cable
	\draw[rotate around = {30 : (axis cs: #1)}] (axis cs: #2 + #5, #3, #4 - 0.5*  #6) .. controls (axis cs: #2 + 1.5* #5, #3, #4 - 0.5*  #6) and (axis cs: #2 + #5, #3, - 0.5*  #6) .. (axis cs: #2 + 1.5* #5, #3, - #6);	
}

%% 2D
\newcommand{\drawCameraTwoD}[5]{ %<rotation origin (x, z)>, <start x>, <start y>, <start z> in axis, <camera width [dx]>, <camera height [dz]> 1, 2, 4, 5, 6 -> 4-> 3, 5->4, 6-> 5
	% base
	\draw[rotate around = {30 : (axis cs: #1)}] (axis cs: #2, #3) rectangle (axis cs: #2 + #4, #3 - #5) ;
	% origin of rotation
	\node[campoint] (rotationorg) at (axis cs: #1) {};
	% "trapezoid" part (tip of the camera)
	\draw[rotate around = {30 : (axis cs: #1)}] (axis cs: #2, #3 - 0.25*#5) -- (axis cs: #2 - 0.5* #5, #3) -- (axis cs: #2 - 0.5* #5, #3-#5) -- (axis cs: #2, #3 - 0.75*#5);	
	% cable
	\draw[rotate around = {30 : (axis cs: #1)}] (axis cs: #2 + #4, #3 - 0.5*  #5) .. controls (axis cs: #2 + 1.5* #4, #3 - 0.5*  #5) and (axis cs: #2 + #4, - 0.5*  #5) .. (axis cs: #2 + 1.5* #4, - #5);	
}


%%%%%%%%%%%%%%%%%%%%%%%%%%%%%%%%%%%%%%%%%%%%%%
%======================================= image macro ====%
%%%%%%%%%%%%%%%%%%%%%%%%%%%%%%%%%%%%%%%%%%%%%%
% TikZ scaling
\newcommand{\inputTikZ}[2]{%<scaling factor>, <name of the tex file>
     \scalebox{#1}{\input{#2}}  
}

% BRP: for error correction animation
\newcommand{\measanimate}[5]{ % <scale size>, <slide page for base pulse>, <slide page for highlighting the pulse 1mm away>, <slide page for highlighting the pulse 2.5mm away>, <slide page for highlighting the pulse 5mm away>
	\scalebox{#1}{
		\begin{tikzpicture}
			\begin{axis}[
					width=6cm, height=6.3cm,  at={(0.7cm,0.3cm)},
					ticks=none, axis lines = center, 
					xmin=-0.5, xmax=10.5, ymin=-0.45, ymax=10,
					xlabel={$x$}, ylabel={$t$},
					y dir=reverse,
					x label style={at={(axis cs: 10.5, 0)}, anchor= west},
					y label style={at={(axis cs: -0.6, 10)}, anchor = north},
			        ]
			        
			        % Defect
			        \node at (axis cs: 5, 4) {\pgftext{\includegraphics[scale=0.15]{images/defect}}};
					% Pulse
					\only<#2>{
					\input{figures/pytikz/1D/coordinates/pulse/pulse_1_blue.tex}
					\input{figures/pytikz/1D/coordinates/pulse/pulse_2_blue.tex}
					\input{figures/pytikz/1D/coordinates/pulse/pulse_3_blue.tex}
					\input{figures/pytikz/1D/coordinates/pulse/pulse_4_blue.tex}
					\input{figures/pytikz/1D/coordinates/pulse/pulse_5_blue.tex}
					\input{figures/pytikz/1D/coordinates/pulse/pulse_6_blue.tex}
					\input{figures/pytikz/1D/coordinates/pulse/pulse_7_blue.tex}
					\input{figures/pytikz/1D/coordinates/pulse/pulse_8_blue.tex}
					\input{figures/pytikz/1D/coordinates/pulse/pulse_9_blue.tex}
					}
					
					% Highlight
					\only<#3>{\input{figures/pytikz/1D/coordinates/pulse/pulse_4_highlight.tex}} %1 mm away
					\only<#4>{\input{figures/pytikz/1D/coordinates/pulse/pulse_3_highlight.tex}} % 2.5mm away
					\only<#5>{\input{figures/pytikz/1D/coordinates/pulse/pulse_1_highlight.tex}} % 5mm away
			
			\end{axis}			
		\end{tikzpicture}
	}
}

\newcommand{\curvefit}[5]{ % <scale size>, <slide page for base pulse>, <slide page for highlighting the pulse 2.5mm away>, <slide page for adding the defect position>, <slide page for adding the scan position>, 
	\scalebox{#1}{
		\begin{tikzpicture}
			\begin{axis}[
					width=6cm, height=6.3cm,  at={(0.7cm,0.3cm)},
					ticks=none, axis lines = center, 
					xmin=-1.5, xmax=10.5, ymin=-1.5, ymax=10,
					xlabel={$x$}, ylabel={$t$},
					y dir=reverse,
					x label style={at={(axis cs: 10.5, 0)}, anchor= west},
					y label style={at={(axis cs: -0.6, 10)}, anchor = north},
			        ]
			        
			        % Defect
			        \node at (axis cs: 5, 4) {\pgftext{\includegraphics[scale=0.15]{images/defect}}};
					% Pulse
					\only<#2>{
					\input{figures/pytikz/1D/coordinates/pulse/pulse_1_blue.tex}
					\input{figures/pytikz/1D/coordinates/pulse/pulse_2_blue.tex}
					\input{figures/pytikz/1D/coordinates/pulse/pulse_3_blue.tex}
					\input{figures/pytikz/1D/coordinates/pulse/pulse_4_blue.tex}
					\input{figures/pytikz/1D/coordinates/pulse/pulse_5_blue.tex}
					\input{figures/pytikz/1D/coordinates/pulse/pulse_6_blue.tex}
					\input{figures/pytikz/1D/coordinates/pulse/pulse_7_blue.tex}
					\input{figures/pytikz/1D/coordinates/pulse/pulse_8_blue.tex}
					\input{figures/pytikz/1D/coordinates/pulse/pulse_9_blue.tex}
					}
					
					% Highlight
					\only<#3>{\input{figures/pytikz/1D/coordinates/pulse/pulse_3_highlight.tex}} %2.5mm away
					% Node: defect positions
					\only<#4>{%
						\node at (axis cs: 5, 0) {$\shortmid$};
						\node at (axis cs: 5, -1) {$x_{\dist}$};
						\node at (axis cs: 0, 4) {$-$};
						\node at (axis cs: -1, 4) {$z_{\dist}$};
					}% 
					% Node: i-th scan
					\only<#5>{%
						% Defect
						\node at (axis cs: 3, 0) {$\shortmid$};
						\node at (axis cs: 3, -1) {$x_{k}$};
						\node at (axis cs: 0, 5) {$-$};
						\node at (axis cs: -1, 5) {$z_{k}$};
					}% 
			
			\end{axis}			
		\end{tikzpicture}
	}
}

\newcommand{\TLSanimate}[3]{ % <scale size>,  <slide page for z_def = 762dz>, <slide page for z_def = 1270dz>, 
	\scalebox{#1}{
		\begin{tikzpicture}
			\begin{axis}[
					width=6cm, height=6.3cm,  at={(0.7cm,0.3cm)},
					ticks=none, axis lines = center, 
					xmin=-1.5, xmax=6, ymin=-1.5, ymax=10,
					xlabel={$x$ [\SI{}{\milli \metre}] }, ylabel={$z$ [\SI{}{\milli \metre}] },
					y dir=reverse,
					x label style={at={(axis cs: 5.7, 0)}, anchor= south},
					y label style={at={(axis cs: -0.6, 10)}, anchor = north},
			        ]
			        
					% z_def = 762dz
					\only<#2>{%
						% Defect
						\node at (axis cs: 3, 4) {\pgftext{\includegraphics[scale=0.15]{images/defect}}};
						% Defect position ticks
						\node at (axis cs: 3, 0) {$\shortmid$};
						\node at (axis cs: 3, -1) {$20$};
						\node at (axis cs: 0, 4) {$-$};
						\node at (axis cs: -1, 4) {$30$};
					}% 
					
					% z_def = 1270dz
					\only<#3>{%
						% Defect
						\node at (axis cs: 3, 6.5) {\pgftext{\includegraphics[scale=0.15]{images/defect}}};
						% Defect position ticks
						\node at (axis cs: 3, 0) {$\shortmid$};
						\node at (axis cs: 3, -1) {$20$};
						\node at (axis cs: 0, 6.5) {$-$};
						\node at (axis cs: -1, 6.5) {$50$};
					}% 

			
			\end{axis}			
		\end{tikzpicture}
	}
}


%%%%%%%%%%%%%%%%%%%%%%%%%%%%%%%%%%%%%%%%%%%%%%%%
%===================== 1D Visualization ======================%
%%%%%%%%%%%%%%%%%%%%%%%%%%%%%%%%%%%%%%%%%%%%%%%%

% ME TLS: 762dz vs 1270dz
\newcommand{\meTLS}[7]{ % <scale size>, <label font size>, <tick font size>,  <slide page to pop up the ME 762dz>, <fname for ME 762dz>, <slide page to pop up the ME 1270dz>, <fname for ME 1270dz>
\scalebox{#1}{
	\pgfplotsset{
			xmin = -0.05, xmax=1.05, 
			ymin = -0.005, ymax=0.15, 
			scaled ticks=false % to avoid formtting with 10^-2 in y tick
	}
	\begin{tikzpicture}
            \begin{axis}[
                width = 10cm, 
            	   height = 6.5cm,
            	   grid=both,
    			   grid style={line width=.1pt, draw=gray!20},
                xlabel = {Tracking error $/ \lambda$ \cigray{(\SI{1.26}{\milli \metre})} },
                ylabel = {$\norm{\pp_{\dist} - \pphat_{\dist}}_{2}$ $/ \lambda$},
                label style = {font = #2},
                tick label style = {
                		/pgf/number format/fixed, % to avoid formtting with 10^-2 in y tick
                		font = #3
                },
                %y dir = reverse,
                %xtick = {0, 10, ..., 30}, %to customize the axis
                %xticklabel = {0, 5, 10, 15}
                extra y ticks={0.125},
                %extra y tick labels = {$\SE^{\dagger}_{\threshold}$}, 
                %extra y tick style={font = #2},
                ]
                \only<#4>{
	                	\input{#5}
	                	% y = 1.25lamda line
		             \addplot[gray, dashed, mark = , line width = 2pt] coordinates{
		            			(0.0, 0.1243)
		            			(1.0, 0.1243)
		            };
                }
                \only<#6>{
                		\input{#7}
                		% y = 0.096lamda line
%                		\addplot[gray, dashed, mark = , line width = 2pt] coordinates{
%		            			(0.0, 0.096)
%		            			(1.0, 0.096)
%		            };
                	} 
            \end{axis}
	\end{tikzpicture}
	}
}

%%%%% Result: evaluation
% API  results with animation
\newcommand{\resultAPIanimate}[6]{ %<scale size>, <font size>, <slide page for mark>, <mark coordinate for Reco_true>, <mark coordinate for Reco_track>, <mark coordinate for Reco_opt> 
\scalebox{#1}{
	\begin{tikzpicture}
            \begin{axis}[
                width = 7cm, height = 4cm,
                xlabel = {ROI depth [\SI{}{\milli \metre}]}, ylabel = {$\MAPI$},
                ymin= 16.5, ymax= 28,
                label style = {font = #2},
                tick label style = {font = #2},
                xtick = {20, 30, ..., 80},
                ytick = {18, 20, ..., 28},
                grid=both, grid style={line width=.1pt, draw=gray!20},
                legend style ={
                	at={(1.5, 0.8)},
                	nodes={scale=0.95, transform shape},
                	font = #2
                }
                ]
                \input{figures/pytikz/1D/api_true_depth.tex} % true
                \input{figures/pytikz/1D/api_track_depth.tex} % track
                \input{figures/pytikz/1D/api_opt_depth.tex} % opt
                
             % x = 20mm line
             \only<#3>{
             		% true
		         \addplot[tui_red, mark = star, mark size = 2pt] coordinates{
		          		#4
		            };
		         % track
		         \addplot[tui_red, mark = star , mark size = 2pt] coordinates{
		          		#5
		            };
		       % opt
		         \addplot[tui_red, mark = star , mark size = 2pt] coordinates{
		          		#6
		            };
		    }   
                
             % legend
             % To insert the legend title
		   	%\addlegendimage{empty legend} 
		   	% Legend entries  
		   	\addlegendentry{Reference}
             \addlegendentry{No correction}
             \addlegendentry{BEC}
             % Title
             %\addlegendentry{$|s_{x} - x|$}        
            \end{axis}
	\end{tikzpicture}
	}
}

% SE results 
\newcommand{\resultSE}[7]{ % <scale size>, <font size>, <xlabel>, <ymax>, <xtick>, <fname for track>, <fname for opt>
\scalebox{#1}{
	\begin{tikzpicture}
            \begin{axis}[
                width = 7cm, height = 4cm,
                xlabel = {#3}, ylabel = {$\MSEdag$},
                ymin= -0.04, ymax= #4,
                label style = {font = #2},
                tick label style = {font = #2},
                xtick = {#5},
                ytick = {0, 0.2, ..., #4}, 
                grid=both, grid style={line width=.1pt, draw=gray!20},
                legend style ={
                	at={(1.5, 0.8)},
                	nodes={scale=0.95, transform shape},
                	font = #2
                }
                ]
                \input{#6} % track
                \input{#7} % opt
                
             % legend
             % To insert the legend title
		   	%\addlegendimage{empty legend} 
		   	% Legend entries  
             \addlegendentry{No correction}
             \addlegendentry{BEC} 
             % Title
             %\addlegendentry{$|s_{x} - x|$}        
            \end{axis}
	\end{tikzpicture}
	}
}

% API results
\newcommand{\resultAPI}[9]{ % <scale size>, <font size>, <xlabel>, <ymax>, <xtick>, <ytick>, <fname for true>, <fname for track>, <fname for opt>
\scalebox{#1}{
	\begin{tikzpicture}
            \begin{axis}[
                width = 7cm, height = 4cm,
                xlabel = {#3}, ylabel = {$\MAPI$},
                ymin= 16.5, ymax= #4,
                label style = {font = #2},
                tick label style = {font = #2},
                xtick = {#5},
                ytick = {#6},
                grid=both, grid style={line width=.1pt, draw=gray!20},
                legend style ={
                	at={(1.5, 0.8)},
                	nodes={scale=0.95, transform shape},
                	font = #2
                }
                ]
                \input{#7} % true
                \input{#8} % track
                \input{#9} % opt
                
             % legend
             % To insert the legend title
		   	%\addlegendimage{empty legend} 
		   	% Legend entries  
		   	\addlegendentry{Reference}
             \addlegendentry{No correction}
             \addlegendentry{BEC}
             % Title
             %\addlegendentry{$|s_{x} - x|$}        
            \end{axis}
	\end{tikzpicture}
	}
}


% GCNR results
\newcommand{\resultGCNR}[9]{ % <scale size>, <font size>, <xlabel>, <ymin>, <xtick>, <ytick>, <fname for true>, <fname for track>, <fname for opt>
\scalebox{#1}{
	\begin{tikzpicture}
            \begin{axis}[
                width = 7cm, height = 4cm,
                xlabel = {#3}, ylabel = {$\MGCNR$},
                ymin= #4, ymax= 0.97,
                label style = {font = #2},
                tick label style = {font = #2},
                xtick = {#5},
                ytick = {#6},  % 1.01 = otherwise the tick does not show up 
                grid=both, grid style={line width=.1pt, draw=gray!20},
                legend style ={
                	at={(1.5, 0.8)},
                	nodes={scale=0.95, transform shape},
                	font = #2
                }
                ]
                \input{#7} % true
                \input{#8} % track
                \input{#9} % opt
                
             % legend
             % To insert the legend title
		   	%\addlegendimage{empty legend} 
		   	% Legend entries  
		   	\addlegendentry{Reference}
             \addlegendentry{No correction}
             \addlegendentry{BEC}
             % Title
             %\addlegendentry{$|s_{x} - x|$}        
            \end{axis}
	\end{tikzpicture}
	}
}
%%%%%%%%%%%%%%%%%%%%%%%%%%%%%%%%%%%%%%%%%%%%%%%%
%===================== 2D Visualization ======================%
%%%%%%%%%%%%%%%%%%%%%%%%%%%%%%%%%%%%%%%%%%%%%%%%

% cimg with both x- & y-labels
\newcommand{\cimgbothlabels}[4]{% <scale size>, <label font size>, <tick font size>, <png file name>
\scalebox{#1}{
	\begin{tikzpicture}
            \begin{axis}[
                enlargelimits = false,
                axis on top = true,
                axis equal image,
                point meta min = -1,   
                point meta max = 1,
                xlabel = {$x$ in \SI{}{\milli\meter}},
                ylabel = {$y$ in \SI{}{\milli\meter}},
                label style = {font = #2},
                tick label style = {font = #3},
                %y dir = reverse,
                %xtick = {0, 10, ..., 30}, %to customize the axis
                %xticklabel = {0, 5, 10, 15}
                ]
                \addplot graphics [
                    xmin = 0,
                    xmax = 20,
                    ymin = 0,
                    ymax = 20
                ]{#4};
            \end{axis}
	\end{tikzpicture}
	}
}




%%%%%%%%%%%%%%%%%%%%%%%%%%%%%%%%%%%%%%%%%%%%%%
%=================== 2D visualization ======================%
%%%%%%%%%%%%%%%%%%%%%%%%%%%%%%%%%%%%%%%%%%%%%%
% cimg for z_def = 20mm
\newcommand{\imgzdefshallow}[4]{% <scale size>, <label font size>, <tick font size>, <png file name>
\scalebox{#1}{
	\begin{tikzpicture}
            \begin{axis}[
            	   width = 5.5cm, 
            	   %height = 6.7cm,
                enlargelimits = false,
                axis on top = true,
                axis equal image,
                %unit vector ratio= 0.3 1, % change aspect ratio, one of them should be 1
                point meta min = -1,   
                point meta max = 1,
                xlabel = {$x$ [\SI{}{\milli \metre}]},
                ylabel = {$z$ [\SI{}{\milli \metre}]},
                label style = {font = #2},
                tick label style = {font = #3},
                xlabel style = {yshift = 0.3cm},
                ylabel style = {yshift = -0.3cm},
                y dir = reverse,
                xtick = {15, 20, 25},
                ytick = {15, 20, 25},
                ]
                \addplot graphics [
                    xmin = 15,
                    xmax = 25,
                    ymin = 15,
                    ymax = 25
                ]{#4};
            \end{axis}
	\end{tikzpicture}
	}
}

% cimg for z_def = 30mm
\newcommand{\imgzdefmiddle}[4]{% <scale size>, <label font size>, <tick font size>, <png file name>,
\scalebox{#1}{
	\begin{tikzpicture}
            \begin{axis}[
            	   width = 5.5cm, 
            	   %height = 6.7cm,
                enlargelimits = false,
                axis on top = true,
                axis equal image,
                %unit vector ratio= 0.3 1, % change aspect ratio, one of them should be 1
                point meta min = -1,   
                point meta max = 1,
                xlabel = {$x$ [\SI{}{\milli \metre}]},
                ylabel = {$z$ [\SI{}{\milli \metre}]},
                label style = {font = #2},
                tick label style = {font = #3},
                xlabel style = {yshift = 0.3cm},
                ylabel style = {yshift = -0.3cm},
                y dir = reverse,
                xtick = {15, 20, 25},
                ytick = {25, 30, 35},%21, 22.5, 24
                ]
                \addplot graphics [
                    xmin = 15,
                    xmax = 25,
                    ymin = 25,
                    ymax = 35
                ]{#4};
            \end{axis}
	\end{tikzpicture}
	}
}

% cimg for z_def = 50mm
\newcommand{\imgzdefdeep}[4]{% <scale size>, <label font size>, <tick font size>, <png file name>,
\scalebox{#1}{
	\begin{tikzpicture}
            \begin{axis}[
            	   width = 5.5cm, 
            	   %height = 6.7cm,
                enlargelimits = false,
                axis on top = true,
                axis equal image,
                %unit vector ratio= 0.3 1, % change aspect ratio, one of them should be 1
                point meta min = -1,   
                point meta max = 1,
                xlabel = {$x$ [\SI{}{\milli \metre}]},
                ylabel = {$z$ [\SI{}{\milli \metre}]},
                label style = {font = #2},
                tick label style = {font = #3},
                xlabel style = {yshift = 0.3cm},
                ylabel style = {yshift = -0.3cm},
                y dir = reverse,
                xtick = {15, 20, 25},
                ytick = {45, 50, 55},%21, 22.5, 24
                ]
                \addplot graphics [
                    xmin = 15,
                    xmax = 25,
                    ymin = 45,
                    ymax = 55
                ]{#4};
            \end{axis}
	\end{tikzpicture}
	}
}

% math operators

\DeclareMathOperator{\sinc}{sinc}
\DeclareMathOperator{\round}{round}
\DeclareMathOperator{\svd}{svd}
\DeclareMathOperator{\argmin}{argmin}
\DeclareMathOperator{\sgn}{sgn}
\DeclareMathOperator{\zeros}{zeros}

\DeclareMathOperator{\dx}{dx}
\DeclareMathOperator{\dy}{dy}
\DeclareMathOperator{\dz}{dz}
\DeclareMathOperator{\dt}{dt}
\DeclareMathOperator{\dist}{d}
\DeclareMathOperator{\pos}{pos}

\DeclareMathOperator{\Expect}{{{\mathbb E}}}

\DeclareMathOperator{\T}{T}

% Metrics
\DeclareMathOperator{\SE}{SE}
\DeclareMathOperator{\MSE}{MSE}
\DeclareMathOperator{\API}{API}
\DeclareMathOperator{\MAPI}{MAPI}
\DeclareMathOperator{\GCNR}{gCNR}
\DeclareMathOperator{\MGCNR}{MgCNR}
\DeclareMathOperator{\OVL}{OVL}

% else
\DeclareMathOperator{\optimized}{opt}
\DeclareMathOperator{\estimated}{est}
\DeclareMathOperator{\thres}{th}
\DeclareMathOperator{\Frob}{F}
\DeclareMathOperator{\Hermit}{H}
\DeclareMathOperator{\TLS}{TLS}
\DeclareMathOperator{\ROI}{ROI}
\DeclareMathOperator{\CF}{CF}



%%%% TikZ setup %%%%

% for SmartInspect
\tikzstyle{scanpath} = [tui_blue, dotted, ->, shorten >=1mm, shorten <=1mm,]
\tikzstyle{scanpoint} = [circle, draw, tui_blue, inner sep = \rCircle, fill = tui_blue]
\tikzstyle{campoint} = [circle, draw, black, inner sep = \rCircleCamera, fill = gray]

% for synthetic aperture
\tikzstyle{griddot} = [circle, draw, black, inner sep = 0.01cm, fill = black]

% for drawing block diagram
\tikzstyle{line} = [draw, thick]
\tikzstyle{line1} = [draw, thick, ->]
\tikzstyle{line3} = [draw, gray, thick, dashed, ->]
\tikzstyle{block} = [rectangle, draw, 
    text width=2cm, text centered, minimum height=0.8cm]
\tikzstyle{block2} = [rectangle, draw, 
    text width=1.8cm, text centered, minimum height=2.8cm]
\tikzstyle{largebox} = [rectangle, draw, dashed, very thick, 
    text width=6.8cm,  minimum height=2cm]
\tikzstyle{block4} = [rectangle, draw, 
    text width=1.5cm, text centered, minimum height=1cm]
\tikzstyle{adder} = [circle, draw, minimum size=2em]
\tikzstyle{junction} = [circle, draw, fill=black, minimum size=1pt]
\tikzstyle{middlepoint} = [circle, draw, fill=gray!30, minimum size=1em]
\tikzstyle{end} = [circle, draw, fill=black!90, minimum size=1em \and circle, draw, minimum size=1.5em]


% for markers in RMSE
\pgfplotsset{
  every axis plot post/.append style={
    every mark/.append style={line width=3pt}
  }
}
% for posscan simulation flow
\tikzstyle{graydashed} = [draw, gray, thick, dashed]

% set color

\usepackage{color}
 
\definecolor{fri_gray}{rgb}{0.8, 0.8, 0.8}
\definecolor{fri_green_light}{rgb}{0.76, 0.95, 0.81}
\definecolor{fri_green}{rgb}{0.51, 0.87, 0.78}
\definecolor{tui_orange}{rgb}{0.94, 0.49, 0}
\definecolor{tui_blue}{rgb}{0, 0.22, 0.39}

\definecolor{box_white}{cmyk}{0.0361,0.0251,0.0166,0}
\definecolor{text_black}{cmyk}{0.7979,0.7417,0.6916,0.6554}
\definecolor{tui_orange_dark}{cmyk}{0,0.6,1,0}
\definecolor{tui_orange_light}{cmyk}{0.0000,0.0876, 0.1474, 0.0157}
\definecolor{tui_green_dark}{cmyk}{1,0,0.5,0.2}
\definecolor{tui_green_light}{cmyk}{0.0576,0.0041, 0.0000, 0.0471}
\definecolor{tui_blue_dark}{cmyk}{1.0000,0.5000,0.0000,0.6000}
\definecolor{tui_blue_light}{cmyk}{0.0920,0.0440,0.0000,0.0196}
\definecolor{tui_red_dark}{cmyk}{0.0000,1.0000,1.0000,0.2000}
\definecolor{tui_red_light}{cmyk}{0.0000,0.1107,0.1107,0.0078}



%%%%%%%%%%%%%%%%%%%%%%%%%%%%%%%%%%%%%%%%%%%%%%%%%%%
%%% Main body of the document
%%%%%%%%%%%%%%%%%%%%%%%%%%%%%%%%%%%%%%%%%%%%%%%%%%%

\begin{document}

\title{\bf "Blind" Iterative SAFT Reconstruction for Manually Acquired Ultrasonic Measurement Data in Nondestructive Testing}

\author{
\textit{Sayako Kodera}\\
Ilmenau University of Technology\\
P. O. Box 100565, D-98684 Ilmenau, Germany \\
Email: sayako.kodera@tu-ilmenau.de
} 

\maketitle

%%%%%%%%%%%%%%%%%%%%%%%%%%%%%%%%%%%%%%%%%%%%%%%%%%%
%%% The abstract
%%%%%%%%%%%%%%%%%%%%%%%%%%%%%%%%%%%%%%%%%%%%%%%%%%%
{\bf {\bf \slshape Abstract \symbol{124}} %
%%% Abstract
There have been developments in nondestructive ultrasonic testing (UT) to assist manual operation for easier and more reliable inspection than conventional ones. Although such system also opens up the possibility to post-process the measurement data for improving imaging quality, manual UT is prone to stochastic observational errors, such as inaccuracy in estimating scan positions or varying coupling, which may cause strong artefacts formation in its reconstruction. %
In an attempt to reduce the positional-inaccuracy induced artefacts, in this work we propose a preprocessing method to correct the unknown positional error from the measurement data and the erroneous positional information. %As the first step of the preprocessing, the positions of the signal sources are estimated via robust polynomial regression. In the next step, the tracking error is estimated and corrected iteratively by modeling the measurement data based on the estimation results from the first step. 
We demonstrate through simulations that the proposed method is more resistant to positional error and can achieve higher resolution, which is comparable to that of the reconstruction with the exact positional information, than the reconstruction without preprocessing.  
}

%%%%%%%%%%%%%%%%%%%%%%%%%%%%%%%%%%%%%%%%%%%%%%%%%%%
%%% Keywords: 2-5, used for indexing the paper
%%%%%%%%%%%%%%%%%%%%%%%%%%%%%%%%%%%%%%%%%%%%%%%%%%%
% Keywords may be selected from the IEEE keyword list found at 
% http://www.ieee.org/organizations/pubs/ani_prod/keywrd98.txt

\smallskip

\begin{IEEEkeywords}
   Nondestructive testing, Ultrasonic testing, SAFT, Manual measurement, Positional inaccuracy, Total least squares, Newton's method
\end{IEEEkeywords}

%\section{Major Changes (v.190814)} 
%\begin{itemize}
%\item Sec. \ref{sec:intro}: modify motivation and background, eliminate \textit{Contributions} ($\rightarrow$ will be included in abstract)
%\item Sec. \ref{sec:pulse_echo}: reduce equations, adjust dimension for $\pp$ and $\scatterer$, modify defect map \ref{eq:defect_map}
%\item Sec. \ref{sec:methods}: modify intro, adjust dimension in \ref{sec:saft_approx}, eliminate long algorithm explanation in \ref{sec:iterative_GD} (to spare the space)
%\item Add sections for simulation \ref{sec:simulation} and results \ref{sec:results}
%\end{itemize}



%%%%%%%%%%%%%%%%%%%%%%%%%%%%%%%%%%%%%%%%%%%%%%%%%%%
%%% The introduction
%%%%%%%%%%%%%%%%%%%%%%%%%%%%%%%%%%%%%%%%%%%%%%%%%%%

\section{Introduction} \label{sec:intro}

%Your introduction motivates the problem again, emphasizing why it is important
%and what are the practical applications. Use references wherever possible to support
%your points. The introduction also introduces the state of the art, citing journal
%papers \cite{Shannon:48}, conference papers \cite{Haykin:07}, books \cite{Mittelbach:2004},  
%standards, RFCs, and other references. The use of BibTeX to organize your references is
%strongly encouraged. Try to avoid using websites as sources since these may change
%over time.
%
%After providing the state of the art, you should emphasize what are the novelties of your proposed 
%solution and how it differs from the existing ones. Why would anyone need your solution?

%%% Intro %%%
%% General idea
% (a)
% Prob(a) : there is a gam b/w automatic & manual UT
% Goal(a) : reduce the gap
% Sol(a) : assistance system -> enables post-processing
% (b)
% Prob(b): conventional reco does not work well w/ manual UT data due to the systematic errors (e.g. tracking error)
% Goal(b): reduce the effect of the tracking error
% Sol(b): correct positions and adjust the reco systems accordingly

%% (1) Background / context
% UT -> Prob(a)
Ultrasonic testing (UT) is a nondestructive testing method to inspect structure of test objects without inducing damage. Conventionally, an UT inspection requires either manual operation by a human technician or automated measurement systems. In manual UT, where a human technician observes the change in the echoed pulse, its inspection quality is highly dependent on the expertise of the technician \cite{Cawley01IMechE}. In automatic UT, on the other hand, measurement data and the corresponding scan positions are recorded, which enables to visualize the inner structure of the test object and further process the data to improve the imaging quality, leading to more reliable inspection quality than its manual counterpart. \par

% Goal(a) -> Sol(a)
Nevertheless, there are still needs for manual UT, when, for instance, a complex structure is inspected, and its inspection reliability has been of great concern. In order to improve the inspection reliability of manual UT, an assistance system can be employed, which records measurement data and recognizes the scan positions through a tracking system. This allows us not only to visualize the measurement data but also to process it further for the better imaging quality \cite{Krieg18SHMNDT}. \par

% Post-processing/ SAFT
Although their application to manual UT data has been typically excluded, several post-processing techniques have been developed and extensively employed for automatic UT data \cite{Hall88} \cite{Krautkraemer90} \cite{Ericsson98ECNDDT}. While the authors of \cite{Ericsson98ECNDDT} apply the signal processing methods widely used in the telecommunication field to the UT data, one of the well established post-processing method is the synthetic aperture focusing technique (SAFT) \cite{Hall88} \cite{Krautkraemer90}. The aim of SAFT is to improve the spatial resolution through performing superposition with respect to the propagation time delay \cite{Lingvall04PhD}. In other words, SAFT regards the measurement region-of-interest (ROI) as a single aperture and each measurement as its spatial sampling, which indicates that the SAFT reconstruction requires accurate positional information. \par

%% (2) Problem statement 
% Prob(b) -> Goal(b)
However, the application of such techniques to manual UT data is so far little studied \cite{Mayer16SAFTwithSmallData} \cite{Krieg18SHMNDT}. The authors of \cite{Krieg18SHMNDT} demonstrate the possibilities of utilizing post-processing with manual measurement data, yet the reconstruction quality is, compared to the reconstruction results of automatic measurement data, significantly degraded. Previously, we identified the possible error sources for such degradation and revealed that several systematic errors, such as varying contact pressure or inaccurate positional information due to the tracking error, can lead to strong artefacts formation \cite{Krieg19IUS}. Since such errors are inevitable in manual measurement, finding the way to reduce those artefacts could improve the reconstruction quality. Unlike other possible error factors which should be entirely estimated from the measurement data, the tracking error can be handled to some extent, as the positional information is, whether accurate or not, available. \par

%% (3) Respons
% Sol(B) + contributions
Our goal in this study is to reduce the position-inaccuracy induced artefacts by correcting the measurement positions and adjusting the reconstruction system accordingly. So far, we are unaware of any other works that deal with this topic. However, expressing the reconstruction process mathematically enables us to approximate the correct model with regard to measurement positions, whereas the traditional regression approaches provide us a tool to estimate the tracking error from the available information. As a possible solution for proper handling of positional inaccuracy, we propose an iterative method combining these mathematical tools, which can be incorporated into reconstruction system. \par
%
%\cite{Hall88} 
%In order to approximate the correct model, we derive the spatial approximation of the SAFT reconstruction matrix and implement an iterative method to estimate the positional error and correct the position for assuring the approximation quality. 
% expressing the reconstruction process mathematically enables us to approximate the correct model with regard to measurement positions
% Our goal in this study is to find a method to properly handle positional inaccuracy for SAFT reconstruction. 

%%%%%%%%%%%%%%%%%%%%%%%%%%%%%%%%%%%%%%%%%%%%%%%%%%%
%%% The main part of the paper
%%%%%%%%%%%%%%%%%%%%%%%%%%%%%%%%%%%%%%%%%%%%%%%%%%%
% Chapters are just an example, feel free to modify.

%%%%%%%%%%%%%%%%%%%%%%%%%%%%%%%%%%%%%%%%%%% sec.2 %%%%
\section{Data Model} \label{sec:data_model}
% Dimension
% M = Nt = Nz
% Nx
% I = Ndefect 
% L = # of all possible scatters
% K = dimension of p (either 1 for 2D measurement, 2 for 3D measurement) 
% Description on data model
% General assumptions
For a measurement setup, we consider a manual contact testing where a handheld transducer is placed directly on the specimen surface at a position $\pp_{k} \in \RR^{3}$.
The transducer inserts an ultrasonic pulse $\pulsesig (t)$ into a specimen and receives the reflected signal, A-Scan, at the same position $\pp_{k}$. The specimen is assumed to be homogenous and isotropic with the constant speed of sound $c_0$ and to have a flat surface. During the measurement, the contact pressure is considered to be constant so that in the measurement data there is no temporal shift or amplitude change caused by improper coupling. The measurement position $\pp$ is arbitrarily selected on the specimen surface and we suppose that there is at least one scatterer inside the specimen, which is regarded as point source. \par

% Pulse model
Conventionally, the inserted pulse $\pulsesig (t)$ is modeled as a real-valued Gabor function \citep{GaborAsymmChirp}, as
\begin{equation} \label{eq:pulse}
	\pulsesig (t) = e^{- \alpha t^2} \cdot \cos (2 \pi f_C t + \phi),
\end{equation}
where $f_C$, $\alpha$  and $\phi$ are the carrier frequency, the window width factor and the phase, respectively.
% ToF
The time which the sound travels from $\pp_{k}$ to a scatterer $\scatterer_{i}$ and reflects back to $\pp_{k}$ is so called \textit{time-of-flight}, ToF, which can be obtained with
\begin{equation} \label{eq:tof}
	\tau_{i}(\pp_{k}) = \frac{2}{c_0} \cdot \norm{\scatterer_{i} - \pp_{k} }_{2}.
\end{equation}
This includes $\norm{\scatterer_{i} - \pp_{k}}_{2}$, which is the $\ell$-2 norm of the difference between $\scatterer_{i}$ and $\pp_{k}$, indicating that the ToF is determined by the position of both measurement and the scatterer. \par

% A-Scan
The obtained A-Scan is the sum of all $\I$ reflected echoes, which are delayed version of the inserted pulse $\pulsesig (t)$
\begin{equation} \label{eq:ascan_continuous}
	\ascansig (t; \pp_{k}) = \sum_{i = 1}^{I} \refcoeff_{\pp_{k}, i} \cdot \pulse (t - \tau_{i} (\pp_{k}) ) + n(t),
\end{equation}
where $\refcoeff_{\pp, i}$ is the reflection coefficient for the position $\pp_{k}$ and a scatterer $s_i$ and $n (t)$ is the measurement noise. Since we process the data digitally with the sampling interval of $\dt = \frac{1}{f_S}$, \eqref{eq:ascan_continuous} can be formulated as a vector $ \ascan{\pp_{k}} \in \RR^{\M}$ with $\M$ representing the number of temporal samples 
\begin{equation} \label{eq:ascan_discrete}
	\left[ \ascan{\pp_{k}} \right]_m = \sum_{i = 1}^{\I} \refcoeff_{\pp_{k}, i} \cdot \pulsesig (m \dt - \tau_{i} (\pp_{k}) ) + \left[ \noisevec \right]_m.
\end{equation}
Here $[ \cdot ]_m$ denotes the $m$-th element of a vector and $\noisevec \in \RR^{\M}$ is the measurement noise in the vector form. \par

% p_{k} -> x_{k}
As the specimen is assumed to be isotropic, the ToF changes symmetric with respect to the scatterer $\scatterer_{i}$ and so does the measurement data. For the sake of simplicity, in this study we consider the case where the measurements are taken along a line on the flat surface, resulting in $\pp_{k} = \begin{bmatrix} x_k & 0 & 0 \end{bmatrix}^{\T}$. This indicates that \eqref{eq:ascan_discrete} now solely depends on $x_k$, denoted as $\ascan{k}$. \par

% FWM for a single A-Scan
By collecting the impulse response at $x_k$ for all possible scatterer positions $\scatterer_{i}$ $\forall i = 1 \ldots \LL$, we can form a measurement dictionary $\SAFT_{k} \in \RR^{\M \times \LL} $ as
\begin{equation} \label{eq:SAFTk_def}
	\SAFT_k = 
	\begin{bmatrix} \SAFTcol{k}{1} & \SAFTcol{k}{2} & \cdots & \SAFTcol{k}{\LL} \end{bmatrix}.
\end{equation}
$\SAFTcol{k}{l} \in \RR^{\M}$ is the $l$-th column vector of $\SAFT_{k}$, corresponding to the $l$-th scatterer position in the specimen which is
\begin{equation} \label{eq:SAFTcol}
	\SAFTcol{k}{l} = \sum_{m = 1}^{\M} \refcoeff_{x_{k}, l} \cdot \pulsesig (m \dt - \tau_{l} (x_{k}) ).
\end{equation} 
This enables us to reformulate \eqref{eq:ascan_discrete} as a vector-matrix product as
\begin{equation} \label{eq:FWM_single}
	\ascan{k} = \SAFT_k \defect + \noisevec,
\end{equation}
where $\defect \in \RR^{\LL}$ is the vectorized \textit{defect map} which represents the scatterer positions and thier amplitudes $\beta_{l}$ \cite{Kirchhof16IUS}. \par

% K A-Scans
After taking $\K$ measurements at the positions $\xx \in \RR^{\K}$, we can obtain the set of measurements $\Ascan \in \RR^{\M \times \K}$ as 
\begin{equation} \label{eq:bscan}
	\Ascan = \left[ \ascan{1} \ascan{2}  \cdots \ascan{\LL} \right].
\end{equation}
Column-wise concatenation of measurement dictionaries for all scan positions yields the complete dictionary $\SAFT \in \RR^{\M \K \times \LL}$
\begin{equation} \label{eq:SAFTdef}
	\SAFT = \begin{bmatrix} \SAFT_1 \\ \SAFT_2 \\ \vdots \\ \SAFT_{\K} \end{bmatrix}.
\end{equation}
This allows us to express $\Ascan$ as linear transform similar to \eqref{eq:FWM_single} 
\begin{equation} \label{eq:FWM}
	\vectorize{ \Ascan} = \SAFT \defect + \vectorize { \bm{N} },
\end{equation}
in which $\vectorize{ \cdot }$ is the vectorize operation of a matrix and $\vectorize { \bm{N} } \in \RR^{\M \K}$ is the concatenation of all noise vectors.\par

% SAFT
The ultimate goal of the inspection is to locate the scatterer positions, which can be recovered from \eqref{eq:FWM} with SAFT by computing
\begin{equation} \label{eq:reco}
	\defecthat = \SAFT^{\T} \vectorize{ \Ascan}.
\end{equation}

%%%%%%%%%%%%%%%%%%%%%%%%%%%%%%%%%%%%%%%%%%% sec.3 %%%%
\section{Employed Optimization Techniques} % change the section name
In the proposed method of this study, two different optimization techniques are used, namely the total least-squares (TLS) and the Newton's method. The former is employed to estimate the scatterer positions, whereas the latter is applied to correct the tracking error  based on the estimated scatterer positions. In this section, the basic ideas and approaches of those techniques are presented. 

\subsection{Total Least-Squares}
% Dimension
% Mone = m x d
% Mtwo = m x n
% Mthree = n x d
% section: TLS
% Intro
The TLS method seeks for an optimal solution of an overdetermined system of equations 
\begin{equation} \label{eq:TLS_approx}
	\Mone \approx \Mtwo \Mthree
\end{equation}
where $\Mone \in \CC^{m \times d}$ and $\Mtwo \in \CC^{m \times n}$ are the given data and $\Mthree \in \CC^{n \times d}$ is unknown with $n, d < m$ \cite{Markovsky07TLS}. \par

% Pertubed eq -> Cost fct
With $m > n$ and $\Mtwo$ being full column rank, there is typically no exact solution, requiring $\Mthree$ to be approximated. Unlike the least-squares approach, where all modeling errors are assumed to be originated from the dependent variables $\Mone$, TLS takes into account the errors in both $\Mone$ and $\Mtwo$.  By incorporating the errors on both sides, the approximation \eqref{eq:TLS_approx} becomes an equality
\begin{equation} \label{eq:TLS_perturb}
	\MonePerturb = \left( \MtwoPerturb \right) \Mthree,
\end{equation} 
with $\MoneDelta \in \CC^{m \times d}$ and $\MtwoDelta \in \CC^{m \times n}$ representing the introduced perturbations on both sides.
% Cost fct 
Under the assumption that $\MoneDelta$ and $\MtwoDelta$ are independent, TLS seeks for the solution which minimizes both perturbations while satisfying the perturbed equation \eqref{eq:TLS_perturb} \cite{Markovsky07TLS} 
\begin{eqnarray} \label{eq:TLS_costfct}
	\hspace*{-0.4cm} \min \norm{\begin{bmatrix}\MtwoDelta & \MoneDelta \end{bmatrix}}_{\Frob}^{2} &
	\mbox{s.t.} & 
	\MonePerturb = \left( \MtwoPerturb \right) \Mthree .
\end{eqnarray} \par

% SVD of measurement matrices
In order to solve \eqref{eq:TLS_costfct}, singular value decomposition (SVD) can be utilized \cite{Markovsky07TLS, VanHuffel07TLS}. Without perturbations $\Mtwo$ and $\Mone$ are assumed to be linearly independent, making their concatenation matrix full column-rank of $n + d$. The SVD of this concatenation matrix can be obtained as 
\begin{align} \label{eq:SVD_meas}
	\begin{bmatrix} \Mtwo & \Mone \end{bmatrix}
	& = \UU \SSigma \VV^{\Hermit} \nonumber \\
	& = \begin{bmatrix} \UU_{1} & \UU_{2} \end{bmatrix} 
	\begin{bmatrix} \SSigma_{1} & \bm{0}_{n \times d} \\ \bm{0}_{(m-n) \times n} & \SSigma_{2} \end{bmatrix} 
	\begin{bmatrix} \VV_{11} & \VV_{12} \\ \VV_{21} & \VV_{22} \end{bmatrix} ^{\Hermit},
\end{align}
where $\UU \in \CC^{m \times m}$ and $\VV \in \CC^{(n + d) \times (n + d)}$ are both unitary matrices and orthogonal to each other, whereas $\SSigma \in \RR^{m \times (n + d)}_{+}$ is a diagonal matrix containing $n + d$ singular values. $\UU$, $\SSigma$ and $\VV$ are further divided into sub matrices $\UU_{1} \in \CC^{m \times n}$ and $\UU_{2} \in \CC^{m \times (m-n)}$, $\SSigma_{1} \in \RR^{n \times n}_{+}$ and  $\SSigma_{2} \in \RR^{(m-n) \times d}_{+}$ and $\VV_{11} \in \CC^{n \times n}$, $\VV_{12} \in \CC^{n \times d}$, $\VV_{21} \in \CC^{d \times n}$ and $\VV_{22} \in \CC^{d \times d}$, respectively. \par

% SVD of pertubed matrices
The perturbed equation \eqref{eq:TLS_perturb} indicates that the perturbed matrices span the same subspace, i.e. $ \Col{\MonePerturb} = \Col{\MtwoPerturb}$ with $\Col{\cdot}$ representing the column space of a matrix. This results in the concatenation matrix of perturbed matrices with the rank of n, whose SVD can be expressed as 
\begin{equation} \label{eq:SVD_perturbed}
	\begin{bmatrix} \MtwoPerturb & \MonePerturb \end{bmatrix}
	= \UU_{1} \SSigma_{1} 
	\begin{bmatrix} \VV_{11}^{\Hermit} & \VV_{21}^{\Hermit} \end{bmatrix}.
\end{equation} \par

% Reformulate the perturbed eq.
With this concatenation matrix, the perturbed equation \eqref{eq:TLS_perturb}, which is the constraint of the cost function \eqref{eq:TLS_costfct}, can be formulated as
\begin{equation} \label{eq:TLS_constraint_matrix}
	\begin{bmatrix} \MtwoPerturb & \MonePerturb \end{bmatrix}
	\begin{bmatrix} \Mthree \\ - \Identity{d} \end{bmatrix}
	= \bm{0}_{m \times d}.
\end{equation}
This suggests that $\begin{bmatrix} \Mthree^{\Hermit} & - \Identity{d} \end{bmatrix}$ lies in the nullspace of the concatenation of the perturbed matrix $\Null{ \begin{bmatrix} \MtwoPerturb & \MonePerturb \end{bmatrix} } $, which is equal to $\Null{ \begin{bmatrix} \VV_{11}^{\Hermit} & \VV_{21}^{\Hermit} \end{bmatrix} } $. Hence, $\begin{bmatrix} \Mthree^{\Hermit} & - \Identity{d} \end{bmatrix}$ spans the same subspace as $\begin{bmatrix} \VV_{12}^{\Hermit} & \VV_{22}^{\Hermit} \end{bmatrix}$, leading to 
%\begin{equation} \label{eq:TLS_colspace}
%	\begin{bmatrix} \Mthree \\ - \Identity{d} \end{bmatrix}
%	= \begin{bmatrix} \VV_{12} \\ \VV_{22} \end{bmatrix} \Mfour
%\end{equation}
%where $\Mfour \in \CC^{d \times d}$. Eq. \eqref{eq:TLS_colspace} yields $- \Identity{d} = \VV_{22} \Mfour$, resulting in $\Mfour = - \VV_{22}^{-1}$. 
%Thus, 
the TLS solution for \eqref{eq:TLS_costfct} %can be obtained from
\begin{equation} \label{eq: TLS_solution}
	\Mthree_{\TLS} = - \VV_{12} \VV_{22}^{-1}.
\end{equation}




\subsection{Newton's Method} \label{sec:Newton}
% Dimension
% f: \RR^{\LL} \rightarrow \RR
% Sec: Newton's method
% Intro
In optimization, the Newton's method is applied to solve unconstrained nonlinear optimization problems iteratively using both first and second derivative of the cost function. Despite its fast convergence, the Newton's method is often less preferred to other gradient-based methods, such as quasi-Newton's method, since it requires to compute the second derivative, which is computationally expensive. Yet, if the second derivative for the particular problem is easy to calculate, it provides more precise and faster convergence than its counterparts which rely only on the first derivative \cite{Nocedal06NumOpt}. \par

% Optimality check
Suppose we want to minimize a twice differentiable cost function $f: \RR^{\LL} \rightarrow \RR$ 
\begin{equation}
	\min_{\xx \in \RR^{\LL}} f (\xx).
\end{equation}
The solution of this problem $\xx_{\optimized}$ should satisfy the following two necessary conditions \cite{Bonnans06NumOpt}:
\begin{eqnarray*} 
	\mbox{NC1:} & \nabla f (\xx_{\optimized}) = \gradf (\xx_{\optimized}) = \bm{0}_{\LL}\\
	\mbox{NC2:} & \frac{\partial^{2} f (\xxopt)}{\partial x_{i} \partial x_{j}} = \Hess (\xxopt) \succeq 0 & \mbox{(positive semi-definite)}.
\end{eqnarray*}
Under the assumption that NC2 is satisfied for all $\xx \in \RR^{\LL}$, the Newton's method converges towards $\xx_{\optimized}$ by seeking for the roots of $\gradf$, which is equivalent to NC1. \par

% Taylor expansion
Each iteration of the Newton's method computes a search direction $\dd{n}$ based on the current iterate $\iter{n}$ and searches for a new iterate $\iter{n + 1}$ whose function value $f (\iter{n+1})$ is lower than the current one $f (\iter{n})$ \cite{Nocedal06NumOpt}. Both the function value and its gradient of the next iterate can be expressed with the current iterate using Taylor expansion as
\begin{align} \label{eq:Taylor_fx}
	f (\iter{n + 1})
	& = f (\iter{n} +  \dd{n}) \nonumber \\
	& \approx f (\iter{n}) + \left[ \gradf (\iter{n}) \right]^{\T} \dd{n} + \frac{1}{2} \dd{n}^{\T} \Hess (\iter{n}) \dd{n}
\end{align}
%
\begin{equation} \label{eq:Taylor_gx}
	\gradf (\iter{n + 1}) = \gradf (\iter{n} +  \dd{n}) \approx \gradf (\iter{n}) + \Hess (\iter{n}) \dd{n}.
\end{equation} \par

% Search direction 
Considering \eqref{eq:Taylor_gx} to satisfy NC1 yields the search direction as
\begin{equation} \label{eq:search_dir}
	\dd{n} = - \left[ \Hess (\iter{n}) \right]^{-1} \left[ \gradf (\iter{n}) \right]^{\T},
\end{equation}
which can be inserted into \eqref{eq:Taylor_fx} 
\begin{equation}
	f (\iter{n + 1}) \approx f (\iter{n}) - \frac{1}{2} \left[ \gradf (\iter{n}) \right]^{\T} \left[ \Hess (\iter{n}) \right]^{-1} \gradf (\iter{n}).
\end{equation}
Since $\Hess$ is assumed to be positive semi-definite, $f (\iter{n + 1}) \leq f (\iter{n})$ is ensured, suggesting that $\dd{n}$ in \eqref{eq:search_dir} is an appropriate choice for the search direction. \par


%%%%%%%%%%%%%%%%%%%%%%%%%%%%%%%%%%%%%%%%%%% sec.4 %%%%
\section{Blind Error Correction (BEC)} \label{sec:methods}  
% Intro
% Sec: spatial approximation in method intro
% Intro
Although SAFT reconstruction is well known for its robustness, computing the measurement matrix from the inaccurate positional information may significantly impair the imaging quality of its reconstruction. However, if the deviation between the actual and the tracked positions is little, the measurement matrix for the actual positions can be spatially approximated via the first order Taylor approximation which we discussed in Section \ref{sec:Newton}. %\par
%
%% Taylor
%Using the first order Taylor approximation the value of an arbitrary function $f$ at the point $u$ can be approximated as   
%\begin{equation}
%	f(u) \approx f (u + \Delta u) - f' (u + \Delta u) \cdot  \Delta u,
%\end{equation}
%where $\Delta u$ represents the deviation and $f'$ denotes the derivative of $f$. If $f(u)$ and $f (u + \Delta u)$ are available, the deviation $ \Delta u$ can be deduced by comparing these two function values. 
The same analogy can be employed for approximating the measurement matrix $\SAFT$ in \eqref{eq:FWM}.\par

% Setup & goal
In this study, we suppose that $\K$ measurements are taken at the positions $\xx \in \RR^{\K}$ which are recognized falsely by the tracking system with the tracking error $\xxdelta \in \RR^{\K}$. This leads the measurement assistance system to provide us the inaccurate positional information $\xxhat = \xx + \xxdelta$. Based on the available information, namely the measurement data $\Ascan$ and the tracked positions $\xxhat$, we aim to estimate and correct the tracking error through comparing our data model with the actual measurement data. \par

% SAFT matrix approximation for ingle A-Scan
For a single A-Scan taken at $x_{k}$, the measurement matrix of the actual position can be approximated with its tracked position $\xhat{k}$ and deviation $\xdelta{k}$ as
\begin{equation} \label{eq:SAFTapproxsingle}
	\SAFT_{k} \approx \SAFThat_{k} - \SAFTdothat_{k} \cdot \xdelta{k}.
\end{equation}
Here, $\SAFT_{k} \in \RR^{\M \times \LL}$ and $\SAFThat_{k} \in \RR^{\M \times \LL}$ are the measurement matrix at the actual position $x_{k}$ and the tracked position $\xhat{k}$, respectively, while $\SAFTdothat_{k} \in \RR^{\M \times \LL}$ denotes the derivative of the measurement matrix with respect to the position at $\xhat{k}$. The derivative of the measurement matrix is obtained from 
\begin{equation} \label{eq:SAFTderivsingle}
	\SAFTdothat_{k} = \begin{bmatrix} \SAFTdothatcol{k}{1} & \SAFTdothatcol{k}{2} & \cdots & \SAFTdothatcol{k}{\LL} \end{bmatrix}, 
\end{equation}
where its column vectors are derived from the corresponding column vectors of the measurement matrix with
\begin{equation}
	\SAFTdothatcol{k}{l} = \frac{\partial \SAFThatcol{k}{l}}{\partial x}.
\end{equation}

% SAFT matrix approximation for K scans
The approximation of the complete measurement matrix $\SAFT$ of the actual measurement positions $\xx$ can be now expressed based on \eqref{eq:SAFTdef} and \eqref{eq:SAFTapproxsingle} as
\begin{align} \label{eq:SAFTapprox}
	\SAFT 
	& \approx
	\begin{bmatrix} 
		\SAFThat_{1} \\ \SAFThat_{2} \\ \vdots \\ \SAFThat_{\K} 
	\end{bmatrix} -
	\begin{bmatrix} 
		\SAFTdothat_{1} \cdot \xdelta{1}\\ \SAFTdothat_{2} \cdot \xdelta{2}\\ \vdots \\ \SAFTdothat_{\K} \cdot \xdelta{\K} 
	\end{bmatrix} \\
	& =
	\SAFThat - \Error \SAFTdothat. \nonumber
\end{align}
Here, $\SAFTdothat \in \RR^{\M \K \times \LL}$ is the complete derivative matrix, whereas $\Error \in \RR^{\M \K \times \M\K}$ is obtained from
\begin{equation} \label{eq:ErrorMat}
	\Error = \ErrorDef
\end{equation}
with the diagonalization operator $\diag{\cdot}$ and $\otimes$ representing the Kronecker product.

% A-Scan approximation ^> cost function
Based on \eqref{eq:FWM} and \eqref{eq:SAFTapprox}, the measurement data can be modeled as 
\begin{equation} \label{eq:Ascanapprox}
	\vectorize{\Ascan} \approx \left[ \SAFThat - \Error \SAFTdothat \right] \defect + \vectorize{\bm{N}}.
\end{equation}
Yet, for estimating the tracking error we still lack the information regarding the scatterer positions $\defect$, resulting in \eqref{eq:Ascanapprox} becoming a joint optimization problem as
\begin{equation} \label{eq:CostFct}
	\min_{\xxdelta} \min_{\defect} 
	\norm{ 
		\vectorize{\Ascan} - \left[ \SAFThat - \Error \SAFTdothat \right] \defect	 
	}_2^{2}.
\end{equation} \par

% Preprocessing steps
Solving \eqref{eq:CostFct} can be divided into two steps shown as Fig. \ref{fig:blockdiagram}. Firstly we estimate the scatterer positions $\defect$ via curve fitting while taking the tracking error into account. Then, we aim to estimate and correct the tracking error iteratively by comparing our data model in \eqref{eq:Ascanapprox} to the measurement data. In the following subsections we demonstrate each step of the preprocessing in detail.
%
\begin{figure} 
	\inputTikZ{0.8}{figures/block_diagram_simulation_flow_base.tex}
	\caption{Block diagram of the blind error correction as preprocessing}
	\label{fig:blockdiagram}
\end{figure}



\subsection{TLS Curve Fitting}
% Sec: Curve Fit via TLS
% Intro
The goal of this preprocessing step is to estimate the scatterer positions based on which we can model the measurement data in the next step. Here we assume that the total of $I$ scatterers are located sufficiently far apart in the specimen so that they can be separated into single source from the measurement data. Thus, we aim to estimate the scan position of a single scatter $\scatterer_{i} = \begin{bmatrix} x^{(s)}_{i} & z^{(s)}_{i} \end{bmatrix}$, which resides in the ROI, from the measurement data corresponding to this particular region $\ascan{}^{\ROI}$ and the erroneous positional information $\xxhat$. \par

% Why curve fitting?
What we know regarding the measurement data is its geometric properties. Due to the symmetric change in the ToF with respect to the scatterer $\scatterer_{i}$, the collection of the measurements, called B-Scan, follows a similar trend as hyperbola. Although they are not the same, a hyperbola can be, with the proper parameterization, well approximated as a parabola within a limited horizontal range. The horizontal range of the UT measurement data is typically very limited as it is confined to the beam spread of the transducer. This enables us to approximate the curve, which a set of measurement data traces, as a parabola, providing the information on the scatterer position. \par 

% Polynomial
One approach to find the best approximate is curve fitting. In the sense of a parabola this can be done via quadratic regression, where we seek to fit the data to a quadratic equation. For the measurement data $\ascan{k}^{\ROI}$ taken at the position $x_{k}$, the polynomial model of the data can be expressed as 
\begin{equation} \label{eq:CF_polynomial}
	z_k = w_0 + w_1 \cdot x_k + w_2 \cdot x_k^{2},
\end{equation}
where $z_k$ denotes the peak position of $\ascan{k}^{\ROI}$ and $w_0$, $w_1$ and $w_2$ are polynomial coefficients. From these coefficients, the scatterer positions can be calculated with
\begin{eqnarray} \label{eq:CF_scatterer}
	x^{(s)}_i =  - \frac{w_1}{2 w_2} & \mbox{,} & z^{(s)}_i  = w_0 - \frac{w_{1}^{2}}{4 w_2}.
\end{eqnarray}

% Vector-Matrix form
Since there are three unknowns in the polynomial, more than three A-Scans are required, and to achieve the sufficient precision of the model the total number of $\K \gg 3$ A-scans are considered to be collected at the scan positions $\xx \in \RR^{\K}$. This lets us formulate \eqref{eq:CF_scatterer} as a vector-matrix product as
\begin{equation} \label{CF_noerror}
	\begin{bmatrix} z_1 \\ z_2 \\ \vdots \\ z_{\K} \end{bmatrix} =
	\begin{bmatrix} 1 & x_1 & x_{1}^{2} \\ 1 & x_2 & x_{2}^{2} \\ & \vdots &  \\ 1 & x_{\K} & x_{\K}^{2} \end{bmatrix}
	\begin{bmatrix} w_0 \\ w_1 \\ w_2\end{bmatrix},
\end{equation}
which can be simply denoted as $\zz = \XX \ww $ with $\zz \in \RR^{\K}$, $\XX \in \RR^{\K \times 3}$ and $\ww \in \RR^{3}$. 

% Errors on both sides -> TLS
The problem here is that both dependent and independent variables, $\zz$ and $\XX$ respectively, contain the error. $\zz$ may include the measurement noise, quantization error or other possible errors, whereas $\XX$ is corrupted due to the tracking error. This results in \eqref{CF_noerror} becoming approximation which can be solved via TLS as these errors are independent. Hence, we can estimate the position of $\scatterer_i$ by solving the following optimization problem 
\begin{eqnarray} \label{eq:CF_costfct}
	\hspace*{-0.4cm} \min \norm{\begin{bmatrix} \Delta \zz & \Delta \XX \end{bmatrix}}_{\Frob}^{2} 
	& \mbox{s.t.}  
	& \zzperturb = \left( \XXperturb \right) \ww.
\end{eqnarray}


\subsection{Iterative Error Correction}
% Sec: Error Correction via Newton
In the second step of the preprocessing, the tracking error should be estimated and corrected. As the scatterer positions are estimated in the prior process, yielding $\defecthat_{\CF}$, measurement data can be modeled with \eqref{eq:Ascanapprox} based on the tracked positions $\xxhat$ and $\defecthat_{\CF}$. We now can formulate an optimization problem to estimate the tracking error as
\begin{equation} \label{eq:ECcostfct}
	\min_{\xxdelta} f (\xxdelta) = 
	\norm{
		\vectorize{\Ascan} - \SAFThat \defecthat_{\CF} + \Error \SAFTdothat \defecthat_{\CF} 
	}_{2}^{2}.
\end{equation} 
As \eqref{eq:ECcostfct} is an unconstrained nonlinear problem, it can be solved iteratively using, for instance, the Newton's method. \par 

%% Prove: Hessian is very easy to compute
% Simplification
Employing the Newton's method is, however, only meaningful, when the Hessian matrix of \eqref{eq:ECcostfct} is proven to be easy to compute. Considering that all $\vectorize{\Ascan}$, $\SAFThat \defecthat_{\CF}$ and $\SAFTdothat \defecthat_{\CF}$ are vectors, $f (\xxdelta)$ in \eqref{eq:ECcostfct} can be symplified as 
\begin{equation}
	f (\xxdelta) 
	= \norm{\voneall + \Error \vtwoall}_{2}^{2}
	= \left[ \voneall + \Error \vtwoall \right]^{\T} \left[ \voneall + \Error \vtwoall \right].
\end{equation}
$\voneall \in \RR^{\M \K}$ represents the difference of the $\vectorize{\Ascan}$ and $\SAFThat \defecthat_{\CF}$, while $\vtwoall \in \RR^{\M \K}$ denotes the vector obtained from $\SAFTdothat \defecthat_{\CF}$, both of which are concatenation of $\M$ length vector related to a single A-Scan and the corresponding data model of each measurement position
\begin{eqnarray}
	\voneall = \begin{bmatrix} \vonepart{1} \\ \vonepart{2} \\ \vdots \\ \vonepart{\K}	\end{bmatrix} &
	, &
	\vtwoall = \begin{bmatrix} \vtwopart{1} \\ \vtwopart{2} \\ \vdots \\ \vtwopart{\K}	\end{bmatrix}.
\end{eqnarray}

%
Since $\Error$ is a diagonal matrix, indicating that the tracking error of each measurement position only accounts for the corresponding  modeled A-Scan, the gradient and the Hessian matrix of $f$ can be calculated very easily with 
\begin{equation}
	\gradf (\xxdelta) = 
	\begin{bmatrix}
		2 \cdot \vonepart{1}^{\T} \vtwopart{1} + \xdelta{1} \cdot \norm{\vtwopart{1}}_{2}^{2}\\
		2 \cdot \vonepart{2}^{\T} \vtwopart{2} + \xdelta{2} \cdot \norm{\vtwopart{2}}_{2}^{2}\\
		\vdots \\
		2 \cdot \vonepart{\K}^{\T} \vtwopart{\K} + \xdelta{\K} \cdot \norm{\vtwopart{\K}}_{2}^{2}\\
	\end{bmatrix},
\end{equation}
and
\begin{equation}
	\Hess (\xxdelta) = \diag{\norm{\vtwopart{k}}_{2}^{2}}_{k = 1}^{\K}.
\end{equation}
This implies that we can benefit from the fast convergence of the Newton's method without expensive calculation of the second derivative. \par

% Iteration step
After we estimate the deviation $\xxdeltahat$ by solving \eqref{eq:ECcostfct} via the Newton's method, we modify the erroneous positional information $\xxhat$ to $\xxhat = \xxhat - \xxdeltahat$ and repeat the same procedure based on the newly set and improved positional information $\xxhat$. As a result, the deviation to the actual measurement positions is reduced with each iteration, improving the accuracy of the measurement matrix in \eqref{eq:SAFTapprox} which is used for the reconstruction in \eqref{eq:reco}.

%%%%%%%%%%%%%%%%%%%%%%%%%%%%%%%%%%%%%%%%%%% sec.5 %%%%
\section{Simulation: Investigation of Error Tolerance and Impact Analysis of the ROI Depth} \label{sec:simulation}  
% Write specific "name" to indicate what kind of s
%
%This section may for instance be a ``simulations'' section if the verification is based
%on computer simulations. In this case, please remember to indicate the simulations setup as completely
%as possible, including all assumptions that were made (e.g., uncorrelated Rayleigh fading
%channels or isotropic antenna elements). The reader should be able to reproduce your 
%simulation results based on your descriptions!
%
%Similarly, if measurements were performed, the measurement setup and equipment 
%as well as your test conditions must be described in detail and the outcome should
%be discussed.
%
%The description of your main results which may of course also span more than one section.
%Point out its main features, discuss its limitations, compare it to alternative solutions.
%
%All results that you show should be interpreted properly -- what do we learn from them?
%
Performance of the proposed method was examined through two simulation studies: the simulation \rom{1} shows the error sensitivity of the proposed method, while the simulation \rom{2} illustrates the impact of the ROI depth. The former seeks to provide the insight to assess whether the proposed method is appropriate for the selected measurement assistance system whose error deviation is known as a part of the device specifications. The simulation \rom{2}, on the other hand, is aimed to determine from and up to which ROI depth it is worthwhile to correct the error, since the impact of the horizontal deviation varies with the ROI depth. \par


\subsection{Assumptions and Test Parameters} 
% Sec: simulation setup
% Specimen + geometry
We conducted sets of Monte Carlo simulations with respect to the tracking error with an aluminum object for which we set the same assumptions as we described in Sec. \ref{sec:data_model}. For the sake of simplicity, the measurement data is regarded as noise free. Our ROI contains one scatterer and is a part of the test object where back and side wall echoes can be neglected. The transducer parameters we assumed are based on the Olympus standard contact transducer SUC 166-1 \cite{OlympusCatalog}. \par

% Measurements
Each measurement data is considered to be taken at a measurement grid point. The resolution of the tracking system, on the other hand, depends on the camera specifications and is assumed to be finer than the measurement grids in order to minimize the quantization error. This means that the tracking positions may be between two grid points, based on which the corresponding measurement matrix is calculated. Since more than three A-Scans are required to estimate a scatterer position, we consider an offline reconstruction process, where each measurement data is first stored in the system before it is reconstructed together with the other A-Scans. Table \ref{tab:params} provides a summary of the test parameters. \par

% Values
\inputTable{1}{tables/table_params.tex}{Summary of the test parameter values for the simulations}{tab:params}
\setlength{\belowcaptionskip}{-10pt} % reduces the sapces b/w figures & captions
% <scaling factor>, <file name for the table>, <caption>, <label>
% Scaling factor does not work currently! 

\subsection{Evaluation Metrics} 
% Sec: Evaluation Metrics
Since the imaging quality is strongly related to the human perception, providing satisfactory numerical evaluations is challenging. For this reason, we used three different metrics: normalized root squared error $\SEdag$, \textit{Array Performance Indicator} ($\API$) \cite{Holmes05API} and \textit{Generalized Contrast-to-Noise Ratio} ($\GCNR$) \cite{Molares19GCNR}. As reference we chose the reconstruction results of the measurement data which is processed at the exact measurement positions. \par

% SE
The normalized root squared error $\SEdag$ shows the difference between the obtained reconstruction results $\Recohat$ and the reference $\Reco$. This is calculated with
\begin{equation} \label{eq:SEdag}
	\SEdag = \frac{\norm{\gamma \Recohat - \Reco}_2}{\norm{\Reco}_2},
\end{equation}
where $\gamma$ is a normalization factor which can be obtained from 
\begin{equation}
	\gamma = \frac{\vectorize{\Reco}^{\T} \vectorize{\Recohat}}{\vectorize{\Recohat}^{\T} \vectorize{\Recohat}}.
\end{equation} \par

% API
The $\API$ evaluates the reconstruction resolution in terms of the area which is above a certain threshold, suggesting that the better resolution is attainable with the smaller $\API$. This is defined by \cite{Holmes05API}
\begin{equation} \label{eq:API}
	\API = \frac{A_{\epsilon}}{\lambda^{2}},
\end{equation}
where $A_{\epsilon}$ and $\lambda$ denote the area which is above the given threshold $\epsilon$ and the wavelength, respectively. In this study we considered the envelop of the reconstruction image and set $\epsilon$ to \SI{-3}{\deci \bel} of its maximum value. \par

% GCNR
On the other hand, the $\GCNR$ accounts more for human perception of the image contrast by considering the overlap of the probability density functions of inside and outside the target area \cite{Molares19GCNR}. %The superiority of this metric over other conventional contrast metrics is that $\GCNR$ remains unaltered when the contrast of an image is intensified after transforming the dynamic range by applying adaptive beamforming methods. 
For the given target area, the overlap of the probability density functions is calculated with
\begin{equation} \label{eq:OVL}
	\OVL = \int \min \{ p_i (x), p_o (x) \} \dx,
\end{equation} 
where $\OVL$ represents the overlap, while $p_i$ and $p_o$ are the probability density function of inside and outside the target area, respectively. Based on this, the $\GCNR$ is obtained from
\begin{equation} \label{eq:GCNR}
	\GCNR = 1 - \OVL,
\end{equation}
revealing that the higher $\GCNR$ value indicates the better resolution of the image. Here we defined the target area from the reference where its envelope is above \SI{-3}{\deci \bel} of its maximum value.\par

Since we conducted sets of Monte Carlo simulations, the average value of each data set, denoted as $\MSEdag$, $\MAPI$ and $\MGCNR$, is calculated and shown in the following sections. 


\subsection{Results of the Simulation \rom{1}} 
% Sec: Results of the simulation 1
% General
Figure \ref{fig:results_tolerance} presents the evaluation of the obtained results for the simulation \rom{1}, and Figure \ref{fig:recoimg_tolerance}, \ref{fig:true30}, \ref{fig:track30} and \ref{fig:opt30} show the corresponding example reconstruction images. We compared three results: reconstruction without error correction, with error correction (BEC) and the reference. \par

% Smaller error ... 0.4 lambda
If the tracking error is very little, up to $0.4 \lambda$ in the selected ROI , $\MAPI$ and $\MGCNR$ of the uncorrected reconstruction remains almost the same, showing that such error has negligible effect on the imaging quality. Although it is within the satisfying level, $\MSEdag$ and $\MGCNR$ of the proposed method is worse than that of the uncorrected one. At the same time, $\MAPI$ remains low, which suggests that for a small range of the error approximating the measurement matrix causes weak but more artefacts in the reconstruction than an erroneous measurement matrix. Correcting the tracking error for this error range is, thus, not necessary. \par

% Larger error
With increasing error, all results illustrate that the tracking error starts making nonnegligible difference between the erroneous and the actual measurement matrices and affecting the imaging quality of the reconstructions. Despite their similar $\MGCNR$ values, the $\MAPI$ values of the uncorrected and the corrected reconstruction vary significantly. These results imply that, while both reconstruction contains the same "amount" of the artefacts, the magnitude of the artefacts in the uncorrected reconstruction is stronger than that in the error-corrected counterpart. Indeed, the imaging quality of the proposed method is comparable to that of the reference (Figure \ref{fig:opt30} and \ref{fig:true30}). This indicates that the proposed method can tolerate larger positional error than reconstructing without preprocessing.  \par

% Results with evaluation
\begin{figure}
\input{figures/fig_results_tolerance.tex}
\setlength{\abovecaptionskip}{-5pt} % reduces the sapces b/w figures & captions
\caption{Results obtained with the simulation \rom{1}: the reconstruction imaging quality for the varying tracking error}
\label{fig:results_tolerance}
\setlength{\belowcaptionskip}{-20pt} % reduces the sapces b/w figures & captions
\end{figure}

%% Reco images
\begin{figure}	
\input{figures/fig_recoimgs_tolerance.tex}
\setlength{\abovecaptionskip}{-10pt} % reduces the sapces b/w figures & captions
\caption{Example reconstruction results of the simulation \rom{1}: \ref{fig:true30tolerance} for reference reconstruction, \ref{fig:track04} and \ref{fig:track08} for reconstruction without correction and \ref{fig:opt04} and \ref{fig:opt08} for reconstruction with BEC. The upper row (except the reference) is with the tracking error of $0.4 \lambda$, while the lower row is with the error of $0.8 \lambda$.}
\label{fig:recoimg_tolerance}
\setlength{\belowcaptionskip}{-20pt} % reduces the sapces b/w figures & captions
\end{figure}


\subsection{Results of the Simulation \rom{2}} 
% Sec: Results of the simulation 2
% General
The evaluation of the simulation \rom{2} is presented in Figure \ref{fig:results_depth}, and the example reconstruction results are provided in Figure \ref{fig:recoimg_depth}.  Both results demonstrate that the ROI depth affects the performance of the proposed method. \par

% Shallower region
Near the object surface, in our case up to \SI{20}{\milli \metre} depth, the imaging quality of the SAFT reconstruction, including the reference, is degraded than the deeper region. This is firstly because of the limited horizontal range due to the transducer beam spread, which generally impairs the spatial resolution of SAFT reconstructions. Furthermore, there is larger change in the ToF of two consecutive scan positions than the deeper region. This not only causes the same horizontal deviation to have more impact on the imaging quality than the deeper region but also reduces the validity range of the spatial approximation. Consequently, the proposed method becomes very susceptive to the horizontal deviation and yields similar or even worse results compared to the reconstruction without error correction. \par

% Middle region
If the ROI lies deeper in the object, the resolution of all SAFT reconstructions becomes better than in the shallower region. This is because the horizontal range captured within the beam spread expands, enhancing the SAFT reconstruction, which can be seen in improved value of $\MAPI$ and $\MGCNR$ in the reference. The most significant improvement in the imaging quality, however, can be observed in our proposed method. As Figure \ref{fig:opt30} demonstrates, artefacts formation is considerably reduced, resulting in the improved evaluation values in Figure \ref{fig:results_depth}. This is due to the fact that there is smaller change in ToF between two neighboring scans than in the shallower region, which increases the validity range of the spatial approximation and enables to correct the tracking error. Decrease in the ToF change also makes the horizontal deviation less impactful compared to the near surface region, however the uncorrected reconstruction shows less significant improvement than our proposed method, revealing that either strong artefacts or horizontal shifts in the reconstruction cannot be avoided in this region (Figure \ref{fig:track30}). Overall, the results suggest that applying error correction is beneficial, if the ROI does not lie near the surface. \par

% Deeper region
In much deeper region, deeper than \SI{50}{\milli \metre} for our setup, the evaluation results in Figure \ref{fig:results_depth} indicates that the resolution of all three methods converges. Notably, $\MGCNR$ of the proposed method converges to that of the reference, demonstrating that in this region the tracking error can be very well corrected. Even without error correction, a satisfying resolution is attainable (Figure \ref{fig:track50}). This implies that the horizontal deviation of $1 \lambda$ has little effect on the imaging quality in this region. Although applying error correction may seem not worthwhile, both $\MAPI$ and $\MGCNR$ show that the uncorrected reconstruction cannot achieve the high resolution as that of other two reconstructions. This may become crucial for reconstructing the actual measurement data, as the attainable resolution is much lower than that of the simulations based on the point source assumption.

% Results with evaluation
\begin{figure}
\input{figures/fig_results_depth.tex}
\setlength{\abovecaptionskip}{-5pt} % reduces the sapces b/w figures & captions
\caption{Results obtained with the simulation \rom{2}: the reconstruction imaging quality for the varying ROI depth}
\label{fig:results_depth}
\end{figure}

%% Reco images
\begin{figure}	
\input{figures/fig_recoimgs_depth.tex}
\caption{Example reconstruction results of the simulation \rom{2}: \ref{fig:true20} to \ref{fig:true50} for reference reconstruction, \ref{fig:track20} to \ref{fig:track50} for reconstruction without correction and \ref{fig:opt20} to \ref{fig:opt50} for reconstruction with BEC}
\label{fig:recoimg_depth}
\end{figure}

%%%%%%%%%%%%%%%%%%%%%%%%%%%%%%%%%%%%%%%%%%%%%%%%%%%
%%% The conclusions
%%%%%%%%%%%%%%%%%%%%%%%%%%%%%%%%%%%%%%%%%%%%%%%%%%%

\section{Conclusion} \label{sec:conclusions}
% Sec: conclusion
In this study we presented a two-step preprocessing method to reduce position-inaccuracy induced artefacts in SAFT reconstructions through correcting the unknown positional error. 
Since the errors in the available information are independent, we devised the first part of the preprocessing as a TLS polynomial regression problem to estimate the scatterer positions. The second part is formulated as a nonlinear optimization problem to estimate and correct the tracking error, which can be solved via Newton's method as its Hessian matrix is very easy to compute. Simulation studies have demonstrated that the proposed method can tolerate larger error and achieve higher resolution than reconstructing without preprocessing. Considering the physical size of the signal source in actual measurements, which alone impairs the resolution of SAFT reconstructions, our method has the potential to greatly reduce error-induced artefacts and realize previously unreachable imaging quality for manual inspection. 

%%%%%%%%%%%%%%%%%%%%%%%%%%%%%%%%%%%%%%%%%%%%%%%%%%%
%%% References
%%%%%%%%%%%%%%%%%%%%%%%%%%%%%%%%%%%%%%%%%%%%%%%%%%%

\small \bibliography{main}

\end{document}


% Some more hints on formatting and avoiding common mistakes can be found here:
%    http://www.ieee-icc.org/2008/template.pdf