% Sec: Evaluation Metrics
Since the imaging quality is strongly related to the human perception, providing satisfactory numerical evaluations is challenging. For this reason, we used three different metrics: normalized root squared error $\SEdag$, \textit{Array Performance Indicator} ($\API$) \cite{Holmes05API} and \textit{Generalized Contrast-to-Noise Ratio} ($\GCNR$) \cite{Molares19GCNR}. As reference we chose the reconstruction results of the measurement data which is processed at the exact measurement positions. \par

% SE
The normalized root squared error $\SEdag$ shows the difference between the obtained reconstruction results $\Recohat$ and the reference $\Reco$. This is calculated with
\begin{equation} \label{eq:SEdag}
	\SEdag = \frac{\norm{\gamma \Recohat - \Reco}_2}{\norm{\Reco}_2},
\end{equation}
where $\gamma$ is a normalization factor which can be obtained from 
\begin{equation}
	\gamma = \frac{\vectorize{\Reco}^{\T} \vectorize{\Recohat}}{\vectorize{\Recohat}^{\T} \vectorize{\Recohat}}.
\end{equation} \par

% API
The $\API$ evaluates the reconstruction resolution in terms of the area which is above a certain threshold, suggesting that the better resolution is attainable with the smaller $\API$. This is defined by \cite{Holmes05API}
\begin{equation} \label{eq:API}
	\API = \frac{A_{\epsilon}}{\lambda^{2}},
\end{equation}
where $A_{\epsilon}$ and $\lambda$ denote the area which is above the given threshold $\epsilon$ and the wavelength, respectively. In this study we considered the envelop of the reconstruction image and set $\epsilon$ to \SI{-3}{\deci \bel} of its maximum value. \par

% GCNR
On the other hand, the $\GCNR$ accounts more for human perception of the image contrast by considering the overlap of the probability density functions of inside and outside the target area \cite{Molares19GCNR}. %The superiority of this metric over other conventional contrast metrics is that $\GCNR$ remains unaltered when the contrast of an image is intensified after transforming the dynamic range by applying adaptive beamforming methods. 
For the given target area, the overlap of the probability density functions is calculated with
\begin{equation} \label{eq:OVL}
	\OVL = \int \min \{ p_i (x), p_o (x) \} \dx,
\end{equation} 
where $\OVL$ represents the overlap, while $p_i$ and $p_o$ are the probability density function of inside and outside the target area, respectively. Based on this, the $\GCNR$ is obtained from
\begin{equation} \label{eq:GCNR}
	\GCNR = 1 - \OVL,
\end{equation}
revealing that the higher $\GCNR$ value indicates the better resolution of the image. Here we defined the target area from the reference where its envelope is above \SI{-3}{\deci \bel} of its maximum value.\par

Since we conducted sets of Monte Carlo simulations, the average value of each data set, denoted as $\MSEdag$, $\MAPI$ and $\MGCNR$, is calculated and shown in the following sections. 
