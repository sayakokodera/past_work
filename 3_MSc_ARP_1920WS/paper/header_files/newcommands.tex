% "newcommand" collections
% !!! In newcommands: numbers cannot be used for the command names!!!
%=====================  for general =======================%
\renewcommand\thesubsection{\thesection.\Alph{subsection}}

\newcommand{\ToDo}[1]{%
	\textcolor{red}{#1}
}

%=====================  for math =======================%
% Math operands
% l2 norms
\newcommand{\norm}[1]{%
	\left\lVert#1\right\rVert
}
% Transpose
\newcommand{\TP}[1]{%
	#1^{\T}
}
% Vectorization
\newcommand{\vectorize}[1]{%
	\operatorname{vec} \{ #1 \}
}
% Diagonal matrix
\newcommand{\diag}[1]{%
	\operatorname{diag} \{ #1 \}
}
% Real part
\newcommand{\Real}{%  
     \operatorname{Re}
}
 % Imaginary part
\newcommand{\Imag}{%  
     \operatorname{Im}
}
% Column space
\newcommand{\Col}[1]{
	\operatorname{col} \{ #1 \}
}
% Nullspace 
\newcommand{\Null}[1]{
	\operatorname{null} \{ #1 \}
} 


% symbols
% General
\newcommand{\Identity}[1]{\bm{I}_{#1}} % Identity matrix
\newcommand{\RR}{{\mathbb{R}}}
\newcommand{\CC}{{\mathbb{C}}}

% TLS relevant
\newcommand{\Mone}{\bm{\Omega}}
\newcommand{\MoneDelta}{\Delta \Mone}
\newcommand{\MonePerturb}{\Mone + \MoneDelta}
\newcommand{\Mtwo}{\bm{\Psi}}
\newcommand{\MtwoDelta}{\Delta \Mtwo}
\newcommand{\MtwoPerturb}{\Mtwo + \MtwoDelta}
\newcommand{\Mthree}{\bm{\Upsilon}}
\newcommand{\Mfour}{\bm{T}}
\newcommand{\UU}{\bm{U}} % SVD: U
\newcommand{\VV}{\bm{V}} % SVD: V
\newcommand{\SSigma}{\bm{\Sigma}} % SVD: Sigma
% Newton relevant
\newcommand{\gradf}{g}
\newcommand{\Hess}{\bm{H_{f}}}
\newcommand{\dd}[1]{\bm{d}_{#1}}
\newcommand{\iter}[1]{\xx_{#1}}

% scan positions
% p = scalar, \pp = vector
\newcommand{\pp}{\bm{p}}
\newcommand{\pphat}{\bm{\hat{p}}}
\newcommand{\ppdelta}{\Delta \pp}
% x = scalar, \xx = vector
\newcommand{\xdelta}[1]{\Delta x_{#1}}
\newcommand{\xhat}[1]{\hat{x}_{#1}}
\newcommand{\xopt}{\xhat_{\optimized}}
\newcommand{\xx}{\bm{x}}
\newcommand{\xxdelta}{\Delta \xx}
\newcommand{\xxhat}{\hat{\xx}}
\newcommand{\xxopt}{\xxhat_{\optimized}}
\newcommand{\xxdeltahat}{\Delta \xxhat}
\newcommand{\xxdeltaopt}{\xxdeltahat_{\optimized}}
% scatter position
\newcommand{\scatterer}{\bm{s}}
% A-Scans
\newcommand{\ascansig}{a}
\newcommand{\ascan}[1]{\bm{a}_{#1}} % with \xx
\newcommand{\ascanhat}[1]{\hat{\ascan}_{#1}} % with \xxhat
\newcommand{\Ascan}{\bm{A}} % with \xx
\newcommand{\Ascanhat}{\hat{\Ascan}} % with \xxhat
% Pulse, SAFT matrix, spatial approximation
\newcommand{\pulsesig}{h}
\newcommand{\pulse}{\bm{h}}
\newcommand{\pulsehat}{\hat{\pulse}}
\newcommand{\pulsederiv}{\bm{j}}
\newcommand{\pulsederivhat}{\hat{\pulsederiv}}
\newcommand{\SAFTcol}[2]{\pulse_{#1}^{(#2)}}
\newcommand{\SAFThatcol}[2]{\pulsehat_{#1}^{(#2)}}
\newcommand{\SAFT}{\bm{H}}
\newcommand{\SAFThat}{\hat{\SAFT}}
\newcommand{\SAFTdot}{\bm{J}}
\newcommand{\SAFTdotcol}[2]{\pulsederiv_{#1}^{(#2)}}
\newcommand{\SAFTdothat}{\hat{\SAFTdot}}
\newcommand{\SAFTdothatcol}[2]{\pulsederivhat_{#1}^{(#2)}}
\newcommand{\Error}{\bm{E}}
\newcommand{\ErrorDef}{\diag{\xxdelta} \otimes \Identity{\M}}
% Defect map
\newcommand{\defect}{\bm{b}} 
\newcommand{\defecthat}{\hat{\defect}}
% Reco
\newcommand{\Reco}{\bm{B}}  
\newcommand{\Recohat}{\hat{\Reco}}  

% TLS curve fit
\newcommand{\zz}{\bm{z}}
\newcommand{\zzperturb}{\zz + \Delta \zz}
\newcommand{\XX}{\bm{X}}
\newcommand{\XXperturb}{\XX + \Delta \XX}
\newcommand{\ww}{\bm{w}}
% Iterative EC via Newton
\newcommand{\vonebase}{\alpha}
\newcommand{\voneall}{\bm{\tilde{\vonebase}}}
\newcommand{\vonepart}[1]{\bm{\vonebase}_{#1}}
\newcommand{\vtwobase}{j}
\newcommand{\vtwoall}{\bm{\tilde{\vtwobase}}}
\newcommand{\vtwopart}[1]{\bm{\vtwobase}_{#1}}
% Metrics
\newcommand{\SEdag}{\SE^{\dagger}} 
\newcommand{\MSEdag}{\MSE^{\dagger}} 

% else
\newcommand{\refcoeff}{\beta}
\newcommand{\noisevec}{\bm{n}} 


% Dimensions
\newcommand{\N}{N}
\newcommand{\M}{M}
\newcommand{\K}{K}
\newcommand{\LL}{L}
\newcommand{\I}{I}

%=====================  for roman numbers =======================%

\newcommand{\rom}[1]{\uppercase\expandafter{\romannumeral #1\relax}}
% from https://tex.stackexchange.com/questions/23487/how-can-i-get-roman-numerals-in-text


%=====================  for tables =======================%
\newcommand{\inputTable}[4]{ % <scaling factor>, <file name for the table>, <caption>, <label>
	%\resizebox{#1}{!}{ -> only works in the beamer setting?
		\begin{table}
		\begin{center}
			\input{#2}
			\caption{#3}
			\label{#4}
		\end{center}
		\end{table}
	%}
}

%=====================  for TikZ =======================%
\newcommand{\inputTikZ}[2]{%  
     \scalebox{#1}{\input{#2}}  
}
%%%%%%%%%%%%%%%%%%%%%%%%%%%%%%%%%%%%%%%%%%%%%
%=================  for data visualization ===================%
%%%%%%%%%%%%%%%%%%%%%%%%%%%%%%%%%%%%%%%%%%%%%

%%%%%%%%%%%%%%%%%%%%%%%%%%%%%%%%%%%%%%%%%%%%%%
%=================== 1D visualization ======================%
%%%%%%%%%%%%%%%%%%%%%%%%%%%%%%%%%%%%%%%%%%%%%%

% SE tolerance
\newcommand{\resultSEtolerance}[6]{ % <scale size>, <font size>,  <fname for track 50mm>, <fname for opt 50mm>,  <fname for track 30mm>, <fname for opt 30mm>
\scalebox{#1}{
	\begin{tikzpicture}
            \begin{axis}[
                width = 7cm, height = 4cm,
                xlabel = {Tracking error / $\lambda$}, ylabel = {$\MSEdag$},
                ymin= -0.04, ymax= 0.65,
                label style = {font = #2},
                tick label style = {font = #2},
                xtick = {0, 0.2, ..., 1.01},
                ytick = {0, 0.2, ..., 0.6}, 
                grid=both, grid style={line width=.1pt, draw=gray!20},
                legend style ={
                	at={(1.45, 0.7)},
                	nodes={scale=0.95, transform shape},
                	font = #2
                }
                ]
                \input{#3} % track 50mm
                \input{#4} % opt 50mm
                \input{#5} % track 30mm
                \input{#6} % opt 30mm
                
             % legend
             % To insert the legend title
		   	%\addlegendimage{empty legend} 
		   	% Legend entries  
             \addlegendentry{No correction}
             \addlegendentry{BEC} 
             % Title
             %\addlegendentry{$|s_{x} - x|$}        
            \end{axis}
	\end{tikzpicture}
	}
}

% SE depth 
\newcommand{\resultSE}[7]{ % <scale size>, <font size>, <xlabel>, <ymax>, <xtick>, <fname for track>, <fname for opt>
\scalebox{#1}{
	\begin{tikzpicture}
            \begin{axis}[
                width = 7cm, height = 4cm,
                xlabel = {#3}, ylabel = {$\MSEdag$},
                ymin= -0.04, ymax= #4,
                label style = {font = #2},
                tick label style = {font = #2},
                xtick = {#5},
                ytick = {0, 0.2, ..., #4}, 
                grid=both, grid style={line width=.1pt, draw=gray!20},
                legend style ={
                	at={(1.45, 0.7)},
                	nodes={scale=0.95, transform shape},
                	font = #2
                }
                ]
                \input{#6} % track
                \input{#7} % opt
                
             % legend
             % To insert the legend title
		   	%\addlegendimage{empty legend} 
		   	% Legend entries  
             \addlegendentry{No correction}
             \addlegendentry{BEC} 
             % Title
             %\addlegendentry{$|s_{x} - x|$}        
            \end{axis}
	\end{tikzpicture}
	}
}

% API
\newcommand{\resultAPI}[9]{ % <scale size>, <font size>, <xlabel>, <ymax>, <xtick>, <ytick>, <fname for true>, <fname for track>, <fname for opt>
\scalebox{#1}{
	\begin{tikzpicture}
            \begin{axis}[
                width = 7cm, height = 4cm,
                xlabel = {#3}, ylabel = {$\MAPI$},
                ymin= 16.5, ymax= #4,
                label style = {font = #2},
                tick label style = {font = #2},
                xtick = {#5},
                ytick = {#6},
                grid=both, grid style={line width=.1pt, draw=gray!20},
                legend style ={
                	at={(1.45, 0.8)},
                	nodes={scale=0.95, transform shape},
                	font = #2
                }
                ]
                \input{#7} % true
                \input{#8} % track
                \input{#9} % opt
                
             % legend
             % To insert the legend title
		   	%\addlegendimage{empty legend} 
		   	% Legend entries  
		   	\addlegendentry{Reference}
             \addlegendentry{No correction}
             \addlegendentry{BEC}
             % Title
             %\addlegendentry{$|s_{x} - x|$}        
            \end{axis}
	\end{tikzpicture}
	}
}


% GCNR
\newcommand{\resultGCNR}[9]{ % <scale size>, <font size>, <xlabel>, <ymin>, <xtick>, <ytick>, <fname for true>, <fname for track>, <fname for opt>
\scalebox{#1}{
	\begin{tikzpicture}
            \begin{axis}[
                width = 7cm, height = 4cm,
                xlabel = {#3}, ylabel = {$\MGCNR$},
                ymin= #4, ymax= 0.97,
                label style = {font = #2},
                tick label style = {font = #2},
                xtick = {#5},
                ytick = {#6},  % 1.01 = otherwise the tick does not show up 
                grid=both, grid style={line width=.1pt, draw=gray!20},
                legend style ={
                	at={(1.45, 0.8)},
                	nodes={scale=0.95, transform shape},
                	font = #2
                }
                ]
                \input{#7} % true
                \input{#8} % track
                \input{#9} % opt
                
             % legend
             % To insert the legend title
		   	%\addlegendimage{empty legend} 
		   	% Legend entries  
		   	\addlegendentry{Reference}
             \addlegendentry{No correction}
             \addlegendentry{BEC}
             % Title
             %\addlegendentry{$|s_{x} - x|$}        
            \end{axis}
	\end{tikzpicture}
	}
}

% GD SE
\newcommand{\gdse}[8]{ %  <scale size>, <label font size>, <tick font size>, <fname for 0.5 lambda>, <fname for 1mm>, <fname for 2.5mm>, <fname for 5mm>, <fname for 7.5mm>
\scalebox{#1}{
	\begin{tikzpicture}
            \begin{axis}[
                width = 7cm, 
            	   height = 4cm,
            	   xmin = -2.1,
            	   xmax = 2.1,
            	   ymin = -0.1,
            	   ymax = 1.1,
                xlabel = {$\xdelta / \lambda$},
                ylabel = {$\SEdag$},
                label style = {font = #2},
                tick label style = {font = #3},
                %y dir = reverse,
                %xtick = {0, 10, ..., 30}, %to customize the axis
                extra x ticks={0.8}, 
                extra x tick style={font = #3, tui_orange},%yshift={-1.2em}
                %xticklabel = {0, 5, 10, 15}
                legend style ={
                	at={(1.4, 0.9)},
                	nodes={scale=0.95, transform shape},
                	font = #3
                }
                ]
                % Reverse the input order, so that 7.5mm away is sent to background
                \input{#8} %7.5mm
                \input{#7} %5mm
                \input{#6} %2.5mm
                \input{#5} %1mm
                \input{#4} %0.5 lambda
                
             % legend
             % To insert the legend title
		   	\addlegendimage{empty legend} 
		   	% Legend entries  
             \addlegendentry{\SI{7.5}{\milli\metre}}
             \addlegendentry{\SI{5}{\milli\metre}}
             \addlegendentry{\SI{2.5}{\milli\metre}} %{$1.98 \lambda$}
             \addlegendentry{\SI{1}{\milli\metre}} %{$0.8 \lambda$}
             \addlegendentry{\SI{0.63}{\milli\metre}} %{$0.5 \lambda$}
             % Title
             \addlegendentry{$| s_{x} - x |$}
             
%             % x = 0.8 line
%	            \addplot[gray, dashed, line width = 1pt, mark = ] coordinates{
%	            			(0.8, 0.2)
%	            			(0.8, -0.1)
%            		};
                         
            \end{axis}
	\end{tikzpicture}
	}
}

%%%%%%%%%%%%%%%%%%%%%%%%%%%%%%%%%%%%%%%%%%%%%%
%=================== 2D visualization ======================%
%%%%%%%%%%%%%%%%%%%%%%%%%%%%%%%%%%%%%%%%%%%%%%
% image with both x- & y-labels
\newcommand{\imgbothlabels}[7]{% <scale size>, <label font size>, <tick font size>,<ytick>, <ymin>, <ymax>, <png file name>
\scalebox{#1}{
	\begin{tikzpicture}
            \begin{axis}[
                enlargelimits = false,
                axis on top = true,
                axis equal image,
                unit vector ratio= 0.5 1, % change aspect ratio, one of them should be 1
                point meta min = -1,   
                point meta max = 1,
                xlabel = {$x$ [\SI{}{\milli \metre}]},
                ylabel = {$z$ [\SI{}{\milli \metre}]},
                label style = {font = #2},
                tick label style = {font = #3},
                y dir = reverse,
                xtick = {15, 20, 25},
                ytick = {#4},
                ]
                \addplot graphics [
                    xmin = 15,
                    xmax = 25,
                    ymin = #5,
                    ymax = #6
                ]{#7};
            \end{axis}
	\end{tikzpicture}
	}
}
            
% img with only x-label
\newcommand{\imgxlabel}[7]{% <scale size>, <label font size>, <tick font size>, <ytick>, <ymin>, <ymax>, <png file name>
\scalebox{#1}{
	\begin{tikzpicture}
            \begin{axis}[
                enlargelimits = false,
                axis on top = true,
                axis equal image,
                unit vector ratio= 0.5 1, % change aspect ratio, one of them should be 1
                point meta min = -1,   
                point meta max = 1,
                xlabel = {$x$ [\SI{}{\milli \metre}]},
                %ylabel = {$y / \dy$},
                label style = {font = #2},
                tick label style = {font = #3},
                xtick = {15, 20, 25},
                ytick = {#4},
                %yticklabel = \empty,
                y dir = reverse,
                ]
                \addplot graphics [
                    xmin = 15,
                    xmax = 25,
                    ymin = #5,
                    ymax = #6
                ]{#7};
            \end{axis}
	\end{tikzpicture}
	}
}



% img with only x-label and cmap
\newcommand{\imgxlabelwithcmap}[7]{% <scale size>, <label font size>, <tick font size>, <ytick>, <ymin>, <ymax>, <png file name>
\scalebox{#1}{
	\begin{tikzpicture}
            \begin{axis}[
                enlargelimits = false,
                axis on top = true,
                axis equal image,
                unit vector ratio= 0.5 1, % change aspect ratio, one of them should be 1
                point meta min = -1,   
                point meta max = 1,
                colorbar,
                colormap = {mymap}{rgb(0.0pt) = (0, 0.22, 0.39) ; rgb(0.43pt) = (0.91, 0.93, 0.96) ; rgb(0.5pt) = (0.97, 0.97, 0.98) ; rgb(0.57pt) = (0.99, 0.93, 0.87) ; rgb(1.0pt) = (0.94, 0.49, 0) ; } ,
                xlabel = {$x$ [\SI{}{\milli \metre}]},
                %ylabel = {$y / \dy$},
                label style = {font = #2},
                tick label style = {font = #3},
                xtick = {15, 20, 25},
                ytick = {#4},
                %yticklabel = \empty,
                y dir = reverse,
                ]
                \addplot graphics [
                    xmin = 15,
                    xmax = 25,
                    ymin = #5,
                    ymax = #6
                ]{#7};
            \end{axis}
	\end{tikzpicture}
	}
}

