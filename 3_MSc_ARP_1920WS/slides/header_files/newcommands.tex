% "newcommand" collections
% important!!!! no numbers in the name of commands!!!!!!!!

%=====================  for math =======================%
% symbols
% scan positions: p
\newcommand{\pp}{\bm{p}}
\newcommand{\pphat}{\bm{\hat{p}}}
\newcommand{\ppdelta}{\Delta \pp}
\newcommand{\ppdeltahat}{\Delta \hat{\pp}}
% scan positions: x
\newcommand{\xdelta}{\Delta x}
\newcommand{\xhat}{\hat{x}}
\newcommand{\xopt}{\hat{x}_{\optimized}}
\newcommand{\xdeltaest}{\xdelta_{\estimated}}
\newcommand{\xdeltaopt}{\xdelta_{\optimized}}
\newcommand{\xdeltahat}{\hat{\xdelta}}
\newcommand{\xvec}{\bm{x}}
\newcommand{\xvechat}{\bm{\hat{x}}}
\newcommand{\xdeltavec}{\Delta \xvec}
\newcommand{\xtrackvec}{\xvechat_{\track}}
\newcommand{\xoptvec}{\xvechat_{\optimized}}
\newcommand{\xdeltaoptvec}{\Delta \xvechat_{\optimized}}
% scatter position
\newcommand{\scatterer}{\bm{s}}
% A-Scans
\newcommand{\ascan}{a}
\newcommand{\ascanvec}{\bm{a}}
\newcommand{\ascank}{\bm{a}_{k}}
\newcommand{\ascanvechat}{\bm{\hat{a}}}
\newcommand{\adothat}{\dot{\bm{\hat{a}}}}
\newcommand{\aopt}{\bm{\tilde{a}}}
\newcommand{\Ascan}{\bm{A}}
% Pulse
\newcommand{\pulse}{h}
\newcommand{\pulsevec}{\bm{h}}
\newcommand{\pulsevecdot}{\bm{h}'}
\newcommand{\SAFTcol}[2]{\pulsevec_{#1}^{( #2)}}
% SAFT matrix
\newcommand{\SAFT}{\bm{H}}
\newcommand{\SAFTp}{\SAFT (\pp)}
\newcommand{\SAFTx}{\SAFT (x)}
\newcommand{\SAFTk}{\SAFT_{k}}
\newcommand{\SAFThat}{\hat{\bm{H}}}
\newcommand{\SAFThatk}{\hat{\bm{H}}_{k}}
\newcommand{\SAFTapproxxvec}{\left( \SAFThat + \bm{E} \Jacobianhat \right)}

% Jacobian relevent
\newcommand{\Jacobian}{\bm{J}} 
\newcommand{\Jacobianhat}{\hat{\Jacobian}} 
\newcommand{\Jacobianpartial}{\bm{J}_{\SAFTcol}}
\newcommand{\Deriv}{\bm{D}} 
\newcommand{\Derivcol}{\bm{d}_{l}} 
% else
\newcommand{\reflectivity}{r}
\newcommand{\refcoeff}{\beta}
\newcommand{\defect}{\bm{b}} % should be modified? 
\newcommand{\defectsingle}{b} % should be modified? 
\newcommand{\defecthat}{\hat{\bm{b}}} % should be modified? 
\newcommand{\zvec}{\bm{z}} 
\newcommand{\noisevec}{\bm{n}} 
\newcommand{\fhatpartial}{\hat{f}_{i}}
\newcommand{\taux}{\tau_{l} (x)}
\newcommand{\Identity}{\bm{I}} 
\newcommand{\SEdag}{\SE^{\dagger}} 
\newcommand{\MSEdag}{\MSE^{\dagger}} 

% Math functions/operators
\newcommand{\norm}[1]{%
	\left\lVert#1\right\rVert
}
\newcommand{\vecoperator}[1]{%
	\vectorize \{ #1 \}
}
\newcommand{\diagoperator}[1]{%  
     \operatorname{diag} \{ #1 \}
}
\newcommand{\Real}{%  
     \operatorname{Re}
}
\newcommand{\Imag}{%  
     \operatorname{Im}
}



% some special characters
\newcommand{\ii}{\mathrm{i}}
\newcommand{\dd}{\mathrm{d}}
\newcommand{\ee}{\mathrm{e}}
\newcommand{\dirac}{\delta}

\newcommand{\RR}{{\mathbb{R}}}
\newcommand{\NN}{{\mathbb{N}}}
\newcommand{\CC}{{\mathbb{C}}}
\newcommand{\OO}{{\mathcal{O}}}


%%%%%%%%%%%%%%%%%%%%%%%%%%%%%%%%%%%%%%%%%%%%%%
%======================================== table macro ====%
%%%%%%%%%%%%%%%%%%%%%%%%%%%%%%%%%%%%%%%%%%%%%%
% table scaling
\newcommand{\inputTable}[2]{%  
	\resizebox{#1}{!}{%
	 \input{#2}
	}
}


%%%%%%%%%%%%%%%%%%%%%%%%%%%%%%%%%%%%%%%%%%%%%%
%========================================== for TikZ  ====%
%%%%%%%%%%%%%%%%%%%%%%%%%%%%%%%%%%%%%%%%%%%%%%
\newcommand{\Nx}{3}
\newcommand{\Ny}{2}
\newcommand{\ddx}{1.5cm}
\newcommand{\ddy}{\ddx}
\newcommand{\ddz}{1.5cm}
\newcommand{\rCircle}{0.11cm}
\newcommand{\rCircleCamera}{0.07cm}

%%% draw transducer %%%
%5 3D
\newcommand{\drawTransducer}[4]{ % scale,scaley, x, y in axis
	\draw[draw = none, fill = white] (axis cs: #3 - #1, #4 - #2, 0) -- (axis cs: #3 + #1, #4 - #2, 0) -- (axis cs: #3 + #1, #4 + #2, 0) --  (axis cs: #3 + #1, #4 + #2, -#1) -- (axis cs: #3 - #1, #4 + #2, -#1) -- (axis cs: #3 - #1, #4 - #2, -#1) -- (axis cs: #3 - #1, #4 - #2, 0);
	\draw[] (axis cs: #3 - #1, #4 - #2, 0) -- (axis cs: #3 + #1, #4 - #2, 0) -- (axis cs: #3 + #1, #4 - #2, -#1) -- (axis cs: #3 - #1, #4 - #2, -#1) -- (axis cs: #3 - #1, #4 - #2, 0);
	\draw[] (axis cs: #3 + #1, #4 - #2, 0) -- (axis cs: #3 + #1, #4 + #2, 0) -- (axis cs: #3 + #1, #4 +  #2, -#1) -- (axis cs: #3 - #1, #4 +  #2, -#1) -- (axis cs: #3 - #1, #4 - #2, -#1);
	\draw[] (axis cs: #3 + #1, #4 - #2, -#1) -- (axis cs: #3 + #1, #4 +  #2, -#1);
}

%% 2D
\newcommand{\drawTransducerTwoD}[2]{ % scale, x  1,3
	\draw[draw = none, fill = white] (axis cs: #2 - #1, 0) -- (axis cs: #2 + #1, 0) --  (axis cs: #2 + #1, -#1) -- (axis cs: #2 - #1, -#1) -- (axis cs: #2 - #1, 0);
	\draw[] (axis cs: #2 - #1, 0) -- (axis cs: #2 + #1, 0) -- (axis cs: #2 + #1, -#1) -- (axis cs: #2 - #1, -#1) -- (axis cs: #2 - #1, 0);
	\draw[] (axis cs: #2 + #1, 0) -- (axis cs: #2 + #1, -#1) -- (axis cs: #2 - #1, -#1);
}

%%% draw video camera %%%
%% 3D
\newcommand{\drawCamera}[6]{ %<rotation origin (x, y, z)>, <start x>, <start y>, <start z> in axis, <camera width [dx]>, <camera height [dz]>
	% base
	\draw[rotate around = {30 : (axis cs: #1)}] (axis cs: #2, #3, #4) rectangle (axis cs: #2 + #5, #3, #4 - #6) ;
	% origin of rotation
	\node[campoint] (rotationorg) at (axis cs: #1) {};
	% "trapezoid" part (tip of the camera)
	\draw[rotate around = {30 : (axis cs: #1)}] (axis cs: #2, #3, #4 - 0.25*#6) -- (axis cs: #2 - 0.5* #6, #3, #4) -- (axis cs: #2 - 0.5* #6, #3, #4-#6) -- (axis cs: #2, #3, #4 - 0.75*#6);	
	% cable
	\draw[rotate around = {30 : (axis cs: #1)}] (axis cs: #2 + #5, #3, #4 - 0.5*  #6) .. controls (axis cs: #2 + 1.5* #5, #3, #4 - 0.5*  #6) and (axis cs: #2 + #5, #3, - 0.5*  #6) .. (axis cs: #2 + 1.5* #5, #3, - #6);	
}

%% 2D
\newcommand{\drawCameraTwoD}[5]{ %<rotation origin (x, z)>, <start x>, <start y>, <start z> in axis, <camera width [dx]>, <camera height [dz]> 1, 2, 4, 5, 6 -> 4-> 3, 5->4, 6-> 5
	% base
	\draw[rotate around = {30 : (axis cs: #1)}] (axis cs: #2, #3) rectangle (axis cs: #2 + #4, #3 - #5) ;
	% origin of rotation
	\node[campoint] (rotationorg) at (axis cs: #1) {};
	% "trapezoid" part (tip of the camera)
	\draw[rotate around = {30 : (axis cs: #1)}] (axis cs: #2, #3 - 0.25*#5) -- (axis cs: #2 - 0.5* #5, #3) -- (axis cs: #2 - 0.5* #5, #3-#5) -- (axis cs: #2, #3 - 0.75*#5);	
	% cable
	\draw[rotate around = {30 : (axis cs: #1)}] (axis cs: #2 + #4, #3 - 0.5*  #5) .. controls (axis cs: #2 + 1.5* #4, #3 - 0.5*  #5) and (axis cs: #2 + #4, - 0.5*  #5) .. (axis cs: #2 + 1.5* #4, - #5);	
}


%%%%%%%%%%%%%%%%%%%%%%%%%%%%%%%%%%%%%%%%%%%%%%
%======================================= image macro ====%
%%%%%%%%%%%%%%%%%%%%%%%%%%%%%%%%%%%%%%%%%%%%%%
% TikZ scaling
\newcommand{\inputTikZ}[2]{%<scaling factor>, <name of the tex file>
     \scalebox{#1}{\input{#2}}  
}

% BRP: for error correction animation
\newcommand{\measanimate}[5]{ % <scale size>, <slide page for base pulse>, <slide page for highlighting the pulse 1mm away>, <slide page for highlighting the pulse 2.5mm away>, <slide page for highlighting the pulse 5mm away>
	\scalebox{#1}{
		\begin{tikzpicture}
			\begin{axis}[
					width=6cm, height=6.3cm,  at={(0.7cm,0.3cm)},
					ticks=none, axis lines = center, 
					xmin=-0.5, xmax=10.5, ymin=-0.45, ymax=10,
					xlabel={$x$}, ylabel={$t$},
					y dir=reverse,
					x label style={at={(axis cs: 10.5, 0)}, anchor= west},
					y label style={at={(axis cs: -0.6, 10)}, anchor = north},
			        ]
			        
			        % Defect
			        \node at (axis cs: 5, 4) {\pgftext{\includegraphics[scale=0.15]{images/defect}}};
					% Pulse
					\only<#2>{
					\input{figures/pytikz/1D/coordinates/pulse/pulse_1_blue.tex}
					\input{figures/pytikz/1D/coordinates/pulse/pulse_2_blue.tex}
					\input{figures/pytikz/1D/coordinates/pulse/pulse_3_blue.tex}
					\input{figures/pytikz/1D/coordinates/pulse/pulse_4_blue.tex}
					\input{figures/pytikz/1D/coordinates/pulse/pulse_5_blue.tex}
					\input{figures/pytikz/1D/coordinates/pulse/pulse_6_blue.tex}
					\input{figures/pytikz/1D/coordinates/pulse/pulse_7_blue.tex}
					\input{figures/pytikz/1D/coordinates/pulse/pulse_8_blue.tex}
					\input{figures/pytikz/1D/coordinates/pulse/pulse_9_blue.tex}
					}
					
					% Highlight
					\only<#3>{\input{figures/pytikz/1D/coordinates/pulse/pulse_4_highlight.tex}} %1 mm away
					\only<#4>{\input{figures/pytikz/1D/coordinates/pulse/pulse_3_highlight.tex}} % 2.5mm away
					\only<#5>{\input{figures/pytikz/1D/coordinates/pulse/pulse_1_highlight.tex}} % 5mm away
			
			\end{axis}			
		\end{tikzpicture}
	}
}

\newcommand{\curvefit}[5]{ % <scale size>, <slide page for base pulse>, <slide page for highlighting the pulse 2.5mm away>, <slide page for adding the defect position>, <slide page for adding the scan position>, 
	\scalebox{#1}{
		\begin{tikzpicture}
			\begin{axis}[
					width=6cm, height=6.3cm,  at={(0.7cm,0.3cm)},
					ticks=none, axis lines = center, 
					xmin=-1.5, xmax=10.5, ymin=-1.5, ymax=10,
					xlabel={$x$}, ylabel={$t$},
					y dir=reverse,
					x label style={at={(axis cs: 10.5, 0)}, anchor= west},
					y label style={at={(axis cs: -0.6, 10)}, anchor = north},
			        ]
			        
			        % Defect
			        \node at (axis cs: 5, 4) {\pgftext{\includegraphics[scale=0.15]{images/defect}}};
					% Pulse
					\only<#2>{
					\input{figures/pytikz/1D/coordinates/pulse/pulse_1_blue.tex}
					\input{figures/pytikz/1D/coordinates/pulse/pulse_2_blue.tex}
					\input{figures/pytikz/1D/coordinates/pulse/pulse_3_blue.tex}
					\input{figures/pytikz/1D/coordinates/pulse/pulse_4_blue.tex}
					\input{figures/pytikz/1D/coordinates/pulse/pulse_5_blue.tex}
					\input{figures/pytikz/1D/coordinates/pulse/pulse_6_blue.tex}
					\input{figures/pytikz/1D/coordinates/pulse/pulse_7_blue.tex}
					\input{figures/pytikz/1D/coordinates/pulse/pulse_8_blue.tex}
					\input{figures/pytikz/1D/coordinates/pulse/pulse_9_blue.tex}
					}
					
					% Highlight
					\only<#3>{\input{figures/pytikz/1D/coordinates/pulse/pulse_3_highlight.tex}} %2.5mm away
					% Node: defect positions
					\only<#4>{%
						\node at (axis cs: 5, 0) {$\shortmid$};
						\node at (axis cs: 5, -1) {$x_{\dist}$};
						\node at (axis cs: 0, 4) {$-$};
						\node at (axis cs: -1, 4) {$z_{\dist}$};
					}% 
					% Node: i-th scan
					\only<#5>{%
						% Defect
						\node at (axis cs: 3, 0) {$\shortmid$};
						\node at (axis cs: 3, -1) {$x_{k}$};
						\node at (axis cs: 0, 5) {$-$};
						\node at (axis cs: -1, 5) {$z_{k}$};
					}% 
			
			\end{axis}			
		\end{tikzpicture}
	}
}

\newcommand{\TLSanimate}[3]{ % <scale size>,  <slide page for z_def = 762dz>, <slide page for z_def = 1270dz>, 
	\scalebox{#1}{
		\begin{tikzpicture}
			\begin{axis}[
					width=6cm, height=6.3cm,  at={(0.7cm,0.3cm)},
					ticks=none, axis lines = center, 
					xmin=-1.5, xmax=6, ymin=-1.5, ymax=10,
					xlabel={$x$ [\SI{}{\milli \metre}] }, ylabel={$z$ [\SI{}{\milli \metre}] },
					y dir=reverse,
					x label style={at={(axis cs: 5.7, 0)}, anchor= south},
					y label style={at={(axis cs: -0.6, 10)}, anchor = north},
			        ]
			        
					% z_def = 762dz
					\only<#2>{%
						% Defect
						\node at (axis cs: 3, 4) {\pgftext{\includegraphics[scale=0.15]{images/defect}}};
						% Defect position ticks
						\node at (axis cs: 3, 0) {$\shortmid$};
						\node at (axis cs: 3, -1) {$20$};
						\node at (axis cs: 0, 4) {$-$};
						\node at (axis cs: -1, 4) {$30$};
					}% 
					
					% z_def = 1270dz
					\only<#3>{%
						% Defect
						\node at (axis cs: 3, 6.5) {\pgftext{\includegraphics[scale=0.15]{images/defect}}};
						% Defect position ticks
						\node at (axis cs: 3, 0) {$\shortmid$};
						\node at (axis cs: 3, -1) {$20$};
						\node at (axis cs: 0, 6.5) {$-$};
						\node at (axis cs: -1, 6.5) {$50$};
					}% 

			
			\end{axis}			
		\end{tikzpicture}
	}
}


%%%%%%%%%%%%%%%%%%%%%%%%%%%%%%%%%%%%%%%%%%%%%%%%
%===================== 1D Visualization ======================%
%%%%%%%%%%%%%%%%%%%%%%%%%%%%%%%%%%%%%%%%%%%%%%%%

% ME TLS: 762dz vs 1270dz
\newcommand{\meTLS}[7]{ % <scale size>, <label font size>, <tick font size>,  <slide page to pop up the ME 762dz>, <fname for ME 762dz>, <slide page to pop up the ME 1270dz>, <fname for ME 1270dz>
\scalebox{#1}{
	\pgfplotsset{
			xmin = -0.05, xmax=1.05, 
			ymin = -0.005, ymax=0.15, 
			scaled ticks=false % to avoid formtting with 10^-2 in y tick
	}
	\begin{tikzpicture}
            \begin{axis}[
                width = 10cm, 
            	   height = 6.5cm,
            	   grid=both,
    			   grid style={line width=.1pt, draw=gray!20},
                xlabel = {Tracking error $/ \lambda$ \cigray{(\SI{1.26}{\milli \metre})} },
                ylabel = {$\norm{\pp_{\dist} - \pphat_{\dist}}_{2}$ $/ \lambda$},
                label style = {font = #2},
                tick label style = {
                		/pgf/number format/fixed, % to avoid formtting with 10^-2 in y tick
                		font = #3
                },
                %y dir = reverse,
                %xtick = {0, 10, ..., 30}, %to customize the axis
                %xticklabel = {0, 5, 10, 15}
                extra y ticks={0.125},
                %extra y tick labels = {$\SE^{\dagger}_{\threshold}$}, 
                %extra y tick style={font = #2},
                ]
                \only<#4>{
	                	\input{#5}
	                	% y = 1.25lamda line
		             \addplot[gray, dashed, mark = , line width = 2pt] coordinates{
		            			(0.0, 0.1243)
		            			(1.0, 0.1243)
		            };
                }
                \only<#6>{
                		\input{#7}
                		% y = 0.096lamda line
%                		\addplot[gray, dashed, mark = , line width = 2pt] coordinates{
%		            			(0.0, 0.096)
%		            			(1.0, 0.096)
%		            };
                	} 
            \end{axis}
	\end{tikzpicture}
	}
}

%%%%% Result: evaluation
% API  results with animation
\newcommand{\resultAPIanimate}[6]{ %<scale size>, <font size>, <slide page for mark>, <mark coordinate for Reco_true>, <mark coordinate for Reco_track>, <mark coordinate for Reco_opt> 
\scalebox{#1}{
	\begin{tikzpicture}
            \begin{axis}[
                width = 7cm, height = 4cm,
                xlabel = {ROI depth [\SI{}{\milli \metre}]}, ylabel = {$\MAPI$},
                ymin= 16.5, ymax= 28,
                label style = {font = #2},
                tick label style = {font = #2},
                xtick = {20, 30, ..., 80},
                ytick = {18, 20, ..., 28},
                grid=both, grid style={line width=.1pt, draw=gray!20},
                legend style ={
                	at={(1.5, 0.8)},
                	nodes={scale=0.95, transform shape},
                	font = #2
                }
                ]
                \input{figures/pytikz/1D/api_true_depth.tex} % true
                \input{figures/pytikz/1D/api_track_depth.tex} % track
                \input{figures/pytikz/1D/api_opt_depth.tex} % opt
                
             % x = 20mm line
             \only<#3>{
             		% true
		         \addplot[tui_red, mark = star, mark size = 2pt] coordinates{
		          		#4
		            };
		         % track
		         \addplot[tui_red, mark = star , mark size = 2pt] coordinates{
		          		#5
		            };
		       % opt
		         \addplot[tui_red, mark = star , mark size = 2pt] coordinates{
		          		#6
		            };
		    }   
                
             % legend
             % To insert the legend title
		   	%\addlegendimage{empty legend} 
		   	% Legend entries  
		   	\addlegendentry{Reference}
             \addlegendentry{No correction}
             \addlegendentry{BEC}
             % Title
             %\addlegendentry{$|s_{x} - x|$}        
            \end{axis}
	\end{tikzpicture}
	}
}

% SE results 
\newcommand{\resultSE}[7]{ % <scale size>, <font size>, <xlabel>, <ymax>, <xtick>, <fname for track>, <fname for opt>
\scalebox{#1}{
	\begin{tikzpicture}
            \begin{axis}[
                width = 7cm, height = 4cm,
                xlabel = {#3}, ylabel = {$\MSEdag$},
                ymin= -0.04, ymax= #4,
                label style = {font = #2},
                tick label style = {font = #2},
                xtick = {#5},
                ytick = {0, 0.2, ..., #4}, 
                grid=both, grid style={line width=.1pt, draw=gray!20},
                legend style ={
                	at={(1.5, 0.8)},
                	nodes={scale=0.95, transform shape},
                	font = #2
                }
                ]
                \input{#6} % track
                \input{#7} % opt
                
             % legend
             % To insert the legend title
		   	%\addlegendimage{empty legend} 
		   	% Legend entries  
             \addlegendentry{No correction}
             \addlegendentry{BEC} 
             % Title
             %\addlegendentry{$|s_{x} - x|$}        
            \end{axis}
	\end{tikzpicture}
	}
}

% API results
\newcommand{\resultAPI}[9]{ % <scale size>, <font size>, <xlabel>, <ymax>, <xtick>, <ytick>, <fname for true>, <fname for track>, <fname for opt>
\scalebox{#1}{
	\begin{tikzpicture}
            \begin{axis}[
                width = 7cm, height = 4cm,
                xlabel = {#3}, ylabel = {$\MAPI$},
                ymin= 16.5, ymax= #4,
                label style = {font = #2},
                tick label style = {font = #2},
                xtick = {#5},
                ytick = {#6},
                grid=both, grid style={line width=.1pt, draw=gray!20},
                legend style ={
                	at={(1.5, 0.8)},
                	nodes={scale=0.95, transform shape},
                	font = #2
                }
                ]
                \input{#7} % true
                \input{#8} % track
                \input{#9} % opt
                
             % legend
             % To insert the legend title
		   	%\addlegendimage{empty legend} 
		   	% Legend entries  
		   	\addlegendentry{Reference}
             \addlegendentry{No correction}
             \addlegendentry{BEC}
             % Title
             %\addlegendentry{$|s_{x} - x|$}        
            \end{axis}
	\end{tikzpicture}
	}
}


% GCNR results
\newcommand{\resultGCNR}[9]{ % <scale size>, <font size>, <xlabel>, <ymin>, <xtick>, <ytick>, <fname for true>, <fname for track>, <fname for opt>
\scalebox{#1}{
	\begin{tikzpicture}
            \begin{axis}[
                width = 7cm, height = 4cm,
                xlabel = {#3}, ylabel = {$\MGCNR$},
                ymin= #4, ymax= 0.97,
                label style = {font = #2},
                tick label style = {font = #2},
                xtick = {#5},
                ytick = {#6},  % 1.01 = otherwise the tick does not show up 
                grid=both, grid style={line width=.1pt, draw=gray!20},
                legend style ={
                	at={(1.5, 0.8)},
                	nodes={scale=0.95, transform shape},
                	font = #2
                }
                ]
                \input{#7} % true
                \input{#8} % track
                \input{#9} % opt
                
             % legend
             % To insert the legend title
		   	%\addlegendimage{empty legend} 
		   	% Legend entries  
		   	\addlegendentry{Reference}
             \addlegendentry{No correction}
             \addlegendentry{BEC}
             % Title
             %\addlegendentry{$|s_{x} - x|$}        
            \end{axis}
	\end{tikzpicture}
	}
}
%%%%%%%%%%%%%%%%%%%%%%%%%%%%%%%%%%%%%%%%%%%%%%%%
%===================== 2D Visualization ======================%
%%%%%%%%%%%%%%%%%%%%%%%%%%%%%%%%%%%%%%%%%%%%%%%%

% cimg with both x- & y-labels
\newcommand{\cimgbothlabels}[4]{% <scale size>, <label font size>, <tick font size>, <png file name>
\scalebox{#1}{
	\begin{tikzpicture}
            \begin{axis}[
                enlargelimits = false,
                axis on top = true,
                axis equal image,
                point meta min = -1,   
                point meta max = 1,
                xlabel = {$x$ in \SI{}{\milli\meter}},
                ylabel = {$y$ in \SI{}{\milli\meter}},
                label style = {font = #2},
                tick label style = {font = #3},
                %y dir = reverse,
                %xtick = {0, 10, ..., 30}, %to customize the axis
                %xticklabel = {0, 5, 10, 15}
                ]
                \addplot graphics [
                    xmin = 0,
                    xmax = 20,
                    ymin = 0,
                    ymax = 20
                ]{#4};
            \end{axis}
	\end{tikzpicture}
	}
}




%%%%%%%%%%%%%%%%%%%%%%%%%%%%%%%%%%%%%%%%%%%%%%
%=================== 2D visualization ======================%
%%%%%%%%%%%%%%%%%%%%%%%%%%%%%%%%%%%%%%%%%%%%%%
% cimg for z_def = 20mm
\newcommand{\imgzdefshallow}[4]{% <scale size>, <label font size>, <tick font size>, <png file name>
\scalebox{#1}{
	\begin{tikzpicture}
            \begin{axis}[
            	   width = 5.5cm, 
            	   %height = 6.7cm,
                enlargelimits = false,
                axis on top = true,
                axis equal image,
                %unit vector ratio= 0.3 1, % change aspect ratio, one of them should be 1
                point meta min = -1,   
                point meta max = 1,
                xlabel = {$x$ [\SI{}{\milli \metre}]},
                ylabel = {$z$ [\SI{}{\milli \metre}]},
                label style = {font = #2},
                tick label style = {font = #3},
                xlabel style = {yshift = 0.3cm},
                ylabel style = {yshift = -0.3cm},
                y dir = reverse,
                xtick = {15, 20, 25},
                ytick = {15, 20, 25},
                ]
                \addplot graphics [
                    xmin = 15,
                    xmax = 25,
                    ymin = 15,
                    ymax = 25
                ]{#4};
            \end{axis}
	\end{tikzpicture}
	}
}

% cimg for z_def = 30mm
\newcommand{\imgzdefmiddle}[4]{% <scale size>, <label font size>, <tick font size>, <png file name>,
\scalebox{#1}{
	\begin{tikzpicture}
            \begin{axis}[
            	   width = 5.5cm, 
            	   %height = 6.7cm,
                enlargelimits = false,
                axis on top = true,
                axis equal image,
                %unit vector ratio= 0.3 1, % change aspect ratio, one of them should be 1
                point meta min = -1,   
                point meta max = 1,
                xlabel = {$x$ [\SI{}{\milli \metre}]},
                ylabel = {$z$ [\SI{}{\milli \metre}]},
                label style = {font = #2},
                tick label style = {font = #3},
                xlabel style = {yshift = 0.3cm},
                ylabel style = {yshift = -0.3cm},
                y dir = reverse,
                xtick = {15, 20, 25},
                ytick = {25, 30, 35},%21, 22.5, 24
                ]
                \addplot graphics [
                    xmin = 15,
                    xmax = 25,
                    ymin = 25,
                    ymax = 35
                ]{#4};
            \end{axis}
	\end{tikzpicture}
	}
}

% cimg for z_def = 50mm
\newcommand{\imgzdefdeep}[4]{% <scale size>, <label font size>, <tick font size>, <png file name>,
\scalebox{#1}{
	\begin{tikzpicture}
            \begin{axis}[
            	   width = 5.5cm, 
            	   %height = 6.7cm,
                enlargelimits = false,
                axis on top = true,
                axis equal image,
                %unit vector ratio= 0.3 1, % change aspect ratio, one of them should be 1
                point meta min = -1,   
                point meta max = 1,
                xlabel = {$x$ [\SI{}{\milli \metre}]},
                ylabel = {$z$ [\SI{}{\milli \metre}]},
                label style = {font = #2},
                tick label style = {font = #3},
                xlabel style = {yshift = 0.3cm},
                ylabel style = {yshift = -0.3cm},
                y dir = reverse,
                xtick = {15, 20, 25},
                ytick = {45, 50, 55},%21, 22.5, 24
                ]
                \addplot graphics [
                    xmin = 15,
                    xmax = 25,
                    ymin = 45,
                    ymax = 55
                ]{#4};
            \end{axis}
	\end{tikzpicture}
	}
}
