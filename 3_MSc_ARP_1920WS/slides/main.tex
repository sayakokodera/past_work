\documentclass[14pt,compress,aspectratio=169]{beamer} % selection of font size = 9, 10, 11, 12, 14, ....pt


% specify the language of the document here
%
\newcommand\setslidelanguage{english}
%\newcommand\setslidelanguage{german}

%
% this flag turns on a basic Fraunhofer CI overwrite
% this just changes colors atm, types and more convenient
% theme handling have to be added
%
%\newcommand\setslidetheme{tui} % - alpha
%\newcommand\setslidetheme{tuiiis} % - alpha
\newcommand\setslidetheme{tuiizfp} % - alpha
%\newcommand\setslidetheme{iis} % - alpha
%\newcommand\setslidetheme{iistui} % - alpha
%\newcommand\setslidetheme{izfp} % - alpha
%\newcommand\setslidetheme{izfptui} % - alpha


%
% load header with all includes
%
%%% input encoding, font encoding
\usepackage[T1]{fontenc}
\usepackage[utf8]{inputenc}
\usepackage{lmodern}


%%% beamerstuff - requirements
\usepackage{ifthen}
\usepackage[absolute,overlay]{textpos}
\setlength{\TPHorizModule}{1cm}
\setlength{\TPVertModule}{1cm}


%%% presentation style stuff
% load a lot of custom macros and sets the
% beamer theme
% when new logos and stuff are required
% look there
%%% presentation style stuff


%%%%%%%%%%%%%%%%%%%%%%%%%%%%%%%%%%%%%%%%%%%%%%%%%%%%%%%%%%%%%%%%%%%%%%%%%%%%%%%%%%%%%%%%%%%%%%%%%%%%%%%%%%%%%%%%%%%%%%%%%%%%%%%%%%%%%
% general stuff

% thanks to https://tex.stackexchange.com/questions/2541/beamer-frame-numbering-in-appendix
% this helps us to organize framenumbering s.t. they do not increase for the appendix
\newcommand{\backupbegin}{
   \newcounter{framenumberappendix}
   \setcounter{framenumberappendix}{\value{framenumber}}
}
\newcommand{\backupend}{
   \addtocounter{framenumberappendix}{-\value{framenumber}}
   \addtocounter{framenumber}{\value{framenumberappendix}} 
}


% this is required to omit problems with tikz-externalize
% as I did all that footer/header stuff using tikz
% so we have to prevent tikz externalize from
% externalizing these
% otherwise framecounters and sections will screw up
\newcommand{\donotexternalize}[1]{
\tikzset{external/export=false}
#1
\tikzset{external/export=true}
}


%%%%%%%%%%%%%%%%%%%%%%%%%%%%%%%%%%%%%%%%%%%%%%%%%%%%%%%%%%%%%%%%%%%%%%%%%%%%%%%%%%%%%%%%%%%%%%%%%%%%%%%%%%%%%%%%%%%%%%%%%%%%%%%%%%%%%
% TU Ilmenau stuff

%% Logos
\newcommand\logotuiwhite{\includegraphics[width=1.86cm]{logos/tu_bw_white.pdf}}
\newcommand\logotuicolor{\includegraphics[width=1.86cm]{logos/tu_col_green.pdf}}

\newcommand\settuitextmacros{
	% Sets macros for text formatting
	\newcommand<>{\cihighlight}[1]{\textcolor##2{tui_orange_dark}{##1}}
	\newcommand<>{\cigray}[1]{\textcolor##2{tui_gray_dark}{##1}}
	\newcommand<>{\cihide}[1]{\textcolor##2{white}{##1}}
}


%% Style
\newcommand{\setTUIStyleTwoLogos}{
	\mode<presentation>
		{
			\usepackage{theme/beamerthemetui}
			\setbeamercovered{transparent}
		}
	}
	
	
	
% for enabling 16:9
\newlength{\ruleswidth}
\setlength{\ruleswidth}{\dimexpr(\paperwidth-20mm)\relax}
	
	
%%%%%%%%%%%%%%%%%%%%%%%%%%%%%%%%%%%%%%%%%%%%%%%%%%%%%%%%%%%%%%%%%%%%%%%%%%%%%%%%%%%%%%%%%%%%%%%%%%%%%%%%%%%%%%%%%%%%%%%%%%%%%%%%%%%%%
% Fraunhofer stuff


%% Logos
\newcommand\logoizfpwhite{\raisebox{-6.68mm}{\includegraphics[width=1.86cm]{logos/izfp_85mm_p334_w.pdf}}}
\newcommand\logoizfpcolor{\raisebox{-6.68mm}{\includegraphics[width=1.86cm]{logos/izfp_85mm_p334-eps-converted-to.pdf}}}
\newcommand\logoiiswhite{\raisebox{-6.72mm}{\includegraphics[width=1.86cm]{logos/iis_85mm_p334.pdf}}}
\newcommand\logoiiscolor{\raisebox{-6.72mm}{\includegraphics[width=1.86cm]{logos/iis_85mm_p334.pdf}}}


\newcommand\setfhitextmacros{
	% Sets macros for text formatting
	\newcommand<>{\cihighlight}[1]{\textcolor##2{fraunhofer_orange}{##1}}
	\newcommand<>{\cigray}[1]{\textcolor##2{fraunhofer_silver!70!black}{##1}}
	\newcommand<>{\cihide}[1]{\textcolor##2{white}{##1}}
	% taken from https://tex.stackexchange.com/questions/23034/how-to-get-larger-item-symbols-for-some-lists-in-a-beamer-	
	% presentation and
	% https://tex.stackexchange.com/questions/87133/changing-the-color-of-itemize-item-in-beamer
	\setbeamertemplate{itemize subsubitem}{\tiny\raise0.5pt\hbox{\color{fraunhofer_silver}$ \blacksquare$}}
}

\newcommand{\setFHIStyleTwoLogos}{
	\mode<presentation>
		{
			\usepackage{theme/beamerthemefhi}
			\setbeamercovered{transparent}
		}
	% Big TODO: Frutiger font
}




%%%%%%%%%%%%%%%%%%%%%%%%%%%%%%%%%%%%%%%%%%%%%%%%%%%%%%%%%%%%%%%%%%%%%%%%%%%%%%%%%%%%%%%%%%%%%%%%%%%%%%%%%%%%%%%%%%%%%%%%%%%%%%%%%%%%%
% set everything up using the macros
%% TUI
\ifthenelse{\equal{\setslidetheme}{tui}}
{
		\newcommand\setfirstlogo{\logotuiwhite}
		\newcommand\setsecondlogo{}
		
		\setTUIStyleTwoLogos
		
		\settuitextmacros
}{}
\ifthenelse{\equal{\setslidetheme}{tuiiis}}
{
		\newcommand\setfirstlogo{\logotuiwhite}
		\newcommand\setsecondlogo{\logoiiswhite}
		
		\setTUIStyleTwoLogos
		
		\settuitextmacros
}{}
\ifthenelse{\equal{\setslidetheme}{tuiizfp}}
{
		\newcommand\setfirstlogo{\logotuiwhite}
		\newcommand\setsecondlogo{\logoizfpwhite}
		
		\setTUIStyleTwoLogos
		
		\settuitextmacros
}{}


%% Fraunhofer 
\ifthenelse{ \equal{\setslidetheme}{iis}}
{
		\newcommand\setfirstlogo{\logoiiscolor}
		\newcommand\setsecondlogo{}

		\setFHIStyleTwoLogos
		
		\setfhitextmacros
}{}
\ifthenelse{ \equal{\setslidetheme}{izfp}}
{
		\newcommand\setfirstlogo{\logoizfpcolor}
		\newcommand\setsecondlogo{}
		
		\setFHIStyleTwoLogos
		
		\setfhitextmacros
}{}
\ifthenelse{ \equal{\setslidetheme}{iistui}}
{
		\newcommand\setfirstlogo{\logoiiscolor}
		\newcommand\setsecondlogo{\logotuicolor}

		\setFHIStyleTwoLogos
		
		\setfhitextmacros
}{}
\ifthenelse{ \equal{\setslidetheme}{izfptui}}
{
		\newcommand\setfirstlogo{\logoizfpcolor}
		\newcommand\setsecondlogo{\logotuicolor}

		\setFHIStyleTwoLogos
		
		\setfhitextmacros
}{}




%%% language settings
\ifthenelse{ \equal{\setslidelanguage}{german}}
	{	\usepackage[ngerman]{babel} }
	{	\usepackage[english]{babel} }


%%% general (useful) packages
% math
\usepackage{amsmath}
\usepackage{bm}
\usepackage{mathtools} % For \coloneqq

% graphics
\usepackage{pgfplots}
\pgfplotsset{compat=newest}
\usepackage{animate}
\usepackage{tikz}
\usepackage{multido}
\usetikzlibrary{patterns}
\usetikzlibrary{matrix,arrows,shapes,trees,fit, decorations, spy}
\usetikzlibrary{calc,positioning,scopes,backgrounds}
\usetikzlibrary{tikzmark,decorations.pathreplacing,calligraphy}
\usepgfplotslibrary{colormaps}
\usetikzlibrary{external} % most useful, but requires --shell-escape
\usetikzlibrary{petri} %for tokens in the block diagram for iterative GD
% this can be used if you want to use tikzexternalize
% it precompiles your plots to a subfolder called
% figures/tikz-ext/ and names them according to
% the section 
% see https://tex.stackexchange.com/questions/177292/chapter-and-figure-numbers-in-tikz-externalize-prefix
%
%
%\tikzexternalize[
%    prefix=figures/tikz-ext/,
%    figure name=plot_sec\thesubsection_no,
%]

%% for putting figures at the specific locations
\usepackage[absolute,overlay]{textpos}
\setlength{\TPHorizModule}{1mm}
\setlength{\TPVertModule}{1mm}

% for checkmark in conclusion
\usepackage{bbding}


% settings and default macros
\usepackage[load-configurations=abbreviations,load-configurations=binary,binary-units=true]{siunitx}	%korrektes Setzen von SI-Einheiten
\DeclareSIUnit{\mms}{\milli\squaremetre}
\DeclareSIUnit{\inch}{in}
\DeclareSIUnit{\inchs}{in\squared}
\DeclareSIUnit{\mil}{mil}
\DeclareSIUnit{\Msps}{Msps}
\DeclareSIUnit{\Mbps}{Mbps}
\DeclareSIUnit{\LSB}{LSB}
\DeclareSIUnit{\pFS}{\percent FS}
\DeclareSIUnit{\dBc}{\deci\bel c}
\DeclareSIUnit{\dBm}{\deci\bel m}
\DeclareSIUnit{\dBFS}{\deci\bel FS}
\DeclareSIUnit{\dB}{\deci\bel}
\DeclareSIUnit{\dBi}{\deci\bel i}
\DeclareSIUnit{\hex}{0x}
\DeclareSIUnit{\vp}{\volt_{\text{p}}}
\DeclareSIUnit{\vpp}{\volt_{\text{pp}}}
\DeclareSIUnit{\kb}{\kilo\bit}
\DeclareSIUnit{\kB}{\kilo\byte}
\DeclareSIUnit{\MB}{\mega\byte}




%
% your settings and includes go here
%
%%%% bibliography %%%%%
\usepackage{url}
\usepackage{natbib} % required for the citation example
\usepackage{bibentry} % required for the citation example
%
\nobibliography* % required for the citation example
\bibliographystyle{plain}

% "newcommand" collections
% !!! In newcommands: numbers cannot be used for the command names!!!
%=====================  for general =======================%
\renewcommand\thesubsection{\thesection.\Alph{subsection}}

\newcommand{\ToDo}[1]{%
	\textcolor{red}{#1}
}

%=====================  for math =======================%
% Math operands
% l2 norms
\newcommand{\norm}[1]{%
	\left\lVert#1\right\rVert
}
% Transpose
\newcommand{\TP}[1]{%
	#1^{\T}
}
% Vectorization
\newcommand{\vectorize}[1]{%
	\operatorname{vec} \{ #1 \}
}
% Diagonal matrix
\newcommand{\diag}[1]{%
	\operatorname{diag} \{ #1 \}
}
% Real part
\newcommand{\Real}{%  
     \operatorname{Re}
}
 % Imaginary part
\newcommand{\Imag}{%  
     \operatorname{Im}
}
% Column space
\newcommand{\Col}[1]{
	\operatorname{col} \{ #1 \}
}
% Nullspace 
\newcommand{\Null}[1]{
	\operatorname{null} \{ #1 \}
} 


% symbols
% General
\newcommand{\Identity}[1]{\bm{I}_{#1}} % Identity matrix
\newcommand{\RR}{{\mathbb{R}}}
\newcommand{\CC}{{\mathbb{C}}}

% TLS relevant
\newcommand{\Mone}{\bm{\Omega}}
\newcommand{\MoneDelta}{\Delta \Mone}
\newcommand{\MonePerturb}{\Mone + \MoneDelta}
\newcommand{\Mtwo}{\bm{\Psi}}
\newcommand{\MtwoDelta}{\Delta \Mtwo}
\newcommand{\MtwoPerturb}{\Mtwo + \MtwoDelta}
\newcommand{\Mthree}{\bm{\Upsilon}}
\newcommand{\Mfour}{\bm{T}}
\newcommand{\UU}{\bm{U}} % SVD: U
\newcommand{\VV}{\bm{V}} % SVD: V
\newcommand{\SSigma}{\bm{\Sigma}} % SVD: Sigma
% Newton relevant
\newcommand{\gradf}{g}
\newcommand{\Hess}{\bm{H_{f}}}
\newcommand{\dd}[1]{\bm{d}_{#1}}
\newcommand{\iter}[1]{\xx_{#1}}

% scan positions
% p = scalar, \pp = vector
\newcommand{\pp}{\bm{p}}
\newcommand{\pphat}{\bm{\hat{p}}}
\newcommand{\ppdelta}{\Delta \pp}
% x = scalar, \xx = vector
\newcommand{\xdelta}[1]{\Delta x_{#1}}
\newcommand{\xhat}[1]{\hat{x}_{#1}}
\newcommand{\xopt}{\xhat_{\optimized}}
\newcommand{\xx}{\bm{x}}
\newcommand{\xxdelta}{\Delta \xx}
\newcommand{\xxhat}{\hat{\xx}}
\newcommand{\xxopt}{\xxhat_{\optimized}}
\newcommand{\xxdeltahat}{\Delta \xxhat}
\newcommand{\xxdeltaopt}{\xxdeltahat_{\optimized}}
% scatter position
\newcommand{\scatterer}{\bm{s}}
% A-Scans
\newcommand{\ascansig}{a}
\newcommand{\ascan}[1]{\bm{a}_{#1}} % with \xx
\newcommand{\ascanhat}[1]{\hat{\ascan}_{#1}} % with \xxhat
\newcommand{\Ascan}{\bm{A}} % with \xx
\newcommand{\Ascanhat}{\hat{\Ascan}} % with \xxhat
% Pulse, SAFT matrix, spatial approximation
\newcommand{\pulsesig}{h}
\newcommand{\pulse}{\bm{h}}
\newcommand{\pulsehat}{\hat{\pulse}}
\newcommand{\pulsederiv}{\bm{j}}
\newcommand{\pulsederivhat}{\hat{\pulsederiv}}
\newcommand{\SAFTcol}[2]{\pulse_{#1}^{(#2)}}
\newcommand{\SAFThatcol}[2]{\pulsehat_{#1}^{(#2)}}
\newcommand{\SAFT}{\bm{H}}
\newcommand{\SAFThat}{\hat{\SAFT}}
\newcommand{\SAFTdot}{\bm{J}}
\newcommand{\SAFTdotcol}[2]{\pulsederiv_{#1}^{(#2)}}
\newcommand{\SAFTdothat}{\hat{\SAFTdot}}
\newcommand{\SAFTdothatcol}[2]{\pulsederivhat_{#1}^{(#2)}}
\newcommand{\Error}{\bm{E}}
\newcommand{\ErrorDef}{\diag{\xxdelta} \otimes \Identity{\M}}
% Defect map
\newcommand{\defect}{\bm{b}} 
\newcommand{\defecthat}{\hat{\defect}}
% Reco
\newcommand{\Reco}{\bm{B}}  
\newcommand{\Recohat}{\hat{\Reco}}  

% TLS curve fit
\newcommand{\zz}{\bm{z}}
\newcommand{\zzperturb}{\zz + \Delta \zz}
\newcommand{\XX}{\bm{X}}
\newcommand{\XXperturb}{\XX + \Delta \XX}
\newcommand{\ww}{\bm{w}}
% Iterative EC via Newton
\newcommand{\vonebase}{\alpha}
\newcommand{\voneall}{\bm{\tilde{\vonebase}}}
\newcommand{\vonepart}[1]{\bm{\vonebase}_{#1}}
\newcommand{\vtwobase}{j}
\newcommand{\vtwoall}{\bm{\tilde{\vtwobase}}}
\newcommand{\vtwopart}[1]{\bm{\vtwobase}_{#1}}
% Metrics
\newcommand{\SEdag}{\SE^{\dagger}} 
\newcommand{\MSEdag}{\MSE^{\dagger}} 

% else
\newcommand{\refcoeff}{\beta}
\newcommand{\noisevec}{\bm{n}} 


% Dimensions
\newcommand{\N}{N}
\newcommand{\M}{M}
\newcommand{\K}{K}
\newcommand{\LL}{L}
\newcommand{\I}{I}

%=====================  for roman numbers =======================%

\newcommand{\rom}[1]{\uppercase\expandafter{\romannumeral #1\relax}}
% from https://tex.stackexchange.com/questions/23487/how-can-i-get-roman-numerals-in-text


%=====================  for tables =======================%
\newcommand{\inputTable}[4]{ % <scaling factor>, <file name for the table>, <caption>, <label>
	%\resizebox{#1}{!}{ -> only works in the beamer setting?
		\begin{table}
		\begin{center}
			\input{#2}
			\caption{#3}
			\label{#4}
		\end{center}
		\end{table}
	%}
}

%=====================  for TikZ =======================%
\newcommand{\inputTikZ}[2]{%  
     \scalebox{#1}{\input{#2}}  
}
%%%%%%%%%%%%%%%%%%%%%%%%%%%%%%%%%%%%%%%%%%%%%
%=================  for data visualization ===================%
%%%%%%%%%%%%%%%%%%%%%%%%%%%%%%%%%%%%%%%%%%%%%

%%%%%%%%%%%%%%%%%%%%%%%%%%%%%%%%%%%%%%%%%%%%%%
%=================== 1D visualization ======================%
%%%%%%%%%%%%%%%%%%%%%%%%%%%%%%%%%%%%%%%%%%%%%%

% SE tolerance
\newcommand{\resultSEtolerance}[6]{ % <scale size>, <font size>,  <fname for track 50mm>, <fname for opt 50mm>,  <fname for track 30mm>, <fname for opt 30mm>
\scalebox{#1}{
	\begin{tikzpicture}
            \begin{axis}[
                width = 7cm, height = 4cm,
                xlabel = {Tracking error / $\lambda$}, ylabel = {$\MSEdag$},
                ymin= -0.04, ymax= 0.65,
                label style = {font = #2},
                tick label style = {font = #2},
                xtick = {0, 0.2, ..., 1.01},
                ytick = {0, 0.2, ..., 0.6}, 
                grid=both, grid style={line width=.1pt, draw=gray!20},
                legend style ={
                	at={(1.45, 0.7)},
                	nodes={scale=0.95, transform shape},
                	font = #2
                }
                ]
                \input{#3} % track 50mm
                \input{#4} % opt 50mm
                \input{#5} % track 30mm
                \input{#6} % opt 30mm
                
             % legend
             % To insert the legend title
		   	%\addlegendimage{empty legend} 
		   	% Legend entries  
             \addlegendentry{No correction}
             \addlegendentry{BEC} 
             % Title
             %\addlegendentry{$|s_{x} - x|$}        
            \end{axis}
	\end{tikzpicture}
	}
}

% SE depth 
\newcommand{\resultSE}[7]{ % <scale size>, <font size>, <xlabel>, <ymax>, <xtick>, <fname for track>, <fname for opt>
\scalebox{#1}{
	\begin{tikzpicture}
            \begin{axis}[
                width = 7cm, height = 4cm,
                xlabel = {#3}, ylabel = {$\MSEdag$},
                ymin= -0.04, ymax= #4,
                label style = {font = #2},
                tick label style = {font = #2},
                xtick = {#5},
                ytick = {0, 0.2, ..., #4}, 
                grid=both, grid style={line width=.1pt, draw=gray!20},
                legend style ={
                	at={(1.45, 0.7)},
                	nodes={scale=0.95, transform shape},
                	font = #2
                }
                ]
                \input{#6} % track
                \input{#7} % opt
                
             % legend
             % To insert the legend title
		   	%\addlegendimage{empty legend} 
		   	% Legend entries  
             \addlegendentry{No correction}
             \addlegendentry{BEC} 
             % Title
             %\addlegendentry{$|s_{x} - x|$}        
            \end{axis}
	\end{tikzpicture}
	}
}

% API
\newcommand{\resultAPI}[9]{ % <scale size>, <font size>, <xlabel>, <ymax>, <xtick>, <ytick>, <fname for true>, <fname for track>, <fname for opt>
\scalebox{#1}{
	\begin{tikzpicture}
            \begin{axis}[
                width = 7cm, height = 4cm,
                xlabel = {#3}, ylabel = {$\MAPI$},
                ymin= 16.5, ymax= #4,
                label style = {font = #2},
                tick label style = {font = #2},
                xtick = {#5},
                ytick = {#6},
                grid=both, grid style={line width=.1pt, draw=gray!20},
                legend style ={
                	at={(1.45, 0.8)},
                	nodes={scale=0.95, transform shape},
                	font = #2
                }
                ]
                \input{#7} % true
                \input{#8} % track
                \input{#9} % opt
                
             % legend
             % To insert the legend title
		   	%\addlegendimage{empty legend} 
		   	% Legend entries  
		   	\addlegendentry{Reference}
             \addlegendentry{No correction}
             \addlegendentry{BEC}
             % Title
             %\addlegendentry{$|s_{x} - x|$}        
            \end{axis}
	\end{tikzpicture}
	}
}


% GCNR
\newcommand{\resultGCNR}[9]{ % <scale size>, <font size>, <xlabel>, <ymin>, <xtick>, <ytick>, <fname for true>, <fname for track>, <fname for opt>
\scalebox{#1}{
	\begin{tikzpicture}
            \begin{axis}[
                width = 7cm, height = 4cm,
                xlabel = {#3}, ylabel = {$\MGCNR$},
                ymin= #4, ymax= 0.97,
                label style = {font = #2},
                tick label style = {font = #2},
                xtick = {#5},
                ytick = {#6},  % 1.01 = otherwise the tick does not show up 
                grid=both, grid style={line width=.1pt, draw=gray!20},
                legend style ={
                	at={(1.45, 0.8)},
                	nodes={scale=0.95, transform shape},
                	font = #2
                }
                ]
                \input{#7} % true
                \input{#8} % track
                \input{#9} % opt
                
             % legend
             % To insert the legend title
		   	%\addlegendimage{empty legend} 
		   	% Legend entries  
		   	\addlegendentry{Reference}
             \addlegendentry{No correction}
             \addlegendentry{BEC}
             % Title
             %\addlegendentry{$|s_{x} - x|$}        
            \end{axis}
	\end{tikzpicture}
	}
}

% GD SE
\newcommand{\gdse}[8]{ %  <scale size>, <label font size>, <tick font size>, <fname for 0.5 lambda>, <fname for 1mm>, <fname for 2.5mm>, <fname for 5mm>, <fname for 7.5mm>
\scalebox{#1}{
	\begin{tikzpicture}
            \begin{axis}[
                width = 7cm, 
            	   height = 4cm,
            	   xmin = -2.1,
            	   xmax = 2.1,
            	   ymin = -0.1,
            	   ymax = 1.1,
                xlabel = {$\xdelta / \lambda$},
                ylabel = {$\SEdag$},
                label style = {font = #2},
                tick label style = {font = #3},
                %y dir = reverse,
                %xtick = {0, 10, ..., 30}, %to customize the axis
                extra x ticks={0.8}, 
                extra x tick style={font = #3, tui_orange},%yshift={-1.2em}
                %xticklabel = {0, 5, 10, 15}
                legend style ={
                	at={(1.4, 0.9)},
                	nodes={scale=0.95, transform shape},
                	font = #3
                }
                ]
                % Reverse the input order, so that 7.5mm away is sent to background
                \input{#8} %7.5mm
                \input{#7} %5mm
                \input{#6} %2.5mm
                \input{#5} %1mm
                \input{#4} %0.5 lambda
                
             % legend
             % To insert the legend title
		   	\addlegendimage{empty legend} 
		   	% Legend entries  
             \addlegendentry{\SI{7.5}{\milli\metre}}
             \addlegendentry{\SI{5}{\milli\metre}}
             \addlegendentry{\SI{2.5}{\milli\metre}} %{$1.98 \lambda$}
             \addlegendentry{\SI{1}{\milli\metre}} %{$0.8 \lambda$}
             \addlegendentry{\SI{0.63}{\milli\metre}} %{$0.5 \lambda$}
             % Title
             \addlegendentry{$| s_{x} - x |$}
             
%             % x = 0.8 line
%	            \addplot[gray, dashed, line width = 1pt, mark = ] coordinates{
%	            			(0.8, 0.2)
%	            			(0.8, -0.1)
%            		};
                         
            \end{axis}
	\end{tikzpicture}
	}
}

%%%%%%%%%%%%%%%%%%%%%%%%%%%%%%%%%%%%%%%%%%%%%%
%=================== 2D visualization ======================%
%%%%%%%%%%%%%%%%%%%%%%%%%%%%%%%%%%%%%%%%%%%%%%
% image with both x- & y-labels
\newcommand{\imgbothlabels}[7]{% <scale size>, <label font size>, <tick font size>,<ytick>, <ymin>, <ymax>, <png file name>
\scalebox{#1}{
	\begin{tikzpicture}
            \begin{axis}[
                enlargelimits = false,
                axis on top = true,
                axis equal image,
                unit vector ratio= 0.5 1, % change aspect ratio, one of them should be 1
                point meta min = -1,   
                point meta max = 1,
                xlabel = {$x$ [\SI{}{\milli \metre}]},
                ylabel = {$z$ [\SI{}{\milli \metre}]},
                label style = {font = #2},
                tick label style = {font = #3},
                y dir = reverse,
                xtick = {15, 20, 25},
                ytick = {#4},
                ]
                \addplot graphics [
                    xmin = 15,
                    xmax = 25,
                    ymin = #5,
                    ymax = #6
                ]{#7};
            \end{axis}
	\end{tikzpicture}
	}
}
            
% img with only x-label
\newcommand{\imgxlabel}[7]{% <scale size>, <label font size>, <tick font size>, <ytick>, <ymin>, <ymax>, <png file name>
\scalebox{#1}{
	\begin{tikzpicture}
            \begin{axis}[
                enlargelimits = false,
                axis on top = true,
                axis equal image,
                unit vector ratio= 0.5 1, % change aspect ratio, one of them should be 1
                point meta min = -1,   
                point meta max = 1,
                xlabel = {$x$ [\SI{}{\milli \metre}]},
                %ylabel = {$y / \dy$},
                label style = {font = #2},
                tick label style = {font = #3},
                xtick = {15, 20, 25},
                ytick = {#4},
                %yticklabel = \empty,
                y dir = reverse,
                ]
                \addplot graphics [
                    xmin = 15,
                    xmax = 25,
                    ymin = #5,
                    ymax = #6
                ]{#7};
            \end{axis}
	\end{tikzpicture}
	}
}



% img with only x-label and cmap
\newcommand{\imgxlabelwithcmap}[7]{% <scale size>, <label font size>, <tick font size>, <ytick>, <ymin>, <ymax>, <png file name>
\scalebox{#1}{
	\begin{tikzpicture}
            \begin{axis}[
                enlargelimits = false,
                axis on top = true,
                axis equal image,
                unit vector ratio= 0.5 1, % change aspect ratio, one of them should be 1
                point meta min = -1,   
                point meta max = 1,
                colorbar,
                colormap = {mymap}{rgb(0.0pt) = (0, 0.22, 0.39) ; rgb(0.43pt) = (0.91, 0.93, 0.96) ; rgb(0.5pt) = (0.97, 0.97, 0.98) ; rgb(0.57pt) = (0.99, 0.93, 0.87) ; rgb(1.0pt) = (0.94, 0.49, 0) ; } ,
                xlabel = {$x$ [\SI{}{\milli \metre}]},
                %ylabel = {$y / \dy$},
                label style = {font = #2},
                tick label style = {font = #3},
                xtick = {15, 20, 25},
                ytick = {#4},
                %yticklabel = \empty,
                y dir = reverse,
                ]
                \addplot graphics [
                    xmin = 15,
                    xmax = 25,
                    ymin = #5,
                    ymax = #6
                ]{#7};
            \end{axis}
	\end{tikzpicture}
	}
}


% set color

\usepackage{color}
 
\definecolor{fri_gray}{rgb}{0.8, 0.8, 0.8}
\definecolor{fri_green_light}{rgb}{0.76, 0.95, 0.81}
\definecolor{fri_green}{rgb}{0.51, 0.87, 0.78}
\definecolor{tui_orange}{rgb}{0.94, 0.49, 0}
\definecolor{tui_blue}{rgb}{0, 0.22, 0.39}

\definecolor{box_white}{cmyk}{0.0361,0.0251,0.0166,0}
\definecolor{text_black}{cmyk}{0.7979,0.7417,0.6916,0.6554}
\definecolor{tui_orange_dark}{cmyk}{0,0.6,1,0}
\definecolor{tui_orange_light}{cmyk}{0.0000,0.0876, 0.1474, 0.0157}
\definecolor{tui_green_dark}{cmyk}{1,0,0.5,0.2}
\definecolor{tui_green_light}{cmyk}{0.0576,0.0041, 0.0000, 0.0471}
\definecolor{tui_blue_dark}{cmyk}{1.0000,0.5000,0.0000,0.6000}
\definecolor{tui_blue_light}{cmyk}{0.0920,0.0440,0.0000,0.0196}
\definecolor{tui_red_dark}{cmyk}{0.0000,1.0000,1.0000,0.2000}
\definecolor{tui_red_light}{cmyk}{0.0000,0.1107,0.1107,0.0078}

%%%% TikZ setup %%%%

% for SmartInspect
\tikzstyle{scanpath} = [tui_blue, dotted, ->, shorten >=1mm, shorten <=1mm,]
\tikzstyle{scanpoint} = [circle, draw, black, inner sep = \rCircle, fill = black]
\tikzstyle{campoint} = [circle, draw, black, inner sep = \rCircleCamera, fill = gray]

% for synthetic aperture
\tikzstyle{griddot} = [circle, draw, black, inner sep = 0.01cm, fill = black]

% for drawing block diagram
\tikzstyle{line} = [draw, thick]
\tikzstyle{line1} = [draw, thick, ->]
\tikzstyle{line3} = [draw, gray, thick, dashed, ->]
\tikzstyle{line4} = [draw, tui_blue, very thick, dashed, -]
% For entire system 
\tikzstyle{largebox} = [rectangle, draw, dashed, very thick, 
text width = 9.6cm,  minimum height = 3.3cm
]
\tikzstyle{block1} = [rectangle, draw,  
	text width=3cm, text centered, minimum height = 2cm
	]
\tikzstyle{block2} = [rectangle, draw,  
	text width=4.5cm, text centered, minimum height = 2cm
	]
\tikzstyle{junction} = [circle, draw, fill=black, minimum size = 1pt, scale = 0.5]
% For FVnet
\tikzstyle{block3} = [rectangle, draw,  
text width=2cm, text centered, minimum height = 1cm
]
\tikzstyle{block4} = [rectangle, draw,  
text width=1.3cm, text centered, minimum height = 3cm
]
\tikzstyle{largecircle} = [circle, draw,  
text width=1.5cm, text centered, minimum size = 1pt
]
\tikzstyle{diamblock} = [diamond, draw,  
text width=1.2cm, text centered
]
\tikzstyle{largebox2} = [rectangle, draw, fri_gray_dark!70, very thick, 
text width = 5.3cm,  minimum height = 4.3cm
]
\tikzstyle{largebox3} = [rectangle, draw, fri_gray_dark!70, very thick, 
text width = 10.8cm,  minimum height = 4.3cm
]
\tikzstyle{largebox4} = [rectangle, draw, dashed, fri_gray_dark!70, very thick, 
text width = 7.5cm,  minimum height = 3.2cm
]


% for markers in RMSE
\pgfplotsset{
  every axis plot post/.append style={
    every mark/.append style={line width=3pt}
  }
}
% for posscan simulation flow
\tikzstyle{graydashed} = [draw, gray, thick, dashed]

% math operators

\DeclareMathOperator{\sinc}{sinc}
\DeclareMathOperator{\round}{round}
\DeclareMathOperator{\svd}{svd}
\DeclareMathOperator{\argmin}{argmin}
\DeclareMathOperator{\sgn}{sgn}
\DeclareMathOperator{\zeros}{zeros}

\DeclareMathOperator{\dx}{dx}
\DeclareMathOperator{\dy}{dy}
\DeclareMathOperator{\dz}{dz}
\DeclareMathOperator{\dt}{dt}
\DeclareMathOperator{\dist}{d}
\DeclareMathOperator{\pos}{pos}

\DeclareMathOperator{\Expect}{{{\mathbb E}}}

\DeclareMathOperator{\T}{T}

% Metrics
\DeclareMathOperator{\SE}{SE}
\DeclareMathOperator{\MSE}{MSE}
\DeclareMathOperator{\API}{API}
\DeclareMathOperator{\MAPI}{MAPI}
\DeclareMathOperator{\GCNR}{gCNR}
\DeclareMathOperator{\MGCNR}{MgCNR}
\DeclareMathOperator{\OVL}{OVL}

% else
\DeclareMathOperator{\optimized}{opt}
\DeclareMathOperator{\estimated}{est}
\DeclareMathOperator{\thres}{th}
\DeclareMathOperator{\Frob}{F}
\DeclareMathOperator{\Hermit}{H}
\DeclareMathOperator{\TLS}{TLS}
\DeclareMathOperator{\ROI}{ROI}
\DeclareMathOperator{\CF}{CF}





% PDF-options
\title{'Blind' Iterative SAFT Reconstruction for Manually Acquired Ultrasonic Measurement Data in Nondestructive Testing} 
\subtitle{CSP Advanced Research Project WS19/20}
\institute{\foreignlanguage{german}{Technische Universität Ilmenau}}
\author{Sayako Kodera}
\date{Aug.07.2019}

% beamer goto-button setup
\setbeamercolor{button}{fg=white,bg=blue}
\renewcommand{\beamergotobutton}[1]{%
    \begingroup% keep color changes local
    \setbeamercolor{button}{fg=white, bg=tui_orange}%
    \beamerbutton{\insertgotosymbol#1}% original definition
    \endgroup
    }

%%%%%%%%%%%%%%%%%%%%%%%%%%%%%%%%%%%%%%%%%%%%%%
%================================= tips to use beamer ======%

% \uncover<page-> = the text will be covered (so displayed in light colors) for the particular pages of the slide
% \ony<page-> = the text/images will be hidden for the particular pages of the slide
% \footnotetext[number] = text for footnote wirh the givien "number"
% \footnotemark = displays the mark (i.e. nuber) of the footnote

%%%%%%%%%%%%%%%%%%%%%%%%%%%%%%%%%%%%%%%%%%%%%%
%%%%%%%%%%%%%%%%%%%%%%%%%%%%%%%%%%%%%%%%%%%%%%

\begin{document}
% display only logos but no contact info in footline
\setbeamertemplate{footline}[light]
% print title slide
\begin{frame}[noframenumbering] % noframenumbering prevents the framecounter from increasing for this single slide
 	\titlepage
\end{frame}

% small footline for more content
\setbeamertemplate{footline}[shrunkplain]
%\begin{frame}[noframenumbering] % noframenumbering prevents the %framecounter from increasing for this single slide
%	\frametitle{Table of contents}
%	\vfill
% 	\tableofcontents[part=1]
%\end{frame}

%%%%%%%%%%%%%%%%%%%%%%%%%%%%%%%%%%%%%%%%%%%%%%
\part{content}

%%%%%%%%%%%%%%%%%%%%%%%%%%%%%%%%%%%%%%%%%%%%%%
%%%%%%%%%%%%%%%%%%%%%%%%%%%%%%%%%%%% Sec. 1.1 %%%%%
\section{Background}

\subsection{Measurement Assistance System} 
% small footline with pagenumbers
\setbeamertemplate{footline}[shrunklight]
% full headline for sections in headline
\setbeamertemplate{frametitle}[full]
\begin{frame}[t]
\frametitle{Measurement Assistance System}
%\fontsize{10pt}
%======================================== content =====%
	\begin{columns}[t]	
	% Text part
	\begin{column}{0.48\textwidth}	
	\vspace{\topsep}
	\hspace*{-0.3cm} \textbf{Features}:
	\vspace*{-0.5cm}
		\begin{itemize}
		\item Position recognition
		\item Data recording
		\item Data visualization
		\item Post-processing
		\end{itemize}
	%	
	\vspace*{0.3cm}
	\textbf{Problem}: Observation errors\\
	\hspace*{0.3cm} e.g. tracking error 
	\end{column}	
	%
	% Image part
	\begin{column}{0.52\textwidth}
		\begin{overprint} 
		\centering
		\vspace*{-0.5cm}
		\only<1>{
		\inputTikZ{0.6}{figures/SmartInspect_2D.tex}
		}
		\only<2->{
		\inputTikZ{0.6}{figures/tracking_error_2D.tex}
		}
		\end{overprint}
	\end{column}
	\end{columns}	
	
%===================================================%
\end{frame}


\subsection{Motivation} 
% small footline with pagenumbers
\setbeamertemplate{footline}[shrunklight]
% full headline for sections in headline
\setbeamertemplate{frametitle}[full]
\begin{frame}[t]
\frametitle{Impact of Positional Inaccuracy} 
%\fontsize{10pt}
%======================================== content =====%
	%%=======  Left: B-Scan =======%%%
	\begin{textblock}{75}(10, 22)
	\centering
	\begin{overprint}
		% B-Scan ref
		\only<1-2>{
		\textbf{Measurement data}\\
		\imgzdefmiddle{1.45}{\scriptsize}{\scriptsize}{figures/pytikz/2D/texpngs/ARP/30mm_Bscan.png}\\
		% <scale size>, <label font size>, <tick font size>, <png file name>
		}
	\end{overprint}
	\end{textblock}
	%
	%
	%%======= Right: Reco =======%%%
	\begin{textblock}{75}(85, 22)
	\centering
	\begin{overprint}
		% Reco ref
		\only<1>{
		\textbf{Reconstruction (no error)}\\
		\vspace*{-0.2cm}
		\imgzdefmiddle{1.45}{\scriptsize}{\scriptsize}{figures/pytikz/2D/texpngs/ARP/30mm_true.png}\\
		% <scale size>, <label font size>, <tick font size>, <png file name>
		}
		% Reco track
		\only<2>{
		\textbf{Reconstruction (with error)}\\
		\vspace*{-0.2cm}
		\imgzdefmiddle{1.45}{\scriptsize}{\scriptsize}{figures/pytikz/2D/texpngs/ARP/30mm_track_1lambda.png}\\
		% <scale size>, <label font size>, <tick font size>, <png file name>
		}
	\end{overprint}
	\end{textblock}
	
%===================================================%
\end{frame}


\subsection{Objective and Contributions}
% small footline with pagenumbers
\setbeamertemplate{footline}[shrunklight]
% full headline for sections in headline
\setbeamertemplate{frametitle}[full]
\begin{frame}[t]
\frametitle{Objective and Contributions}
%======================================== content =====%
	%%========= Left: Text =========%%
	\begin{textblock}{80}(10, 25) %-> using textblock avoid shifting of figures b/w slides
	\textbf{Objective} 	
	\begin{itemize}
		\item Reduce error-induced artefacts in SAFT reconstructions
	\end{itemize}
	%\vspace*{0.2cm}
	
	\textbf{Contributions}
	\begin{itemize}
		\item Data model considering the positional inaccuracy 
		\item Preprocessing method to estimate and correct positional error 
	\end{itemize}
	\end{textblock}
	
	%%========= Right: Image =========%%
	\begin{textblock}{50}(95, 25)
	\centering
	\begin{overprint}
	% Reco track
	\only<1>{
	\imgzdefmiddle{1.2}{\scriptsize}{\scriptsize}{figures/pytikz/2D/texpngs/ARP/30mm_track_1lambda.png}\\
	%<scale size>, <label font size>, <tick font size>, <png file name>, <ytick min>, <ytick center>, <ytick max>
	}
	% Reco opt
	\only<2>{
	\imgzdefmiddle{1.2}{\scriptsize}{\scriptsize}{figures/pytikz/2D/texpngs/ARP/30mm_opt_1lambda.png}\\
	}
	\end{overprint}
	\end{textblock}
	
%===================================================%	
\end{frame}

%%%%%%%%%%%%%%%%%%%%%%%%%%%%%%%%%%%%%%%%%%%%%%%%%%%%%%%%%%%%%%%%%%%%%%
%%%%%%%%%%%%%%%%%%%%%%%%%%%%%%%%%%%%%%%%%%%%%%%%%%%%%%%%%%%%%%%%%%%%%%
\section{Method}
% small footline with pagenumbers
\setbeamertemplate{footline}[shrunklight]
% full headline for sections in headline
\setbeamertemplate{frametitle}[full]
\begin{frame}[t]
\frametitle{Blind Error Correction (BEC)}
%======================================== content =====%
	\textbf{Data model based on spatial approximation}\\
	\begin{itemize}
		\item Signal source positions
		\item Tracking error
	\end{itemize}
	%\vspace*{0.2cm}
	%
	\textbf{Preprocessing in 2 steps}\\
	(1) Estimate the signal source positions\\
	\hspace*{0.6cm} Known: data structure\\
	\hspace*{0.6cm} $\rightarrow$ Robust polynomial regression\\
	\vspace*{0.2cm}
	%
	(2) Estimate and correct the tracking error\\
	\hspace*{0.6cm} $\rightarrow$ Nonlinear programming\\
%===================================================%	
\end{frame}

%%%%%%%%%%%%%%%%%%%%%%%%%%%%%%%%%%%%%%%%%%%%%%%%%%%%%%%%%%%%%%%%%%%%%%
%%%%%%%%%%%%%%%%%%%%%%%%%%%%%%%%%%%%%%%%%%%%%%%%%%%%%%%%%%%%%%%%%%%%%%
\section{Simulations}
\subsection{Setup}
% small footline with pagenumbers
\setbeamertemplate{footline}[shrunklight]
% full headline for sections in headline
\setbeamertemplate{frametitle}[full]
\begin{frame}[t]
\frametitle{BEC Performance Analysis} 
%======================================== content =====%
	\only<1-2>{
		\textbf{Simulation studies}\\
		\begin{itemize}
			\item \cigray<2>{Error tolerance}
			\item\cihighlight<2>{Impact of the ROI depth}
		\end{itemize}
		%
		\textbf{Scenario and Assumptions}
		\begin{itemize}
			\item Linear contact scanning (\SI{0.5}{\milli \metre} grids)
			\item One point source in ROI
			\item Tracking error $= - \lambda \ldots + \lambda$
		\end{itemize}
		\vspace*{0.2cm}
	}
	\only<3-4>{
		\begin{textblock}{130}(10, 25) 
			\textbf{Evaluation methods}
			\begin{itemize}
				\item \cigray<4>{Normalized squared error}
				\item \cigray<4>{\textit{Generalized Contrast-to-Noise Ratio} (gCNR)}
				\item \cihighlight<4>{\textit{Array Performance Indicator} (API)}
			\end{itemize}
		\end{textblock}
		%
		\begin{textblock}{90}(10, 53) 
			\only<4>{
				\hspace*{0.8cm} API = area $> \epsilon$ (normalized with $\lambda^{2}$)\\
				\hspace*{1.4cm} $\Rightarrow$ smaller API $\hat{=}$ better resolution
			}
		\end{textblock}
	}
%===================================================%	
\end{frame}

\subsection{Results} 
% small footline with pagenumbers
\setbeamertemplate{footline}[shrunklight]
% full headline for sections in headline
\setbeamertemplate{frametitle}[full]
\begin{frame}[t]
\frametitle{Impact of the ROI Depth} 
%\fontsize{10pt}
%======================================== content =====%
	\only<1-2>{
		\resultAPIanimate{1.55}{\scriptsize}{2}{(20.0, 18.1)}{(20.0, 26.4)}{(20.0, 20.5)}
		%<scale size>, <font size>, <slide page for mark>, <mark coordinate for Reco_true>, <mark coordinate for Reco_track>, <mark coordinate for Reco_opt>
	}
	\only<4>{
		\resultAPIanimate{1.55}{\scriptsize}{4}{(30.0, 17.79)}{(30.0, 24.97)}{(30.0, 18.69)}
		%<scale size>, <font size>, <slide page for mark>, <mark coordinate for Reco_true>, <mark coordinate for Reco_track>, <mark coordinate for Reco_opt> 
		}
	\only<6>{
		\resultAPIanimate{1.55}{\scriptsize}{6}{(50.0, 17.63)}{(50.0, 24.74)}{(50.0,18.48)}
		%<scale size>, <font size>, <slide page for mark>, <mark coordinate for Reco_true>, <mark coordinate for Reco_track>, <mark coordinate for Reco_opt> 
	}
	%%=======  Left: Reco true =======%%%
	\begin{textblock}{50}(5, 22)
	\centering
	\begin{overprint}
		%% 20mm
		\only<3>{
			\textbf{Reference}\\
			\imgzdefshallow{1.2}{\scriptsize}{\scriptsize}{figures/pytikz/2D/texpngs/ARP/20mm_true.png}\\
			% <scale size>, <label font size>, <tick font size>, <png file name>
		}
		%% 30mm
		\only<5>{
			\textbf{Reference}\\
			\imgzdefmiddle{1.2}{\scriptsize}{\scriptsize}{figures/pytikz/2D/texpngs/ARP/30mm_true.png}\\
			% <scale size>, <label font size>, <tick font size>, <png file name>
		}
		%% 50mm
		\only<7>{
			\textbf{Reference}\\
			\imgzdefdeep{1.2}{\scriptsize}{\scriptsize}{figures/pytikz/2D/texpngs/ARP/50mm_true.png}\\
			% <scale size>, <label font size>, <tick font size>, <png file name>
		}
	\end{overprint}
	\end{textblock}
	%
	%
	%%======= Middle: Reco track =======%%%
	\begin{textblock}{50}(55, 22)
	\centering
	\begin{overprint}
		%% 20mm
		\only<3>{
			\textbf{No error correction}\\
			\imgzdefshallow{1.2}{\scriptsize}{\scriptsize}{figures/pytikz/2D/texpngs/ARP/20mm_track_1lambda.png}\\
			% <scale size>, <label font size>, <tick font size>, <png file name>
		}
		%% 30mm
		\only<5>{
			\textbf{No error correction}\\
			\imgzdefmiddle{1.2}{\scriptsize}{\scriptsize}{figures/pytikz/2D/texpngs/ARP/30mm_track_1lambda.png}\\
			% <scale size>, <label font size>, <tick font size>, <png file name>
		}
		%% 50mm
		\only<7>{
			\textbf{No error correction}\\
			\imgzdefdeep{1.2}{\scriptsize}{\scriptsize}{figures/pytikz/2D/texpngs/ARP/50mm_track_1lambda.png}\\
			% <scale size>, <label font size>, <tick font size>, <png file name>
		}
	\end{overprint}
	\end{textblock}
	%
	%
	%%======= Right: Reco opt =======%%%
	\begin{textblock}{50}(105, 22)
	\centering
	\begin{overprint}
		%% 20mm
		\only<3>{
			\textbf{With BEC}\\
			\imgzdefshallow{1.2}{\scriptsize}{\scriptsize}{figures/pytikz/2D/texpngs/ARP/20mm_opt_1lambda.png}\\
			% <scale size>, <label font size>, <tick font size>, <png file name>
		}
		%% 30mm
		\only<5>{
			\textbf{With BEC}\\
			\imgzdefmiddle{1.2}{\scriptsize}{\scriptsize}{figures/pytikz/2D/texpngs/ARP/30mm_opt_1lambda.png}\\
			% <scale size>, <label font size>, <tick font size>, <png file name>
		}
		%% 50mm
		\only<7>{
			\textbf{With BEC}\\
			\imgzdefdeep{1.2}{\scriptsize}{\scriptsize}{figures/pytikz/2D/texpngs/ARP/50mm_opt_1lambda.png}\\
			% <scale size>, <label font size>, <tick font size>, <png file name>
		}
	\end{overprint}
	\end{textblock}
	
%===================================================%
\end{frame}


%%%%%%%%%%%%%%%%%%%%%%%%%%%%%%%%%%%%%%%%%%%%%%%%%%%%%%%%%%%%%%%%%%%%%%
%%%%%%%%%%%%%%%%%%%%%%%%%%%%%%%%%%%%%%%%%%%%%%%%%%%%%%%%%%%%%%%%%%%%%%
\section{Summary} 

\subsection{Conclusion}
% small footline with pagenumbers
\setbeamertemplate{footline}[shrunklight]
% full headline for sections in headline
\setbeamertemplate{frametitle}[full]
\begin{frame}
\frametitle{Conclusion}
%\vspace*{1cm}
%\footnotesize
%======================================== content =====%
	\textbf{BEC $\Rightarrow$ artefacts reduction}
	\begin{table}
		\inputTable{0.9\textwidth}{tables/table_conclusion.tex}
	\end{table}
	%
	\vspace*{0.3cm}
	\textbf{Future Work}
	\begin{itemize}
		\item Extension to 3D and/or \textit{gridless} cases
		\item Error correction via minimax estimator
	\end{itemize}
%===================================================%
\end{frame}


%%%%%%%%%%%%%%%%%%%%%%%%%%%%%%%%%%%%%%%%%%%%%%
%%%%%%%%%%%%%%%%%%%%%%%%%%%%%%%%% Bibliography %%%%%
%\bibliography{main}

%%%%%%%%%%%%%%%%%%%%%%%%%%%%%%%%%%%%%%%%%%%%%%
%%%%%%%%%%%%%%%%%%%%%%%%%%%%%%%%%%%% Backup %%%%%
\appendix
\backupbegin
\section{Backup}
\subsection{Backup Start}
% small footline with pagenumbers
\setbeamertemplate{footline}[shrunklight]
% full headline for sections in headline
\setbeamertemplate{frametitle}[full]
\begin{frame}
	%\frametitle{ } 
	%\vspace*{1cm}
	%\footnotesize
	%======================================== content =====%
	\centering
	\huge Backup
	%===================================================%
\end{frame}

%%% Appendix %%%
\subsection{Parameters (1)}
% small footline with pagenumbers
\setbeamertemplate{footline}[shrunklight]
% full headline for sections in headline
\setbeamertemplate{frametitle}[full]
\begin{frame}
	\frametitle{Parameters w.r.t. Test Object}
	%\vspace*{1cm}
	%\footnotesize
	%======================================== content =====%
	\centering
	\begin{table}
	\begin{center}
		\inputTable{0.85\textwidth}{tables/table_constant_params_EN.tex}
	\end{center}
	\end{table}
	%===================================================%
\end{frame}


\subsection{Parameters (2)}
% small footline with pagenumbers
\setbeamertemplate{footline}[shrunklight]
% full headline for sections in headline
\setbeamertemplate{frametitle}[full]
\begin{frame}
	\frametitle{Parameters w.r.t. Pulse}
	%\vspace*{1cm}
	%\footnotesize
	%======================================== content =====%
	\centering
	\begin{columns}
	% table
	\begin{column}{0.5\textwidth}
	\begin{table}
	\begin{center}
		\inputTable{1\textwidth}{tables/table_constant_params_pulse_EN.tex}
	\end{center}
	\end{table}
	\end{column}
	% FIG : Gabot
	\begin{column}{0.5\textwidth}
	\inputTikZ{0.6}{figures/Gabor_pulse.tex}
	\end{column}
	\end{columns}	
	%===================================================%
\end{frame}

\subsection{Motivation}
% small footline with pagenumbers
\setbeamertemplate{footline}[shrunklight]
% full headline for sections in headline
\setbeamertemplate{frametitle}[full]
\begin{frame}[t]
\frametitle{Reconstructing UT Data}
%======================================== content =====%
	%%% autom = Muse %%%
	\only<1>{
	Automatic measurement\\
	
	\vspace*{0.5cm}
	\centering
	\begin{columns}[c]	
	% Data
	\begin{column}{0.5\textwidth}
		\centering
		Measurement data\footnotemark \addtocounter{footnote}{-1} \\  \vspace*{0.1cm} 
		\inputTikZ{0.8}{figures/Krieg_18SHMNDT/reference_reco_data.tex}
	\end{column}	
	% Reco
	\begin{column}{0.5\textwidth}
		\centering
		 SAFT Reconstruction\footnotemark \addtocounter{footnote}{-1} \\ \vspace*{0.1cm}
		\inputTikZ{0.8}{figures/Krieg_18SHMNDT/reference_reco_reco.tex}
	\end{column}	
	\end{columns}	
	}
	
	%%% manual = aviation SCAN 5-3 %%%
	\only<2>{
	Manual measurement\\
	
	\centering
	\begin{columns}[t]	
	% Data
	\begin{column}{0.5\textwidth}
		\centering
		Measurement data\footnotemark \addtocounter{footnote}{-1} \\ \vspace*{0.3cm} %\hspace*{0.3cm}
		\inputTikZ{0.7}{figures/Krieg_18SHMNDT/aviation_SCAN5_3_data_gauss_2.tex}
	\end{column}
	% Reco
	\begin{column}{0.5\textwidth}
		\centering
		SAFT Reconstruction\footnotemark \\ \vspace*{0.3cm}
		\inputTikZ{0.7}{figures/Krieg_18SHMNDT/aviation_SCAN5_3.tex}
	\end{column}	
	\end{columns}	
	}	
	%\vspace*{0.3cm}
	% footnote
	\only<1>{
	\footnotetext[1]{F. Krieg et al., SAFT processing for manually acquired ultrasonic measurement data with 3D SmartInspect, \textsl{SHM-NDT}, 2018}
	}
%===================================================%	
\end{frame}


\subsection{Previous Work}
% small footline with pagenumbers
\setbeamertemplate{footline}[shrunklight]
% full headline for sections in headline
\setbeamertemplate{frametitle}[full]
\begin{frame}[t]
\frametitle{Impact of Systematic Errors}
%======================================== content =====%
%%% positional error from BA %%%
%\centering
\begin{columns}[t]
	%% % Left: Text %%%
	\begin{column}{0.5\textwidth}
	\textbf{Vertical shift}
		\begin{itemize}
		\item High impact
		\item No accessible information \\
				$\rightarrow$ only estimation through data
		\end{itemize}
	\textbf{Positional inaccuracy}
		\begin{itemize}
		\item Less impact
		\item Critical when sparsely scanned
		\item Information available
		\end{itemize}
	\end{column}
	
	%%% Right: Image %%%
	\begin{column}{0.5\textwidth}
	%\vspace*{-0.3cm} 
	\centering
	\only<1>{
	% "Automatic" measurement
	\small Equidistant 1600 positions\\
	\small Without systematic errors\\
	\vspace*{0.02cm} % because SI in other parts caue vertical shift
	}
	\only<2>{
	% Zscan error 1
	\small Equidistant 1600 positions\\
	\small Vertical shift $=$ \SI{0.3}{\milli\metre} $(0.24 \lambda)$\\
	}
	\only<3>{
	% Position error  1 -> better with lambda?
	\small 1600 random positions\\
	\small Positional error $=$ \SI{2}{\milli\metre} $(1.6 \lambda)$\\
	}
	\only<4>{
	% Position error 2
	\small 160 random positions\\
	\small Positional error $=$ \SI{2}{\milli\metre} $(1.6 \lambda)$\\
	}
	\vspace*{0.1cm}
	% Image
	\begin{overprint} % to keep the figures in the place! 
	\only<1>{
	\cimgbothlabels{0.65}{\small}{\small}{figures/pytikz/2D/texpngs/grid/cimg_reco_max_oa_0.png}
	}
	%
	\only<2>{
	\cimgbothlabels{0.65}{\small}{\small}{figures/pytikz/2D/texpngs/zscan/oa_0_cimg_reco_max_030.png}
	}
	\only<3>{
	\cimgbothlabels{0.65}{\small}{\small}{figures/pytikz/2D/texpngs/posscan/dr_np_5_cimg_max_sigma_400_00.png}
	}
	\only<4>{
	\cimgbothlabels{0.65}{\small}{\small}{figures/pytikz/2D/texpngs/posscan/dr_np_2_cimg_max_sigma_400_00.png}
	}
	\end{overprint}
	\end{column}	
\end{columns}

%===================================================%	
\end{frame}


%%%%%%%%%%%%%%%%%%%%%%%%%%%%%%%%%%%%%%%%%%%%%%
%%%%%%%%%%%%%%%%%%%%%%%%%%%%%%%%%%% Sec. 1.2 %%%%%


\subsection{SAFT}
% small footline with pagenumbers
\setbeamertemplate{footline}[shrunklight]
% full headline for sections in headline
\setbeamertemplate{frametitle}[full]
\begin{frame}[t]
\frametitle{Post-processing Method} 
%======================================== content =====%
	Synthetic Aperture Focusing Technique (SAFT)
	\begin{columns}[t]
	% Text part
	\begin{column}{0.48\textwidth}
	\begin{center}
		\begin{itemize}
		\item Superposition according to propagation time delay
		\item Spatial sampling of the specimen
		\end{itemize}
	\end{center}
	\end{column}	
	%	
	% Image part
	\begin{column}{0.52\textwidth}
	\vspace*{-0.3cm}
	\begin{center}
		\only<1>{\inputTikZ{1.1}{figures/SAFT_superposition.tex}}
		\only<2>{\inputTikZ{0.9}{figures/Synthetic_aperture.tex}}
	\end{center}
	\end{column}
	%
	\end{columns}	
%===================================================%		
\end{frame}

\subsection{FWM}
% small footline with pagenumbers
\setbeamertemplate{footline}[shrunklight]
% full headline for sections in headline
\setbeamertemplate{frametitle}[full]
\begin{frame}[t]
\frametitle{SAFT as Linear Transform} 
%======================================== content =====%
	% Defect map
	\only<1>{
	\textit{defect map} for ROI = \cihighlight{$\M \times \N_{x}$}\\
	\centering
	\inputTikZ{1.1}{figures/ROI_sketch.tex}
	}
	%
	% Linear transform
	\only<2->{
	\textbf{Transform}: \textit{defect map} $\rightarrow$ A-Scan\\
	\only<2>{
	% Equation
	\begin{equation*}
	\ascanvec (\pp) = \SAFTp \cdot \defect + \noisevec
	\end{equation*}\\ % "\\" avoids to shift the next text horizontaly
	%
	\vspace*{0.2cm}
	% Description
	\begin{itemize}
	\item $\ascanvec (\pp)$: measured A-Scan at $\pp$ (\cihighlight{$\M$})
	\item $\SAFTp$: SAFT matrix at $\pp$ (\cihighlight{$\M \times \LL = \M \times \M \N_{x}$})\\
			Containing pulse information for $\scatterer_{l} = (x_i, z_j)$, $l = 1$...$\LL$
	\item $\defect$: vectorized \textit{defect map} (\cihighlight{$\LL = \M \N_{x}$})
	\item $\noisevec$: noise (\cihighlight{$\M$})
	\end{itemize}
	}
	%
	\only<3>{
	% Equation
	\begin{equation*}
	\ascanvechat (\pp) = \SAFTp \cdot \defect 
	\end{equation*}\\
	%
	\vspace*{0.2cm}
	% Description
	\begin{itemize}
	\item $\ascanvechat (\pp)$: modeled A-Scan  at $\pp$ (\cihighlight{$\M$})
	\item \cigray{$\SAFTp$: SAFT matrix at $\pp$ ($\M \times \LL$)\\
			Containing pulse information for $\scatterer_{l} = (x_i, z_j)$, $l = 1$...$\LL$}
	\item \cigray{$\defect$: vectorized \textit{defect map} ($\LL$)}
	\end{itemize}
	}
	%
	\only<4>{
	% Equation
	\begin{equation*}
	\SAFTcol (\pp) = \SAFTp \cdot  \defect^{(l)}
	\end{equation*}\\
	%
	\vspace*{0.2cm}
	% Description
	\begin{itemize}
	\item $\SAFTcol (\pp)$: $l$-th column vector of $\SAFTp$ (\cihighlight{$\M$})
	\item \cigray{$\SAFTp$: SAFT matrix at $\pp$ ($\M \times \LL$)\\
			Containing pulse information for $\scatterer_{l} = (x_i, z_j)$, $l = 1$...$\LL$}
	\item $\defect^{(l)}$: vectorized \textit{defect map} (\cihighlight{$\LL$})\\
			Containing single non-zero element in $l$-th row
	\end{itemize}
	}
	}
	%
%===================================================%		
\end{frame}

\subsection{Spatial Approximation}
% small footline with pagenumbers
\setbeamertemplate{footline}[shrunklight]
% full headline for sections in headline
\setbeamertemplate{frametitle}[full]
\begin{frame}[t]
\frametitle{Spatial Approximation of SAFT Matrix} 
%======================================== content =====%
	% SAFT as linear transformation : reference
	\textbf{Transform}: \textit{defect map} $\rightarrow$ A-Scan  
	\begin{equation*}
	\SAFTcol (\pp) = \SAFTp \cdot \defect^{(l)}
	\end{equation*}
	\hspace*{5.2cm} \cihighlight{$\M$} \hspace*{0.8cm} \cihighlight{$\M \times \LL $} \hspace*{0.3cm} \cihighlight{$\LL$} \\
	\vspace*{0.2cm}
	%
	% Goal
	\only<1>{
	\textbf{Goal}: spatial approximation of SAFT matrix
	\begin{equation*}
	\SAFTp \approx f \left( \SAFT (\pp + \ppdelta); \SAFTdot (\pp + \ppdelta); \ppdelta \right)
	\end{equation*}
	}
	%
	\only<2-4>{
	% Non-linear transformation w.r.t. scan positions
	\textbf{Transform}: scan position $\rightarrow$ A-Scan
	% Local linearlity w/ Jacobian matrix
	\only<2>{
	\begin{equation*}
	\SAFTcol (\pp) \approx \SAFTcol (\pp + \ppdelta) - \Jacobianpartial (\pp + \ppdelta)\cdot \ppdelta
	\end{equation*}
	\hspace*{3cm} \cihighlight{$\M$} \hspace*{1.5cm} \cihighlight{$\M$} \hspace*{1.7cm} \cihighlight{$\M \times \K $} \hspace*{0.9cm} \cihighlight{$\K$} \\
	}
	%
	% w/ "Comprehensive" Jacobian matrix
	\only<3>{
	\begin{equation*}
	\SAFTcol (\pp) \approx \SAFTcol (\pp + \ppdelta) - \vectorize^{-1}_{\M, \LL} \{ \Jacobian (\pp + \ppdelta)\cdot \ppdelta \} \cdot \defect^{(l)}
	\end{equation*}
	\hspace*{2cm} \cihighlight{$\M$} \hspace*{1.5cm} \cihighlight{$\M$} \hspace*{2.5cm} \cihighlight{$\M \LL \times \K $} \hspace*{0.9cm} \cihighlight{$\K$} \hspace*{0.5cm} \cihighlight{$\LL$} \\
	}
	%
	% w/ approximated SAFT matrix
	\only<4>{
	\begin{equation*}
	\SAFTp \cdot \defect_l \approx \left[ \SAFT (\pp + \ppdelta) - \vectorize^{-1}_{\M, \LL} \{ \Jacobian (\pp + \ppdelta) \cdot \ppdelta \} \right] \cdot \defect^{(l)}
	\end{equation*}
	\hspace*{1.1cm} \cihighlight{$\M \times \LL $} \hspace*{0.2cm} \cihighlight{$\LL$} \hspace*{1cm} \cihighlight{$\M \times \LL$} \hspace*{2.1cm} \cihighlight{$\M \LL \times \K $} \hspace*{0.9cm} \cihighlight{$\K$} \hspace*{0.5cm} \cihighlight{$\LL$} \\
	}
	}
	\only<5>{
	% Spatial approximation of SAFT matrix
	\textbf{Spatial approximation of SAFT matrix}
	% express vec^{-1} as Kronecker product
	\begin{equation*}
	\SAFTp \approx \SAFT (\pp + \ppdelta) - \vectorize^{-1}_{\M, \LL} \{ \Jacobian (\pp + \ppdelta) \cdot \ppdelta \} 
	\end{equation*}
	\hspace*{2.2cm} \cihighlight{$\M \times \LL $} \hspace*{0.8cm} \cihighlight{$\M \times \LL$} \hspace*{2.3cm} \cihighlight{$\M \LL \times \K $} \hspace*{0.6cm} \cihighlight{$\K$} \\
	}
	\only<6>{
	% Spatial approximation of SAFT matrix
	\textbf{Spatial approximation of SAFT matrix \cihighlight{$\pp =$  $[x, 0]$}} 
	% express vec^{-1} as Kronecker product
	\begin{equation*}
	\SAFTx \approx \SAFT (x + \xdelta) - \vectorize^{-1}_{\M, \LL} \{ \Jacobian (x + \xdelta)\} \cdot \xdelta  
	\end{equation*}
	\hspace*{2.2cm} \cihighlight{$\M \times \LL $} \hspace*{0.8cm} \cihighlight{$\M \times \LL$} \hspace*{2.3cm} \cihighlight{$\M \LL \times 1 $} \hspace*{0.6cm} \cihighlight{$1$} \\
	}
	\only<7>{
	% Spatial approximation of SAFT matrix in 2D
	\textbf{Spatial approximation of SAFT matrix \cihighlight{$\pp =$  $[x, 0]$}}
	% express vec^{-1} as Kronecker product
	\begin{equation*}
	\SAFTx \approx \SAFT (x + \xdelta) - \SAFTdot (x + \xdelta) \cdot \xdelta  
	\end{equation*}
	\hspace*{2.8cm} \cihighlight{$\M \times \LL $} \hspace*{0.8cm} \cihighlight{$\M \times \LL$} \hspace*{1.2cm} \cihighlight{$\M \times \LL $} \hspace*{0.6cm} \cihighlight{$1$} \\
	}
%===================================================%		
\end{frame}


\subsection{Inverse vec Operator}
% small footline with pagenumbers
\setbeamertemplate{footline}[shrunklight]
% full headline for sections in headline
\setbeamertemplate{frametitle}[full]
\begin{frame}
\frametitle{Inverse vec Operator}
%\vspace*{1cm}
%\footnotesize
%======================================== content =====%
	\begin{equation*}
	\vectorize^{-1}_{\M, \LL} \{ \Jacobian (\pp + \ppdelta)\cdot \ppdelta \} = \left[ (\vectorize \{ \Identity_{\LL} \}^{\T} \otimes \Identity_{\M} \right] \cdot \left[ \Identity_{\LL} \otimes ( \Jacobian (\pp + \ppdelta)\cdot \ppdelta ) \right]
	\end{equation*}
%===================================================%
\end{frame}


\subsection{Iterative Update}
% small footline with pagenumbers
\setbeamertemplate{footline}[shrunklight]
% full headline for sections in headline
\setbeamertemplate{frametitle}[full]
\begin{frame}[t]
\frametitle{Iterative Matrix Improvement} 
%======================================== content =====%
	\only<1>{
	1st order Taylor approximation of A-Scan \cihighlight{$\pp = [x, 0]$}\\
	\begin{equation*}
	\SAFTcol (x) \approx \SAFTcol (x + \xdelta) - \SAFTcoldot (x + \xdelta) \cdot \xdelta
	\end{equation*}\\
	%
	% Problems
	\textbf{Problems}:
	\begin{itemize}
	\item Acquisition of $\xdelta$
	\item Limited validity range
	\end{itemize}
	%
	\vspace*{0.2cm}
	%
	$\Rightarrow$ iterative estimation and correction of $\xdelta$\\
	\begin{equation*}
	\min_{\xdelta} \| \SAFTcol (x + \xdelta) - \SAFTcol (x) -  \SAFTcoldot (x + \xdelta) \cdot \xdelta \|_{2}
	\end{equation*}
	}
	%
	% Block diagram
%	\only<3>{
%	Iterative process\\
%	\inputTikZ{1}{figures/blockdiagram_iterativeGD.tex} 
%	}
	%
	% Animation
	\only<2->{
	\vspace*{-0.3cm}
	% Figure part
	Iterative estimation and correction of $\xdelta$
	\begin{center}
		\inputTikZ{1}{figures/GD_Positions.tex} % add p_opt!!!!!
	\end{center}
	%
	\vspace*{-0.5cm}
	% Text part
	(1) Calculate $\SAFTcol (\xhat_{\optimized})$ \\
	(2) Solve the least square problem and obtain $\xdeltaest$ \\
	(3) Update $\xopt = \xopt - \xdeltaest$\\
	\vspace*{0.1cm}
	** Break, when $\| \SAFTcol (x) - \pulsevec_{l, \optimized} (\xopt ; \xdeltaopt ) \|_{2} \leq$ target
	}
%===================================================%		
\end{frame}


\subsection{Normalized Squared Error} 
% small footline with pagenumbers
\setbeamertemplate{footline}[shrunklight]
% full headline for sections in headline
\setbeamertemplate{frametitle}[full]
\begin{frame}
\frametitle{Normalized Squared Error $\SEdag$}
%\vspace*{1cm}
%\footnotesize
%======================================== content =====%
	\begin{equation*}
	\SEdag = \frac{ \| \gamma \aopt - \ascanvec \|_{2}}{\| \ascanvec \|_{2}}
	\end{equation*}\\
	%
	\vspace*{0.5cm}
	$\gamma$: normalization factor \\
	\begin{equation*}
	\gamma = \frac{\ascanvec^{\T} \cdot \aopt}{\aopt^{\T} \cdot \aopt}.
	\end{equation*}
%===================================================%
\end{frame}



\subsection{Effect of Error on A-Scan Modeling}
% small footline with pagenumbers
\setbeamertemplate{footline}[shrunklight]
% full headline for sections in headline
\setbeamertemplate{frametitle}[full]
\begin{frame}[t]
\frametitle{Effect of Error on A-Scan Modeling} 
%======================================== content =====%
	\centering
	% Position illustration
	\vspace*{-0.1cm}
	\inputTikZ{0.8}{figures/Results_PositionIllustration.tex}\\
	%
	% Figures
	\begin{overprint} % to keep the figures in the place! 
	\only<1>{ % 7.5mm away
	\seoffset{1}{\normalsize}{\normalsize}{figures/pytikz/1D/coordinates/se_offset_7_5mm_away.tex}
	}
	%
	\only<2>{ % 5mm away
	\seoffset{1}{\normalsize}{\normalsize}{figures/pytikz/1D/coordinates/se_offset_5mm_away.tex}
	}
	%
	\only<3>{ % 1mm away
	\seoffset{1}{\normalsize}{\normalsize}{figures/pytikz/1D/coordinates/se_offset_1mm_away.tex}
	}	
	\only<4>{ % half lambda away
	\seoffset{1}{\normalsize}{\normalsize}{figures/pytikz/1D/coordinates/se_offset_halflambda_away.tex}
	}	
	\only<5>{ % 0mm away
	\seoffset{1}{\normalsize}{\normalsize}{figures/pytikz/1D/coordinates/se_offset_0mm_away.tex}
	}	
	\end{overprint}		 
%===================================================%		
\end{frame}


\subsection{Results: Error Correction}
% small footline with pagenumbers
\setbeamertemplate{footline}[shrunklight]
% full headline for sections in headline
\setbeamertemplate{frametitle}[full]
\begin{frame}[t]
\frametitle{Results: Error Correction} 
%======================================== content =====%
	\centering
	% errmax_5lambda
	\gdpe{1}{\normalsize}{\normalsize}{figures/pytikz/1D/coordinates/errmax_5lambda/gd_pe_halflambda_away.tex}{figures/pytikz/1D/coordinates/errmax_5lambda/gd_pe_1mm_away.tex}{figures/pytikz/1D/coordinates/errmax_5lambda/gd_pe_2_5mm_away.tex}{figures/pytikz/1D/coordinates/errmax_5lambda/gd_pe_5mm_away.tex}{figures/pytikz/1D/coordinates/errmax_5lambda/gd_pe_7_5mm_away.tex}
	%
	% errmax_2lambda
	%\gdpe{1}{\normalsize}{\large}{figures/pytikz/1D/coordinates/errmax_2lambda/gd_pe_halflambda_away.tex}{figures/pytikz/1D/coordinates/errmax_2lambda/gd_pe_1mm_away.tex}{figures/pytikz/1D/coordinates/errmax_2lambda/gd_pe_2_5mm_away.tex}{figures/pytikz/1D/coordinates/errmax_2lambda/gd_pe_5mm_away.tex}{figures/pytikz/1D/coordinates/errmax_2lambda/gd_pe_7_5mm_away.tex}
	 
%===================================================%		
\end{frame}


\subsection{Results: SE}
% small footline with pagenumbers
\setbeamertemplate{footline}[shrunklight]
% full headline for sections in headline
\setbeamertemplate{frametitle}[full]
\begin{frame}[t]
\frametitle{Results: A-Scan Approximation} 
%======================================== content =====%
	\centering
	%errmax_5lambda
	\gdse{1}{\normalsize}{\normalsize}{figures/pytikz/1D/coordinates/errmax_5lambda/gd_se_halflambda_away.tex}{figures/pytikz/1D/coordinates/errmax_5lambda/gd_se_1mm_away.tex}{figures/pytikz/1D/coordinates/errmax_5lambda/gd_se_2_5mm_away.tex}{figures/pytikz/1D/coordinates/errmax_5lambda/gd_se_5mm_away.tex}{figures/pytikz/1D/coordinates/errmax_5lambda/gd_se_7_5mm_away.tex}
	%
	%errmax_2lambda	
	%\gdse{1}{\normalsize}{\large}{figures/pytikz/1D/coordinates/errmax_2lambda/gd_se_halflambda_away.tex}{figures/pytikz/1D/coordinates/errmax_2lambda/gd_se_1mm_away.tex}{figures/pytikz/1D/coordinates/errmax_2lambda/gd_se_2_5mm_away.tex}{figures/pytikz/1D/coordinates/errmax_2lambda/gd_se_5mm_away.tex}{figures/pytikz/1D/coordinates/errmax_2lambda/gd_se_7_5mm_away.tex}
			 
%===================================================%		
\end{frame}


\subsection{SAFT reconstruction} 
% small footline with pagenumbers
\setbeamertemplate{footline}[shrunklight]
% full headline for sections in headline
\setbeamertemplate{frametitle}[full]
\begin{frame}
\frametitle{SAFT reconstruction}
%\vspace*{1cm}
%\footnotesize
%======================================== content =====%
	% SAFT as linear transformation : reference
	\only<1>{
	% FWM: A-Scan
	\textbf{FWM}: \textit{defect map} $\rightarrow$ A-Scan  
	\begin{equation*}
	\ascanvechat (\pp) = \SAFTp \cdot \defect
	\end{equation*}
	\hspace*{5.2cm} \cihighlight{$\M$} \hspace*{0.8cm} \cihighlight{$\M \times \LL $} \hspace*{0.3cm} \cihighlight{$\LL$} \\
	\vspace*{0.2cm}
	%
	% FWM: B-Scan
	\textbf{FWM}: \textit{defect map} $\rightarrow$ all A-Scans
	\begin{equation*}
	\vectorize \{ \Ascan \} = \SAFTcomp \cdot \defect
	\end{equation*}
	\hspace*{5.5cm} \cihighlight{$\LL$} \hspace*{0.8cm} \cihighlight{$\LL \times \LL $} \hspace*{0.3cm} \cihighlight{$\LL$} \\
	\vspace*{0.2cm}
	}
	%
	% SAFT as linear transformation : reference
	\only<2>{
	% FWM
	\textbf{FWM}: \textit{defect map} $\rightarrow$ all A-Scans
	\begin{equation*}
	\vectorize \{ \Ascan \} = \SAFTcomp \cdot \defect
	\end{equation*}
	\hspace*{5.5cm} \cihighlight{$\LL$} \hspace*{0.8cm} \cihighlight{$\LL \times \LL $} \hspace*{0.3cm} \cihighlight{$\LL$} \\
	\vspace*{0.2cm}
	%
	% Reco
	\textbf{Reconstruction}: all A-Scans $\rightarrow$ reconstructed \textit{defect map}
	\begin{equation*}
	\tilde{\defect} = \SAFTcomp^{\T} \cdot \vectorize \{ \Ascan \} 
	\end{equation*}
	\hspace*{5cm} \cihighlight{$\LL$} \hspace*{0.4cm} \cihighlight{$\LL \times \LL $} \hspace*{0.3cm} \cihighlight{$\LL$} \\
	\vspace*{0.4cm}
	}
%===================================================%
\end{frame}
 

\backupend
%%%%%%%%%%%%%%%%%%%%%%%%%%%%%%%%%%%%%%%%%%%%%%
\end{document}
