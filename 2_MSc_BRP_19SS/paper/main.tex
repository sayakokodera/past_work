%%%%%%%%%%%%%%%%%%%%%%%%%%%%%%%%%%%%%%%%%%%%%%%%%%%
%%% Header: declarations, packages, definitions
%%%%%%%%%%%%%%%%%%%%%%%%%%%%%%%%%%%%%%%%%%%%%%%%%%%

\documentclass[10pt,twocolumn,a4paper,conference]{IEEEtran}

% package collection

%%% Language %%%
\usepackage[english]{babel}

%%% pgfplots and TikZ %%%
\usepackage{pgfplots}
\pgfplotsset{compat = 1.14}
\usepgfplotslibrary{colormaps}

\usepackage{tikz}
\usetikzlibrary{patterns}
\usetikzlibrary{positioning}
\usetikzlibrary{arrows}
\usetikzlibrary{snakes}
\usetikzlibrary{petri} %for tokens in the block diagram for iterative GD

%\usepackage{fullpage}

%%% Math %%%
\usepackage{amsmath,amssymb,amsfonts}   % ams packages, useful for typing math
\usepackage{bm}     % bm provides \boldmath for bold fonts in equations (matrices, vectors)

%%% EPS graphics -> unneccesary? %%%
%\usepackage{graphicx}                   % needed to insert EPS graphics via \includegraphics
%\usepackage{psfrag}                     % needed only if text in EPS graphics shall be replaced (e.g., formulas in graphics)

%%% else %%%
\usepackage{siunitx}
\usepackage[font = {small}]{caption}
\usepackage{subcaption}
\usepackage{multicol}
\usepackage{gensymb}

%%% Bibliography %%%
\usepackage{url}
\usepackage[sort]{natbib}
\bibliographystyle{unsrt} %plain -> order by alphabet, unsrt -> order by appearance 
%\bibliographystyle{IEEEsort}

\usepackage{hyperref}
%\usepackage[breaklinks,hidelinks]{hyperref}







% "newcommand" collections
% important!!!! no numbers in the name of commands!!!!!!!!

%=====================  for math =======================%
% symbols
% scan positions: p
\newcommand{\pp}{\bm{p}}
\newcommand{\pphat}{\bm{\hat{p}}}
\newcommand{\ppdelta}{\Delta \pp}
\newcommand{\ppdeltahat}{\Delta \hat{\pp}}
% scan positions: x
\newcommand{\xdelta}{\Delta x}
\newcommand{\xhat}{\hat{x}}
\newcommand{\xopt}{\hat{x}_{\optimized}}
\newcommand{\xdeltaest}{\xdelta_{\estimated}}
\newcommand{\xdeltaopt}{\xdelta_{\optimized}}
\newcommand{\xdeltahat}{\hat{\xdelta}}
\newcommand{\xvec}{\bm{x}}
\newcommand{\xvechat}{\bm{\hat{x}}}
\newcommand{\xdeltavec}{\Delta \xvec}
\newcommand{\xtrackvec}{\xvechat_{\track}}
\newcommand{\xoptvec}{\xvechat_{\optimized}}
\newcommand{\xdeltaoptvec}{\Delta \xvechat_{\optimized}}
% scatter position
\newcommand{\scatterer}{\bm{s}}
% A-Scans
\newcommand{\ascan}{a}
\newcommand{\ascanvec}{\bm{a}}
\newcommand{\ascank}{\bm{a}_{k}}
\newcommand{\ascanvechat}{\bm{\hat{a}}}
\newcommand{\adothat}{\dot{\bm{\hat{a}}}}
\newcommand{\aopt}{\bm{\tilde{a}}}
\newcommand{\Ascan}{\bm{A}}
% Pulse
\newcommand{\pulse}{h}
\newcommand{\pulsevec}{\bm{h}}
\newcommand{\pulsevecdot}{\bm{h}'}
\newcommand{\SAFTcol}[2]{\pulsevec_{#1}^{( #2)}}
% SAFT matrix
\newcommand{\SAFT}{\bm{H}}
\newcommand{\SAFTp}{\SAFT (\pp)}
\newcommand{\SAFTx}{\SAFT (x)}
\newcommand{\SAFTk}{\SAFT_{k}}
\newcommand{\SAFThat}{\hat{\bm{H}}}
\newcommand{\SAFThatk}{\hat{\bm{H}}_{k}}
\newcommand{\SAFTapproxxvec}{\left( \SAFThat + \bm{E} \Jacobianhat \right)}

% Jacobian relevent
\newcommand{\Jacobian}{\bm{J}} 
\newcommand{\Jacobianhat}{\hat{\Jacobian}} 
\newcommand{\Jacobianpartial}{\bm{J}_{\SAFTcol}}
\newcommand{\Deriv}{\bm{D}} 
\newcommand{\Derivcol}{\bm{d}_{l}} 
% else
\newcommand{\reflectivity}{r}
\newcommand{\refcoeff}{\beta}
\newcommand{\defect}{\bm{b}} % should be modified? 
\newcommand{\defectsingle}{b} % should be modified? 
\newcommand{\defecthat}{\hat{\bm{b}}} % should be modified? 
\newcommand{\zvec}{\bm{z}} 
\newcommand{\noisevec}{\bm{n}} 
\newcommand{\fhatpartial}{\hat{f}_{i}}
\newcommand{\taux}{\tau_{l} (x)}
\newcommand{\Identity}{\bm{I}} 
\newcommand{\SEdag}{\SE^{\dagger}} 
\newcommand{\MSEdag}{\MSE^{\dagger}} 

% Math functions/operators
\newcommand{\norm}[1]{%
	\left\lVert#1\right\rVert
}
\newcommand{\vecoperator}[1]{%
	\vectorize \{ #1 \}
}
\newcommand{\diagoperator}[1]{%  
     \operatorname{diag} \{ #1 \}
}
\newcommand{\Real}{%  
     \operatorname{Re}
}
\newcommand{\Imag}{%  
     \operatorname{Im}
}



% some special characters
\newcommand{\ii}{\mathrm{i}}
\newcommand{\dd}{\mathrm{d}}
\newcommand{\ee}{\mathrm{e}}
\newcommand{\dirac}{\delta}

\newcommand{\RR}{{\mathbb{R}}}
\newcommand{\NN}{{\mathbb{N}}}
\newcommand{\CC}{{\mathbb{C}}}
\newcommand{\OO}{{\mathcal{O}}}


%%%%%%%%%%%%%%%%%%%%%%%%%%%%%%%%%%%%%%%%%%%%%%
%======================================== table macro ====%
%%%%%%%%%%%%%%%%%%%%%%%%%%%%%%%%%%%%%%%%%%%%%%
% table scaling
\newcommand{\inputTable}[2]{%  
	\resizebox{#1}{!}{%
	 \input{#2}
	}
}


%%%%%%%%%%%%%%%%%%%%%%%%%%%%%%%%%%%%%%%%%%%%%%
%========================================== for TikZ  ====%
%%%%%%%%%%%%%%%%%%%%%%%%%%%%%%%%%%%%%%%%%%%%%%
\newcommand{\Nx}{3}
\newcommand{\Ny}{2}
\newcommand{\ddx}{1.5cm}
\newcommand{\ddy}{\ddx}
\newcommand{\ddz}{1.5cm}
\newcommand{\rCircle}{0.11cm}
\newcommand{\rCircleCamera}{0.07cm}

%%% draw transducer %%%
%5 3D
\newcommand{\drawTransducer}[4]{ % scale,scaley, x, y in axis
	\draw[draw = none, fill = white] (axis cs: #3 - #1, #4 - #2, 0) -- (axis cs: #3 + #1, #4 - #2, 0) -- (axis cs: #3 + #1, #4 + #2, 0) --  (axis cs: #3 + #1, #4 + #2, -#1) -- (axis cs: #3 - #1, #4 + #2, -#1) -- (axis cs: #3 - #1, #4 - #2, -#1) -- (axis cs: #3 - #1, #4 - #2, 0);
	\draw[] (axis cs: #3 - #1, #4 - #2, 0) -- (axis cs: #3 + #1, #4 - #2, 0) -- (axis cs: #3 + #1, #4 - #2, -#1) -- (axis cs: #3 - #1, #4 - #2, -#1) -- (axis cs: #3 - #1, #4 - #2, 0);
	\draw[] (axis cs: #3 + #1, #4 - #2, 0) -- (axis cs: #3 + #1, #4 + #2, 0) -- (axis cs: #3 + #1, #4 +  #2, -#1) -- (axis cs: #3 - #1, #4 +  #2, -#1) -- (axis cs: #3 - #1, #4 - #2, -#1);
	\draw[] (axis cs: #3 + #1, #4 - #2, -#1) -- (axis cs: #3 + #1, #4 +  #2, -#1);
}

%% 2D
\newcommand{\drawTransducerTwoD}[2]{ % scale, x  1,3
	\draw[draw = none, fill = white] (axis cs: #2 - #1, 0) -- (axis cs: #2 + #1, 0) --  (axis cs: #2 + #1, -#1) -- (axis cs: #2 - #1, -#1) -- (axis cs: #2 - #1, 0);
	\draw[] (axis cs: #2 - #1, 0) -- (axis cs: #2 + #1, 0) -- (axis cs: #2 + #1, -#1) -- (axis cs: #2 - #1, -#1) -- (axis cs: #2 - #1, 0);
	\draw[] (axis cs: #2 + #1, 0) -- (axis cs: #2 + #1, -#1) -- (axis cs: #2 - #1, -#1);
}

%%% draw video camera %%%
%% 3D
\newcommand{\drawCamera}[6]{ %<rotation origin (x, y, z)>, <start x>, <start y>, <start z> in axis, <camera width [dx]>, <camera height [dz]>
	% base
	\draw[rotate around = {30 : (axis cs: #1)}] (axis cs: #2, #3, #4) rectangle (axis cs: #2 + #5, #3, #4 - #6) ;
	% origin of rotation
	\node[campoint] (rotationorg) at (axis cs: #1) {};
	% "trapezoid" part (tip of the camera)
	\draw[rotate around = {30 : (axis cs: #1)}] (axis cs: #2, #3, #4 - 0.25*#6) -- (axis cs: #2 - 0.5* #6, #3, #4) -- (axis cs: #2 - 0.5* #6, #3, #4-#6) -- (axis cs: #2, #3, #4 - 0.75*#6);	
	% cable
	\draw[rotate around = {30 : (axis cs: #1)}] (axis cs: #2 + #5, #3, #4 - 0.5*  #6) .. controls (axis cs: #2 + 1.5* #5, #3, #4 - 0.5*  #6) and (axis cs: #2 + #5, #3, - 0.5*  #6) .. (axis cs: #2 + 1.5* #5, #3, - #6);	
}

%% 2D
\newcommand{\drawCameraTwoD}[5]{ %<rotation origin (x, z)>, <start x>, <start y>, <start z> in axis, <camera width [dx]>, <camera height [dz]> 1, 2, 4, 5, 6 -> 4-> 3, 5->4, 6-> 5
	% base
	\draw[rotate around = {30 : (axis cs: #1)}] (axis cs: #2, #3) rectangle (axis cs: #2 + #4, #3 - #5) ;
	% origin of rotation
	\node[campoint] (rotationorg) at (axis cs: #1) {};
	% "trapezoid" part (tip of the camera)
	\draw[rotate around = {30 : (axis cs: #1)}] (axis cs: #2, #3 - 0.25*#5) -- (axis cs: #2 - 0.5* #5, #3) -- (axis cs: #2 - 0.5* #5, #3-#5) -- (axis cs: #2, #3 - 0.75*#5);	
	% cable
	\draw[rotate around = {30 : (axis cs: #1)}] (axis cs: #2 + #4, #3 - 0.5*  #5) .. controls (axis cs: #2 + 1.5* #4, #3 - 0.5*  #5) and (axis cs: #2 + #4, - 0.5*  #5) .. (axis cs: #2 + 1.5* #4, - #5);	
}


%%%%%%%%%%%%%%%%%%%%%%%%%%%%%%%%%%%%%%%%%%%%%%
%======================================= image macro ====%
%%%%%%%%%%%%%%%%%%%%%%%%%%%%%%%%%%%%%%%%%%%%%%
% TikZ scaling
\newcommand{\inputTikZ}[2]{%<scaling factor>, <name of the tex file>
     \scalebox{#1}{\input{#2}}  
}

% BRP: for error correction animation
\newcommand{\measanimate}[5]{ % <scale size>, <slide page for base pulse>, <slide page for highlighting the pulse 1mm away>, <slide page for highlighting the pulse 2.5mm away>, <slide page for highlighting the pulse 5mm away>
	\scalebox{#1}{
		\begin{tikzpicture}
			\begin{axis}[
					width=6cm, height=6.3cm,  at={(0.7cm,0.3cm)},
					ticks=none, axis lines = center, 
					xmin=-0.5, xmax=10.5, ymin=-0.45, ymax=10,
					xlabel={$x$}, ylabel={$t$},
					y dir=reverse,
					x label style={at={(axis cs: 10.5, 0)}, anchor= west},
					y label style={at={(axis cs: -0.6, 10)}, anchor = north},
			        ]
			        
			        % Defect
			        \node at (axis cs: 5, 4) {\pgftext{\includegraphics[scale=0.15]{images/defect}}};
					% Pulse
					\only<#2>{
					\input{figures/pytikz/1D/coordinates/pulse/pulse_1_blue.tex}
					\input{figures/pytikz/1D/coordinates/pulse/pulse_2_blue.tex}
					\input{figures/pytikz/1D/coordinates/pulse/pulse_3_blue.tex}
					\input{figures/pytikz/1D/coordinates/pulse/pulse_4_blue.tex}
					\input{figures/pytikz/1D/coordinates/pulse/pulse_5_blue.tex}
					\input{figures/pytikz/1D/coordinates/pulse/pulse_6_blue.tex}
					\input{figures/pytikz/1D/coordinates/pulse/pulse_7_blue.tex}
					\input{figures/pytikz/1D/coordinates/pulse/pulse_8_blue.tex}
					\input{figures/pytikz/1D/coordinates/pulse/pulse_9_blue.tex}
					}
					
					% Highlight
					\only<#3>{\input{figures/pytikz/1D/coordinates/pulse/pulse_4_highlight.tex}} %1 mm away
					\only<#4>{\input{figures/pytikz/1D/coordinates/pulse/pulse_3_highlight.tex}} % 2.5mm away
					\only<#5>{\input{figures/pytikz/1D/coordinates/pulse/pulse_1_highlight.tex}} % 5mm away
			
			\end{axis}			
		\end{tikzpicture}
	}
}

\newcommand{\curvefit}[5]{ % <scale size>, <slide page for base pulse>, <slide page for highlighting the pulse 2.5mm away>, <slide page for adding the defect position>, <slide page for adding the scan position>, 
	\scalebox{#1}{
		\begin{tikzpicture}
			\begin{axis}[
					width=6cm, height=6.3cm,  at={(0.7cm,0.3cm)},
					ticks=none, axis lines = center, 
					xmin=-1.5, xmax=10.5, ymin=-1.5, ymax=10,
					xlabel={$x$}, ylabel={$t$},
					y dir=reverse,
					x label style={at={(axis cs: 10.5, 0)}, anchor= west},
					y label style={at={(axis cs: -0.6, 10)}, anchor = north},
			        ]
			        
			        % Defect
			        \node at (axis cs: 5, 4) {\pgftext{\includegraphics[scale=0.15]{images/defect}}};
					% Pulse
					\only<#2>{
					\input{figures/pytikz/1D/coordinates/pulse/pulse_1_blue.tex}
					\input{figures/pytikz/1D/coordinates/pulse/pulse_2_blue.tex}
					\input{figures/pytikz/1D/coordinates/pulse/pulse_3_blue.tex}
					\input{figures/pytikz/1D/coordinates/pulse/pulse_4_blue.tex}
					\input{figures/pytikz/1D/coordinates/pulse/pulse_5_blue.tex}
					\input{figures/pytikz/1D/coordinates/pulse/pulse_6_blue.tex}
					\input{figures/pytikz/1D/coordinates/pulse/pulse_7_blue.tex}
					\input{figures/pytikz/1D/coordinates/pulse/pulse_8_blue.tex}
					\input{figures/pytikz/1D/coordinates/pulse/pulse_9_blue.tex}
					}
					
					% Highlight
					\only<#3>{\input{figures/pytikz/1D/coordinates/pulse/pulse_3_highlight.tex}} %2.5mm away
					% Node: defect positions
					\only<#4>{%
						\node at (axis cs: 5, 0) {$\shortmid$};
						\node at (axis cs: 5, -1) {$x_{\dist}$};
						\node at (axis cs: 0, 4) {$-$};
						\node at (axis cs: -1, 4) {$z_{\dist}$};
					}% 
					% Node: i-th scan
					\only<#5>{%
						% Defect
						\node at (axis cs: 3, 0) {$\shortmid$};
						\node at (axis cs: 3, -1) {$x_{k}$};
						\node at (axis cs: 0, 5) {$-$};
						\node at (axis cs: -1, 5) {$z_{k}$};
					}% 
			
			\end{axis}			
		\end{tikzpicture}
	}
}

\newcommand{\TLSanimate}[3]{ % <scale size>,  <slide page for z_def = 762dz>, <slide page for z_def = 1270dz>, 
	\scalebox{#1}{
		\begin{tikzpicture}
			\begin{axis}[
					width=6cm, height=6.3cm,  at={(0.7cm,0.3cm)},
					ticks=none, axis lines = center, 
					xmin=-1.5, xmax=6, ymin=-1.5, ymax=10,
					xlabel={$x$ [\SI{}{\milli \metre}] }, ylabel={$z$ [\SI{}{\milli \metre}] },
					y dir=reverse,
					x label style={at={(axis cs: 5.7, 0)}, anchor= south},
					y label style={at={(axis cs: -0.6, 10)}, anchor = north},
			        ]
			        
					% z_def = 762dz
					\only<#2>{%
						% Defect
						\node at (axis cs: 3, 4) {\pgftext{\includegraphics[scale=0.15]{images/defect}}};
						% Defect position ticks
						\node at (axis cs: 3, 0) {$\shortmid$};
						\node at (axis cs: 3, -1) {$20$};
						\node at (axis cs: 0, 4) {$-$};
						\node at (axis cs: -1, 4) {$30$};
					}% 
					
					% z_def = 1270dz
					\only<#3>{%
						% Defect
						\node at (axis cs: 3, 6.5) {\pgftext{\includegraphics[scale=0.15]{images/defect}}};
						% Defect position ticks
						\node at (axis cs: 3, 0) {$\shortmid$};
						\node at (axis cs: 3, -1) {$20$};
						\node at (axis cs: 0, 6.5) {$-$};
						\node at (axis cs: -1, 6.5) {$50$};
					}% 

			
			\end{axis}			
		\end{tikzpicture}
	}
}


%%%%%%%%%%%%%%%%%%%%%%%%%%%%%%%%%%%%%%%%%%%%%%%%
%===================== 1D Visualization ======================%
%%%%%%%%%%%%%%%%%%%%%%%%%%%%%%%%%%%%%%%%%%%%%%%%

% ME TLS: 762dz vs 1270dz
\newcommand{\meTLS}[7]{ % <scale size>, <label font size>, <tick font size>,  <slide page to pop up the ME 762dz>, <fname for ME 762dz>, <slide page to pop up the ME 1270dz>, <fname for ME 1270dz>
\scalebox{#1}{
	\pgfplotsset{
			xmin = -0.05, xmax=1.05, 
			ymin = -0.005, ymax=0.15, 
			scaled ticks=false % to avoid formtting with 10^-2 in y tick
	}
	\begin{tikzpicture}
            \begin{axis}[
                width = 10cm, 
            	   height = 6.5cm,
            	   grid=both,
    			   grid style={line width=.1pt, draw=gray!20},
                xlabel = {Tracking error $/ \lambda$ \cigray{(\SI{1.26}{\milli \metre})} },
                ylabel = {$\norm{\pp_{\dist} - \pphat_{\dist}}_{2}$ $/ \lambda$},
                label style = {font = #2},
                tick label style = {
                		/pgf/number format/fixed, % to avoid formtting with 10^-2 in y tick
                		font = #3
                },
                %y dir = reverse,
                %xtick = {0, 10, ..., 30}, %to customize the axis
                %xticklabel = {0, 5, 10, 15}
                extra y ticks={0.125},
                %extra y tick labels = {$\SE^{\dagger}_{\threshold}$}, 
                %extra y tick style={font = #2},
                ]
                \only<#4>{
	                	\input{#5}
	                	% y = 1.25lamda line
		             \addplot[gray, dashed, mark = , line width = 2pt] coordinates{
		            			(0.0, 0.1243)
		            			(1.0, 0.1243)
		            };
                }
                \only<#6>{
                		\input{#7}
                		% y = 0.096lamda line
%                		\addplot[gray, dashed, mark = , line width = 2pt] coordinates{
%		            			(0.0, 0.096)
%		            			(1.0, 0.096)
%		            };
                	} 
            \end{axis}
	\end{tikzpicture}
	}
}

%%%%% Result: evaluation
% API  results with animation
\newcommand{\resultAPIanimate}[6]{ %<scale size>, <font size>, <slide page for mark>, <mark coordinate for Reco_true>, <mark coordinate for Reco_track>, <mark coordinate for Reco_opt> 
\scalebox{#1}{
	\begin{tikzpicture}
            \begin{axis}[
                width = 7cm, height = 4cm,
                xlabel = {ROI depth [\SI{}{\milli \metre}]}, ylabel = {$\MAPI$},
                ymin= 16.5, ymax= 28,
                label style = {font = #2},
                tick label style = {font = #2},
                xtick = {20, 30, ..., 80},
                ytick = {18, 20, ..., 28},
                grid=both, grid style={line width=.1pt, draw=gray!20},
                legend style ={
                	at={(1.5, 0.8)},
                	nodes={scale=0.95, transform shape},
                	font = #2
                }
                ]
                \input{figures/pytikz/1D/api_true_depth.tex} % true
                \input{figures/pytikz/1D/api_track_depth.tex} % track
                \input{figures/pytikz/1D/api_opt_depth.tex} % opt
                
             % x = 20mm line
             \only<#3>{
             		% true
		         \addplot[tui_red, mark = star, mark size = 2pt] coordinates{
		          		#4
		            };
		         % track
		         \addplot[tui_red, mark = star , mark size = 2pt] coordinates{
		          		#5
		            };
		       % opt
		         \addplot[tui_red, mark = star , mark size = 2pt] coordinates{
		          		#6
		            };
		    }   
                
             % legend
             % To insert the legend title
		   	%\addlegendimage{empty legend} 
		   	% Legend entries  
		   	\addlegendentry{Reference}
             \addlegendentry{No correction}
             \addlegendentry{BEC}
             % Title
             %\addlegendentry{$|s_{x} - x|$}        
            \end{axis}
	\end{tikzpicture}
	}
}

% SE results 
\newcommand{\resultSE}[7]{ % <scale size>, <font size>, <xlabel>, <ymax>, <xtick>, <fname for track>, <fname for opt>
\scalebox{#1}{
	\begin{tikzpicture}
            \begin{axis}[
                width = 7cm, height = 4cm,
                xlabel = {#3}, ylabel = {$\MSEdag$},
                ymin= -0.04, ymax= #4,
                label style = {font = #2},
                tick label style = {font = #2},
                xtick = {#5},
                ytick = {0, 0.2, ..., #4}, 
                grid=both, grid style={line width=.1pt, draw=gray!20},
                legend style ={
                	at={(1.5, 0.8)},
                	nodes={scale=0.95, transform shape},
                	font = #2
                }
                ]
                \input{#6} % track
                \input{#7} % opt
                
             % legend
             % To insert the legend title
		   	%\addlegendimage{empty legend} 
		   	% Legend entries  
             \addlegendentry{No correction}
             \addlegendentry{BEC} 
             % Title
             %\addlegendentry{$|s_{x} - x|$}        
            \end{axis}
	\end{tikzpicture}
	}
}

% API results
\newcommand{\resultAPI}[9]{ % <scale size>, <font size>, <xlabel>, <ymax>, <xtick>, <ytick>, <fname for true>, <fname for track>, <fname for opt>
\scalebox{#1}{
	\begin{tikzpicture}
            \begin{axis}[
                width = 7cm, height = 4cm,
                xlabel = {#3}, ylabel = {$\MAPI$},
                ymin= 16.5, ymax= #4,
                label style = {font = #2},
                tick label style = {font = #2},
                xtick = {#5},
                ytick = {#6},
                grid=both, grid style={line width=.1pt, draw=gray!20},
                legend style ={
                	at={(1.5, 0.8)},
                	nodes={scale=0.95, transform shape},
                	font = #2
                }
                ]
                \input{#7} % true
                \input{#8} % track
                \input{#9} % opt
                
             % legend
             % To insert the legend title
		   	%\addlegendimage{empty legend} 
		   	% Legend entries  
		   	\addlegendentry{Reference}
             \addlegendentry{No correction}
             \addlegendentry{BEC}
             % Title
             %\addlegendentry{$|s_{x} - x|$}        
            \end{axis}
	\end{tikzpicture}
	}
}


% GCNR results
\newcommand{\resultGCNR}[9]{ % <scale size>, <font size>, <xlabel>, <ymin>, <xtick>, <ytick>, <fname for true>, <fname for track>, <fname for opt>
\scalebox{#1}{
	\begin{tikzpicture}
            \begin{axis}[
                width = 7cm, height = 4cm,
                xlabel = {#3}, ylabel = {$\MGCNR$},
                ymin= #4, ymax= 0.97,
                label style = {font = #2},
                tick label style = {font = #2},
                xtick = {#5},
                ytick = {#6},  % 1.01 = otherwise the tick does not show up 
                grid=both, grid style={line width=.1pt, draw=gray!20},
                legend style ={
                	at={(1.5, 0.8)},
                	nodes={scale=0.95, transform shape},
                	font = #2
                }
                ]
                \input{#7} % true
                \input{#8} % track
                \input{#9} % opt
                
             % legend
             % To insert the legend title
		   	%\addlegendimage{empty legend} 
		   	% Legend entries  
		   	\addlegendentry{Reference}
             \addlegendentry{No correction}
             \addlegendentry{BEC}
             % Title
             %\addlegendentry{$|s_{x} - x|$}        
            \end{axis}
	\end{tikzpicture}
	}
}
%%%%%%%%%%%%%%%%%%%%%%%%%%%%%%%%%%%%%%%%%%%%%%%%
%===================== 2D Visualization ======================%
%%%%%%%%%%%%%%%%%%%%%%%%%%%%%%%%%%%%%%%%%%%%%%%%

% cimg with both x- & y-labels
\newcommand{\cimgbothlabels}[4]{% <scale size>, <label font size>, <tick font size>, <png file name>
\scalebox{#1}{
	\begin{tikzpicture}
            \begin{axis}[
                enlargelimits = false,
                axis on top = true,
                axis equal image,
                point meta min = -1,   
                point meta max = 1,
                xlabel = {$x$ in \SI{}{\milli\meter}},
                ylabel = {$y$ in \SI{}{\milli\meter}},
                label style = {font = #2},
                tick label style = {font = #3},
                %y dir = reverse,
                %xtick = {0, 10, ..., 30}, %to customize the axis
                %xticklabel = {0, 5, 10, 15}
                ]
                \addplot graphics [
                    xmin = 0,
                    xmax = 20,
                    ymin = 0,
                    ymax = 20
                ]{#4};
            \end{axis}
	\end{tikzpicture}
	}
}




%%%%%%%%%%%%%%%%%%%%%%%%%%%%%%%%%%%%%%%%%%%%%%
%=================== 2D visualization ======================%
%%%%%%%%%%%%%%%%%%%%%%%%%%%%%%%%%%%%%%%%%%%%%%
% cimg for z_def = 20mm
\newcommand{\imgzdefshallow}[4]{% <scale size>, <label font size>, <tick font size>, <png file name>
\scalebox{#1}{
	\begin{tikzpicture}
            \begin{axis}[
            	   width = 5.5cm, 
            	   %height = 6.7cm,
                enlargelimits = false,
                axis on top = true,
                axis equal image,
                %unit vector ratio= 0.3 1, % change aspect ratio, one of them should be 1
                point meta min = -1,   
                point meta max = 1,
                xlabel = {$x$ [\SI{}{\milli \metre}]},
                ylabel = {$z$ [\SI{}{\milli \metre}]},
                label style = {font = #2},
                tick label style = {font = #3},
                xlabel style = {yshift = 0.3cm},
                ylabel style = {yshift = -0.3cm},
                y dir = reverse,
                xtick = {15, 20, 25},
                ytick = {15, 20, 25},
                ]
                \addplot graphics [
                    xmin = 15,
                    xmax = 25,
                    ymin = 15,
                    ymax = 25
                ]{#4};
            \end{axis}
	\end{tikzpicture}
	}
}

% cimg for z_def = 30mm
\newcommand{\imgzdefmiddle}[4]{% <scale size>, <label font size>, <tick font size>, <png file name>,
\scalebox{#1}{
	\begin{tikzpicture}
            \begin{axis}[
            	   width = 5.5cm, 
            	   %height = 6.7cm,
                enlargelimits = false,
                axis on top = true,
                axis equal image,
                %unit vector ratio= 0.3 1, % change aspect ratio, one of them should be 1
                point meta min = -1,   
                point meta max = 1,
                xlabel = {$x$ [\SI{}{\milli \metre}]},
                ylabel = {$z$ [\SI{}{\milli \metre}]},
                label style = {font = #2},
                tick label style = {font = #3},
                xlabel style = {yshift = 0.3cm},
                ylabel style = {yshift = -0.3cm},
                y dir = reverse,
                xtick = {15, 20, 25},
                ytick = {25, 30, 35},%21, 22.5, 24
                ]
                \addplot graphics [
                    xmin = 15,
                    xmax = 25,
                    ymin = 25,
                    ymax = 35
                ]{#4};
            \end{axis}
	\end{tikzpicture}
	}
}

% cimg for z_def = 50mm
\newcommand{\imgzdefdeep}[4]{% <scale size>, <label font size>, <tick font size>, <png file name>,
\scalebox{#1}{
	\begin{tikzpicture}
            \begin{axis}[
            	   width = 5.5cm, 
            	   %height = 6.7cm,
                enlargelimits = false,
                axis on top = true,
                axis equal image,
                %unit vector ratio= 0.3 1, % change aspect ratio, one of them should be 1
                point meta min = -1,   
                point meta max = 1,
                xlabel = {$x$ [\SI{}{\milli \metre}]},
                ylabel = {$z$ [\SI{}{\milli \metre}]},
                label style = {font = #2},
                tick label style = {font = #3},
                xlabel style = {yshift = 0.3cm},
                ylabel style = {yshift = -0.3cm},
                y dir = reverse,
                xtick = {15, 20, 25},
                ytick = {45, 50, 55},%21, 22.5, 24
                ]
                \addplot graphics [
                    xmin = 15,
                    xmax = 25,
                    ymin = 45,
                    ymax = 55
                ]{#4};
            \end{axis}
	\end{tikzpicture}
	}
}

% math operators

\DeclareMathOperator{\sinc}{sinc}
\DeclareMathOperator{\round}{round}
\DeclareMathOperator{\svd}{svd}
\DeclareMathOperator{\argmin}{argmin}
\DeclareMathOperator{\sgn}{sgn}
\DeclareMathOperator{\zeros}{zeros}

\DeclareMathOperator{\dx}{dx}
\DeclareMathOperator{\dy}{dy}
\DeclareMathOperator{\dz}{dz}
\DeclareMathOperator{\dt}{dt}
\DeclareMathOperator{\dist}{d}
\DeclareMathOperator{\pos}{pos}

\DeclareMathOperator{\Expect}{{{\mathbb E}}}

\DeclareMathOperator{\T}{T}

% Metrics
\DeclareMathOperator{\SE}{SE}
\DeclareMathOperator{\MSE}{MSE}
\DeclareMathOperator{\API}{API}
\DeclareMathOperator{\MAPI}{MAPI}
\DeclareMathOperator{\GCNR}{gCNR}
\DeclareMathOperator{\MGCNR}{MgCNR}
\DeclareMathOperator{\OVL}{OVL}

% else
\DeclareMathOperator{\optimized}{opt}
\DeclareMathOperator{\estimated}{est}
\DeclareMathOperator{\thres}{th}
\DeclareMathOperator{\Frob}{F}
\DeclareMathOperator{\Hermit}{H}
\DeclareMathOperator{\TLS}{TLS}
\DeclareMathOperator{\ROI}{ROI}
\DeclareMathOperator{\CF}{CF}



% set color

\usepackage{color}
 
\definecolor{fri_gray}{rgb}{0.8, 0.8, 0.8}
\definecolor{fri_green_light}{rgb}{0.76, 0.95, 0.81}
\definecolor{fri_green}{rgb}{0.51, 0.87, 0.78}
\definecolor{tui_orange}{rgb}{0.94, 0.49, 0}
\definecolor{tui_blue}{rgb}{0, 0.22, 0.39}

\definecolor{box_white}{cmyk}{0.0361,0.0251,0.0166,0}
\definecolor{text_black}{cmyk}{0.7979,0.7417,0.6916,0.6554}
\definecolor{tui_orange_dark}{cmyk}{0,0.6,1,0}
\definecolor{tui_orange_light}{cmyk}{0.0000,0.0876, 0.1474, 0.0157}
\definecolor{tui_green_dark}{cmyk}{1,0,0.5,0.2}
\definecolor{tui_green_light}{cmyk}{0.0576,0.0041, 0.0000, 0.0471}
\definecolor{tui_blue_dark}{cmyk}{1.0000,0.5000,0.0000,0.6000}
\definecolor{tui_blue_light}{cmyk}{0.0920,0.0440,0.0000,0.0196}
\definecolor{tui_red_dark}{cmyk}{0.0000,1.0000,1.0000,0.2000}
\definecolor{tui_red_light}{cmyk}{0.0000,0.1107,0.1107,0.0078}



%%%%%%%%%%%%%%%%%%%%%%%%%%%%%%%%%%%%%%%%%%%%%%%%%%%
%%% Main body of the document
%%%%%%%%%%%%%%%%%%%%%%%%%%%%%%%%%%%%%%%%%%%%%%%%%%%

\begin{document}

\title{\bf Iterative SAFT Reconstruction for Manually Acquired Ultrasonic Measurement Data in Nondestructive Testing}

\author{
\textit{Sayako Kodera}\\
Ilmenau University of Technology\\
P. O. Box 100565, D-98684 Ilmenau, Germany \\
Email: sayako.kodera@tu-ilmenau.de
} 

\maketitle

%%%%%%%%%%%%%%%%%%%%%%%%%%%%%%%%%%%%%%%%%%%%%%%%%%%
%%% The abstract
%%%%%%%%%%%%%%%%%%%%%%%%%%%%%%%%%%%%%%%%%%%%%%%%%%%
{\bf {\bf \slshape Abstract \symbol{124}} %
%%% Abstract
There have been developments in nondestructive ultrasonic testing (UT) to assist manual operation for easier and more reliable inspection than conventional ones. Although such system also opens up the possibility to post-process the measurement data for improving imaging quality, manual UT is prone to stochastic observational errors, such as inaccuracy in estimating scan positions or varying coupling, which may cause strong artefacts formation in its reconstruction. %
In an attempt to reduce the positional-inaccuracy induced artefacts, in this work we propose a preprocessing method to correct the unknown positional error from the measurement data and the erroneous positional information. %As the first step of the preprocessing, the positions of the signal sources are estimated via robust polynomial regression. In the next step, the tracking error is estimated and corrected iteratively by modeling the measurement data based on the estimation results from the first step. 
We demonstrate through simulations that the proposed method is more resistant to positional error and can achieve higher resolution, which is comparable to that of the reconstruction with the exact positional information, than the reconstruction without preprocessing.  
}

%%%%%%%%%%%%%%%%%%%%%%%%%%%%%%%%%%%%%%%%%%%%%%%%%%%
%%% Keywords: 2-5, used for indexing the paper
%%%%%%%%%%%%%%%%%%%%%%%%%%%%%%%%%%%%%%%%%%%%%%%%%%%
% Keywords may be selected from the IEEE keyword list found at 
% http://www.ieee.org/organizations/pubs/ani_prod/keywrd98.txt

\smallskip

\begin{IEEEkeywords}
   Nondestructive testing, Ultrasonic testing, SAFT, Manual measurement, Positional inaccuracy
\end{IEEEkeywords}

%\section{Major Changes (v.190814)} 
%\begin{itemize}
%\item Sec. \ref{sec:intro}: modify motivation and background, eliminate \textit{Contributions} ($\rightarrow$ will be included in abstract)
%\item Sec. \ref{sec:pulse_echo}: reduce equations, adjust dimension for $\pp$ and $\scatterer$, modify defect map \ref{eq:defect_map}
%\item Sec. \ref{sec:methods}: modify intro, adjust dimension in \ref{sec:saft_approx}, eliminate long algorithm explanation in \ref{sec:iterative_GD} (to spare the space)
%\item Add sections for simulation \ref{sec:simulation} and results \ref{sec:results}
%\end{itemize}



%%%%%%%%%%%%%%%%%%%%%%%%%%%%%%%%%%%%%%%%%%%%%%%%%%%
%%% The introduction
%%%%%%%%%%%%%%%%%%%%%%%%%%%%%%%%%%%%%%%%%%%%%%%%%%%

\section{Introduction} \label{sec:intro}

%Your introduction motivates the problem again, emphasizing why it is important
%and what are the practical applications. Use references wherever possible to support
%your points. The introduction also introduces the state of the art, citing journal
%papers \cite{Shannon:48}, conference papers \cite{Haykin:07}, books \cite{Mittelbach:2004},  
%standards, RFCs, and other references. The use of BibTeX to organize your references is
%strongly encouraged. Try to avoid using websites as sources since these may change
%over time.
%
%After providing the state of the art, you should emphasize what are the novelties of your proposed 
%solution and how it differs from the existing ones. Why would anyone need your solution?

%%% Intro %%%
%% General idea
% (a)
% Prob(a) : there is a gam b/w automatic & manual UT
% Goal(a) : reduce the gap
% Sol(a) : assistance system -> enables post-processing
% (b)
% Prob(b): conventional reco does not work well w/ manual UT data due to the systematic errors (e.g. tracking error)
% Goal(b): reduce the effect of the tracking error
% Sol(b): correct positions and adjust the reco systems accordingly

%% (1) Background / context
% UT -> Prob(a)
Ultrasonic testing (UT) is a nondestructive testing method to inspect structure of test objects without inducing damage. Conventionally, an UT inspection requires either manual operation by a human technician or automated measurement systems. In manual UT, where a human technician observes the change in the echoed pulse, its inspection quality is highly dependent on the expertise of the technician \cite{Cawley01IMechE}. In automatic UT, on the other hand, measurement data and the corresponding scan positions are recorded, which enables to visualize the inner structure of the test object and further process the data to improve the imaging quality, leading to more reliable inspection quality than its manual counterpart. \par

% Goal(a) -> Sol(a)
Nevertheless, there are still needs for manual UT, when, for instance, a complex structure is inspected, and its inspection reliability has been of great concern. In order to improve the inspection reliability of manual UT, an assistance system can be employed, which records measurement data and recognizes the scan positions through a tracking system. This allows us not only to visualize the measurement data but also to process it further for the better imaging quality \cite{Krieg18SHMNDT}. \par

% Post-processing/ SAFT
Although their application to manual UT data has been typically excluded, several post-processing techniques have been developed and extensively employed for automatic UT data \cite{Hall88} \cite{Krautkraemer90} \cite{Ericsson98ECNDDT}. While the authors of \cite{Ericsson98ECNDDT} apply the signal processing methods widely used in the telecommunication field to the UT data, one of the well established post-processing method is the synthetic aperture focusing technique (SAFT) \cite{Hall88} \cite{Krautkraemer90}. The aim of SAFT is to improve the spatial resolution through performing superposition with respect to the propagation time delay \cite{Lingvall04PhD}. In other words, SAFT regards the measurement region-of-interest (ROI) as a single aperture and each measurement as its spatial sampling, which indicates that the SAFT reconstruction requires accurate positional information. \par

%% (2) Problem statement 
% Prob(b) -> Goal(b)
However, the application of such techniques to manual UT data is so far little studied \cite{Mayer16SAFTwithSmallData} \cite{Krieg18SHMNDT}. The authors of \cite{Krieg18SHMNDT} demonstrate the possibilities of utilizing post-processing with manual measurement data, yet the reconstruction quality is, compared to the reconstruction results of automatic measurement data, significantly degraded. Previously, we identified the possible error sources for such degradation and revealed that several systematic errors, such as varying contact pressure or inaccurate positional information due to the tracking error, can lead to strong artefacts formation \cite{Krieg19IUS}. Since such errors are inevitable in manual measurement, finding the way to reduce those artefacts could improve the reconstruction quality. Unlike other possible error factors which should be entirely estimated from the measurement data, the tracking error can be handled to some extent, as the positional information is, whether accurate or not, available. \par

%% (3) Respons
% Sol(B) + contributions
Our goal in this study is to reduce the position-inaccuracy induced artefacts by correcting the measurement positions and adjusting the reconstruction system accordingly. So far, we are unaware of any other works that deal with this topic. However, expressing the reconstruction process mathematically enables us to approximate the correct model with regard to measurement positions, whereas the traditional regression approaches provide us a tool to estimate the tracking error from the available information. As a possible solution for proper handling of positional inaccuracy, we propose an iterative method combining these mathematical tools, which can be incorporated into reconstruction system. \par
%
%\cite{Hall88} 
%In order to approximate the correct model, we derive the spatial approximation of the SAFT reconstruction matrix and implement an iterative method to estimate the positional error and correct the position for assuring the approximation quality. 
% expressing the reconstruction process mathematically enables us to approximate the correct model with regard to measurement positions
% Our goal in this study is to find a method to properly handle positional inaccuracy for SAFT reconstruction. 



%%%%%%%%%%%%%%%%%%%%%%%%%%%%%%%%%%%%%%%%%%%%%%%%%%%
%%% The main part of the paper
%%%%%%%%%%%%%%%%%%%%%%%%%%%%%%%%%%%%%%%%%%%%%%%%%%%
% Chapters are just an example, feel free to modify.

%%%%%%%%%%%%%%%%%%%%%%%%%%%%%%%%%%%%%%%%%%% sec.2 %%%%
\section{Data model}
% Dimension
% M = Nt = Nz
% Nx
% I = Ndefect 
% L = # of all possible scatters
% K = dimension of p (either 1 for 2D measurement, 2 for 3D measurement) 
\subsection{Pulse-Echo Setup} \label{sec:pulse_echo}
% Description on pulse-echo model
% General assumptions
For a measurement setup, we consider a manual contact testing where a handheld transducer is placed directly on the specimen surface at a position $\pp \in \RR^{\K}$.% as depicted in Fig. \ref{fig:pulseecho}. 
The transducer inserts an ultrasonic pulse $h(t)$ into a specimen and receives the reflected pulse, A-Scan,  $\ascan_{\pp} (t)$ at the same position $\pp$. The specimen is assumed to be homogenous and isotropic with the constant speed of sound $c_0$ and have a flat surface. During the measurement, the contact pressure is considered to be constant so that in the measurement data there is no temporal shift or amplitude change caused by improper coupling. The measurement position $\pp$ is arbitrarily selected on the specimen surface and we suppose that there is at least one scatterer inside the specimen, which is regarded as point source. \par

% Convolution model
The measured A-Scan $\ascan_{\pp} (t)$ can be considered as a convolution of the inserted pulse and the reflectivity of the specimen
\begin{equation} \label{eq:ascan_base}
\ascan_{\pp} (t) = \pulse (t) \ast \reflectivity (t) + n(t).
\end{equation}
$\reflectivity (t)$ denotes the reflectivity of the specimen and $n(t)$ the additive measurement noise, respectively, which is assumed to be zero-mean i.i.d. Gaussian noise with variance $\sigma_{N}^2$. \par

% Pulse model
Conventionally, the inserted pulse $\pulse (t)$ is modeled as a real-valued Gabor function \citep{GaborAsymmChirp}, as
\begin{equation} \label{eq:pulse}
\pulse (t) = e^{- \alpha t^2} \cdot \cos (2 \pi f_C t + \phi),
\end{equation}
where $f_C$, $\alpha$  and $\phi$ are the carrier frequency, the window width factor and the phase, respectively.\par

% Reflectivity as delta pulse
Since we consider the scatterers as point sources, the reflectivity $\reflectivity (t)$ can be expressed as a sum of time-shifted delta for all $\I$ scatterers as
\begin{equation} \label{eq:reflectivity}
\reflectivity (t; \tau) = \sum_{i = 1}^{I} \refcoeff_{\pp, i} \cdot \delta (t - \tau_{i}).
\end{equation}
$\refcoeff_{\pp, i}$ is the reflection coefficient for the position $\pp$ and a scatterer $s_i$, whereas $\tau_{i}$ is the time-of-flight (ToF) which the ultrasonic pulse needs to travel for way forth and back from $\pp$ to $s_i$. 
% ToF
The ToF can be obtained with 
\begin{equation} \label{eq:tof}
\tau_{i}(\pp) = \frac{2}{c_0} \cdot \norm{\scatterer_{i} - \pp }_{2},
\end{equation}
where $\norm{\scatterer_{i} - \pp }_{2}$ the $\ell$-2 norm of $\scatterer_{i}$ and $\pp$. Eq. \eqref{eq:tof} shows that the ToF depends on the position of both measurement and the scaterrer, resulting in the reflectivity as a function of time $t$ and position $\pp$ as well.  \par

% A-Scan = time-shifted pulse
By inserting \eqref{eq:reflectivity} into \eqref{eq:ascan_base}, we obtain the A-Scan as the time-shifted input pulse as
\begin{equation} \label{eq:ascan_conv}
\ascan (t; \pp) = \sum_{i = 1}^{I} \refcoeff_{\pp, i} \cdot \pulse (t - \tau_{i} (\pp) ) + n(t).
\end{equation}
Since we process the data digitally with the sampling interval of $\dt = \frac{1}{f_S}$, \eqref{eq:ascan_conv} becomes
\begin{equation} \label{eq:ascan_discrete}
\ascan (t; \pp) = \sum_{m = 1}^{\M} \sum_{i = 1}^{\I} \refcoeff_{\pp, i} \cdot \pulse (m \dt - \tau_{i} (\pp) ) + n(m \dt),
\end{equation}
where $\M$ is the number of temporal samples. \par

% Fig: measurement setup
%\begin{figure}
%\begin{center}
%\inputTikZ{0.8}{figures/pulse_echo_2D.tex}
%\caption{Measurement setup ***"Transducer" cannot be added to the figure $\rightarrow$ why??"***}
%\label{fig:pulseecho}
%\end{center}
%\end{figure}

\subsection{Forward Model with SAFT} \label{sec:fwm_saft}
%% FWM w/ SAFT
% General description on SAFT
The goal of SAFT reconstruction is to determine the location of scatterers in a solid test object. As \eqref{eq:tof} indicates, the distance between the scatterer and the measurement position, $\norm{\scatterer_{i} - \pp }_{2}$, can be obtained through the ToF, suggesting that $s_i$ is likely to be on the semicircle with $\pp$ in the center and the radius of $\norm{\scatterer_{i} - \pp }_{2}$. By computing such semicircles at different scan positions, we can specify the position of $s_i$. Based on this idea, SAFT extracts the ToF information, performs the superposition of the multiple measurement data and achieves a spatial focus \cite{Lingvall04PhD}. \par

% Linear transform
In order to compute SAFT efficiently, it is desirable to express \eqref{eq:ascan_discrete} as a linear transform. As \eqref{eq:ascan_discrete} demonstrates, the obtained A-Scan can be modeled as a sum of the time-shifted input pulse $\pulse (t)$, enabling to form a matrix $\SAFTp$ from the impulse response at the measurement position $\pp$ for all possible scatterer positions. We call $\SAFTp$ a SAFT matrix which is tied to the measurement position and has a dimension of $\RR^{\M \times \LL}$, when there are $\LL$ possible scatterer positions in our ROI. A column vector of the SAFT matrix $\pulsevec \in \RR^{\M}$ can be expressed as
\begin{equation} \label{eq:saft_colvec}
[\SAFTp]_{(:, l)} = \SAFTcol (\pp) = \sum_{m = 1}^{\M} \refcoeff_{\pp, l} \cdot \pulse (m \dt - \tau_{l} (\pp)),
\end{equation}
where $l$ is the column index which corresponds to the scatterer positions.
%
This allows us to rewrite \eqref{eq:ascan_discrete} as a linear transform
\begin{equation} \label{eq:saft_LT}
\ascanvec (\pp) = \SAFTp \cdot \defect + \noisevec = \ascanvechat (\pp) + \noisevec
\end{equation}
where $\ascanvec (\pp) \in \RR^{\M}$, $\ascanvechat (\pp) \in \RR^{\M}$ and $\noisevec \in \RR^{\M}$ are the A-Scan, its model and the measurement noise, respectively, as vector form and $\defect \in \RR^{\LL}$ is the vectorized "defect map" which represents the scatterer positions \cite{Kirchhof16IUS}.
%
When there is only one scatterer located at the $l$-th position of our ROI, each element of the vectorized defect map $b_{q}$ can be expressed as 
\begin{equation} \label{eq:defect_map}
b_{q} = 
\begin{cases}
\refcoeff_{\pp, l}, & \text{for } q = l\\
0 & \text{else}
\end{cases},
\end{equation}
where the index $q$ satisfies $q = 1$, $2$, ..., $\LL$. 
%
% The goal of SAFT
Consequently, SAFT reconstruction becomes the following optimization problem \cite{Kirchhof16IUS}
\begin{equation} \label{eq:saft_optimization}
\min_{\defect} \| \ascanvec (\pp) -  \SAFTp \cdot \defect \|_{2} .
\end{equation}



%%%%%%%%%%%%%%%%%%%%%%%%%%%%%%%%%%%%%%%%%%% sec.3 %%%%
\section{Methods} \label{sec:methods} 
%%% Methods %%%
% Short intro: probelm and introducing the notation: \pphat, \SAFTphat etc...
Since the inaccurate positional information can lead the SAFT matrix to fail in proper modeling of the measured A-scan, the quality of the SAFT reconstruction strongly depends on the accuracy of the input measurement positions. Suppose an A-Scan $\ascanvec (\pp)$ is taken at the position $\pp \in \RR^{\K}$, and the tracking system recognizes the measurement position as $\pphat = \pp + \ppdelta$ with the tracking error $\ppdelta$. Based on this falsely recognized measurement position $\pphat$, we compute a SAFT matrix $\SAFTphat$ which corresponds to an incorrect A-Scan model $\ascanvechat (\pphat)$. When the deviation between $\ascanvechat (\pphat)$ and the correct A-Scan model $\ascanvechat (\pp)$ is large, reconstructing the measurement data $\ascanvec (\pp)$ with the improper SAFT matrix $\SAFTphat$ results in a significant degradation of reconstruction quality. \par

% Our goal and solution
For improving the reconstruction quality despite the positional inaccuracy, it is essential to bridge the gap between the proper SAFT matrix $\SAFTp$ and the falsely computed matrix $\SAFTphat$, such that the resulting A-Scan model becomes very similar to the correct model. In order to minimize the deviation in A-Scan modeling, we aim to approximate the correct SAFT matrix from the available information, i.e. the measurement data $\ascanvec (\pp)$ and the positional information $\pphat$. This is done based on the first Taylor approximation of the function $f$ 
\begin{equation}
\begin{split}
f (u) & \approx f (v) + f' (v) \cdot (u - v) \\
        & \approx f (v) - f' (v) \cdot (v - u),
\end{split}
\end{equation} 
where $f(u)$ for $u$ near $v$ can be approximated with $f(v)$ and its derivative $f'(v)$ when $|u - v| \ll u$. \par

% Description of each subsection
In the following subsections, the spatial approximation of the SAFT matrix is firstly derived (Sec.\ref{sec:saft_approx}), and an iterative method, which estimates the tracking error $\ppdelta$ and improves the positional accuracy, is presented (Sec.\ref{sec:iterative_GD}). Although SAFT matrix is meant to be used for solving the optimization problem shown in \eqref{eq:saft_optimization}, in this section we will be focusing on modeling A-Scan properly by assuming that the position of scatterers, i.e. the defect map $\defect$, is known. Moreover, for the sake of simplicity we consider a noise free scenario, i.e. the obtained measurement data $\ascanvec (\pp)$ becomes identical to our A-Scan model $\ascanvechat (\pp)$. \par


%% Spatial approximation of SAFT
\subsection{Spatial Approximation of SAFT Matrix} \label{sec:saft_approx}
% non-linear transformation f_{i} = column of the SAFT matrix
In order to obtain a spatial approximation of SAFT matrix, we should model A-Scans in terms of measurement positions, which is, as \eqref{eq:pulse}, \eqref{eq:tof} and \eqref{eq:ascan_discrete} suggest, a non-linear transformation $f: \pp \in \RR^{\K} \rightarrow f (\pp) \in \RR^{\M}$. Yet, this non-linear transformation can be considered as a superposition of each scatterer position in the same manner we model an A-Scan in \eqref{eq:ascan_discrete}, resulting in the non-linear transformation for a single scatterer becoming identical to the corresponding column of the SAFT matrix as
\begin{equation} \label{eq:ascan_model}
f_{i} (\pp) = \SAFTcol (\pp) = \SAFTp \cdot \defect^{(l)}.
\end{equation}
$f_{i}$ denotes the non-linear transformation of the measurement position $\pp$ for the $i$-the scatterer which is located at the $l$-th position in our ROI and $\SAFTcol$ is the corresponding $l$-th column vector of the SAFT matrix. $\defect^{(l)}$ is a vectorized defect map containing only one non-zero element, e.g. 1, in the $l$-th row. \par

% Local linearity  and Jacobian
When we consider the positional error $\ppdelta$ in a small range, i.e. $\ppdelta \ll \pp$, $\SAFTcol$ becomes locally linear and can be linearly approximated as 
\begin{equation} \label{eq:local_linearity}
\SAFTcol (\pp) \approx \SAFTcol (\pp + \ppdelta) - \Jacobianpartial (\pp + \ppdelta) \cdot \ppdelta, 
\end{equation}
where $\Jacobianpartial \in \RR^{\M \times \K} $ is the Jacobian matrix of $\SAFTcol$, which can be expressed as
\begin{equation} \label{eq:jacobian_partial}
\Jacobianpartial (\pp) =  \left[ \frac{\partial \SAFTcol (\pp)}{\partial \pp} \right] .
\end{equation} \par

% Comprehensive Jacobian matrix
Since $\Jacobianpartial$ is associated with the $l$-th column of the SAFT matrix, we can form a comprehensive Jacobian matrix $\Jacobian \in \RR^{\M \LL \times \K}$, in the similar manner as \eqref{eq:saft_LT}, which contains the derivative of $\SAFTcol$ for all possible scatterer positions $\LL$ as
\begin{equation} \label{eq:jacobian_full}
\Jacobian (\pp) = \left[ {\Jacobian_{\pulsevec_{1}} (\pp) }^{\T} {\Jacobian_{\pulsevec_{2}} (\pp)}^{\T}  \text{...} {\Jacobian_{\pulsevec_{\LL}} (\pp)}^{\T}  \right]^{\T}.
\end{equation} 
%
% Matricize the resulting vector
Consequently, the inner product of $\Jacobian$ and $\ppdelta$ in \eqref{eq:local_linearity} yields a vector $\in \RR^{\M \LL}$, from which a matrix $\Deriv \in \RR^{\M \times \LL}$ can be formed with inverse $\vectorize$ operation as 
\begin{equation} \label{eq:deriv_matrix}
\begin{split}
\Deriv (\pphat ; \ppdelta) &= \vectorize^{-1}_{\M, \LL} \{ \Jacobian (\pphat) \cdot \ppdelta \} \\
                                           &= \left[ (\vectorize \{ \Identity_{\LL} \}^{\T} \otimes \Identity_{\M} \right] \cdot \left[ \Identity_{\LL} \otimes ( \Jacobian (\pphat)\cdot \ppdelta ) \right].
\end{split}
\end{equation}\par

% Matrix-vector product
Since $\Deriv$ has the same dimension as our SAFT matrix $\SAFT$, we can express $\SAFTcol$ by applying the same approach in \eqref{eq:saft_LT} to \eqref{eq:local_linearity} as 
\begin{equation} \label{eq:local_linearity_jacobianfull}
\SAFTcol (\pp) \approx \SAFTcol (\pphat) - \defectsingle_{l} \cdot \Derivcol (\pphat; \ppdelta),
\end{equation}
where $\defectsingle_{l}$ and $\Derivcol$ are the $l$-th element of $\defect$ and the $l$-th column vector of $\Deriv$, respectively.
%
Inserting \eqref{eq:ascan_model} and \eqref{eq:local_linearity_jacobianfull} into \eqref{eq:local_linearity} yields
\begin{equation}
\SAFTp \cdot \defect^{(l)} \approx \left[ \SAFT (\pphat) - \Deriv (\pphat ; \ppdelta ) \right] \cdot \defect^{(l)},
\end{equation}
which indicates that we can approximate the correct SAFT matrix through the falsely computed SAFT matrix as
\begin{equation} \label{eq:saft_approx}
\SAFTp \approx \SAFTphat - \Deriv (\pphat ; \ppdelta ).
\end{equation} \par
%
%%% Fig: position notations %%%
%\begin{figure}
%\begin{center}
%\inputTikZ{1}{figures/PositionNotations.tex}
%\caption{Position notations: the correct measurement position $\pp$, the falsely recognized position $\pphat$ and their deviation $\ppdelta$}
%\label{fig:position_notation}
%\end{center}
%\end{figure}


%% Iterative GD %%
\subsection{Iterative Position Correction} \label{sec:iterative_GD}
% Intro: problems
For properly approximating the correct SAFT matrix, \eqref{eq:saft_approx} indicates that we need the information about the tracking error $\ppdelta$. However, as we only know the falsely tracked position $\pphat$, the tracking error should be estimated from the obtained measurement data $\ascanvec (\pp)$ and the positional information $\pphat$. Moreover, although our approximation is less susceptive to the positional error than just calculating a SAFT matrix at the wrong position, the validity range of our approximation is limited. In order to tackle these problems, we incorporate an iterative method into our approximation process, so that we can estimate the tracking error $\ppdelta$ and improve the positional accuracy. In this subsection, we consider a simple measurement scenario where there is only one scatterer located at the $l$-th position in our ROI, i.e. the measured A-Scan $\ascan$ is identical to $\SAFTcol$, and measurements are taken along the x-axis.\par

% A-Scan approximation -> iterative process
Eq. \eqref{eq:local_linearity_jacobianfull} indicates that the tracking error can be estimated by comparing the measurement data and the falsely modeled A-Scan. Since the transducer is further assumed to be placed directly on the object surface with the constant contact pressure, the vertical component $z$ becomes 0 and remains constant, resulting in a measurement position becoming $\pp = [x, 0]$. This enables us to express both measured and modeled A-Scans as a function of $x$ and remove the tracking error $\ppdelta = \xdelta$ from $\vectorize^{-1}$ operator in \eqref{eq:deriv_matrix}. As a result, \eqref{eq:local_linearity_jacobianfull} can be formulated with the model derivative $\SAFTcoldot$ into a least squares problem as
\begin{equation} \label{eq:LS}
\min_{\xdelta} \| \SAFTcol (\xhat) - \SAFTcol (x) - \SAFTcoldot (\xhat) \cdot \xdelta \|_{2},
\end{equation}
through which we can obtain the estimated positional error $\xdeltaest$. \par

% Iterative process  
If we take into account this estimated error and update the positional information as $\xopt = \xhat - \xdeltaest$, we can further reduce the positional error. With the improved scan position $\xopt$, \eqref{eq:LS} can be solved again, which provides a \textit{new} estimated error, realizing the better approximation than the previous one with $\xhat$. In other words, repeating this procedure can improve the positional accuracy and consequently realize the reliable approximation for the SAFT matrix in \eqref{eq:saft_approx}. \par

% Description of the iteration
This iterative process is depicted in Fig. \ref{fig:blockdiagram_GD}. The breaking condition is either (a) the squared error of the approximated A-Scan $\aopt (\xopt; \xdeltaest)$ compared to the measured A-Scan $\ascanvec (x)$ (in our scenario, identical to the correct model $\ascanvechat (x)$) reaches the given target value or (b) the maximal number of iteration is carried out. After the iteration break, $\xopt$ and $\xdeltaest$ are returned as output. 


%%% Fig: block diagram -> should I simplify the diagram? only upto the break condition? 
\begin{figure}
\begin{center}
\inputTikZ{1.2}{figures/blockdiagram_iterativeGD.tex}
\caption{Block diagram of the iterative method to estimate and improve the positional error}
\label{fig:blockdiagram_GD} 
\end{center}
\end{figure}











%%%%%%%%%%%%%%%%%%%%%%%%%%%%%%%%%%%%%%%%%%% sec.4 %%%%
\section{Simulation \rom{1}: Iterative Position Correction} \label{sec:simulation_GD} 
%
%This section may for instance be a ``simulations'' section if the verification is based
%on computer simulations. In this case, please remember to indicate the simulations setup as completely
%as possible, including all assumptions that were made (e.g., uncorrelated Rayleigh fading
%channels or isotropic antenna elements). The reader should be able to reproduce your 
%simulation results based on your descriptions!
%
%Similarly, if measurements were performed, the measurement setup and equipment 
%as well as your test conditions must be described in detail and the outcome should
%be discussed.
%
%The description of your main results which may of course also span more than one section.
%Point out its main features, discuss its limitations, compare it to alternative solutions.
%
%All results that you show should be interpreted properly -- what do we learn from them?
%
%%% simulation %%%
% Short intro
Performance of the proposed method was examined through simulations where we applied the algorithm presented in \ref{sec:iterative_GD} to simulated data sets. The goal of the simulations was to illustrate the error sensitivity of the proposed method, which can ultimately lead us to determine how we should incorporate our method into reconstruction process.

% Parameters
\subsection{Assumptions and Test Parameters} 
For the simulations we chose an aluminum object for which we set the same assumptions as we described in Sec.\ref{sec:pulse_echo}. For the sake of simplicity, the measurement data is regarded as noise free. Our ROI contains one scatterer and is a part of the test object where back and side wall echoes can be neglected. In order to illuminate the position-dependency of the results, the transducer is assumed to be ominidirectional. Table \ref{tab:params}  provides a summary of the test parameters.

% Param table
\begin{table}
\begin{center}
%%%%%% table of the constant parameters (EN)
\begin{tabular}{ | c | c | } 
\hline
Parameter & Value \\
\hline 
% N: dimension of the A-Scan
Measurement size $\N$ & 220\\
\hline
% K: sparsity
Sparsity $\K$ & 2\\
\hline
% M: dimension of the kernel
Rows in kernel $\M$ & 50\\
\hline
% fS
Sampling frequency $f_S$ & \SI{100}{\hertz} \\
\hline
% f_C
Carrier frequency $f_{C}$ & \SI{20}{\hertz}\\
\hline
% bandwidth factor alpha
Gabor pulse bandwidth $\B$ & $0.01$ (\SI{}{\hertz})$^{-2}$ \\
\hline
\end{tabular}


\caption{Summary of the test parameters for the simulations}
\label{tab:params}
\end{center}
\end{table}

% Fig: positions
\begin{figure}
\centering
\inputTikZ{0.8}{figures/GD_ScanPositions.tex}
\caption{Illustration of the measurement positions used for the simulation \rom{1} in relation to the scatterer position $s_{x}$}
\label{fig:GD_positions}
\setlength{\belowcaptionskip}{-10pt} % reduces the sapces b/w figures & captions
\end{figure}

% Criterion
\subsection{Evaluation Criterion and Variables}
In order to evaluate the simulation results, we chose two criterion: the position correction and the approximation quality. The position correction $\Delta x_{\optimized} = x - \xopt$ shows how close we can correct the position $\xhat$ through the proposed method. The approximation quality is assessed with the modified squared error $\SEdag$ of the approximated A-Scan $\aopt (\xopt; \xdeltaest)$ compared to the measurement data $\ascanvec (x)$. $\SEdag$ can be expressed as
\begin{equation}
\SEdag = \frac{ \| \gamma \aopt - \ascanvec \|_{2}}{\| \ascanvec \|_{2}}, 
\end{equation}
where $\gamma$ is a normalization factor obtained through
\begin{equation}
\gamma = \frac{\ascanvec^{\T} \cdot \aopt}{\aopt^{\T} \cdot \aopt}.
\end{equation} \par

% Variables
As we aimed to illustrate the error sensitivity of our method, the simulations were carried out with different tracking error $\xdelta$. Furthermore, since the measurement data is tied to its scan position, the error sensitivity is expected to vary with scan position $x$, which we set as our second variable in the simulations. Considering the symmetry in the measurement setup, we used the \textit{scatterer-scan} distance, $| s_{x} - x|$, as a variable for $x$. Fig. \ref{fig:GD_positions} illustrates the measurement positions used in simulations. For the break condition, we set our target to $\SEdag = 0.01$ and the maximal number of iterations to 15. 


\subsection{Results} 
%% Text 
% General
Fig. \ref{fig:results_PE} and Fig. \ref{fig:results_SE} show the positional error correction and the squared error of the approximated A-Scan for four different measurement positions depicted in Fig. \ref{fig:GD_positions}. In general, position-dependency can be well observed, and Fig. \ref{fig:results_SE} shows that our approximation can tolerate the positional error for certain range. \par

% 0.5 lambda away
When the measurement position is only \SI{0.63}{\milli \metre}, which is equivalent to half wavelength, away from the scatterer, the error up to $- 0.5 \lambda$ can be well corrected, leading to successful approximations. However, when the error is within the range of $- 0.5 \lambda$ and $1.5 \lambda$, the error is not corrected. This is because there is almost no difference in the falsely computed A-Scan and the correct model. This no-correction range is related to the symmetry in the measurement setup, which results in the doubled width in the range. That is to say, in order for the error correction to successfully function, the deviation in A-Scan between the correct model and its falsely computed counterpart should be large enough. On the other hand, as the deviation in A-Scan is very little within the no-correlation range, the correct model can be very well approximated even without error correction. For the error above $0.5 \lambda$ the difference in A-Scan becomes larger, prompting the error correction. Yet, the position is corrected to the opposite side of the measurement position with respect to the scatterer, where we can obtain the identical A-Scan model. \par

% 1mm away
Likewise, for the measurement position which is \SI{1}{\milli \metre} away from the scatterer we can observe both the correction and the no-correction range, and the both results show very similar progression, except two points. One is the "sudden" improvement in the error correction, when the error is equal to $0.76 \lambda$. This is because the scan position is now located further away from the scatterer than the previous result, making A-Scan modeling more sensitive to the error. As a result, the deviation in A-Scans becomes large enough to prompt the error correction. The other is the "sudden" worsening in the approximation with the error of $0.8 \lambda$, where the falsely tracked position is directly above the scatterer, i.e. $\xhat = s_{x}$. When an A-Scan is modeled directly above the scatterer, the progression in error sensitivity of A-Scan modeling becomes convex  as shown in Fig. \ref{fig:se_offset}. Solving the least squares problem \eqref{eq:LS} based on $\xhat = s_{x}$ "optimizes" the position to $s_{x}$, failing to correct $\xhat$ to $x$. Since A-Scan modeling is more sensitive to the error than the previous result, the correct A-Scan model can be no longer well approximated without correcting the error. \par

% 2.5mm, 5mm away
On the contrary, as the measurement position moves away from the scatterer, the change in position results in the larger deviation in A-Scan, making the no-correction range negligible. Consequently, the error can be well corrected within the certain correction range. This correction range narrows with the increasing distance between the scatterer and the measurement position, since our approximation becomes more susceptive to the positional error. We've found that the positional error can be successfully corrected up to the first local minima after the global minimum (Fig. \ref{fig:se_offset}). In case of $| s_{x} - x| =$ \SI{2.5}{\milli \metre}, which is equivalent to $1.98 \lambda$, we can also observe that the error correction impairs when $\xhat$ approaches to $s_{x}$. \par

%% Figures: GD PE and GD SE
\begin{figure}
\centering
% Fig: PE
\begin{subfigure}[T]{0.5\textwidth}
	\caption{ } 
	\label{fig:results_PE}
	\gdpe{1}{\scriptsize}{\scriptsize}{figures/pytikz/1D/coordinates/errmax_5lambda/gd_pe_halflambda_away.tex}{figures/pytikz/1D/coordinates/errmax_5lambda/gd_pe_1mm_away.tex}{figures/pytikz/1D/coordinates/errmax_5lambda/gd_pe_2_5mm_away.tex}{figures/pytikz/1D/coordinates/errmax_5lambda/gd_pe_5mm_away.tex}{figures/pytikz/1D/coordinates/errmax_5lambda/gd_pe_7_5mm_away.tex}
 % <scale size>, <label font size>, <tick font size>, <fname for 0.5 lambda>, <fname for 1mm>, <fname for 2.5mm>, <fname for 5mm>, <fname for 7.5mm>
\end{subfigure}
%
% Fig: SE
\begin{subfigure}[T]{0.5\textwidth}
	\caption{ }
	\label{fig:results_SE}
	\gdse{1}{\scriptsize}{\scriptsize}{figures/pytikz/1D/coordinates/errmax_5lambda/gd_se_halflambda_away.tex}{figures/pytikz/1D/coordinates/errmax_5lambda/gd_se_1mm_away.tex}{figures/pytikz/1D/coordinates/errmax_5lambda/gd_se_2_5mm_away.tex}{figures/pytikz/1D/coordinates/errmax_5lambda/gd_se_5mm_away.tex}{figures/pytikz/1D/coordinates/errmax_5lambda/gd_se_7_5mm_away.tex}
 % <scale size>, <label font size>, <tick font size>, <fname for 0.5 lambda>, <fname for 1mm>, <fname for 2.5mm>, <fname for 5mm>, <fname for 7.5mm>
\end{subfigure}
%
\caption{Results obtained with simulation \rom{1}: (a) position correction normalized with the wavelength $\lambda$ and (b) normalized squared error of the approximated A-Scans compared to their correct models}
\end{figure}
%
%% Figures : SE offset
\begin{figure}
\setlength{\abovecaptionskip}{-10pt} % reduces the sapces b/w figures & captions
\centering
\seoffset{1}{\scriptsize}{\scriptsize}{figures/pytikz/1D/coordinates/se_offset/se_offset_0mm_away.tex}{figures/pytikz/1D/coordinates/se_offset/se_offset_2_5mm_away.tex}{figures/pytikz/1D/coordinates/se_offset/se_offset_5mm_away.tex}{figures/pytikz/1D/coordinates/se_offset/se_offset_7_5mm_away.tex}
%<scale size>, <label font size>, <tick font size>, <fname for xdelta = 0mm>, <fname for xdelta = 2.5mm>, <fname for xdelta = 5mm>
\caption{Error sensitivity of A-Scan modeling at different scan positions} %
\label{fig:se_offset} 
\end{figure}

%% interpretation
Above all, the obtained results show that the proposed method could improve the accuracy of A-scan modeling with the inaccurate positional information, which ultimately leads us to a better SAFT reconstruction. In order to assure the reconstruction quality, we should find the proper countermeasures for the no-correction range and the error sensitivity which increases with the measurement-scatterer distance. \par

For the no-correction range, we could set the threshold $\SEdag_{\thres}$ to initiate the error correction. Since the deviation in A-Scan is very little within the no-correction range, there is very likely no need for the approximation. When $\SEdag$ of the falsely computed A-scan is smaller than $\SEdag_{\thres}$, then the modeled A-Scan, as well as the corresponding SAFT matrix, will remain unchanged, which reduces the overall computational time. When $\SEdag > \SEdag_{\thres}$ and $\SEdag$ of the resulting approximation is not well improved, which is very likely the case where the tracked position is directly above the scatterer, then this data can be discarded.\par

In order to tackle the increasing error sensitivity with the scatterer-scan distance, it is desirable to suppress the contribution of the measurement data taken at the position far away from the scatterer.For this purpose, we can apply a spatial filter which varies the reflection coefficients according to the positions. A proper filter can be selected through comparing the measurement data with its neighboring data. When the change in the measurement data is large, which indicates that the position is located far away from the scatterer, we can accordingly choose the smaller reflection coefficient. In fact, transducers employed in real measurements have a limited angular sensitivity range which can be regarded as a form of spatial filters. \par
 

%%%%%%%%%%%%%%%%%%%%%%%%%%%%%%%%%%%%%%%%%%% sec.5 %%%%
\section{Simulation \rom{2}: Reconstruction with Iterative Position Correction} \label{sec:simulation_reco} 
%
%%% Simulation 2 : example reco
%% Text
% Assumptions & test parameters
\subsection{Simulation Setup}
In order to verify the proposed method, an example measurement data set is reconstructed. The same assumptions as the simulation \rom{1} are made, and a summary of the test parameters are provided in Table \ref{tab:params}. We selected the transducer opening angle as a substitution of a spatial filter, which makes the scatterer \textit{invisible} from the scan positions more than \SI{5}{\milli \metre} away from the scatterer. Each measurement data is considered to be taken at a measurement grid point. The tracking system is, on the other hand, assumed to be capable of providing the positional information between two grid points. With that said, the tracking error is regarded not as quantization error but as mere false recognition due to, for instance, the sudden move of the transducer. Initially, the SAFT matrix is pre-calculated for all measurement grid points based on \textit{a priori} knowledge regarding the test object.\par
The simulations are carried out with the following steps: first, the measurement data is taken at a particular grid point $x$ which is recognized as $\xhat$. When the reconstruction process is accompanied with position correction, $\xhat$ is corrected to $\xopt$, and the SAFT matrix is modified accordingly, whereas without correction both $\xhat$ and the SAFT matrix remain same. Then, either $\xhat$ or $\xopt$ is rounded to the nearest grid point at which the obtained measurement data is stored in the system. After completing the measurement, the stored data is reconstructed. 

% Results
\subsection{Results}
Fig. \ref{fig:results_reco} shows the simulation results. As a reference, the measurement data (Fig. \ref{fig:bscan_ref}) is stored in the system at the correct positions and reconstructed with the pre-calculated SAFT reconstruction matrix (Fig. \ref{fig:reco_ref}). When the positional information is incorrect and, as a result, A-Scans are assigned to the false positions to be stored in the system, the resulting data becomes incomplete and incoherent (Fig. \ref{fig:bscan_track}), which degrades the quality of the reconstruction with the given SAFT matrix (Fig. \ref{fig:reco_track}). However, when the tracked positions are corrected with the proposed method and the SAFT matrix is modified, more A-Scans are assigned to the correct positions (Fig. \ref{fig:bscan_opt}), and the reconstruction quality is significantly improved (Fig. \ref{fig:reco_opt}). 

%% Figure
\begin{figure}
\setlength{\abovecaptionskip}{-5pt} % reduces the sapces b/w figures & captions
\setlength{\belowcaptionskip}{-5pt} % reduces the sapces b/w figures & captions
\input{figures/pytikz2D_results_data_reco.tex }
\caption{Results obtained with simulation \rom{2}: (a) measurement data at the correct positions, (b) reference reconstruction, (c) measurement data according to the tracked positions with the tracking error of \SI{1.26}{\milli \metre}, (d) reconstruction of the measurement data, (e) measurement data after position correction, (f) reconstruction of the  position-corrected measurement data}
\label{fig:results_reco} 
\end{figure} 

%%%%%%%%%%%%%%%%%%%%%%%%%%%%%%%%%%%%%%%%%%%%%%%%%%%
%%% The conclusions
%%%%%%%%%%%%%%%%%%%%%%%%%%%%%%%%%%%%%%%%%%%%%%%%%%%

\section{Conclusions} \label{sec:conclusions}
In this study, a possible solution to handle positional inaccuracy for SAFT reconstruction was presented. We have derived the spatial approximation of SAFT reconstruction matrix and proposed an iterative process to improve positional accuracy. The performance of the proposed method was examined through simulations, revealing two major problems to overcome for incorporating this approach into reconstruction process. First, for successful error correction the deviation between the falsely computed A-Scan model and the measurement data should be large enough. This suggests that we should initiate error correction, only when the deviation in A-Scan is large enough. Second, the proposed method becomes susceptive to positional error when the scan position moves away from a scatterer. As a countermeasure, we can apply a spatial filter to suppress the contribution of the measurement data taken far away from the scatterer. Based on these findings, an example measurement data set was reconstructed, in which artefacts are significantly reduced with the proposed method. 

%%%%%%%%%%%%%%%%%%%%%%%%%%%%%%%%%%%%%%%%%%%%%%%%%%%
%%% References
%%%%%%%%%%%%%%%%%%%%%%%%%%%%%%%%%%%%%%%%%%%%%%%%%%%

\small \bibliography{main}

\end{document}


% Some more hints on formatting and avoiding common mistakes can be found here:
%    http://www.ieee-icc.org/2008/template.pdf