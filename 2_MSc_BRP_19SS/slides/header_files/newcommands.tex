% "newcommand" collections
% important!!!! no numbers in the name of commands!!!!!!!!

%=====================  for math =======================%
% symbols
% scan positions
\newcommand{\pp}{\bm{p}}
\newcommand{\pphat}{\bm{\hat{p}}}
\newcommand{\ppdelta}{\Delta \pp}
\newcommand{\xdelta}{\Delta x}
\newcommand{\xhat}{\hat{x}}
\newcommand{\xopt}{\hat{x}_{\optimized}}
\newcommand{\xdeltaest}{\xdelta_{\estimated}}
\newcommand{\xdeltaopt}{\xdelta_{\optimized}}
% scatter position
\newcommand{\scatterer}{\bm{s}}
% A-Scans
\newcommand{\ascan}{a}
\newcommand{\ascanvec}{\bm{a}}
\newcommand{\ascanvechat}{\bm{\hat{a}}}
\newcommand{\adothat}{\dot{\bm{\hat{a}}}}
\newcommand{\aopt}{\bm{\tilde{a}}}
\newcommand{\Ascan}{\bm{A}}
% Pulse
\newcommand{\pulse}{h}
\newcommand{\pulsevec}{\bm{h}}
\newcommand{\pulsevecdot}{\bm{h}'}
% SAFT matrix
\newcommand{\SAFT}{\bm{H}}
\newcommand{\SAFTp}{\SAFT (\pp)}
\newcommand{\SAFTx}{\SAFT (x)}
\newcommand{\SAFTphat}{\SAFT (\pphat)}
\newcommand{\SAFTdot}{\SAFT'}
\newcommand{\SAFTdotphat}{\SAFTdot (\pphat)}
\newcommand{\SAFTcomp}{\bm{\mathcal{H}}} 
\newcommand{\SAFTcol}{\pulsevec_{l}}
\newcommand{\SAFTcoldot}{\pulsevec'_{l}}

% Jacobian relevent
\newcommand{\Jacobian}{\bm{J}} 
\newcommand{\Jacobianpartial}{\bm{J}_{\SAFTcol}}
\newcommand{\Deriv}{\bm{D}} 
\newcommand{\Derivcol}{\bm{d}_{l}} 
% else
\newcommand{\reflectivity}{r}
\newcommand{\refcoeff}{\beta}
\newcommand{\defect}{\bm{b}} % should be modified? 
\newcommand{\defectsingle}{b} % should be modified? 
\newcommand{\noisevec}{\bm{n}} 
\newcommand{\fhatpartial}{\hat{f}_{i}}
\newcommand{\taux}{\tau_{l} (x)}
\newcommand{\gipx}{g_{\ip} (x)}
\newcommand{\gqrx}{g_{\qr} (x)}
\newcommand{\Identity}{\bm{I}} 
\newcommand{\SEdag}{\SE^{\dagger}} 


% some special characters
\newcommand{\ii}{\mathrm{i}}
\newcommand{\dd}{\mathrm{d}}
\newcommand{\ee}{\mathrm{e}}
\newcommand{\dirac}{\delta}

\newcommand{\RR}{{\mathbb{R}}}
\newcommand{\NN}{{\mathbb{N}}}
\newcommand{\CC}{{\mathbb{C}}}
\newcommand{\OO}{{\mathcal{O}}}


\newcommand{\Real}{%  
     \operatorname{Re}
}
\newcommand{\Imag}{%  
     \operatorname{Im}
}


% for two norms (NEW 01.12.18 cf. https://tex.stackexchange.com/questions/107186/how-to-write-norm-which-adjusts-its-size)
\newcommand\norm[1]{\left\lVert#1\right\rVert }

%%%%%%%%%%%%%%%%%%%%%%%%%%%%%%%%%%%%%%%%%%%%%%
%======================================== table macro ====%
%%%%%%%%%%%%%%%%%%%%%%%%%%%%%%%%%%%%%%%%%%%%%%
% table scaling
\newcommand{\inputTable}[2]{%  
	\resizebox{#1}{!}{%
	 \input{#2}
	}
}


%%%%%%%%%%%%%%%%%%%%%%%%%%%%%%%%%%%%%%%%%%%%%%
%========================================== for TikZ  ====%
%%%%%%%%%%%%%%%%%%%%%%%%%%%%%%%%%%%%%%%%%%%%%%
\newcommand{\Nx}{3}
\newcommand{\Ny}{2}
\newcommand{\ddx}{1.5cm}
\newcommand{\ddy}{\ddx}
\newcommand{\ddz}{1.5cm}
\newcommand{\rCircle}{0.11cm}
\newcommand{\rCircleCamera}{0.07cm}

%%% draw transducer %%%
%5 3D
\newcommand{\drawTransducer}[4]{ % scale,scaley, x, y in axis
	\draw[draw = none, fill = white] (axis cs: #3 - #1, #4 - #2, 0) -- (axis cs: #3 + #1, #4 - #2, 0) -- (axis cs: #3 + #1, #4 + #2, 0) --  (axis cs: #3 + #1, #4 + #2, -#1) -- (axis cs: #3 - #1, #4 + #2, -#1) -- (axis cs: #3 - #1, #4 - #2, -#1) -- (axis cs: #3 - #1, #4 - #2, 0);
	\draw[] (axis cs: #3 - #1, #4 - #2, 0) -- (axis cs: #3 + #1, #4 - #2, 0) -- (axis cs: #3 + #1, #4 - #2, -#1) -- (axis cs: #3 - #1, #4 - #2, -#1) -- (axis cs: #3 - #1, #4 - #2, 0);
	\draw[] (axis cs: #3 + #1, #4 - #2, 0) -- (axis cs: #3 + #1, #4 + #2, 0) -- (axis cs: #3 + #1, #4 +  #2, -#1) -- (axis cs: #3 - #1, #4 +  #2, -#1) -- (axis cs: #3 - #1, #4 - #2, -#1);
	\draw[] (axis cs: #3 + #1, #4 - #2, -#1) -- (axis cs: #3 + #1, #4 +  #2, -#1);
}

%% 2D
\newcommand{\drawTransducerTwoD}[2]{ % scale, x  1,3
	\draw[draw = none, fill = white] (axis cs: #2 - #1, 0) -- (axis cs: #2 + #1, 0) --  (axis cs: #2 + #1, -#1) -- (axis cs: #2 - #1, -#1) -- (axis cs: #2 - #1, 0);
	\draw[] (axis cs: #2 - #1, 0) -- (axis cs: #2 + #1, 0) -- (axis cs: #2 + #1, -#1) -- (axis cs: #2 - #1, -#1) -- (axis cs: #2 - #1, 0);
	\draw[] (axis cs: #2 + #1, 0) -- (axis cs: #2 + #1, -#1) -- (axis cs: #2 - #1, -#1);
}

%%% draw video camera %%%
%% 3D
\newcommand{\drawCamera}[6]{ %<rotation origin (x, y, z)>, <start x>, <start y>, <start z> in axis, <camera width [dx]>, <camera height [dz]>
	% base
	\draw[rotate around = {30 : (axis cs: #1)}] (axis cs: #2, #3, #4) rectangle (axis cs: #2 + #5, #3, #4 - #6) ;
	% origin of rotation
	\node[campoint] (rotationorg) at (axis cs: #1) {};
	% "trapezoid" part (tip of the camera)
	\draw[rotate around = {30 : (axis cs: #1)}] (axis cs: #2, #3, #4 - 0.25*#6) -- (axis cs: #2 - 0.5* #6, #3, #4) -- (axis cs: #2 - 0.5* #6, #3, #4-#6) -- (axis cs: #2, #3, #4 - 0.75*#6);	
	% cable
	\draw[rotate around = {30 : (axis cs: #1)}] (axis cs: #2 + #5, #3, #4 - 0.5*  #6) .. controls (axis cs: #2 + 1.5* #5, #3, #4 - 0.5*  #6) and (axis cs: #2 + #5, #3, - 0.5*  #6) .. (axis cs: #2 + 1.5* #5, #3, - #6);	
}

%% 2D
\newcommand{\drawCameraTwoD}[5]{ %<rotation origin (x, z)>, <start x>, <start y>, <start z> in axis, <camera width [dx]>, <camera height [dz]> 1, 2, 4, 5, 6 -> 4-> 3, 5->4, 6-> 5
	% base
	\draw[rotate around = {30 : (axis cs: #1)}] (axis cs: #2, #3) rectangle (axis cs: #2 + #4, #3 - #5) ;
	% origin of rotation
	\node[campoint] (rotationorg) at (axis cs: #1) {};
	% "trapezoid" part (tip of the camera)
	\draw[rotate around = {30 : (axis cs: #1)}] (axis cs: #2, #3 - 0.25*#5) -- (axis cs: #2 - 0.5* #5, #3) -- (axis cs: #2 - 0.5* #5, #3-#5) -- (axis cs: #2, #3 - 0.75*#5);	
	% cable
	\draw[rotate around = {30 : (axis cs: #1)}] (axis cs: #2 + #4, #3 - 0.5*  #5) .. controls (axis cs: #2 + 1.5* #4, #3 - 0.5*  #5) and (axis cs: #2 + #4, - 0.5*  #5) .. (axis cs: #2 + 1.5* #4, - #5);	
}

%%%	 SAFT dictionary %%%
% matrix separations
\newcommand{\linewidththick}{0.1}
\newcommand{\linewidththin}{0.1}
\newcommand{\matrixspacesmall}{0.2}
\newcommand{\matrixspacebig}{0.35}
% single element
\newcommand{\matrixwidth}{2}
\newcommand{\matrixheight}{1.5}
\newcommand{\vectorwidth}{0.5}

% result vector
\newcommand{\resultcolorone}{orange}
\newcommand{\resultcolortwo}{green}

% matrix -1
\newcommand{\matrixonecolorone}{cyan}
\newcommand{\matrixonecolortwo}{yellow}

% matrix 0
\newcommand{\matrixtwocolorone}{violet}
\newcommand{\matrixtwocolortwo}{orange}

% matrix 1
\newcommand{\matrixthreecolorone}{blue}
\newcommand{\matrixthreecolortwo}{green}

% data vector
\newcommand{\datacolorone}{black}
\newcommand{\datacolortwo}{white}


%%%%%%%%%%%%%%%%%%%%%%%%%%%%%%%%%%%%%%%%%%%%%%
%======================================= image macro ====%
%%%%%%%%%%%%%%%%%%%%%%%%%%%%%%%%%%%%%%%%%%%%%%
% TikZ scaling
\newcommand{\inputTikZ}[2]{%<scaling factor>, <name of the tex file>
     \scalebox{#1}{\input{#2}}  
}


\newcommand{\measanimate}[5]{ % <scale size>, <slide page for base pulse>, <slide page for highlighting the pulse 1mm away>, <slide page for highlighting the pulse 2.5mm away>, <slide page for highlighting the pulse 5mm away>
	\scalebox{#1}{
		\begin{tikzpicture}
			\begin{axis}[
					width=6cm, height=6.3cm,  at={(0.7cm,0.3cm)},
					ticks=none, axis lines = center, 
					xmin=-0.5, xmax=10.5, ymin=-0.45, ymax=10,
					xlabel={$x$}, ylabel={$t$},
					y dir=reverse,
					x label style={at={(axis cs: 10.5, 0)}, anchor= west},
					y label style={at={(axis cs: -0.6, 10)}, anchor = north},
			        ]
			        
			        % Defect
			        \node at (axis cs: 5, 4) {\pgftext{\includegraphics[scale=0.15]{images/defect}}};
					% Pulse
					\only<#2>{
					\input{figures/pytikz/1D/coordinates/pulse/pulse_1_blue.tex}
					\input{figures/pytikz/1D/coordinates/pulse/pulse_2_blue.tex}
					\input{figures/pytikz/1D/coordinates/pulse/pulse_3_blue.tex}
					\input{figures/pytikz/1D/coordinates/pulse/pulse_4_blue.tex}
					\input{figures/pytikz/1D/coordinates/pulse/pulse_5_blue.tex}
					\input{figures/pytikz/1D/coordinates/pulse/pulse_6_blue.tex}
					\input{figures/pytikz/1D/coordinates/pulse/pulse_7_blue.tex}
					\input{figures/pytikz/1D/coordinates/pulse/pulse_8_blue.tex}
					\input{figures/pytikz/1D/coordinates/pulse/pulse_9_blue.tex}
					}
					
					% Highlight
					\only<#3>{\input{figures/pytikz/1D/coordinates/pulse/pulse_4_highlight.tex}} %1 mm away
					\only<#4>{\input{figures/pytikz/1D/coordinates/pulse/pulse_3_highlight.tex}} % 2.5mm away
					\only<#5>{\input{figures/pytikz/1D/coordinates/pulse/pulse_1_highlight.tex}} % 5mm away
			
			\end{axis}			
		\end{tikzpicture}
	}
}


%%%%%%%%%%%%%%%%%%%%%%%%%%%%%%%%%%%%%%%%%%%%%%%%
%===================== 1D Visualization ======================%
%%%%%%%%%%%%%%%%%%%%%%%%%%%%%%%%%%%%%%%%%%%%%%%%

% SE offset 
\newcommand{\seoffset}[4]{ % <scale size>, <label font size>, <tick font size>, <fname for offset coord>, <fname for approx coord>
\scalebox{#1}{
	\begin{tikzpicture}
            \begin{axis}[
                width = 12cm, height = 5cm,
            	   xmin = -2.2, xmax = 2.2,
            	   ymin = -0.2, ymax = 1.2,
                xlabel = {$\xdelta / \lambda$},
                ylabel = {$\SE^{\dagger}$},
                label style = {font = #2},
                tick label style = {font = #3},
                %y dir = reverse,
                %xtick = {0, 10, ..., 30}, %to customize the axis
                %xticklabel = {0, 5, 10, 15}
                ]
                \input{#4}            
            \end{axis}
	\end{tikzpicture}
	}
}



% SE offset vs approx
\newcommand{\seoffsetapprox}[5]{ % <scale size>, <label font size>, <tick font size>, <fname for offset coord>, <fname for approx coord>
\scalebox{#1}{
	\begin{tikzpicture}
            \begin{axis}[
                width = 12cm, 
            	   height = 6cm,
                xlabel = {$\ppdelta_{\track} / \lambda$},
                ylabel = {$\SE^{\dagger}$},
                label style = {font = #2},
                tick label style = {font = #3},
                %y dir = reverse,
                %xtick = {0, 10, ..., 30}, %to customize the axis
                %xticklabel = {0, 5, 10, 15}
                extra y ticks={0.22},
                extra y tick labels = {$\SE^{\dagger}_{\threshold}$}, 
                extra y tick style={font = #2},
                legend style ={
                	at={(1.25, 0.6)},
                	nodes={scale=0.85, transform shape},
                	font = #3
                }
                ]
                \input{#4}
                \input{#5}
              
             % mse = 0.25 line
             \addplot[gray, dashed, line width = 3pt, mark = ] coordinates{
            			(-2.0, 0.22)
            			(2.0, 0.22)
            };  
              % legend
           	 %\addlegendentry{$\N_{\point}$}
            	 \addlegendentry{Model}
              \addlegendentry{Approx.}
                         
            \end{axis}
	\end{tikzpicture}
	}
}

% GD PE w/ legend
\newcommand{\gdpe}[8]{ % <scale size>, <label font size>, <tick font size>, <fname for offset coord>, <fname for approx coord>
\scalebox{#1}{
	\begin{tikzpicture}
            \begin{axis}[
                width = 12cm, 
            	   height = 6.3cm,
                xlabel = {$\xdelta / \lambda$},
                ylabel = {$\xdeltaopt / \lambda$}, % \Delta \pp_{\optimized}
                label style = {font = #2},
                tick label style = {font = #3},
                %y dir = reverse,
                %xtick = {0, 10, ..., 30}, %to customize the axis
                %xticklabel = {0, 5, 10, 15}
                legend style ={
                	at={(1.25, 0.8)},
                	nodes={scale=0.85, transform shape},
                	font = #3
                }
                ]
              % Reverse the input order, so that 7.5mm away is sent to background
                \input{#8} 
                \input{#7}
                \input{#6}
                \input{#5}
                \input{#4}
                
             % legend
             % To insert the legend title
		   	\addlegendimage{empty legend} 
		   	% Legend entries  
             \addlegendentry{\SI{7.5}{\milli\metre}}
             \addlegendentry{\SI{5}{\milli\metre}}
             \addlegendentry{$1.98 \lambda$}
             \addlegendentry{$0.8 \lambda$}
             \addlegendentry{$0.5 \lambda$}
             % Title
             \addlegendentry{$| s_{x} - x |$}
             
            \end{axis}
	\end{tikzpicture}
	}
}


% GD SE w/ legend
\newcommand{\gdse}[8]{ % <scale size>, <label font size>, <tick font size>, <fname for offset coord>, <fname for approx coord>
\scalebox{#1}{
	\begin{tikzpicture}
            \begin{axis}[
                width = 12cm, 
            	   height = 6.3cm,
            	   ymin = 0,
            	   ymax = 1.1,
                xlabel = {$\xdelta / \lambda$},
                ylabel = {$\SE^{\dagger}$},
                label style = {font = #2},
                tick label style = {font = #3},
                %y dir = reverse,
                %xtick = {0, 10, ..., 30}, %to customize the axis
                %xticklabel = {0, 5, 10, 15}
                legend style ={
                	at={(1.25, 0.8)},
                	nodes={scale=0.85, transform shape},
                	font = #3
                }
                ]
                % Reverse the input order, so that 7.5mm away is sent to background
                \input{#8} 
                \input{#7}
                \input{#6}
                \input{#5}
                \input{#4}
                
             % legend
             % To insert the legend title
		   	\addlegendimage{empty legend} 
		   	% Legend entries  
             \addlegendentry{\SI{7.5}{\milli\metre}}
             \addlegendentry{\SI{5}{\milli\metre}}
             \addlegendentry{$1.98 \lambda$}
             \addlegendentry{$0.8 \lambda$}
             \addlegendentry{$0.5 \lambda$}
             % Title
             \addlegendentry{$| s_{x} - x |$}
                         
            \end{axis}
	\end{tikzpicture}
	}
}

%% GD PE animation
\newcommand{\gdpeanimate}[6]{ % <scale size>, <label font size>, <tick font size>, <fname for offset coord>, <fname for approx coord>
\scalebox{#1}{
	\begin{tikzpicture}
            \begin{axis}[
                width = 8cm, height = 6.3cm,
            	   xmin = -2.2, xmax = 2.2,
            	   ymin = -5, ymax = 8.5,
            	   % labels and ticks
                xlabel = {$\xdelta / \lambda$},
                ylabel = {$\xdeltaopt / \lambda$}, % \Delta \pp_{\optimized}
                label style = {font = #2},
                tick label style = {font = #3},
                %y dir = reverse,
                ytick = {-5, 0, 2, 5}, %to customize the axis
                extra x ticks={0.76}, 
                extra x tick style={font = #2, yshift={-1em}},
                ]
                
               % GD PE values
                \only<3->{\input{#4}} % 1mm
                \only<4->{\input{#5}} % 2.5mm
                \only<5->{\input{#6}} % 5mm
             
             % y = 2 line
               \only<3>{
	            \addplot[gray, dashed, line width = 2pt, mark = ] coordinates{
	            			(-2.2, 2)
	            			(1.95, 2)
            		};
            		}
                          
             % x = 0.76 line
               \only<3>{
	            \addplot[gray, dashed, line width = 2pt, mark = ] coordinates{
	            			(0.76, 0)
	            			(0.76, -5)
            		};
            		}
             
            \end{axis}
	\end{tikzpicture}
	}
}

% GD SE animation
\newcommand{\gdseanimate}[6]{ % <scale size>, <label font size>, <tick font size>, <fname for offset coord>, <fname for approx coord>
\scalebox{#1}{
	\begin{tikzpicture}
            \begin{axis}[
                width = 8cm, height = 6.3cm,
            	   xmin = -2.2, xmax = 2.2,
            	   ymin = -0.1, ymax = 1.1,
            	   % Labels and ticks 
                xlabel = {$\xdelta  / \lambda$},
                ylabel = {$\SE^{\dagger}$},
                label style = {font = #2},
                tick label style = {font = #3},
                %y dir = reverse,
                %xtick = {0, 10, ..., 30}, %to customize the axis
                %xticklabel = {0, 5, 10, 15}
                extra x ticks={0.8}, 
                extra x tick style={font = #2, yshift={-1em}},
                ]
                
                % GD SE values
                \input{#4} % 1mm
                \only<2->{\input{#5}} % 2.5mm
                \only<3->{\input{#6}} % 5mm
             
             % x = 0.8 line
               \only<1>{
	            \addplot[gray, dashed, line width = 2pt, mark = ] coordinates{
	            			(0.8, 0.2)
	            			(0.8, -0.1)
            		};
            		}
                         
            \end{axis}
	\end{tikzpicture}
	}
}


%% GD PE animation FULL
\newcommand{\gdpeanimatefull}[8]{ % <scale size>, <label font size>, <tick font size>, <fname for offset coord>, <fname for approx coord>
\scalebox{#1}{
	\begin{tikzpicture}
            \begin{axis}[
                width = 12cm, 
            	   height = 6.7cm,
            	   xmin = -2.2,
            	   xmax = 2.2,
            	   ymin = -5,
            	   ymax = 8.5,
            	   % labels and ticks
                xlabel = {$\ppdelta_{\track} / \lambda$},
                ylabel = {$\Delta \pp_{\optimized} / \lambda$}, % \Delta \pp_{\optimized}
                label style = {font = #2},
                tick label style = {font = #3},
                %y dir = reverse,
                %xtick = {0, 10, ..., 30}, %to customize the axis
                %xticklabel = {0, 5, 10, 15}
                extra x ticks={0.76}, 
                extra x tick style={font = #2, yshift={-1.2em}},
                extra y ticks={1}, 
                extra y tick style={font = #2, xshift={-1.2em}},
                % legend
                legend style ={
                	at={(1.25, 0.8)},
                	nodes={scale=0.85, transform shape},
                	font = #3
                }
                ]
                
               % GD PE values
                \input{#4}
                \only<2->{\input{#5}}
                \only<3->{\input{#6}}
                \only<4->{\input{#7}}
                \only<5->{\input{#8}}
                  
             % legend
             % To insert the legend title
		   	\addlegendimage{empty legend} 
		   	% Legend entries  
		   	\addlegendentry{$0.5 \lambda$}
		   	\only<2->{\addlegendentry{$0.8 \lambda$} }
		   	\only<3->{\addlegendentry{$1.98 \lambda$} }
		   	\only<4->{\addlegendentry{\SI{5}{\milli\metre}} }		   	
             \only<5->{\addlegendentry{\SI{7.5}{\milli\metre}} }
             
             % Title
             \addlegendentry{$\| \pp - \scatterer \|_{2}$}
             
             % y = 1 line
               \only<1>{
	            \addplot[gray, dashed, line width = 2pt, mark = ] coordinates{
	            			(-2.2, 1)
	            			(1.5, 1)
            		};
            		}
                          
             % x = 0.76 line
               \only<2>{
	            \addplot[gray, dashed, line width = 2pt, mark = ] coordinates{
	            			(0.76, 0)
	            			(0.76, -5)
            		};
            		}
             
            \end{axis}
	\end{tikzpicture}
	}
}

% GD SE animation FULL
\newcommand{\gdseanimatefull}[8]{ % <scale size>, <label font size>, <tick font size>, <fname for offset coord>, <fname for approx coord>
\scalebox{#1}{
	\begin{tikzpicture}
            \begin{axis}[
                width = 12cm, 
            	   height = 6.7cm,
            	   xmin = -2.2,
            	   xmax = 2.2,
            	   ymin = -0.1,
            	   ymax = 1.1,
            	   % Labels and ticks 
                xlabel = {$\ppdelta_{\track} / \lambda$},
                ylabel = {$\SE^{\dagger}$},
                label style = {font = #2},
                tick label style = {font = #3},
                %y dir = reverse,
                %xtick = {0, 10, ..., 30}, %to customize the axis
                %xticklabel = {0, 5, 10, 15}
                extra x ticks={0.8}, 
                extra x tick style={font = #2, yshift={-1.2em}},
                % Legend
                legend style ={
                	at={(1.25, 0.8)},
                	nodes={scale=0.85, transform shape},
                	font = #3
                }
                ]
                
                % GD SE values
                \input{#4}
                \only<2->{\input{#5}}
                \only<3->{\input{#6}}
                \only<4->{\input{#7}}
                \only<5->{\input{#8}}
                  
             % legend
             % To insert the legend title
		   	\addlegendimage{empty legend} 
		   	% Legend entries  
		   	\addlegendentry{$0.5 \lambda$}
		   	\only<2->{\addlegendentry{$0.8 \lambda$} }
		   	\only<3->{\addlegendentry{$1.98 \lambda$} }
		   	\only<4->{\addlegendentry{\SI{5}{\milli\metre}} }		   	
             \only<5->{\addlegendentry{\SI{7.5}{\milli\metre}} }
             
             % Title
             \addlegendentry{$\| \pp - \scatterer \|_{2}$}
             
             % x = 0.8 line
               \only<2>{
	            \addplot[gray, dashed, line width = 2pt, mark = ] coordinates{
	            			(0.8, 0.2)
	            			(0.8, -0.1)
            		};
            		}
                         
            \end{axis}
	\end{tikzpicture}
	}
}

%%%%%%%%%%%%%%%%%%%%%%%%%%%%%%%%%%%%%%%%%%%%%%%%
%===================== 2D Visualization ======================%
%%%%%%%%%%%%%%%%%%%%%%%%%%%%%%%%%%%%%%%%%%%%%%%%

% cimg with both x- & y-labels
\newcommand{\cimgbothlabels}[4]{% <scale size>, <label font size>, <tick font size>, <png file name>
\scalebox{#1}{
	\begin{tikzpicture}
            \begin{axis}[
                enlargelimits = false,
                axis on top = true,
                axis equal image,
                point meta min = -1,   
                point meta max = 1,
                xlabel = {$x$ in \SI{}{\milli\meter}},
                ylabel = {$y$ in \SI{}{\milli\meter}},
                label style = {font = #2},
                tick label style = {font = #3},
                %y dir = reverse,
                %xtick = {0, 10, ..., 30}, %to customize the axis
                %xticklabel = {0, 5, 10, 15}
                ]
                \addplot graphics [
                    xmin = 0,
                    xmax = 20,
                    ymin = 0,
                    ymax = 20
                ]{#4};
            \end{axis}
	\end{tikzpicture}
	}
}


% for cimg of Aviation Scan 5-3
% important!!!! no numbers in the name of the command!!!!!!!!
\newcommand{\cimgAviationScanBothlabels}[4]{% <scale size>, <label font size>, <tick font size>, <png file name>
\scalebox{#1}{
	\begin{tikzpicture}
            \begin{axis}[
                enlargelimits = false,
                axis on top = true,
                axis equal image,
                point meta min = 0,   
                point meta max = 26.829943,
                xlabel = {$x$ in \SI{}{\centi\meter}},
                ylabel = {$y$ in \SI{}{\centi\meter}},
                label style = {font = #2},
                tick label style = {font = #3},
                %y dir = reverse,
                %xtick = {0, 20, ..., 100} to customize the axis
                %xticklabel = {0, 20, 40, 60, 80, 100}
                ]
                \addplot graphics [
        				xmin = 10.000000,
        				xmax = 15.735000,
        				ymin = 20.833333,
        				ymax = 28.774500
    				]{#4};
            \end{axis}
	\end{tikzpicture}
	}
}


% cimgs with both labels and cmaps for 10th lambda differences
\newcommand{\cimgfortenthlambda}[4]{% <scale size>, <label font size>, <tick font size>, <png file name>
\scalebox{#1}{
\begin{tikzpicture}
            \begin{axis}[
            	  enlargelimits = false,
                axis on top = true,
                axis equal image,
                point meta min = 0,   
                point meta max = 0.08,
                colorbar,
                colormap = {mymap}{rgb(0.0pt) = (0.97, 0.97, 0.98) ; rgb(1.0pt) = (0.76, 0.95, 0.91) ; },
                xlabel = {$x$ in \SI{}{\milli\meter}},
                ylabel = {$y$ in \SI{}{\milli\meter}},
                label style = {font = #2},
                tick label style = {font = #3},
                %y dir = reverse,
                ]
                %\addplot[tui_red, thick, mark = x] coordinates{
                %};
                \addplot graphics [
                    xmin = 0,
                    xmax = 20,
                    ymin = 0,
                    ymax = 20
                ]{#4};
            \end{axis}
            \end{tikzpicture}
            
	}
}


%%%%%%%%%%%%%%%%%%%%%%%%%%%%%%%%%%%%%%%%%%%%%%
%=================== 2D visualization ======================%
%%%%%%%%%%%%%%%%%%%%%%%%%%%%%%%%%%%%%%%%%%%%%%
% image with both x- & y-labels
\newcommand{\imgbothlabels}[4]{% <scale size>, <label font size>, <tick font size>, <png file name>
\scalebox{#1}{
	\begin{tikzpicture}
            \begin{axis}[
            	   width = 5.5cm, 
            	   %height = 6.7cm,
                enlargelimits = false,
                axis on top = true,
                axis equal image,
                unit vector ratio= 0.3 1, % change aspect ratio, one of them should be 1
                point meta min = -1,   
                point meta max = 1,
                xlabel = {$x$ [\SI{}{\milli \metre}]},
                ylabel = {$z$ [\SI{}{\milli \metre}]},
                label style = {font = #2},
                tick label style = {font = #3},
                xlabel style = {yshift = 0.3cm},
                ylabel style = {yshift = -0.3cm},
                y dir = reverse,
                xtick = {5, 10, 15},
                ytick = {21, 22.5, 24},
                ]
                \addplot graphics [
                    xmin = 5,
                    xmax = 15,
                    ymin = 20.908125,
                    ymax = 24.4125
                ]{#4};
            \end{axis}
	\end{tikzpicture}
	}
}

% img with both labels and cmap
\newcommand{\imgbothlabelswithcmap}[4]{% <scale size>, <label font size>, <tick font size>, <png file name>
\scalebox{#1}{
	\begin{tikzpicture}
            \begin{axis}[
                %width = 5.5cm, 
            	   %height = 6.7cm,
                enlargelimits = false,
                axis on top = true,
                axis equal image,
                unit vector ratio= 0.3 1, % change aspect ratio, one of them should be 1
                point meta min = -1,   
                point meta max = 1,
                xlabel = {$x$ [\SI{}{\milli \metre}]},
                ylabel = {$z$ [\SI{}{\milli \metre}]},
                label style = {font = #2},
                tick label style = {font = #3},
                y dir = reverse,
                xtick = {5, 10, 15},
                ytick = {21, 22.5, 24},
                ]
                \addplot graphics [
                    xmin = 5,
                    xmax = 15,
                    ymin = 20.908125,
                    ymax = 24.4125
                ]{#4};
            \end{axis}
	\end{tikzpicture}
	}
}
